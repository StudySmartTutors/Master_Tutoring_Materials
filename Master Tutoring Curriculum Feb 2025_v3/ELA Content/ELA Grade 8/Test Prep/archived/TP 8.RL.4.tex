\documentclass[12pt]{article}

\usepackage[a4paper, top=0.8in, bottom=0.7in, left=0.7in, right=0.7in]{geometry}

\usepackage{amsmath}
\usepackage{graphicx}
\usepackage{fancyhdr}
\usepackage{tcolorbox}
\usepackage{multicol}
\usepackage{pifont} % For checkboxes
\usepackage[defaultfam,tabular,lining]{montserrat} %% Option 'defaultfam'
\usepackage[T1]{fontenc}
\renewcommand*\oldstylenums[1]{{\fontfamily{Montserrat-TOsF}\selectfont #1}}
\renewcommand{\familydefault}{\sfdefault}
\usepackage{enumitem}
\usepackage{setspace}
\usepackage{parcolumns}
\usepackage{tabularx}

\setlength{\parindent}{0pt}
\hyphenpenalty=10000
\exhyphenpenalty=10000

\pagestyle{fancy}
\fancyhf{}
\fancyhead[L]{\textbf{8.RL.4: Understanding Word Meaning and Usage}}
\fancyhead[R]{\includegraphics[width=1cm]{Round Logo.png}}
\fancyfoot[C]{\footnotesize Study Smart Tutors}

\begin{document}

\onehalfspacing

% Passage
\subsection*{Passage: Running Late for School}

\begin{tcolorbox}[colframe=black!40, colback=gray!5]
\begin{spacing}{1.15}
    Mia woke up with a start, her heart pounding in her chest. The sun was already high in the sky, and she had overslept. Her alarm clock, which she had forgotten to set the night before, had failed her. Panicking, she jumped out of bed, quickly dressing in whatever clothes she could grab from her closet. The morning was a blur as she raced through the house, grabbing her backpack, and trying to eat something—anything—before running out the door.

    By the time she reached the bus stop, the bus was already pulling away. "No!" she thought, watching it disappear down the street. She was going to be late. Again.

    As she walked to school, Mia’s mind raced with excuses she could give her teachers. She had been trying to be on time, but it seemed like something always went wrong. Her thoughts were jumbled and filled with frustration, yet she knew that she needed to stay calm. The morning was already a mess; getting upset wouldn’t help.

    When she finally reached school, she rushed to her first class, hoping that her teacher would understand. She was late, but she wasn’t sure how much of an excuse she could give. It wasn’t as if she had slept in on purpose. But she had learned by now that excuses didn’t matter much—what mattered was making sure she got everything done.

    As she sat down at her desk, Mia couldn’t help but reflect on the mornings she had wasted. She promised herself that the next day would be different.
\end{spacing}
\end{tcolorbox}

% Worksheet Questions
\subsection*{Questions}

\begin{enumerate}

    \item What does the phrase "heart pounding in her chest" suggest about Mia’s feelings?
    \begin{enumerate}[label=\Alph*.]
        \item She is excited.
        \item She is scared and anxious.
        \item She is confused.
        \item She is happy.
    \end{enumerate}
    \vspace{0.5cm}

    \item What is the meaning of the word "panicking" in the context of the story?
    \begin{enumerate}[label=\Alph*.]
        \item Calm and relaxed.
        \item Frightened and worried.
        \item Angry and upset.
        \item Happy and excited.
    \end{enumerate}
    \vspace{0.5cm}

    \item Which of the following is a synonym for the word "blur" in the sentence, "The morning was a blur"?
    \begin{enumerate}[label=\Alph*.]
        \item Clear
        \item Hazy
        \item Fun
        \item Bright
    \end{enumerate}
    \vspace{0.5cm}

    \item What does Mia mean when she says, "The morning was already a mess"?
    \begin{enumerate}[label=\Alph*.]
        \item The morning went perfectly.
        \item Her morning had a lot of problems.
        \item She had a very fun morning.
        \item She had everything planned perfectly.
    \end{enumerate}
    \vspace{0.5cm}

    \item What is the meaning of "jumbled" in the sentence, "Her thoughts were jumbled"?
    \begin{enumerate}[label=\Alph*.]
        \item Calm and clear.
        \item Mixed up and confused.
        \item Organized and focused.
        \item Happy and excited.
    \end{enumerate}
    \vspace{0.5cm}

    \item What does Mia’s reflection on her mornings tell us about her character?
    \begin{enumerate}[label=\Alph*.]
        \item She is always late and doesn’t care.
        \item She is aware of her mistakes and wants to improve.
        \item She doesn’t mind being late.
        \item She likes making excuses for being late.
    \end{enumerate}
    \vspace{0.5cm}

    \item Which of the following phrases is an example of figurative language from the passage?
    \begin{enumerate}[label=\Alph*.]
        \item "Her thoughts were jumbled."
        \item "The bus was already pulling away."
        \item "She was late, but she wasn’t sure how much of an excuse she could give."
        \item "The sun was already high in the sky."
    \end{enumerate}
    \vspace{0.5cm}

    \item How does the setting (Mia’s home) affect the plot?
    \begin{enumerate}[label=\Alph*.]
        \item Mia is calm and relaxed in her home.
        \item The home is where Mia starts her morning, and it sets up her rush to school.
        \item Mia spends the entire day at home.
        \item The home is where Mia reflects on her day.
    \end{enumerate}
    \vspace{0.5cm}

    \item What does Mia hope for when she gets to school?
    \begin{enumerate}[label=\Alph*.]
        \item She hopes that her teachers will be angry with her.
        \item She hopes that her teacher will understand why she is late.
        \item She hopes to sleep through the entire class.
        \item She hopes to be late every day.
    \end{enumerate}
    \vspace{0.5cm}

    \item What does the phrase “excuses didn’t matter much” imply about Mia’s attitude?
    \begin{enumerate}[label=\Alph*.]
        \item She believes that excuses are the best way to solve problems.
        \item She understands that actions are more important than explanations.
        \item She is always able to make excuses and avoid responsibility.
        \item She doesn’t care about being late.
    \end{enumerate}
    \vspace{0.5cm}

    \item What does the word "reflection" mean in the context of the story?
    \begin{enumerate}[label=\Alph*.]
        \item Thinking deeply about something.
        \item Looking at yourself in a mirror.
        \item A physical act of seeing your reflection.
        \item Running late to school.
    \end{enumerate}
    \vspace{0.5cm}

    \item Why does Mia think "getting upset wouldn’t help"?
    \begin{enumerate}[label=\Alph*.]
        \item She believes getting upset would make the situation worse.
        \item She loves to get upset and make things harder.
        \item She enjoys being late to school.
        \item She likes to create problems for herself.
    \end{enumerate}
    \vspace{0.5cm}

    \item Which of the following best describes the tone of the passage?
    \begin{enumerate}[label=\Alph*.]
        \item Cheerful and excited.
        \item Serious and frustrated.
        \item Calm and peaceful.
        \item Sad and depressing.
    \end{enumerate}
    \vspace{0.5cm}

    \item In what way does the setting of Mia's school contribute to the story?
    \begin{enumerate}[label=\Alph*.]
        \item It is where Mia hopes to be forgiven for being late.
        \item It makes Mia feel more stressed about being late.
        \item Mia feels that school is a place where she cannot be herself.
        \item Mia feels calm and at peace when she reaches school.
    \end{enumerate}
    \vspace{0.5cm}

    \item What is the main lesson Mia learns at the end of the passage?
    \begin{enumerate}[label=\Alph*.]
        \item That waking up early doesn’t matter.
        \item That making excuses is always okay.
        \item That it’s important to plan ahead and avoid making the same mistakes.
        \item That being late every day is fine.
    \end{enumerate}
\end{enumerate}

\end{document}
