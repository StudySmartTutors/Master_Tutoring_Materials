\documentclass[12pt]{article}

\usepackage[a4paper, top=0.8in, bottom=0.7in, left=0.7in, right=0.7in]{geometry}

\usepackage{amsmath}
\usepackage{graphicx}
\usepackage{fancyhdr}
\usepackage{tcolorbox}
\usepackage{multicol}
\usepackage{pifont} % For checkboxes
\usepackage[defaultfam,tabular,lining]{montserrat} %% Option 'defaultfam'
\usepackage[T1]{fontenc}
\renewcommand*\oldstylenums[1]{{\fontfamily{Montserrat-TOsF}\selectfont #1}}
\renewcommand{\familydefault}{\sfdefault}
\usepackage{enumitem}
\usepackage{setspace}
\usepackage{parcolumns}
\usepackage{tabularx}

\setlength{\parindent}{0pt}
\hyphenpenalty=10000
\exhyphenpenalty=10000

\pagestyle{fancy}
\fancyhf{}
\fancyhead[L]{\textbf{8.RL.2: Analyzing Text Structure}}
\fancyhead[R]{\includegraphics[width=1cm]{Round Logo.png}}
\fancyfoot[C]{\footnotesize Study Smart Tutors}

\begin{document}

\onehalfspacing

% Passage
\subsection*{Passage: The Nightmare}

\begin{tcolorbox}[colframe=black!40, colback=gray!5]
\begin{spacing}{1.15}
    Sarah awoke with a start, her heart racing. She had been dreaming—no, *living*—in a nightmare. The room was pitch dark, except for the faint glow of moonlight coming through the window. In her dream, she was running through a dense forest, the trees stretching high above her. She could hear footsteps behind her, too close, too fast. A shadow loomed in the distance, getting larger and larger until it was almost upon her.

    She tried to scream, but no sound came out. Her legs felt heavy, as if they were made of stone. The faster she ran, the slower she seemed to move. The shadow grew bigger, and she could feel its cold breath on the back of her neck. Just as it was about to touch her, she woke up, gasping for air.

    Her room was silent. The darkness seemed less threatening now, but the feeling of dread still clung to her. She sat up in bed, trying to shake off the remnants of the nightmare. It had felt so real, so terrifying. She could still hear the faint sound of footsteps, echoing in her mind. Sarah shuddered and pulled the covers tightly around her, hoping the feeling of fear would disappear with the light of day. But the image of the shadow lingered, like a dark stain in her memory, refusing to fade.
\end{spacing}
\end{tcolorbox}

% Worksheet Questions
\subsection*{Questions}

\begin{enumerate}

    \item What is the main conflict in the story?
    \begin{enumerate}[label=\Alph*.]
        \item Sarah is lost in a forest.
        \item Sarah is being chased by a shadow in her nightmare.
        \item Sarah is trying to fall back asleep.
        \item Sarah is afraid of the dark.
    \end{enumerate}
    \vspace{0.5cm}

    \item What is the significance of the darkness in the room when Sarah wakes up?
    \begin{enumerate}[label=\Alph*.]
        \item It suggests that Sarah is still trapped in her nightmare.
        \item It represents the fear Sarah feels after the nightmare.
        \item It symbolizes the forest from her dream.
        \item It shows that the nightmare is over and there is nothing to fear.
    \end{enumerate}
    \vspace{0.5cm}

    \item How does Sarah’s feeling of fear change throughout the passage?
    \begin{enumerate}[label=\Alph*.]
        \item Her fear intensifies as she wakes up.
        \item Her fear decreases when she wakes up and realizes it was just a dream.
        \item Her fear remains the same throughout the passage.
        \item Her fear disappears as soon as she wakes up.
    \end{enumerate}
    \vspace{0.5cm}

    \item What does the shadow represent in the story?
    \begin{enumerate}[label=\Alph*.]
        \item A real person chasing Sarah.
        \item Sarah’s anxiety or fear.
        \item The moonlight coming through the window.
        \item A friend trying to help Sarah escape.
    \end{enumerate}
    \vspace{0.5cm}

    \item What is the effect of the imagery of “trees stretching high above her” in the nightmare?
    \begin{enumerate}[label=\Alph*.]
        \item It shows that Sarah is trapped in the forest.
        \item It makes the forest feel vast and overwhelming.
        \item It emphasizes the safety of Sarah’s surroundings.
        \item It makes Sarah feel calm and secure.
    \end{enumerate}
    \vspace{0.5cm}

    \item How does the structure of the passage help build suspense?
    \begin{enumerate}[label=\Alph*.]
        \item It starts with Sarah waking up, leaving the reader unsure if she is still dreaming.
        \item The passage reveals the nightmare in its entirety before Sarah wakes up.
        \item It provides Sarah’s feelings of relief before describing the nightmare.
        \item It alternates between Sarah’s dream and reality, making the reader unsure of what’s real.
    \end{enumerate}
    \vspace{0.5cm}

    \item Why is the detail “she could hear footsteps behind her, too close, too fast” important?
    \begin{enumerate}[label=\Alph*.]
        \item It shows that Sarah is being chased in the nightmare.
        \item It reveals that Sarah is running away from someone.
        \item It indicates that Sarah is alone and can’t escape.
        \item It demonstrates Sarah’s ability to move quickly.
    \end{enumerate}
    \vspace{0.5cm}

    \item What is the impact of the sentence “Just as it was about to touch her, she woke up”?
    \begin{enumerate}[label=\Alph*.]
        \item It provides a sense of resolution and safety for Sarah.
        \item It builds suspense by delaying the resolution.
        \item It marks the climax of Sarah’s nightmare.
        \item It shows that Sarah is no longer afraid.
    \end{enumerate}
    \vspace{0.5cm}

    \item What role does Sarah’s inability to scream in the nightmare play in the story?
    \begin{enumerate}[label=\Alph*.]
        \item It makes the nightmare more frightening by emphasizing her helplessness.
        \item It shows that Sarah is not afraid of the shadow.
        \item It reveals that Sarah is trying to fight the shadow.
        \item It symbolizes Sarah’s desire to escape her fears.
    \end{enumerate}
    \vspace{0.5cm}

    \item What does the phrase “the image of the shadow lingered, like a dark stain in her memory” suggest about Sarah’s experience?
    \begin{enumerate}[label=\Alph*.]
        \item Sarah has already forgotten about the nightmare.
        \item The nightmare will continue to haunt her even after she wakes up.
        \item The nightmare was not important to Sarah.
        \item Sarah is no longer afraid of the shadow.
    \end{enumerate}
    \vspace{0.5cm}

    \item What does the detail “she could still hear the faint sound of footsteps” suggest about Sarah’s state of mind?
    \begin{enumerate}[label=\Alph*.]
        \item She is still hearing things from her dream and is not fully awake.
        \item She feels safe now that the nightmare is over.
        \item She has become numb to the fear.
        \item She is imagining the footsteps in the dark.
    \end{enumerate}
    \vspace{0.5cm}

    \item How does the passage suggest that Sarah is trying to escape her fear?
    \begin{enumerate}[label=\Alph*.]
        \item She attempts to scream but cannot.
        \item She pulls the covers tightly around herself, trying to feel safe.
        \item She ignores the nightmare and goes back to sleep.
        \item She tries to convince herself that the nightmare was not real.
    \end{enumerate}
    \vspace{0.5cm}

    \item What does the description of the “faint glow of moonlight” contribute to the atmosphere of the passage?
    \begin{enumerate}[label=\Alph*.]
        \item It creates a sense of warmth and comfort.
        \item It makes the darkness in the room feel more ominous and unsettling.
        \item It makes Sarah feel safer in her room.
        \item It suggests that Sarah is dreaming and not awake.
    \end{enumerate}
    \vspace{0.5cm}

    \item Why is the structure of alternating between Sarah’s nightmare and reality important?
    \begin{enumerate}[label=\Alph*.]
        \item It allows the reader to see the difference between what’s real and what’s imagined.
        \item It makes the story easier to follow.
        \item It provides a sense of closure at the end of the passage.
        \item It suggests that Sarah will wake up from the nightmare at the end.
    \end{enumerate}
    \vspace{0.5cm}

    \item What lesson can be inferred from Sarah’s experience in the passage?
    \begin{enumerate}[label=\Alph*.]
        \item That nightmares are always a sign of real danger.
        \item That fear can linger even after waking up from a nightmare.
        \item That Sarah should not sleep at night.
        \item That nightmares are not important and should be ignored.
    \end{enumerate}

\end{enumerate}

\end{document}
