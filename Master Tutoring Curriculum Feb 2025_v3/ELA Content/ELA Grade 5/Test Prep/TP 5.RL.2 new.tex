\documentclass[12pt]{article}

\usepackage[a4paper, top=0.8in, bottom=0.7in, left=0.7in, right=0.7in]{geometry}
\usepackage{amsmath}
\usepackage{graphicx}
\usepackage{fancyhdr}
\usepackage{tcolorbox}
\usepackage{multicol}
\usepackage{pifont} % For checkboxes
\usepackage[defaultfam,tabular,lining]{montserrat} %% Option 'defaultfam'
\usepackage[T1]{fontenc}
\renewcommand*\oldstylenums[1]{{\fontfamily{Montserrat-TOsF}\selectfont #1}}
\renewcommand{\familydefault}{\sfdefault}
\usepackage{enumitem}
\usepackage{setspace}
\usepackage{parcolumns}
\usepackage{tabularx}

\setlength{\parindent}{0pt}
\hyphenpenalty=10000
\exhyphenpenalty=10000

\pagestyle{fancy}
\fancyhf{}
%\fancyhead[L]{\textbf{5.RL.2: Theme and Summary Practice}}
\fancyhead[R]{\includegraphics[width=1cm]{Round Logo.png}}
\fancyfoot[C]{\footnotesize Study Smart Tutors}

\begin{document}

\subsection*{Theme and Summary Assessment}
\onehalfspacing

\begin{tcolorbox}[colframe=black!40, colback=gray!0, title=Learning Objective]
\textbf{Objective:} Determine the theme of a story, drama, or poem and summarize the text.
\end{tcolorbox}

\subsection*{Part 1: Multiple-Choice Questions}

1. \textbf{What is the theme of the following story?\\}
"Rita loved to win, but she often avoided games if she thought she might lose. One day, her teacher announced a school-wide board game tournament. Rita hesitated to sign up, fearing embarrassment if she failed. However, after encouragement from her best friend, she decided to give it a try. During the tournament, Rita lost several games but was surprised by how much fun she had. She met new friends, learned new strategies, and found herself enjoying the experience of playing rather than focusing solely on winning. By the end, Rita didn’t win the tournament, but she felt proud of herself for stepping outside her comfort zone. Her classmates cheered for her perseverance, and her teacher praised her for her effort. Rita realized that trying new things and enjoying the journey was more important than the outcome. Her experience taught her the value of personal growth through challenges."\\
\begin{enumerate}[label=\Alph*.]
    \item Winning is everything.  
    \item It’s okay to avoid challenges.  
    \item Trying new things can lead to personal growth.  
    \item Friends are not important.  
\end{enumerate}

\vspace{1em}
\newpage
2. \textbf{Which detail supports the theme of the passage below?}\\
"A forest community thrived for decades, with animals living peacefully. However, one year, a drought struck, drying up water sources and leaving food scarce. The animals faced challenges but decided to work together to survive. The elephants dug deeper into dried-up riverbeds to uncover hidden water, which they shared with the smaller animals. Birds flew long distances to find seeds and nuts, bringing food back for the community. Predators even agreed to a temporary truce, ensuring peace during the difficult time. By working together, the animals made it through the drought and learned the importance of cooperation. When the rains finally returned, the animals celebrated their resilience with a festival. They understood that kindness and teamwork had helped them overcome a nearly impossible \\challenge. The story shows how communities can thrive when they unite during hardships, demonstrating the value of collaboration and mutual support."\\
\begin{enumerate}[label=\Alph*.]
    \item The animals hoarded resources.  
    \item The drought caused division among the animals.  
    \item The animals survived by working together.  
    \item The animals moved to a new forest.  
\end{enumerate}

\vspace{1em}

3. \textbf{What theme is shown in the following story?\\}
"Luis spent all his time playing video games and ignoring his homework. At first, he didn’t see a problem—his grades were average, and he enjoyed the thrill of leveling up in his favorite game. However, as assignments piled up, his grades began to drop, and his parents grew concerned. When Luis failed an important math test, he realized he needed to make a change. With help from his parents, Luis created a study schedule and started asking his teacher for extra help. It was tough at first, but Luis stuck to the plan. Over time, his grades improved, and he felt proud of his progress. By the end of the school year, Luis balanced his hobbies with his \\responsibilities. He learned that hard work, discipline, and asking for help could lead to success. His experience taught him the value of making responsible choices and staying committed to his goals."\\
\begin{enumerate}[label=\Alph*.]
    \item Hard work and responsibility lead to success.  
    \item Video games are harmful to students.  
    \item Failing tests is inevitable.  
    \item Parents should not be involved in homework.  
\end{enumerate}

\vspace{1cm}

\subsection*{Part 2: Select All That Apply Questions}

4. Select \textbf{all} themes demonstrated in the story about the animals during the drought from question 2:\\
\begin{enumerate}[label=\Alph*.]
    \item Kindness makes communities stronger.  
    \item Teamwork helps overcome challenges.  
    \item Selfishness leads to success.  
    \item Sharing resources benefits everyone.  
\end{enumerate}

\vspace{1cm}

5. What details from the story about Luis in question 3 support the theme of \\responsibility?\\
\begin{enumerate}[label=\Alph*.]
    \item Luis ignored his homework.  
    \item Luis created a study schedule with his parents.  
    \item Luis improved his grades with hard work.  
    \item Luis decided to stop playing video games.  
\end{enumerate}

\vspace{1cm}

6. What lessons can be learned from Rita’s experience in the board game \\tournament from question 1?\\
\begin{enumerate}[label=\Alph*.]
    \item Winning is not the only way to grow.  
    \item Personal growth happens when you avoid challenges.  
    \item Perseverance is more important than the outcome.  
    \item Trying new things can lead to unexpected joy.  
\end{enumerate}

\vspace{1cm}
\newpage
\subsection*{Part 3: Short Answer Questions}

7. How did teamwork help the animals in the story from question 2 survive the drought? Use details from the text.\\
\vspace{4cm}

8. Summarize the story about Luis from question 3. What lesson can readers learn from his experience?\\
\vspace{4cm}

\subsection*{Part 4: Fill in the Blank Questions}
\vspace{1em
}
9. A story’s \underline{\hspace{4cm}} teaches an important lesson or message about life.

\vspace{3cm}

10. Summarizing a text means identifying the \underline{\hspace{4cm}} and key details.

\vspace{3cm}
\newpage
% \section*{Answer Key}

% \subsection*{Part 1: Multiple-Choice Questions}

% 1. **C.** Trying new things can lead to personal growth.  

% 2. **C.** The animals survived by working together.  

% 3. **A.** Hard work and responsibility lead to success.  

% \subsection*{Part 2: Select All That Apply Questions}

% 4. **A, B, D.**  
%    - Kindness makes communities stronger.  
%    - Teamwork helps overcome challenges.  
%    - Sharing resources benefits everyone.  

% 5. **B, C.**  
%    - Luis created a study schedule with his parents.  
%    - Luis improved his grades with hard work.  

% 6. **A, C, D.**  
%    - Winning is not the only way to grow.  
%    - Perseverance is more important than the outcome.  
%    - Trying new things can lead to unexpected joy.  

% \subsection*{Part 3: Short Answer Questions}

% 7. **Sample Answer:** Teamwork helped the animals survive by allowing them to share resources and support each other. Elephants dug for water, birds brought back seeds, and predators agreed to a truce. These cooperative efforts ensured the community’s survival during the drought.  

% 8. **Sample Answer:** Luis initially ignored his homework, which caused his grades to drop. With the help of his parents, he created a study schedule and worked hard to improve his grades. Luis balanced his hobbies and responsibilities, learning that hard work and discipline lead to success.  

% \subsection*{Part 4: Fill in the Blank Questions}

% 9. A story’s \underline{theme} teaches an important lesson or message about life.

% 10. Summarizing a text means identifying the \underline{main idea} and key details.

\end{document}
