\documentclass[12pt]{article}

\usepackage[a4paper, top=0.8in, bottom=0.7in, left=0.7in, right=0.7in]{geometry}
\usepackage{amsmath}
\usepackage{graphicx}
\usepackage{fancyhdr}
\usepackage{tcolorbox}
\usepackage[defaultfam,tabular,lining]{montserrat} %% Option 'defaultfam'
\usepackage[T1]{fontenc}
\renewcommand*\oldstylenums[1]{{\fontfamily{Montserrat-TOsF}\selectfont #1}}
\renewcommand{\familydefault}{\sfdefault}
\usepackage{enumitem}
\usepackage{setspace}

\setlength{\parindent}{0pt}
\hyphenpenalty=10000
\exhyphenpenalty=10000

\pagestyle{fancy}
\fancyhf{}
%\fancyhead[L]{\textbf{3.W.2: Informative Writing Practice}}
\fancyhead[R]{\includegraphics[width=1cm]{Round Logo.png}}
\fancyfoot[C]{\footnotesize Study Smart Tutors}

\begin{document}

\subsection*{Informative Writing: Why Do Animals Hibernate?}
\onehalfspacing

\begin{tcolorbox}[colframe=black!40, colback=gray!0, title=Learning Objective]
\textbf{Objective:} Write an informative text that examines a topic, includes facts and details, and explains the information clearly.
\end{tcolorbox}

\subsection*{Prompt}

After reading the passages below, write an informative text in response to the \\question:  
"Why do animals hibernate?"  
Use details from the passages to \\explain what hibernation is and why animals do it.

\subsection*{Passage 1: What Is Hibernation?}

Hibernation is a special way some animals survive during the winter. When it gets cold and food is hard to find, animals like bears, bats, and turtles go into hibernation. During hibernation, an animal's body slows down. Their heartbeat and breathing become much slower, and they use less energy. This helps them stay alive without eating for a long time. Before hibernating, animals eat a lot of food to store fat in their bodies. This fat gives them energy while they rest. Hibernation is like a long, deep sleep that helps animals survive until spring, when food is easier to find.

\subsection*{Passage 2: Why Do Animals Hibernate?}

Animals hibernate to stay safe and survive when it is cold. In winter, snow covers the ground, and plants stop growing, making it hard for animals to find food. Some animals, like squirrels, store food in their nests to eat later. Others, like bears, hibernate so they don’t have to search for food in the cold. Hibernation also keeps animals safe from the freezing weather. Instead of facing the cold, animals like frogs and turtles hide in warm places and hibernate until spring. Hibernation is an important way for animals to stay alive during the hardest time of the year.
\newpage
\subsection*{Instructions for Students}

\begin{enumerate}
    \item **Understand the question.** Explain why animals hibernate using details from the passages.
    \item **Plan your writing.** Organize your ideas and include:
    \begin{itemize}
        \item An introduction that explains what hibernation is.
        \item Details from the texts that describe why animals hibernate.
        \item A conclusion that explains why hibernation is important for animals.
    \end{itemize}
    \item **Write your informative text.** Use complete sentences and explain the topic clearly.
    \item **Revise and edit.** Check your writing for spelling, grammar, and punctuation.
\end{enumerate}

\subsection*{Checklist for Success}

Make sure your writing includes:
\begin{itemize}
    \item A clear introduction explaining the topic.
    \item At least two facts supported by details from the passages.
    \item A conclusion that explains the importance of hibernation.
    \item Correct grammar, spelling, and punctuation.
\end{itemize}

\end{document}
