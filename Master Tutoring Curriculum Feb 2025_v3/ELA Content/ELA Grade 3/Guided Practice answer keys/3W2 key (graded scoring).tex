\documentclass[12pt]{article}
\usepackage[a4paper, top=0.8in, bottom=0.7in, left=0.8in, right=0.8in]{geometry}
\usepackage{amsmath}
\usepackage{amsfonts}
\usepackage{graphicx}
\usepackage{fancyhdr}
\usepackage{enumitem}
\usepackage{setspace}
\usepackage{tcolorbox}
\usepackage[defaultfam,tabular,lining]{montserrat}

\setlength{\parindent}{0pt}
\pagestyle{fancy}

\fancyhf{}
\fancyhead[L]{\textbf{Scoring Guide: Informative Writing}}
\fancyfoot[C]{\footnotesize \textcopyright Study Smart Tutors}

\begin{document}

\section*{Scoring Guide: Informative Writing Assignment}

This scoring guide evaluates student responses based on the Arizona Grade 3 Writing Standards. Each category is scored on a scale of \textbf{4} (Exceeds Expectations) to \textbf{1} (Does Not Meet Expectations).

\subsection*{Scoring Categories}

\begin{tcolorbox}[colframe=black!60, colback=white, title=Purpose, Focus, and Organization]
\textbf{Criteria:}
\begin{itemize}
    \item The response has a clear introduction, body, and conclusion.
    \item The writing is focused on the assigned topic.
    \item The organization enhances readability, using logical transitions between sections.
\end{itemize}
\textbf{Scoring Levels:}
\begin{itemize}[label=\textbullet]
    \item \textbf{4 (Exceeds Expectations):} The response is well-structured with a compelling introduction, logically organized body paragraphs, and a strong conclusion. Transitions are varied and enhance readability.
    \item \textbf{3 (Meets Expectations):} The response includes a clear introduction, body, and conclusion. Organization is effective with basic transitions between ideas.
    \item \textbf{2 (Approaching Expectations):} The response is partially organized but may lack a clear introduction, body, or conclusion. Transitions may be limited or repetitive.
    \item \textbf{1 (Does Not Meet Expectations):} The response lacks clear organization and transitions. It may be off-topic or incomplete.
\end{itemize}
\end{tcolorbox}

\begin{tcolorbox}[colframe=black!60, colback=white, title=Evidence and Elaboration]
\textbf{Criteria:}
\begin{itemize}
    \item The response provides accurate and relevant information from the sources.
    \item Details and examples are included to support main ideas.
    \item Linking words and phrases connect ideas and evidence effectively.
\end{itemize}
\textbf{Scoring Levels:}
\begin{itemize}[label=\textbullet]
    \item \textbf{4 (Exceeds Expectations):} The response uses multiple, specific details from both sources. Ideas are thoroughly explained and connected using varied and effective linking words.
    \item \textbf{3 (Meets Expectations):} The response includes details from both sources. Ideas are explained with adequate elaboration and appropriate linking words.
    \item \textbf{2 (Approaching Expectations):} The response includes limited or partially accurate details. Elaboration may be minimal, and linking words are basic or repetitive.
    \item \textbf{1 (Does Not Meet Expectations):} The response lacks relevant or accurate details. Elaboration is missing, and linking words are ineffective or absent.
\end{itemize}
\end{tcolorbox}

\begin{tcolorbox}[colframe=black!60, colback=white, title=Conventions]
\textbf{Criteria:}
\begin{itemize}
    \item The response demonstrates grade-level spelling, punctuation, and grammar.
    \item Sentences are complete and varied in structure.
\end{itemize}
\textbf{Scoring Levels:}
\begin{itemize}[label=\textbullet]
    \item \textbf{4 (Exceeds Expectations):} Minimal or no errors in spelling, punctuation, or grammar. Sentence structure is varied and enhances the writing.
    \item \textbf{3 (Meets Expectations):} Few errors in spelling, punctuation, or grammar that do not interfere with readability. Sentence structure is adequate.
    \item \textbf{2 (Approaching Expectations):} Frequent errors in spelling, punctuation, or grammar that may interfere with readability. Sentence structure is repetitive or awkward.
    \item \textbf{1 (Does Not Meet Expectations):} Errors in spelling, punctuation, or grammar significantly interfere with readability. Sentence structure is incomplete or incorrect.
\end{itemize}
\end{tcolorbox}

\subsection*{Final Score}
\begin{tcolorbox}[colframe=black!60, colback=white, title=Total Score Calculation]
Add the scores from the three categories to determine the total score. \textbf{Maximum Score: 12}
\begin{itemize}
    \item 10-12: Exceeds Expectations
    \item 7-9: Meets Expectations
    \item 4-6: Approaching Expectations
    \item 3 or below: Does Not Meet Expectations
\end{itemize}
\end{tcolorbox}

\subsection*{Teacher Notes}
\begin{itemize}
    \item Provide specific feedback for students scoring below \textbf{3} in any category.
    \item Highlight areas of strength to encourage improvement.
    \item Allow students scoring below \textbf{7} overall to revise and resubmit.
\end{itemize}

\end{document}
