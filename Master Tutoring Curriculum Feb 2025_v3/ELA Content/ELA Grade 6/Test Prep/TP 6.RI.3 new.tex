\documentclass[12pt]{article}

\usepackage[a4paper, top=0.8in, bottom=0.7in, left=0.7in, right=0.7in]{geometry}
\usepackage{amsmath}
\usepackage{graphicx}
\usepackage{fancyhdr}
\usepackage{tcolorbox}
\usepackage[defaultfam,tabular,lining]{montserrat} %% Option 'defaultfam'
\usepackage[T1]{fontenc}
\renewcommand*\oldstylenums[1]{{\fontfamily{Montserrat-TOsF}\selectfont #1}}
\renewcommand{\familydefault}{\sfdefault}
\usepackage{enumitem}
\usepackage{setspace}

\setlength{\parindent}{0pt}
\hyphenpenalty=10000
\exhyphenpenalty=10000

\pagestyle{fancy}
\fancyhf{}
%\fancyhead[L]{\textbf{6.RI.3: Analyzing Text Details Practice}}
\fancyhead[R]{\includegraphics[width=1cm]{Round Logo.png}}
\fancyfoot[C]{\footnotesize Study Smart Tutors}

\begin{document}

\subsection*{How Individuals, Events, and Ideas are Developed in Texts}
\onehalfspacing

\begin{tcolorbox}[colframe=black!40, colback=gray!0, title=Learning Objective]
\textbf{Objective:} Analyze how individuals, events, or ideas are introduced, illustrated, and elaborated in a text.
\end{tcolorbox}

\subsection*{Part 1: Multiple-Choice Questions}

1. \textbf{How does the text introduce the topic of using animals in circuses?\\}
"For centuries, animals have been a central feature of circuses, with lions, elephants, and bears performing tricks to entertain audiences. The text begins by describing the historical popularity of animal performances and their appeal to families. However, it transitions to the ethical concerns raised by animal rights organizations. Evidence is provided about the harsh conditions animals often face, such as confinement in small cages, lack of proper medical care, and the use of punishment to train them. Real-life examples, including investigations into well-known circuses, highlight these issues. The text also explores legal changes, such as bans on using wild animals in circuses in countries like the UK and India. Alternatives, such as human-only performances like Cirque du Soleil, are discussed as growing trends that still captivate audiences. By combining historical context, ethical debates, and recent changes, the text illustrates the evolving perception of animals in the circus industry."  
\begin{enumerate}[label=\Alph*.]
    \item By emphasizing the entertainment value of animal performances.  
    \item By discussing the ethical concerns and alternative circus models.  
    \item By focusing solely on the decline of circuses.  
    \item By presenting laws regulating circus performances.  
\end{enumerate}

\vspace{1cm}
\newpage
2. \textbf{How does the text elaborate on the harvesting of sea silk?\\}
"Sea silk, one of the rarest and most luxurious fibers in the world, is made from the silky filaments of the pen shell, a large marine mollusk. The text explains that harvesting sea silk requires diving to great depths to collect the shells, a dangerous task traditionally performed by skilled divers in the Mediterranean. Examples from Sardinia, where this craft has survived for generations, illustrate its cultural importance. The text describes the intricate process of extracting, cleaning, and spinning the filaments into golden threads, which are prized for their shine and rarity. However, it also highlights challenges, including overharvesting of pen shells and environmental changes threatening their habitats. Efforts to protect the species and preserve this ancient craft are introduced, such as promoting sustainable practices and raising awareness of its cultural value. Through vivid descriptions and examples, the text conveys the artistry and fragility of sea silk production."  
\begin{enumerate}[label=\Alph*.]
    \item By describing the environmental threats to pen shells.  
    \item By detailing the process and cultural significance of sea silk.  
    \item By emphasizing the economic value of sea silk.  
    \item By discussing modern substitutes for sea silk.  
\end{enumerate}

\vspace{1cm}
\
3. \textbf{How does the text illustrate the lives of Japan’s female ama divers?\\}
"The ama, or 'women of the sea,' are Japanese divers renowned for their ability to free dive for seafood and pearls without modern equipment. The text begins by describing their unique skills, honed over years of training, which allow them to hold their breath for extended periods while diving in cold waters. Historical context is provided, showing how the tradition of ama diving dates back over 2,000 years and has been passed down through generations. The text elaborates on their daily routines, from preparing their equipment to sharing meals with their diving \\communities. Challenges, such as the physical toll of the work and the decline in young people joining the profession, are highlighted. Efforts to preserve the tradition, including featuring ama diving in tourism and documentaries, are discussed. The combination of cultural history, vivid personal stories, and ongoing challenges \\illustrates the resilience and significance of the ama divers."  
\begin{enumerate}[label=\Alph*.]
    \item By focusing on the physical challenges of free diving.  
    \item By describing their skills, history, and role in preserving cultural traditions.  
    \item By emphasizing the economic benefits of ama diving.  
    \item By discussing the use of modern technology in ama diving.  
\end{enumerate}

\vspace{1cm}

\subsection*{Part 2: Select All That Apply Questions}

4. Select \textbf{all} details that explain the ethical concerns of using animals in circuses:  
\begin{enumerate}[label=\Alph*.]
    \item Animals often face confinement and poor medical care.  
    \item Animal performances bring joy to families worldwide.  
    \item Legal changes ban the use of wild animals in some countries.  
    \item Training often involves punishment.  
\end{enumerate}

\vspace{1cm}

5. Which details elaborate on the challenges faced by sea silk production?  
\begin{enumerate}[label=\Alph*.]
    \item Overharvesting of pen shells threatens their population.  
    \item Extracting sea silk threads is an intricate process.  
    \item Environmental changes are affecting pen shell habitats.  
    \item Sea silk is produced exclusively in modern factories.  
\end{enumerate}

\vspace{1cm}

6. Select \textbf{all} details that illustrate the lives of Japan’s ama divers:  
\begin{enumerate}[label=\Alph*.]
    \item Ama divers use modern scuba equipment to collect seafood.  
    \item Ama diving is a tradition dating back over 2,000 years.  
    \item Ama communities are featured in tourism efforts to preserve their culture.  
    \item Ama divers endure cold waters and physical challenges without equipment.  
\end{enumerate}

\vspace{1em}

\subsection*{Part 3: Short Answer Questions}

7. How does the text from question 2 illustrate the artistry and challenges of sea \\silk harvesting? Include specific examples.  
\vspace{4cm}

\newpage
8. Based on the passage about ama divers from question 3, explain how their \\cultural significance is preserved despite modern challenges. Use textual evidence.  
\vspace{4cm}

\subsection*{Part 4: Fill in the Blank Questions}
\vspace{1cm}
9. Some examples of the type of details an author might use to elaborate on a \\topic include \underline{\hspace{4cm}} and \underline{\hspace{4cm}} .
\vspace{2cm}

10. An introduction to a topic should include enough \underline{\hspace{4cm}} so that \\the reader has the context to understand what is being said.  
\vspace{2cm}
% \newpage
% \subsection*{Answer Key}
% \textbf{Part 1: Multiple-Choice Questions}  
% 1. B  
% 2. B  
% 3. B  

% \textbf{Part 2: Select All That Apply Questions}  
% 4. A, C, D  
% 5. A, C  
% 6. B, C, D  

% \textbf{Part 3: Short Answer Questions}  
% 7. Answers will vary but should reference the artistry in extracting and spinning filaments and the challenges of environmental threats.  
% 8. Answers will vary but should mention the preservation of ama culture through tourism and documentaries, as well as the traditional skills passed down.  

% \textbf{Part 4: Fill in the Blank Questions}  
% 9. \textit{Examples, anecdotes, and data.}  
% 10. \textit{Background information.}  

\end{document}

