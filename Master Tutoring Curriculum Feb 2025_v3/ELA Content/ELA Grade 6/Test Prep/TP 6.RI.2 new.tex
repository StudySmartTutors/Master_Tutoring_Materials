\documentclass[12pt]{article}

\usepackage[a4paper, top=0.8in, bottom=0.7in, left=0.7in, right=0.7in]{geometry}
\usepackage{amsmath}
\usepackage{graphicx}
\usepackage{fancyhdr}
\usepackage{tcolorbox}
\usepackage[defaultfam,tabular,lining]{montserrat} %% Option 'defaultfam'
\usepackage[T1]{fontenc}
\renewcommand*\oldstylenums[1]{{\fontfamily{Montserrat-TOsF}\selectfont #1}}
\renewcommand{\familydefault}{\sfdefault}
\usepackage{enumitem}
\usepackage{setspace}

\setlength{\parindent}{0pt}
\hyphenpenalty=10000
\exhyphenpenalty=10000

\pagestyle{fancy}
\fancyhf{}
%\fancyhead[L]{\textbf{6.RI.2: Central Idea Practice}}
\fancyhead[R]{\includegraphics[width=1cm]{Round Logo.png}}
\fancyfoot[C]{\footnotesize Study Smart Tutors}

\begin{document}

\subsection*{Determining the Central Idea of Informational Texts}
\onehalfspacing

\begin{tcolorbox}[colframe=black!40, colback=gray!0, title=Learning Objective]
\textbf{Objective:} Determine the central idea of a text and explain how it is conveyed through supporting details.
\end{tcolorbox}

\subsection*{Part 1: Multiple-Choice Questions}

1. \textbf{What is the central idea of the passage below? } 

"Deforestation has far-reaching impacts on the environment. When forests are cut down, the carbon stored in trees is released into the atmosphere, contributing to climate change. Forests also play a critical role in regulating water cycles. Without trees, the soil becomes prone to erosion, reducing the land’s fertility and causing sediment to pollute rivers. This impacts aquatic life and water quality. Furthermore, deforestation destroys the habitats of countless species, leading to a decline in \\biodiversity. Indigenous communities who depend on forests for their way of life are also severely affected. To address deforestation, reforestation projects are being implemented globally, aiming to restore ecosystems and combat climate change. Governments and environmental organizations are working together to create policies that protect existing forests and promote sustainable land use. By raising awareness and adopting responsible practices, individuals and communities can help mitigate the harmful effects of deforestation and preserve natural resources for future \\generations."  
\begin{enumerate}[label=\Alph*.]
    \item Forests are important for biodiversity.  
    \item Deforestation contributes to climate change and harms ecosystems.  
    \item Indigenous communities rely on forests for survival.  
    \item Reforestation projects are necessary for combating climate change.  
\end{enumerate}

\vspace{1cm}
\newpage
2. \textbf{Which statement best represents the central idea of the passage?}  

"Plastic pollution is one of the most pressing environmental challenges of our time. Each year, millions of tons of plastic waste end up in oceans, harming marine life. Sea turtles often mistake plastic bags for jellyfish and ingest them, causing fatal blockages. Fish and birds are also affected, with plastic debris becoming entangled around their bodies or entering their digestive systems. Microplastics, tiny pieces of degraded plastic, are now found in drinking water and the food chain, posing potential health risks to humans. Despite these challenges, solutions are emerging. Many countries have banned single-use plastics, while recycling initiatives are \\reducing waste. Innovative technologies are being developed to clean up existing pollution, such as machines that collect plastic from oceans. Public awareness \\campaigns encourage people to reduce plastic use by adopting reusable alternatives. Addressing plastic pollution requires global cooperation, but small changes at the individual level can make a significant difference."  
\begin{enumerate}[label=\Alph*.]
    \item Microplastics are the most harmful form of plastic waste.  
    \item Plastic pollution harms marine life and human health, but solutions exist.  
    \item Single-use plastics are the main source of pollution in oceans.  
    \item Plastic pollution is impossible to clean up entirely.  
\end{enumerate}

\vspace{1em}
\newpage
3. \textbf{What central idea is conveyed in the passage below?  }

"Renewable energy is transforming the global energy landscape. Technologies like solar panels, wind turbines, and hydroelectric dams provide sustainable alternatives to fossil fuels. Unlike coal and oil, renewable energy sources produce little to no greenhouse gas emissions, making them critical in the fight against climate change. They also reduce reliance on nonrenewable resources, ensuring energy security for future generations. However, transitioning to renewable energy comes with \\challenges. Building solar farms and wind turbines requires significant upfront \\investment, and some argue that renewable technologies can impact local \\ecosystems. Despite these concerns, the benefits outweigh the drawbacks. Many countries are setting ambitious targets for renewable energy adoption, driven by both environmental and economic incentives. As costs for these technologies \\decrease, renewables are becoming more accessible to individuals and businesses alike. Public policies, such as tax credits and subsidies, are accelerating this shift. Renewable energy is paving the way for a cleaner and more sustainable future."  
\begin{enumerate}[label=\Alph*.]
    \item Renewable energy is expensive to implement.  
    \item Renewable energy has minimal impact on the environment.  
    \item Renewable energy offers sustainable solutions for climate change and energy security.  
    \item Renewable energy relies solely on public policies to succeed.  
\end{enumerate}


\subsection*{Part 2: Select All That Apply Questions}

4. Select \textbf{all} supporting details from the passage about deforestation:  
\begin{enumerate}[label=\Alph*.]
    \item Deforestation releases carbon into the atmosphere, contributing to climate \\change.  
    \item Reforestation projects are being implemented globally.  
    \item Deforestation has no impact on water cycles or soil quality.  
    \item Governments and organizations are creating policies to protect forests.  
\end{enumerate}

\vspace{1em}

5. Which details from the passage about plastic pollution explain its impact?  
\begin{enumerate}[label=\Alph*.]
    \item Microplastics are found in drinking water and food.  
    \item Recycling initiatives are reducing waste.  
    \item Plastic pollution primarily affects marine life.  
    \item Plastic pollution has no solutions.  
\end{enumerate}

\vspace{1cm}

6. Select \textbf{all} reasons why renewable energy is beneficial:  
\begin{enumerate}[label=\Alph*.]
    \item Renewable energy reduces greenhouse gas emissions.  
    \item Renewable energy requires no investment.  
    \item Renewable energy ensures energy security for future generations.  
    \item Renewable energy reduces reliance on nonrenewable resources.  
\end{enumerate}



\subsection*{Part 3: Short Answer Questions}

7. Based on the passage about deforestation, what are two key reasons forests \\are essential?  
\vspace{4cm}

8. Explain how renewable energy benefits both the environment and the \\economy. Cite specific evidence from the text.  
\vspace{3cm}

\subsection*{Part 4: Fill in the Blank Question}
\vspace{1cm}
9. The central idea of a text is supported by \underline{\hspace{4cm}} that explain or \\expand upon the main concept.  
\vspace{2cm}

10. Personal opinions or judgments in a summary is called  \underline{\hspace{4cm}}  \\and should be avoided.
% \newpage
% \subsection*{Answer Key}

% \begin{itemize}
%     \item \textbf{Question 1:} B. Deforestation contributes to climate change and harms ecosystems.
%     \item \textbf{Question 2:} B. Plastic pollution harms marine life and human health, but solutions exist.
%     \item \textbf{Question 3:} C. Renewable energy offers sustainable solutions for climate change and energy security.
%     \item \textbf{Question 4:} A, B, D.
%     \item \textbf{Question 5:} A, C.
%     \item \textbf{Question 6:} A, C, D.
%     \item \textbf{Question 7:} Forests absorb carbon dioxide and release oxygen (benefit to climate); forests prevent soil erosion and support biodiversity.
%     \item \textbf{Question 8:} Renewable energy reduces greenhouse gas emissions and creates jobs in manufacturing, installation, and maintenance.
%     \item \textbf{Question 9:} Supporting details.
%     \item \textbf{Question 10:} Bias.
% \end{itemize}

\end{document}

