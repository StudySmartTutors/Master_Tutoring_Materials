\documentclass[12pt]{article}

\usepackage[a4paper, top=0.8in, bottom=0.7in, left=0.7in, right=0.7in]{geometry}
\usepackage{amsmath}
\usepackage{graphicx}
\usepackage{fancyhdr}
\usepackage{tcolorbox}
\usepackage[defaultfam,tabular,lining]{montserrat} %% Option 'defaultfam'
\usepackage[T1]{fontenc}
\renewcommand*\oldstylenums[1]{{\fontfamily{Montserrat-TOsF}\selectfont #1}}
\renewcommand{\familydefault}{\sfdefault}
\usepackage{enumitem}
\usepackage{setspace}

\setlength{\parindent}{0pt}
\hyphenpenalty=10000
\exhyphenpenalty=10000

\pagestyle{fancy}
\fancyhf{}
%\fancyhead[L]{\textbf{6.RL.3: Analyzing Plot and Character Response Practice}}
\fancyhead[R]{\includegraphics[width=1cm]{Round Logo.png}}
\fancyfoot[C]{\footnotesize Study Smart Tutors}

\begin{document}

\subsection*{Analyzing Plot Development and Character Responses}
\onehalfspacing

\begin{tcolorbox}[colframe=black!40, colback=gray!0, title=Learning Objective]
\textbf{Objective:} Describe how a particular story’s plot unfolds and how the characters respond or change as the plot moves toward a resolution.
\end{tcolorbox}

\subsection*{Part 1: Multiple-Choice Questions}

1. \textbf{How does Mia’s response to the argument with her best friend show her character growth?\\}
"Mia and Clara had always been close, sharing their love for art and spending weekends painting together. One day, as Mia was finishing her most detailed watercolor painting yet, Clara accidentally bumped the table, spilling juice all over the artwork. Mia’s face turned red with anger, and without saying a word, she stormed out of the room, leaving Clara standing there, speechless. That evening, Mia sat in her room, staring at the ruined painting. As her initial anger faded, she thought about how upset Clara must have felt. Later, she saw that Clara had spent hours trying to fix the painting by blending her own artistic touches into it. It wasn’t perfect, but it was heartfelt. Mia called Clara, apologizing for her outburst and thanking her for trying to help. They laughed together over how the painting now looked like a modern art masterpiece. From that day on, Mia learned to approach conflicts with patience and understanding, strengthening their friendship."  
\begin{enumerate}[label=\Alph*.]
    \item Mia remained angry at Clara.  
    \item Mia learned to forgive and approach situations with patience.  
    \item Mia avoided Clara after the argument.  
    \item Mia stopped painting altogether.  
\end{enumerate}

\vspace{1cm}
\newpage
2. \textbf{How does the plot develop tension in the following story?}\\
"Jonah had been preparing for the school talent show for weeks. Every afternoon, he would sit at his piano, playing his chosen piece repeatedly until he had memorized every note. On the day of the show, he felt confident—until he realized he had left his sheet music at home. Panic set in as he paced backstage, clutching his hands together. His best friend, Max, found him and said, ‘You’ve played this a hundred times. You don’t need the music.’ Jonah wasn’t convinced, but with Max’s encouragement, he took a deep breath and walked onto the stage. At first, his fingers hesitated on the keys, his mind racing to recall the notes. Slowly, the melody began to flow, and he relaxed, letting his muscle memory guide him. By the end of the performance, the audience erupted into applause. Jonah smiled, realizing that his weeks of preparation had paid off. He learned that sometimes, even when things don’t go as planned, he could rely on his hard work and confidence to succeed."  
\begin{enumerate}[label=\Alph*.]
    \item Jonah realized he could memorize any piece instantly.  
    \item The tension comes from Jonah’s fear of forgetting his performance.  
    \item Jonah was upset with Max for not bringing his music.  
    \item The tension arises when Jonah struggles to choose a song.  
\end{enumerate}

\vspace{0.5cm}

3. \textbf{What causes the resolution in the story?\\}
"Sophia often felt overshadowed by her older sister, Lily, who excelled at everything she did. Lily was the captain of her school’s basketball team, while Sophia struggled to even dribble a ball. One afternoon, during a neighborhood basketball game, Sophia reluctantly joined a team short on players. At first, she stayed out of the action, passing the ball to others and trying to avoid mistakes. But as the game went on, her teammates encouraged her to take more chances. With seconds left and the score tied, Sophia unexpectedly found herself holding the ball. Her hands shook as she glanced at Lily, who shouted, ‘You’ve got this!’ Taking a deep breath, Sophia shot the ball—and it went in. The crowd erupted in cheers, and Lily ran to hug her. ‘I’ve always believed in you,’ Lily said. From that moment, Sophia stopped seeing Lily as competition and realized her sister had always been her biggest supporter. Their bond grew stronger as they began encouraging each other to shine in their own ways."  
\begin{enumerate}[label=\Alph*.]
    \item Sophia and Lily continued to compete with each other.  
    \item Sophia realized her sister had always supported her.  
    \item Sophia stopped playing basketball.  
    \item Sophia and Lily argued about the game.  
\end{enumerate}

\vspace{1cm}

\subsection*{Part 2: Select All That Apply Questions}

4. Select \textbf{all} details that show Mia’s character growth in the story from question 1:  
\begin{enumerate}[label=\Alph*.]
    \item Mia apologized to Clara for her outburst.  
    \item Mia laughed about the painting with Clara.  
    \item Mia refused to talk to Clara again.  
    \item Mia learned to handle her emotions with patience.  
\end{enumerate}

\vspace{1cm}

5. Which details build tension in Jonah’s story from question 2?  
\begin{enumerate}[label=\Alph*.]
    \item Jonah left his sheet music at home.  
    \item Jonah’s best friend encouraged him.  
    \item Jonah walked on stage feeling calm and prepared.  
    \item Jonah hesitated but soon played the melody from memory.  
\end{enumerate}

\vspace{1cm}

6. Select \textbf{all} details that show the resolution in Sophia’s story from question 3:  
\begin{enumerate}[label=\Alph*.]
    \item Sophia scored the winning basket.  
    \item Lily praised Sophia and expressed her pride.  
    \item Sophia realized Lily was her biggest supporter.  
    \item Sophia decided not to play basketball anymore.  
\end{enumerate}

\vspace{1cm}

\subsection*{Part 3: Short Answer Questions}

7. How does Jonah overcome his fear during the talent show in the passage from question 2? Use specific details from the text to explain.  
\vspace{4cm}
\newpage
8. What changes in Sophia and Lily’s relationship by the end of the story from question 3? Provide evidence from the text.  
\vspace{4cm}

\subsection*{Part 4: Fill in the Blank Questions}
\vspace{1cm}
9. A plot unfolds as characters face \underline{\hspace{4cm}} and respond to challenges.  
\vspace{2cm}

10. The resolution of a story often shows how characters have \underline{\hspace{4cm}} or how their situations have changed.  
\vspace{2cm}
% \newpage

% \section*{Answer Key}

% \subsection*{Part 1: Multiple-Choice Questions}
% 1. \textbf{B} - Mia learned to forgive and approach situations with patience.  
% 2. \textbf{B} - The tension comes from Jonah’s fear of forgetting his performance.  
% 3. \textbf{B} - Sophia realized her sister had always supported her.  

% \subsection*{Part 2: Select All That Apply Questions}
% 4. \textbf{A, B, D} - Mia apologized to Clara for her outburst, laughed about the painting with Clara, and learned to handle her emotions with patience.  
% 5. \textbf{A, D} - Jonah left his sheet music at home, Jonah hesitated but soon played the melody from memory.  
% 6. \textbf{A, B, C} - Sophia scored the winning basket, Lily praised Sophia and expressed her pride, and Sophia realized Lily was her biggest supporter.  

% \subsection*{Part 3: Short Answer Questions}
% 7. Jonah overcomes his fear during the talent show by relying on his preparation and his best friend’s encouragement. He was initially nervous but was able to play the piece from memory once he let go of his anxiety.  
% 8. By the end of the story, Sophia and Lily’s relationship becomes stronger as they no longer view each other as competitors. Sophia realizes that Lily has always supported her, and they both encourage each other to shine in their own ways.

% \subsection*{Part 4: Fill in the Blank Questions}
% 9. conflict  
% 10. changed  




\end{document}

