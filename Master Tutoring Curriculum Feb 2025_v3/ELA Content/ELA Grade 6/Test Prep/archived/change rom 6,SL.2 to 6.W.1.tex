\documentclass[12pt]{article}

\usepackage[a4paper, top=0.8in, bottom=0.7in, left=0.7in, right=0.7in]{geometry}
\usepackage{amsmath}
\usepackage{graphicx}
\usepackage{fancyhdr}
\usepackage{tcolorbox}
\usepackage{multicol}
\usepackage{pifont} % For checkboxes
%\usepackage{tgadventor}
\usepackage[defaultfam,tabular,lining]{montserrat} %% Option 'defaultfam'
\usepackage[T1]{fontenc}
\renewcommand*\oldstylenums[1]{{\fontfamily{Montserrat-TOsF}\selectfont #1}}
\renewcommand{\familydefault}{\sfdefault}
\usepackage{enumitem}
\usepackage{setspace}
\usepackage{parcolumns}
\usepackage{tabularx}

\setlength{\parindent}{0pt}
\hyphenpenalty=10000
\exhyphenpenalty=10000

\pagestyle{fancy}
\fancyhf{}
\fancyhead[L]{\textbf{6.SL.2: Speaking and Listening Worksheet}}
\fancyhead[R]{\includegraphics[width=1cm]{Round Logo.png}}
\fancyfoot[C]{\footnotesize Study Smart Tutors}

\begin{document}

\onehalfspacing

% Fictional Text
\subsection*{Fictional Text: The Brave Little Fox}

\begin{tcolorbox}[colframe=black!40, colback=gray!5]
\begin{spacing}{1.15}
Once upon a time, in a dense forest, there lived a young fox named Finn. He was smaller than the other foxes, but he had a big heart. One day, a storm swept through the forest, knocking down trees and causing floods. Finn’s home was destroyed, and he had to find a new place to live. 

Despite his fear, Finn ventured deep into the woods to find shelter. Along the way, he met a group of animals who had also lost their homes. Finn decided to lead them to safety, using his quick thinking and bravery. In the end, Finn became the hero of the forest, known for his courage and kindness.

\end{spacing}
\end{tcolorbox}

\vspace{1cm}

% Informational Text
\subsection*{Informational Text: The Water Cycle}

\begin{tcolorbox}[colframe=black!40, colback=gray!5]
\begin{spacing}{1.15}
The water cycle is the process by which water moves around the Earth. It starts when the sun heats water in rivers, lakes, and oceans, causing it to evaporate into the air. This water vapor rises and cools, forming clouds. When the clouds become heavy with water, they release it as precipitation, which can be rain, snow, or hail. 

The water then flows back into rivers and lakes, where it is eventually evaporated again. This cycle repeats constantly, providing water to plants, animals, and humans. The water cycle is essential for life on Earth and helps maintain a balance in nature.

\end{spacing}
\end{tcolorbox}

\vspace{1cm}

% Graph (Placeholder)
\subsection*{Graph: The Average Monthly Rainfall in Millimeters}

\begin{center}
\includegraphics[width=0.7\textwidth]{rainfall_graph.png} % Replace with your actual graph image file
\end{center}

\vspace{1cm}

% Worksheet Questions
\subsection*{Questions}

\begin{enumerate}

    % Question 1
    \item What did Finn do after the storm destroyed his home?

    \begin{enumerate}[label=\Alph*.]
        \item He gave up and left the forest.
        \item He went on a journey to find a new place to live.
        \item He built a new home by himself.
        \item He found a new family to live with.
    \end{enumerate}

    \vspace{0.5cm}

    % Question 2
    \item How did Finn help the other animals in the forest?

    \begin{enumerate}[label=\Alph*.]
        \item He told them to find shelter elsewhere.
        \item He led them to safety by using his quick thinking.
        \item He gave them food and water.
        \item He built shelters for each animal.
    \end{enumerate}

    \vspace{0.5cm}

    % Question 3
    \item What lesson can be learned from Finn’s actions?

    \begin{enumerate}[label=\Alph*.]
        \item It’s important to be brave and help others.
        \item It’s okay to run away from problems.
        \item Only the strong can help others.
        \item It’s better to be alone than to help others.
    \end{enumerate}

    \vspace{0.5cm}

    % Question 4
    \item According to the informational text, what is the first step in the water cycle?

    \begin{enumerate}[label=\Alph*.]
        \item Precipitation
        \item Evaporation
        \item Condensation
        \item Runoff
    \end{enumerate}

    \vspace{0.5cm}

    % Question 5
    \item What happens to the water vapor in the water cycle after it rises into the air?

    \begin{enumerate}[label=\Alph*.]
        \item It cools and forms clouds.
        \item It turns into snow.
        \item It flows back into the ground.
        \item It is absorbed by plants.
    \end{enumerate}

    \vspace{0.5cm}

    % Question 6
    \item How does the water cycle affect life on Earth?

    \begin{enumerate}[label=\Alph*.]
        \item It helps plants grow and animals survive.
        \item It makes the weather unpredictable.
        \item It causes storms and floods.
        \item It only affects humans.
    \end{enumerate}

    \vspace{0.5cm}

    % Question 7
    \item According to the graph, which month has the highest rainfall?

    \begin{enumerate}[label=\Alph*.]
        \item January
        \item April
        \item August
        \item December
    \end{enumerate}

    \vspace{0.5cm}

    % Question 8
    \item What season likely has the most rainfall, based on the graph?

    \begin{enumerate}[label=\Alph*.]
        \item Spring
        \item Summer
        \item Fall
        \item Winter
    \end{enumerate}

    \vspace{0.5cm}

    % Question 9
    \item What is the purpose of the graph showing rainfall?

    \begin{enumerate}[label=\Alph*.]
        \item To show how much rainfall occurs in each month.
        \item To predict the weather for the year.
        \item To measure the temperature each month.
        \item To show the amount of sunshine each month.
    \end{enumerate}

    \vspace{0.5cm}

    % Question 10
    \item If the rainfall in a certain month was much lower than expected, what might happen?

    \begin{enumerate}[label=\Alph*.]
        \item There could be a drought.
        \item There could be a flood.
        \item The temperature would drop.
        \item The weather would become warmer.
    \end{enumerate}

\end{enumerate}

\end{document}
