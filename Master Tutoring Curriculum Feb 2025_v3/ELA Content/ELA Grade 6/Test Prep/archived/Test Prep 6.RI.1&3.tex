\documentclass[12pt]{article}
\usepackage[a4paper, top=0.8in, bottom=0.7in, left=0.7in, right=0.7in]{geometry}
\usepackage{amsmath}
\usepackage{graphicx}
\usepackage{fancyhdr}
\usepackage{tcolorbox}
\usepackage{multicol}
\usepackage{pifont} % For checkboxes
\usepackage[defaultfam,tabular,lining]{montserrat} %% Option 'defaultfam'
\usepackage[T1]{fontenc}
\renewcommand*\oldstylenums[1]{{\fontfamily{Montserrat-TOsF}\selectfont #1}}
\renewcommand{\familydefault}{\sfdefault}
\usepackage{enumitem}
\usepackage{setspace}
\usepackage{parcolumns}
\usepackage{tabularx}

\setlength{\parindent}{0pt}

\hyphenpenalty=10000
\exhyphenpenalty=10000

\pagestyle{fancy}
\fancyhf{}
%\fancyhead[L]{\textbf{6.RL.1, 6.RL.3L: Analyzing Development of Events or Ideas}}
\fancyhead[R]{\includegraphics[width=1cm]{Round Logo.png}}
\fancyfoot[C]{\footnotesize Study Smart Tutors}

\begin{document}

\onehalfspacing

% Passage
\subsection*{The California Gold Rush: A Search for Riches}
\begin{tcolorbox}[colframe=black!40, colback=gray!5]
\begin{spacing}{1.15}
    The California Gold Rush began in 1848 when gold was discovered at Sutter's Mill, in Coloma, California. Word spread quickly, and thousands of people from all over the United States, and even other countries, rushed to California in hopes of striking it rich. This event transformed California from a sparsely populated area into a bustling center of activity.

    Most of the gold seekers were known as "forty-niners" because many arrived in 1849. These miners faced many challenges, including harsh weather, dangerous mining conditions, and limited resources. Some lived in tents or makeshift cabins, while others worked in crowded camps. Despite the hardships, many miners continued to hope they would find large quantities of gold.

    Not all who came to California struck gold. While some did find gold, many left disappointed. The impact of the Gold Rush was also felt in other areas. As people flocked to California, the demand for supplies, food, and services grew. This led to the development of new towns and businesses. It also contributed to California's statehood in 1850.

    The Gold Rush did not last long. By the mid-1850s, the gold was harder to find, and many miners moved on. However, the legacy of the Gold Rush remains today. It led to significant economic growth and helped shape the state's development, making California one of the most populous and wealthiest states in the U.S.
\end{spacing}
\end{tcolorbox}

% Worksheet Questions
\subsection*{Questions}
\begin{enumerate}

    % Question 1
    \item What event started the California Gold Rush?
    \begin{enumerate}[label=\Alph*.]
        \item The discovery of gold at Sutter's Mill
        \item The arrival of the first miners in California
        \item California's admission to the United States
        \item The construction of the Transcontinental Railroad
    \end{enumerate}

    \vspace{0.5cm}

    % Question 2
    \item What were the people who went to California to search for gold called?
    \begin{enumerate}[label=\Alph*.]
        \item Gold miners
        \item Forty-niners
        \item Gold rushers
        \item California pioneers
    \end{enumerate}

    \vspace{0.5cm}

    % Question 3
    \item What was the main goal of the people who participated in the Gold Rush?
    \begin{enumerate}[label=\Alph*.]
        \item To build new towns in California
        \item To find gold and become rich
        \item To claim land for farming
        \item To help establish California as a state
    \end{enumerate}

    \vspace{0.5cm}

    % Question 4
    \item What was one of the challenges miners faced during the Gold Rush?
    \begin{enumerate}[label=\Alph*.]
        \item Extreme weather and dangerous conditions
        \item Lack of food and supplies
        \item Unfriendly relationships with Native Americans
        \item Long travel times from Europe
    \end{enumerate}

    \vspace{0.5cm}

    % Question 5
    \item How did the Gold Rush affect the population of California?
    \begin{enumerate}[label=\Alph*.]
        \item California became less populated
        \item California's population increased rapidly
        \item The population remained the same
        \item The number of settlers decreased
    \end{enumerate}

    \vspace{0.5cm}

    % Question 6
    \item Why were the miners called "forty-niners"?
    \begin{enumerate}[label=\Alph*.]
        \item They arrived in 1849
        \item They all came from the state of Florida
        \item They were part of a group of forty miners
        \item They wanted to find gold in 1849
    \end{enumerate}

    \vspace{0.5cm}

    % Question 7
    \item What contributed to the development of new towns and businesses in California during the Gold Rush?
    \begin{enumerate}[label=\Alph*.]
        \item The discovery of oil
        \item The demand for supplies, food, and services
        \item The arrival of new settlers for farming
        \item The construction of new railroads
    \end{enumerate}

    \vspace{0.5cm}

    % Question 8
    \item What was one of the reasons the Gold Rush ended by the mid-1850s?
    \begin{enumerate}[label=\Alph*.]
        \item The government banned gold mining
        \item Gold became harder to find
        \item Most miners became wealthy
        \item New technology made mining easier
    \end{enumerate}

    \vspace{0.5cm}

    % Question 9
    \item What happened to many miners who searched for gold during the Gold Rush?
    \begin{enumerate}[label=\Alph*.]
        \item Most found large quantities of gold
        \item Many left disappointed without finding gold
        \item All miners became rich
        \item Many became farmers in California
    \end{enumerate}

    \vspace{0.5cm}

    % Question 10
    \item What impact did the Gold Rush have on California's statehood?
    \begin{enumerate}[label=\Alph*.]
        \item It delayed California's statehood
        \item It contributed to California's statehood in 1850
        \item It made California the most populated state
        \item It caused California to become a part of Mexico
    \end{enumerate}

    \vspace{0.5cm}

    % Question 11
    \item Which of the following is a lasting impact of the Gold Rush on California?
    \begin{enumerate}[label=\Alph*.]
        \item California became a major mining state
        \item The state’s population remained small
        \item Gold mining is illegal in California today
        \item California became a major agricultural state
    \end{enumerate}

    \vspace{0.5cm}

    % Question 12
    \item What was the main reason people left California after the Gold Rush ended?
    \begin{enumerate}[label=\Alph*.]
        \item The discovery of gold in other areas
        \item The gold was harder to find
        \item The weather became too harsh
        \item California became overcrowded
    \end{enumerate}

    \vspace{0.5cm}

    % Question 13
    \item What was one of the physical living conditions of the miners during the Gold Rush?
    \begin{enumerate}[label=\Alph*.]
        \item They lived in well-established cities
        \item Many lived in tents or makeshift cabins
        \item They were provided with comfortable homes
        \item They worked from dawn till dusk in large factories
    \end{enumerate}

    \vspace{0.5cm}

    % Question 14
    \item Why did the Gold Rush attract people from other countries?
    \begin{enumerate}[label=\Alph*.]
        \item To build railroads
        \item To escape economic hardships
        \item To establish farms in California
        \item To join the U.S. military
    \end{enumerate}

    \vspace{0.5cm}

    % Question 15
    \item What did the discovery of gold at Sutter’s Mill lead to?
    \begin{enumerate}[label=\Alph*.]
        \item The creation of new towns in California
        \item The decline of mining in California
        \item A decrease in California’s population
        \item The development of California’s oil industry
    \end{enumerate}

    \vspace{0.5cm}

    % Question 16
    \item How did the Gold Rush contribute to the economy of California?
    \begin{enumerate}[label=\Alph*.]
        \item It helped create many new industries and businesses
        \item It decreased the number of jobs in agriculture
        \item It led to fewer people coming to California
        \item It caused businesses to close
    \end{enumerate}

    \vspace{0.5cm}

    % Question 17
    \item What led to California becoming one of the wealthiest states in the U.S.?
    \begin{enumerate}[label=\Alph*.]
        \item The discovery of gold
        \item The establishment of large farms
        \item The arrival of settlers from Europe
        \item The completion of the Transcontinental Railroad
    \end{enumerate}

    \vspace{0.5cm}

    % Question 18
    \item How did the discovery of gold in California affect the rest of the United States?
    \begin{enumerate}[label=\Alph*.]
        \item It led to fewer people moving west
        \item It caused people to flock to California for riches
        \item It caused the economy to collapse
        \item It decreased the population in other states
    \end{enumerate}

    \vspace{0.5cm}

    % Question 19
    \item Which group of people did NOT participate in the Gold Rush?
    \begin{enumerate}[label=\Alph*.]
        \item Native Americans
        \item Immigrants from Europe and China
        \item People from other states in the U.S.
        \item California farmers
    \end{enumerate}

    \vspace{0.5cm}

    % Question 20
    \item What happened to the population of California after the Gold Rush ended?
    \begin{enumerate}[label=\Alph*.]
        \item It decreased significantly
        \item It stayed the same
        \item It continued to grow
        \item It returned to the pre-Gold Rush levels
    \end{enumerate}

\end{enumerate}

\end{document}
