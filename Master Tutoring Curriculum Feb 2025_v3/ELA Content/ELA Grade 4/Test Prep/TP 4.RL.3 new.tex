\documentclass[12pt]{article}

\usepackage[a4paper, top=0.8in, bottom=0.7in, left=0.7in, right=0.7in]{geometry}
\usepackage{amsmath}
\usepackage{graphicx}
\usepackage{fancyhdr}
\usepackage{tcolorbox}
\usepackage{multicol}
\usepackage{pifont} % For checkboxes
\usepackage[defaultfam,tabular,lining]{montserrat} %% Option 'defaultfam'
\usepackage[T1]{fontenc}
\renewcommand*\oldstylenums[1]{{\fontfamily{Montserrat-TOsF}\selectfont #1}}
\renewcommand{\familydefault}{\sfdefault}
\usepackage{enumitem}
\usepackage{setspace}
\usepackage{parcolumns}
\usepackage{tabularx}

\setlength{\parindent}{0pt}
\hyphenpenalty=10000
\exhyphenpenalty=10000

\pagestyle{fancy}
\fancyhf{}
%\fancyhead[L]{\textbf{4.RL.3: Story Events and Character Practice}}
\fancyhead[R]{\includegraphics[width=1cm]{Round Logo.png}}
\fancyfoot[C]{\footnotesize Study Smart Tutors}

\begin{document}

\subsection*{Understanding Events, Characters, and Their Influence}
\onehalfspacing

\begin{tcolorbox}[colframe=black!40, colback=gray!0, title=Learning Objective]
\textbf{Objective:} Explain events in a story and analyze how characters influence or are influenced by these events.
\end{tcolorbox}

\subsection*{Part 1: Multiple-Choice Questions}

1. How did the character's actions influence the story?\\
"Maria found an injured kitten in the park. She decided to care for it, feeding it and taking it to the vet. Over time, the kitten grew healthy and playful. One day, the kitten alerted Maria to a gas leak in her home by meowing loudly near the stove. Maria realized that her decision to care for the kitten had saved her life. She shared the story with her neighbors, who were inspired by her act of kindness. Many of them began helping stray animals, and the community became more compassionate toward animals. Maria’s simple act of caring for the kitten created a ripple effect of kindness, changing not only her life but also her neighborhood. This story shows how even small actions can lead to significant positive changes in unexpected ways."\\
\begin{enumerate}[label=\Alph*.]
    \item Maria ignored the kitten’s needs.  
    \item Maria’s care for the kitten led to an event that saved her life.  
    \item The kitten caused trouble for Maria.  
    \item Maria returned the kitten to the park.  
\end{enumerate}

\vspace{1cm}
\newpage
2. What role did the setting play in the story?\\
"A small village near a river faced frequent flooding during the rainy season. One year, the villagers decided to build a system of canals to direct the water away from their homes. The canals worked well, and the village stayed safe. The river, once a source of danger, became a resource for irrigation and fishing. Over time, the villagers adapted further by planting trees along the riverbanks to prevent soil erosion. They also developed a schedule to maintain the canals. Their efforts not only protected their homes but also improved their quality of life. Festivals celebrating the river became annual traditions, fostering a stronger sense of community. The story shows how the villagers transformed a challenge into an opportunity, demonstrating \\resilience and ingenuity in the face of natural challenges."\\
\begin{enumerate}[label=\Alph*.]
    \item The village’s location near the river caused its decline.  
    \item The river provided only challenges for the villagers.  
    \item The setting influenced the villagers to find a solution to their problem.  
    \item The setting was irrelevant to the story’s events.  
\end{enumerate}

\vspace{1cm}

3. How did the character change by the end of the story?\\
"Liam was nervous about speaking in front of his class. During the school year, his teacher encouraged him to participate in small group discussions. Slowly, Liam gained confidence. He began practicing at home and watching videos of great speakers for inspiration. By the end of the year, he volunteered to give a speech at the school assembly. Despite feeling a little nervous, Liam stood on stage and spoke clearly and proudly. His classmates cheered and congratulated him afterward. Liam realized he had overcome his fear of public speaking and felt proud of his growth. He even decided to join the debate team the following year. This transformation shows how encouragement, practice, and perseverance can help overcome fears and lead to personal growth."\\
\begin{enumerate}[label=\Alph*.]
    \item Liam remained shy and avoided public speaking.  
    \item Liam overcame his fear of speaking and became more confident.  
    \item Liam decided not to participate in the school assembly.  
    \item Liam was unaffected by his teacher’s encouragement.  
\end{enumerate}

\vspace{1cm}

\subsection*{Part 2: Select All That Apply Questions}

4. Which details show how Maria influenced the kitten in the story from question 1?\\
\begin{enumerate}[label=\Alph*.]
    \item Maria fed the kitten and took it to the vet.  
    \item Maria ignored the kitten’s cries for help.  
    \item The kitten became healthy under Maria’s care.  
    \item Maria taught the kitten to alert her to danger.  
\end{enumerate}

\vspace{1cm}

5. Select \textbf{all} the ways the villagers adapted to their setting in the story from \\question 2:\\
\begin{enumerate}[label=\Alph*.]
    \item They built canals to manage flooding.  
    \item They moved away from the river.  
    \item They used the river for irrigation.  
    \item They learned to fish in the river.  
\end{enumerate}

\vspace{1cm}

6. What actions helped Liam overcome his fear of public speaking in the \\story from question 3?\\
\begin{enumerate}[label=\Alph*.]
    \item His teacher encouraged him to join discussions.  
    \item He avoided speaking in front of others.  
    \item He volunteered to give a speech.  
    \item His classmates supported him during his speech.  
\end{enumerate}

\vspace{1cm}
\newpage
\subsection*{Part 3: Short Answer Questions}

7. Based on the story about Maria and the kitten, how did their relationship benefit both of them?\\
\vspace{4cm}

8. How did the villagers in the story from question 2 turn a challenge in their setting into an opportunity?\\
\vspace{4cm}

\subsection*{Part 4: Fill in the Blank Questions}
\vspace{1cm}

9. A character’s \underline{\hspace{4cm}} often changes because of events in the story.

\vspace{3cm}

10. The \underline{\hspace{4cm}} can shape the challenges and opportunities characters face in a story.

% \vspace{3cm}
% \section*{Answer Key}

% \subsection*{Part 1: Multiple-Choice Questions}

% B. Maria’s care for the kitten led to an event that saved her life.

% C. The setting influenced the villagers to find a solution to their problem.

% B. Liam overcame his fear of speaking and became more confident.

% \subsection*{Part 2: Select All That Apply Questions}

% A, C.

% Maria fed the kitten and took it to the vet.
% The kitten became healthy under Maria’s care.
% A, C.

% They built canals to manage flooding.
% They used the river for irrigation.
% A, C, D.

% His teacher encouraged him to join discussions.
% He volunteered to give a speech.
% His classmates supported him during his speech.
% \subsection*{Part 3: Short Answer Questions}

% Answer: Maria and the kitten both benefited from their relationship. Maria’s decision to care for the kitten saved her life when the kitten alerted her to a gas leak. In return, Maria’s kindness gave the kitten a chance to live and thrive, while also inspiring her neighbors to show kindness to animals.

% Answer: The villagers turned the challenge of flooding into an opportunity by building canals to manage the water. They also used the river for irrigation and fishing, and planted trees to prevent soil erosion. These actions helped protect their homes, improve their quality of life, and fostered a stronger sense of community.

% \subsection*{Part 4: Fill in the Blank Questions}

% A character’s \underline{development} often changes because of events in the story.

% The \underline{setting} can shape the challenges and opportunities characters face in a story.


\end{document}
