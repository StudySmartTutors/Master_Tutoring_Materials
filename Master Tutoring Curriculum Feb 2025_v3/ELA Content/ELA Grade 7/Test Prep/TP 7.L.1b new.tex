\documentclass[12pt]{article}

\usepackage[a4paper, top=0.8in, bottom=0.7in, left=0.7in, right=0.7in]{geometry}
\usepackage{amsmath}
\usepackage{graphicx}
\usepackage{fancyhdr}
\usepackage{tcolorbox}
\usepackage[defaultfam,tabular,lining]{montserrat} %% Option 'defaultfam'
\usepackage[T1]{fontenc}
\renewcommand*\oldstylenums[1]{{\fontfamily{Montserrat-TOsF}\selectfont #1}}
\renewcommand{\familydefault}{\sfdefault}
\usepackage{enumitem}
\usepackage{setspace}

\setlength{\parindent}{0pt}
\hyphenpenalty=10000
\exhyphenpenalty=10000

\pagestyle{fancy}
\fancyhf{}
\fancyhead[L]{\textbf{7.L.1b: Sentence Structure Practice}}
\fancyhead[R]{\includegraphics[width=1cm]{Round Logo.png}}
\fancyfoot[C]{\footnotesize Study Smart Tutors}

\begin{document}

\subsection*{Understanding Sentence Structure and Relationships Among Ideas}
\onehalfspacing

\begin{tcolorbox}[colframe=black!40, colback=gray!0, title=Learning Objective]
\textbf{Objective:} Demonstrate command of the conventions of standard English grammar by choosing among simple, compound, complex, and compound-complex sentences to signal differing relationships among ideas.
\end{tcolorbox}

\subsection*{Part 1: Multiple-Choice Questions}

1. Which sentence is a **compound** sentence?  
\begin{enumerate}[label=\Alph*.]
    \item The cat slept on the couch, and the dog napped on the floor.  
    \item The bird in the tree sang a cheerful melody.  
    \item Although it was raining, the children played outside.  
    \item The book on the table belongs to my sister.  
\end{enumerate}

\vspace{1cm}

2. Which sentence is a **complex** sentence?  
\begin{enumerate}[label=\Alph*.]
    \item The sun set, but the sky remained bright.  
    \item When the bell rang, the students packed their bags and left.  
    \item The team practiced every afternoon for the big game.  
    \item She cooked dinner and cleaned the kitchen.  
\end{enumerate}

\vspace{1cm}

3. Which sentence is a **compound-complex** sentence?  
\begin{enumerate}[label=\Alph*.]
    \item After the meeting ended, the group stayed behind to discuss the project, \\and they decided to schedule another session.  
    \item The flowers in the garden are blooming beautifully.  
    \item He likes to read mystery novels and write poetry.  
    \item The weather was perfect for a picnic, so we packed lunch and headed \\to the park.  
\end{enumerate}

\vspace{1cm}

\subsection*{Part 2: Select All That Apply Questions}

4. Select \textbf{all} sentences that are **simple** sentences:  
\begin{enumerate}[label=\Alph*.]
    \item The car stopped suddenly at the red light.  
    \item Before she could respond, the phone rang again.  
    \item He studies hard and dreams big.  
    \item The lake shimmered under the moonlight.  
\end{enumerate}

\vspace{1cm}

5. Select \textbf{all} sentences that use **compound** structure:  
\begin{enumerate}[label=\Alph*.]
    \item The chef prepared the meal, and the waiter served it.  
    \item If the bus arrives on time, we will make it to the show.  
    \item The dog barked loudly, but no one was home to hear it.  
    \item She enjoys hiking and biking on the weekends.  
\end{enumerate}

\vspace{1cm}

6. Which sentences are **complex**? (Select \textbf{all} that apply.)  
\begin{enumerate}[label=\Alph*.]
    \item The movie started late because the projector was broken.  
    \item Since we had extra time, we decided to explore the city.  
    \item The book was fascinating, so I finished it in one day.  
    \item After the rain stopped, a rainbow appeared in the sky.  
\end{enumerate}

\vspace{1cm}
\newpage
\subsection*{Part 3: Short Answer Questions}

7. Write an example of a **compound-complex** sentence and explain why it fits this category.  
\vspace{4cm}

8. Why is it important to use different types of sentences when writing? Provide examples of how varying sentence structures improve writing.  
\vspace{4cm}

\subsection*{Part 4: Fill in the Blank Questions}
\vspace{1cm}
9. A sentence with one independent clause and one or more dependent clauses is a \underline{\hspace{4cm}} sentence.  
\vspace{2cm}

10. A sentence with two or more independent clauses joined by a conjunction is a \underline{\hspace{4cm}} sentence.  
\vspace{2cm}
\newpage
\subsection*{Answer Key}

\textbf{Part 1: Multiple-Choice Questions}

1. \textbf{A} – The cat slept on the couch, and the dog napped on the floor. (This is a compound sentence because it has two independent clauses joined by "and.")

2. \textbf{B} – When the bell rang, the students packed their bags and left. (This is a complex sentence because it has one independent clause and one dependent clause starting with "when.")

3. \textbf{A} – After the meeting ended, the group stayed behind to discuss the project, and they decided to schedule another session. (This is a compound-complex sentence because it has two independent clauses joined by "and" and one dependent clause starting with "after.")

\textbf{Part 2: Select All That Apply Questions}

4. \textbf{A, D} – The car stopped suddenly at the red light. (Simple sentence: one independent clause.)  
The lake shimmered under the moonlight. (Simple sentence: one independent clause.)

5. \textbf{A, C} – The chef prepared the meal, and the waiter served it. (Compound sentence: two independent clauses joined by "and.")  
The dog barked loudly, but no one was home to hear it. (Compound sentence: two independent clauses joined by "but.")

6. \textbf{A, B, D} – The movie started late because the projector was broken. (Complex sentence: one independent clause and one dependent clause.)  
Since we had extra time, we decided to explore the city. (Complex sentence: one independent clause and one dependent clause.)  
After the rain stopped, a rainbow appeared in the sky. (Complex sentence: one independent clause and one dependent clause.)

\textbf{Part 4: Fill in the Blank Questions}

9. Complex  
10. Compound

\end{document}

