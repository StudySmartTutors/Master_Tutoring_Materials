\documentclass[12pt]{article}

\usepackage[a4paper, top=0.8in, bottom=0.7in, left=0.7in, right=0.7in]{geometry}
\usepackage{amsmath}
\usepackage{graphicx}
\usepackage{fancyhdr}
\usepackage{tcolorbox}
\usepackage[defaultfam,tabular,lining]{montserrat} %% Option 'defaultfam'
\usepackage[T1]{fontenc}
\renewcommand*\oldstylenums[1]{{\fontfamily{Montserrat-TOsF}\selectfont #1}}
\renewcommand{\familydefault}{\sfdefault}
\usepackage{enumitem}
\usepackage{setspace}

\setlength{\parindent}{0pt}
\hyphenpenalty=10000
\exhyphenpenalty=10000

\pagestyle{fancy}
\fancyhf{}
%\fancyhead[L]{\textbf{6.RI.3: Analyzing Text Details Practice}}
\fancyhead[R]{\includegraphics[width=1cm]{Round Logo.png}}
\fancyfoot[C]{\footnotesize Study Smart Tutors}

\begin{document}

\subsection*{How Individuals, Events, and Ideas are Developed in Texts}
\onehalfspacing

\begin{tcolorbox}[colframe=black!40, colback=gray!0, title=Learning Objective]
\textbf{Objective:} Analyze how individuals, events, or ideas are introduced, illustrated, and elaborated in a text.
\end{tcolorbox}


\subsection*{Answer Key}
\textbf{Part 1: Multiple-Choice Questions}  
1. B  
2. B  
3. B  

\textbf{Part 2: Select All That Apply Questions}  
4. A, C, D  
5. A, C  
6. B, C, D  

\textbf{Part 3: Short Answer Questions}  
7. Answers will vary but should reference the artistry in extracting and spinning filaments and the challenges of environmental threats.  
8. Answers will vary but should mention the preservation of ama culture through tourism and documentaries, as well as the traditional skills passed down.  

\textbf{Part 4: Fill in the Blank Questions}  
9. \textit{Examples, anecdotes, and data.}  
10. \textit{Background information.}  

\end{document}

