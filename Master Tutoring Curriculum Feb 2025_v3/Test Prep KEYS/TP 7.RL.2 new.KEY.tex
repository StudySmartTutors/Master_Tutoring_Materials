\documentclass[12pt]{article}
\usepackage[a4paper, top=0.8in, bottom=0.7in, left=0.7in, right=0.7in]{geometry}
\usepackage{amsmath}
\usepackage{graphicx}
\usepackage{fancyhdr}
\usepackage{tcolorbox}
\usepackage[defaultfam,tabular,lining]{montserrat} %% Option 'defaultfam'
\usepackage[T1]{fontenc}
\renewcommand*\oldstylenums[1]{{\fontfamily{Montserrat-TOsF}\selectfont #1}}
\renewcommand{\familydefault}{\sfdefault}
\usepackage{enumitem}
\usepackage{setspace}

\setlength{\parindent}{0pt}
\hyphenpenalty=10000
\exhyphenpenalty=10000

\pagestyle{fancy}
\fancyhf{}
\fancyhead[L]{\textbf{7.RL.2: Determining Themes and Central Ideas Practice}}
\fancyhead[R]{\includegraphics[width=1cm]{Round Logo.png}}
\fancyfoot[C]{\footnotesize Study Smart Tutors}

\begin{document}

\subsection*{Understanding Themes and Central Ideas}
\onehalfspacing

\begin{tcolorbox}[colframe=black!40, colback=gray!0, title=Learning Objective]
\textbf{Objective:} Determine the theme or central idea of a text and analyze its development over the course of the text.
\end{tcolorbox}


\section*{Answer Key}

\subsection*{Part 1: Multiple-Choice Questions}

1. \textbf{What is the central idea of the passage?}  
\textbf{Answer:} B. Growth and improvement are more important than victory.  
\textbf{Explanation:} The passage emphasizes Emma’s realization that growth mattered more than winning, as she was proud of her improvement and determination.

\vspace{1cm}
2. \textbf{What theme is revealed in this passage?}  
\textbf{Answer:} A. Natural disasters bring people together.  
\textbf{Explanation:} The passage shows how the townspeople came together to rebuild after the storm, demonstrating unity in the face of adversity.

\vspace{1cm}
3. \textbf{What central idea emerges in the story?}  
\textbf{Answer:} B. Perseverance and learning from failure lead to success.  
\textbf{Explanation:} The central idea is that Alex’s persistence and ability to learn from his mistakes allowed him to eventually succeed with his robot.

\subsection*{Part 2: Select All That Apply Questions}

4. \textbf{Select all details that support the central idea in Emma’s story from question 1:}  
\textbf{Answer:} A. Emma studied her mistakes and improved her technique. \\
C. Emma realized growth was more important than winning. \\
D. Emma trained harder for her next race.  
\textbf{Explanation:} These details demonstrate Emma’s dedication to improvement and her realization that personal growth mattered more than victory.

\vspace{1cm}
5. \textbf{Which details reveal the theme in the story from question 2 about the town recovering from a flood?}  
\textbf{Answer:} A. Families opened their homes to displaced neighbors. \\
B. Volunteers worked tirelessly to rebuild. \\
C. Mr. Hughes organized a fundraiser for supplies.  
\textbf{Explanation:} These actions highlight the theme of unity and collaboration in times of hardship.

\vspace{1cm}
6. \textbf{Select all details that convey the central idea in Alex’s story from question 3:}  
\textbf{Answer:} A. Alex never gave up despite early failures. \\
C. Alex studied robotics to improve his design. \\
D. Alex learned that perseverance was valuable.  
\textbf{Explanation:} These details show how Alex's perseverance, learning from mistakes, and continual effort led to success.

\subsection*{Part 3: Short Answer Questions}

7. \textbf{How does the story of Emma demonstrate the value of growth over victory? Use evidence in the passage from question 1 to support your answer.}  
\textbf{Answer:} Emma demonstrates the value of growth over victory through her determination to improve after placing fourth. She studied her mistakes, trained harder, and realized that personal growth was more meaningful than winning. She felt pride in her progress, not just in her race results.

\vspace{1cm}
8. \textbf{Based on the story about Alex, explain how failure contributed to his ultimate success. Provide textual evidence in your response from the passage from question 3.}  
\textbf{Answer:} Alex’s failures were crucial in helping him improve his design. After his robot’s initial failure, he studied robotics and worked tirelessly to fix it. His perseverance and ability to learn from his mistakes ultimately led him to success, as his robot was perfect by the time of the science fair.

\subsection*{Part 4: Fill in the Blank Questions}

9. The theme of a story is a general statement about life, people, or society and should not be about a specific \underline{event}.  
\textbf{Answer:} event.

10. A summary of a fictional text should be \underline{objective} and not biased.  
\textbf{Answer:} objective.






\end{document}
