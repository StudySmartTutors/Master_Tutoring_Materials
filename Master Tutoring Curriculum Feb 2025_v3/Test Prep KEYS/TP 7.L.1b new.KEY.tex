\documentclass[12pt]{article}

\usepackage[a4paper, top=0.8in, bottom=0.7in, left=0.7in, right=0.7in]{geometry}
\usepackage{amsmath}
\usepackage{graphicx}
\usepackage{fancyhdr}
\usepackage{tcolorbox}
\usepackage[defaultfam,tabular,lining]{montserrat} %% Option 'defaultfam'
\usepackage[T1]{fontenc}
\renewcommand*\oldstylenums[1]{{\fontfamily{Montserrat-TOsF}\selectfont #1}}
\renewcommand{\familydefault}{\sfdefault}
\usepackage{enumitem}
\usepackage{setspace}

\setlength{\parindent}{0pt}
\hyphenpenalty=10000
\exhyphenpenalty=10000

\pagestyle{fancy}
\fancyhf{}
\fancyhead[L]{\textbf{7.L.1b: Sentence Structure Practice}}
\fancyhead[R]{\includegraphics[width=1cm]{Round Logo.png}}
\fancyfoot[C]{\footnotesize Study Smart Tutors}

\begin{document}

\subsection*{Understanding Sentence Structure and Relationships Among Ideas}
\onehalfspacing

\begin{tcolorbox}[colframe=black!40, colback=gray!0, title=Learning Objective]
\textbf{Objective:} Demonstrate command of the conventions of standard English grammar by choosing among simple, compound, complex, and compound-complex sentences to signal differing relationships among ideas.
\end{tcolorbox}


\subsection*{Answer Key}

\textbf{Part 1: Multiple-Choice Questions}

1. \textbf{A} – The cat slept on the couch, and the dog napped on the floor. (This is a compound sentence because it has two independent clauses joined by "and.")

2. \textbf{B} – When the bell rang, the students packed their bags and left. (This is a complex sentence because it has one independent clause and one dependent clause starting with "when.")

3. \textbf{A} – After the meeting ended, the group stayed behind to discuss the project, and they decided to schedule another session. (This is a compound-complex sentence because it has two independent clauses joined by "and" and one dependent clause starting with "after.")

\textbf{Part 2: Select All That Apply Questions}

4. \textbf{A, D} – The car stopped suddenly at the red light. (Simple sentence: one independent clause.)  
The lake shimmered under the moonlight. (Simple sentence: one independent clause.)

5. \textbf{A, C} – The chef prepared the meal, and the waiter served it. (Compound sentence: two independent clauses joined by "and.")  
The dog barked loudly, but no one was home to hear it. (Compound sentence: two independent clauses joined by "but.")

6. \textbf{A, B, D} – The movie started late because the projector was broken. (Complex sentence: one independent clause and one dependent clause.)  
Since we had extra time, we decided to explore the city. (Complex sentence: one independent clause and one dependent clause.)  
After the rain stopped, a rainbow appeared in the sky. (Complex sentence: one independent clause and one dependent clause.)

\textbf{Part 4: Fill in the Blank Questions}

9. Complex  
10. Compound





\end{document}

