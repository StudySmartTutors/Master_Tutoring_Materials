\documentclass[12pt]{article}
\usepackage[a4paper, top=0.8in, bottom=0.7in, left=0.7in, right=0.7in]{geometry}
\usepackage{amsmath}
\usepackage{graphicx}
\usepackage{fancyhdr}
\usepackage{tcolorbox}
\usepackage[defaultfam,tabular,lining]{montserrat} %% Option 'defaultfam'
\usepackage[T1]{fontenc}
\renewcommand*\oldstylenums[1]{{\fontfamily{Montserrat-TOsF}\selectfont #1}}
\renewcommand{\familydefault}{\sfdefault}
\usepackage{enumitem}
\usepackage{setspace}


\setlength{\parindent}{0pt}
\setlength{\emergencystretch}{1.75em}
\hyphenpenalty=10000
\exhyphenpenalty=10000

\pagestyle{fancy}
\fancyhf{}
\fancyhead[L]{\textbf{7.RI.3: Interactions in Informational Text Practice}}
\fancyhead[R]{\includegraphics[width=1cm]{Round Logo.png}}
\fancyfoot[C]{\footnotesize Study Smart Tutors}

\begin{document}

\subsection*{Understanding Interactions Between Individuals, Events, and Ideas}
\onehalfspacing

\begin{tcolorbox}[colframe=black!40, colback=gray!0, title=Learning Objective]
\textbf{Objective:} Analyze the interactions between individuals, events, and ideas in a text to explain how they contribute to the development of the text.
\end{tcolorbox}


\subsection*{Answer Key}

\textbf{Part 1: Multiple-Choice Questions}

1. B. Farming sea urchins preserves wild populations and creates economic opportunities.  
2. A. Carnivorous plants capture prey using sticky, snapping, or tubular traps.  
3. B. Mushrooms decompose organic matter and support plant growth.  

\textbf{Part 2: Select All That Apply Questions}

4. A, B, C.  
5. A, B, C.  
6. A, B, C.  

\textbf{Part 3: Short Answer Questions}

7. Sea urchin farming helps balance environmental and economic needs by reducing pressure on wild sea urchin populations and providing a consistent supply of sea urchins to consumers. The passage mentions that "farming sea urchins allows for sustainable harvesting, reducing pressure on wild populations" and "creates economic opportunities for coastal communities."

8. Carnivorous plants adapt to nutrient-poor environments by developing specialized mechanisms to capture and digest insects. For example, Venus flytraps "snap shut when triggered by prey," pitcher plants have "tubular leaves filled with digestive fluids," and sundews "produce sticky, glandular hairs that ensnare prey."

\textbf{Part 4: Fill in the Blank Questions}

9. support  
10. relevant  

\end{document}

