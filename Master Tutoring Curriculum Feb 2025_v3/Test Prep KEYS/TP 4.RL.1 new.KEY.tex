\documentclass[12pt]{article}

\usepackage[a4paper, top=0.8in, bottom=0.7in, left=0.7in, right=0.7in]{geometry}
\usepackage{amsmath}
\usepackage{graphicx}
\usepackage{fancyhdr}
\usepackage{tcolorbox}
\usepackage{multicol}
\usepackage{pifont} % For checkboxes
\usepackage[defaultfam,tabular,lining]{montserrat} %% Option 'defaultfam'
\usepackage[T1]{fontenc}
\renewcommand*\oldstylenums[1]{{\fontfamily{Montserrat-TOsF}\selectfont #1}}
\renewcommand{\familydefault}{\sfdefault}
\usepackage{enumitem}
\usepackage{setspace}
\usepackage{parcolumns}
\usepackage{tabularx}

\setlength{\parindent}{0pt}
\hyphenpenalty=10000
\exhyphenpenalty=10000

\pagestyle{fancy}
\fancyhf{}
%\fancyhead[L]{\textbf{4.RL.1: Making Inferences and Citing Evidence Practice}}
\fancyhead[R]{\includegraphics[width=1cm]{Round Logo.png}}
\fancyfoot[C]{\footnotesize Study Smart Tutors}

\begin{document}

\subsection*{Making Inferences and Citing Evidence}
\onehalfspacing

\begin{tcolorbox}[colframe=black!40, colback=gray!0, title=Learning Objective]
\textbf{Objective:} Refer explicitly to the text when explaining what it says and when making inferences.
\end{tcolorbox}


% \vspace{3cm}
% \newpage
\section*{Answer Key}

\subsection*{Part 1: Multiple-Choice Questions}

B. Lila enjoys reading more than playing games.

B. The stranger taught them how to collect and store rainwater.

B. The crow used stones to raise the water level.

B. The story is set in a desert or dry area.

\subsection*{Part 2: Select All That Apply Questions}

A, B, C.

Emma helped her neighbor with grocery bags.
Emma helped her classmates with homework.
Emma volunteered at an animal shelter.
A, C.

Excitement
Determination
A, B, C.

It is rare and beautiful.
It represents Mia’s goal and determination.
It teaches Mia about patience.
\subsection*{Part 3: Short Answer Questions}

Answer: Mia’s reaction to seeing the golden bird reveals her patience, determination, and focus. Her desire to capture the bird with her camera shows her dedication to her goal, and her ability to remain still when she spotted the bird indicates her attentiveness and carefulness.

Answer: The author uses the setting to create a sense of mystery by describing the quiet, eerie atmosphere of the forest. The unusual silence, along with Mia tiptoeing along the trail, builds suspense and anticipation. The description of the golden bird as rare and fleeting adds to the mystery and intrigue of the setting.

\subsection*{Part 4: Fill in the Blank Questions}

The setting of a story helps the reader understand where and \underline{when} the events take place.
\end{document}
