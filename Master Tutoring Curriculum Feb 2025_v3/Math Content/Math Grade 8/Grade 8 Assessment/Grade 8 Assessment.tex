\documentclass[12pt]{article}
\usepackage[a4paper, top=0.8in, bottom=0.7in, left=0.8in, right=0.8in]{geometry}
\usepackage{amsmath}
\usepackage{amsfonts}
\usepackage{graphicx}
\usepackage{fancyhdr}
\usepackage{enumitem}
\usepackage{setspace}
\usepackage{tcolorbox}
\usepackage{xcolor}
\usepackage{tikz}
\usepackage[defaultfam,tabular,lining]{montserrat}
\usepackage[T1]{fontenc}
\renewcommand{\familydefault}{\sfdefault}

\setlength{\headheight}{27.11148pt}

% Header Configuration
\pagestyle{fancy}
\fancyhf{}
\fancyhead[L]{AASA Practice Exam I}
\fancyhead[R]{\includegraphics[width=0.8cm]{Round Logo.png}} % Replace 'RoundLogo.png' with your logo file
\fancyfoot[C]{\footnotesize © Study Smart Tutors}

\begin{document}

\subsection*{Assessment I: Math Pre/Post-Assessment}
\onehalfspacing

\begin{tcolorbox}[colframe=black!50, colback=white, title=Assessment Directions]
\textbf{Directions:} Solve each question carefully. For multiple-choice questions, circle the best answer. For "select all that apply" questions, mark all the correct answers. For performance tasks, explain your reasoning clearly.
\end{tcolorbox}

% Problem 1: Simplifying Exponents
\begin{tcolorbox}[colframe=black!50, colback=white, title=\textbf{Problem 1 (8.EE.A.1)}]
Simplify the expression \(2^5 \cdot 2^3\).

\begin{enumerate}[label=(\Alph*)]
    \item \(2^8\)
    \item \(2^{15}\)
    \item \(2^{10}\)
    \item \(2^6\)
\end{enumerate}
\end{tcolorbox}

% Problem 2: Exponent Properties
\begin{tcolorbox}[colframe=black!50, colback=white, title=\textbf{Problem 2 (8.EE.A.1)}]
Simplify the expression \(\frac{3^7}{3^2}\).

\begin{enumerate}[label=(\Alph*)]
    \item \(3^5\)
    \item \(3^9\)
    \item \(3^3\)
    \item \(3^{10}\)
\end{enumerate}
\end{tcolorbox}

% Problem 3: Solving Cube Root Equation
\begin{tcolorbox}[colframe=black!50, colback=white, title=\textbf{Problem 3 (8.EE.A.2)}]
Solve for \(x\): \(x^3 = 64\). Show your work.

\vspace{3cm}
\textbf{Answer:} \_\_\_\_\_\_\_\_\_\_\_\_\_\_\_\_\_\_\_\_\_\_
\end{tcolorbox}

% Problem 4: Identifying Irrational Numbers
\begin{tcolorbox}[colframe=black!50, colback=white, title=\textbf{Problem 4 (8.NS.A.1)}]
Which of the following numbers is irrational? Select one:

\begin{enumerate}[label=(\Alph*)]
    \item \( \sqrt{25} \)
    \item \( \pi \)
    \item \( \frac{3}{4} \)
    \item \( 0.5 \)
\end{enumerate}
\end{tcolorbox}

% Problem 5: Approximating Irrational Numbers
\begin{tcolorbox}[colframe=black!50, colback=white, title=\textbf{Problem 5 (8.NS.A.2)}]
Approximate \(\sqrt{45}\) to the nearest tenth. Explain your reasoning.

\vspace{2cm}
\textbf{Answer:} \_\_\_\_\_\_\_\_\_\_\_\_\_\_\_\_\_\_\_\_\_\_
\end{tcolorbox}

% Problem 6: Comparing Irrational Numbers
\begin{tcolorbox}[colframe=black!50, colback=white, title=\textbf{Problem 6 (8.NS.A.2)}]
Which is greater: \( \sqrt{50} \) or \( 7 \)? Use reasoning to support your answer.

\vspace{1.75cm}
\textbf{Answer:} \_\_\_\_\_\_\_\_\_\_\_\_\_\_\_\_\_\_\_\_\_\_
\end{tcolorbox}

% Problem 7: Solving Linear Equations
\begin{tcolorbox}[colframe=black!50, colback=white, title=\textbf{Problem 7 (8.EE.C.7)}]
Solve the equation \(3(x - 2) = 9x + 4\). Show your work and write your solution.

\vspace{2cm}
\textbf{Answer:} \_\_\_\_\_\_\_\_\_\_\_\_\_\_\_\_\_\_\_\_\_\_
\end{tcolorbox}

% Problem 8: Linear Equations
\begin{tcolorbox}[colframe=black!50, colback=white, title=\textbf{Problem 8 (8.EE.C.7)}]
Which of the following equations have no solution? Select all that apply:

\begin{enumerate}[label=(\Alph*)]
    \item \(5x + 2 = 5x + 4\)
    \item \(3(x - 1) = 3x - 3\)
    \item \(x + 1 = x + 1\)
    \item \(2x - 4 = 2x + 4\)
\end{enumerate}
\end{tcolorbox}

% Problem 9: Interpreting Slope
\begin{tcolorbox}[colframe=black!50, colback=white, title=\textbf{Problem 9 (8.EE.B.5)}]
The graph below shows the relationship between distance (in miles) and time (in hours) for a biker riding at a constant speed. Select all the true statements about the slope of the graph.

\begin{center}
\begin{tikzpicture}[scale=0.8]
    \draw[thick,->] (0,0) -- (6,0) node[right]{Time (hours)};
    \draw[thick,->] (0,0) -- (0,6) node[above]{Distance (miles)};
    \draw[thick] (0,0) -- (5,5);
    \draw[fill=black] (4,4) circle (2pt);
    \node[below right] at (4,4) {(4,4)};
    \foreach \x in {0,1,...,5} {
        \draw (\x,0.1) -- (\x,-0.1) node[below]{\x};
    }
    \foreach \y in {0,1,...,5} {
        \draw (0.1,\y) -- (-0.1,\y) node[left]{\y};
    }
\end{tikzpicture}
\end{center}

\begin{enumerate}[label=(\Alph*)]
    \item The slope represents the total distance traveled
    \item The slope is \(  1 \).
    \item The slope is \( \frac{1}{4} \).
    \item The slope is constant.
\end{enumerate}
\end{tcolorbox}

% Problem 10: Graphing a Line
\begin{tcolorbox}[colframe=black!50, colback=white, title=\textbf{Problem 10 (8.EE.B.5)}]
Graph the equation \(y = 2x\). Identify the slope and explain its meaning.

\begin{center}
\begin{tikzpicture}[scale=0.8]
    \draw[thick,->] (0,0) -- (6,0) node[right]{\(x\)};
    \draw[thick,->] (0,0) -- (0,6) node[above]{\(y\)};
    \foreach \x in {0,1,...,5} {
        \draw (\x,0.1) -- (\x,-0.1) node[below]{\x};
    }
    \foreach \y in {0,1,...,5} {
        \draw (0.1,\y) -- (-0.1,\y) node[left]{\y};
    }
\end{tikzpicture}
\end{center}

\vspace{1cm}
\textbf{Answer:} \_\_\_\_\_\_\_\_\_\_\_\_\_\_\_\_\_\_\_\_\_\_
\end{tcolorbox}

\end{document}
