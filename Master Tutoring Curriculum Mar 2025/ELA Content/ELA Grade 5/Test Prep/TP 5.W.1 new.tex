\documentclass[12pt]{article}

\usepackage[a4paper, top=0.8in, bottom=0.7in, left=0.7in, right=0.7in]{geometry}
\usepackage{amsmath}
\usepackage{graphicx}
\usepackage{fancyhdr}
\usepackage{tcolorbox}
\usepackage[defaultfam,tabular,lining]{montserrat} %% Option 'defaultfam'
\usepackage[T1]{fontenc}
\renewcommand*\oldstylenums[1]{{\fontfamily{Montserrat-TOsF}\selectfont #1}}
\renewcommand{\familydefault}{\sfdefault}
\usepackage{enumitem}
\usepackage{setspace}

\setlength{\parindent}{0pt}
\hyphenpenalty=10000
\exhyphenpenalty=10000

\pagestyle{fancy}
\fancyhf{}
%\fancyhead[L]{\textbf{5.W.1: Argumentative Writing Practice}}
\fancyhead[R]{\includegraphics[width=1cm]{Round Logo.png}}
\fancyfoot[C]{\footnotesize Study Smart Tutors}

\begin{document}

\subsection*{Argumentative Writing: Exploring Homework Policies}
\onehalfspacing

\begin{tcolorbox}[colframe=black!40, colback=gray!0, title=Learning Objective]
\textbf{Objective:} Write an argumentative essay that introduces and supports a claim with reasons and evidence.
\end{tcolorbox}

\subsection*{Prompt}

After reading the passages below, write an argumentative essay responding to the question:  
"Should schools give students homework?"  
Use evidence from the texts to support your position, address counterclaims, and provide a strong conclusion.

\subsection*{Passage 1: The Benefits of Homework}

Homework helps students practice what they learn in class and build important study habits. When students review lessons at home, they remember the material better and improve their skills. Homework also teaches responsibility and time \\management because students must complete their assignments on time. For example, practicing math problems at home can help students feel more confident during tests. Parents can also get involved by helping with homework, which strengthens the connection between school and home. While too much homework can be overwhelming, giving a reasonable amount helps students prepare for future success in school and beyond.

\subsection*{Passage 2: The Problems with Homework}

Some people believe that homework causes more harm than good. Many students already spend hours in school, and adding homework can leave them feeling stressed and tired. Instead of relaxing or spending time with family, students may feel pressured to finish their assignments. For example, a student who struggles with math might feel frustrated if they don’t understand their homework. In addition, some families don’t have the resources or time to help with assignments. Too much homework can make learning less enjoyable and take away from activities like sports, music, or playing with friends.
\newpage
\subsection*{Passage 3: Finding a Balance with Homework}

Many teachers and parents believe the best approach is to give a balanced amount of homework. For example, assigning 20 minutes of reading each night helps students develop a love for books without feeling overwhelmed. Short assignments in math or writing can reinforce lessons without taking too much time. Teachers can also give students choices about when and how to complete their homework, making it more flexible. By focusing on quality over quantity, schools can make homework a positive experience that supports learning and still gives students time for other activities.

\subsection*{Instructions for Students}

\begin{enumerate}
    \item **Choose a side.** Decide whether you believe schools should give homework, reduce it, or find a balanced approach.
    \item **Plan your essay.** Organize your ideas and include:
    \begin{itemize}
        \item A clear claim that states your position.
        \item Reasons and evidence from the texts to support your argument.
        \item Acknowledgment of the other side’s perspective.
        \item A strong conclusion that reinforces your position.
    \end{itemize}
    \item **Write your essay.** Use formal language and logical reasoning to present your argument.
    \item **Revise and edit.** Check your essay for grammar, clarity, and organization.
\end{enumerate}

\subsection*{Scoring Guide}

Your essay will be evaluated on the following criteria:
\begin{enumerate}
    \item \textbf{Content and Ideas}: Strength of argument and use of evidence.
    \item \textbf{Organization}: Clear introduction, logical transitions, and structured paragraphs.
    \item \textbf{Style and Tone}: Formal style, precise language, and strong voice.
    \item \textbf{Conventions}: Proper grammar, punctuation, and spelling.
\end{enumerate}

\end{document}
