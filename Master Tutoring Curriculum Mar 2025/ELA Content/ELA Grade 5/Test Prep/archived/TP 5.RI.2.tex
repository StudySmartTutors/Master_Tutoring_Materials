\documentclass[12pt]{article}

\usepackage[a4paper, top=0.8in, bottom=0.7in, left=0.7in, right=0.7in]{geometry}
\usepackage{amsmath}
\usepackage{graphicx}
\usepackage{fancyhdr}
\usepackage{tcolorbox}
\usepackage{multicol}
\usepackage{pifont} % For checkboxes
\usepackage[defaultfam,tabular,lining]{montserrat} %% Option 'defaultfam'
\usepackage[T1]{fontenc}
\renewcommand*\oldstylenums[1]{{\fontfamily{Montserrat-TOsF}\selectfont #1}}
\renewcommand{\familydefault}{\sfdefault}
\usepackage{enumitem}
\usepackage{setspace}
\usepackage{parcolumns}
\usepackage{tabularx}

\setlength{\parindent}{0pt}
\hyphenpenalty=10000
\exhyphenpenalty=10000

\pagestyle{fancy}
\fancyhf{}
%\fancyhead[L]{\textbf{5.RI.2: Main Idea and Key Details}}
\fancyhead[R]{\includegraphics[width=1cm]{Round Logo.png}}
\fancyfoot[C]{\footnotesize Study Smart Tutors}

\begin{document}

\onehalfspacing

\subsection*{Informational Text: Pluto Reclassified as a Moon}

\begin{tcolorbox}[colframe=black!40, colback=gray!5]

\begin{spacing}{1.15}

In 2006, Pluto was officially reclassified from a planet to a dwarf planet by the International Astronomical Union (IAU). However, new findings in the years that followed have led to another reconsideration. In 2023, scientists proposed that Pluto should be reclassified once again — this time as a moon of Neptune.

Pluto was discovered in 1930 by astronomer Clyde Tombaugh and for many decades, it was considered the ninth planet in our solar system. Pluto is much smaller than the eight planets in our solar system. It is about one-sixth the size of Earth’s moon and has five known moons of its own, with Charon being the largest.

Scientists have long debated Pluto's classification because of its unusual orbit. Unlike the planets, which travel in nearly circular orbits, Pluto’s orbit is much more elliptical, or oval-shaped. It even crosses into the orbit of Neptune for part of its journey around the Sun. This behavior, combined with its relatively small size and location in the Kuiper Belt, led to the IAU's decision to reclassify Pluto as a dwarf planet in 2006.

But the proposal to reclassify Pluto as a moon of Neptune stems from recent studies that show how closely Pluto and Neptune are connected. Pluto’s orbit is locked into a gravitational dance with Neptune’s, meaning that they can never collide, even though their paths cross. Some scientists argue that Pluto's size, orbit, and the relationship with Neptune make it more appropriate to consider it as one of Neptune’s moons rather than a dwarf planet.

The idea of reclassifying Pluto as a moon of Neptune is still being discussed by astronomers, and a final decision has not been made. Regardless of its classification, Pluto remains a fascinating object in our solar system, and scientists continue to learn more about it through missions like NASA's New Horizons.

\end{spacing}

\end{tcolorbox}

\subsection*{Multiple Choice Questions}

\begin{enumerate}

    \item Who discovered Pluto?

    \begin{enumerate}[label=\Alph*.]
        \item Galileo Galilei
        \item Carl Sagan
        \item Clyde Tombaugh
        \item Isaac Newton
    \end{enumerate}
    
    \vspace{0.5cm}

    \item In what year was Pluto reclassified as a dwarf planet?

    \begin{enumerate}[label=\Alph*.]
        \item 1990
        \item 2006
        \item 2015
        \item 2020
    \end{enumerate}
    
    \vspace{0.5cm}

    \item What is the new classification proposed for Pluto in 2023?

    \begin{enumerate}[label=\Alph*.]
        \item A planet
        \item A star
        \item A moon of Neptune
        \item A dwarf planet
    \end{enumerate}
    
    \vspace{0.5cm}

    \item Who proposed the new idea of classifying Pluto as a moon of Neptune?

    \begin{enumerate}[label=\Alph*.]
        \item NASA scientists
        \item The International Astronomical Union
        \item Astronomers in 2023
        \item Clyde Tombaugh
    \end{enumerate}
    
    \vspace{0.5cm}

    \item What is Pluto's most noticeable feature compared to other planets?

    \begin{enumerate}[label=\Alph*.]
        \item Its large size
        \item Its unusual orbit
        \item Its many moons
        \item Its rings
    \end{enumerate}
    
    \vspace{0.5cm}

    \item What is Pluto's relationship with Neptune?

    \begin{enumerate}[label=\Alph*.]
        \item They never cross paths
        \item Pluto's orbit crosses Neptune's orbit
        \item They are in separate orbits
        \item Pluto orbits Neptune
    \end{enumerate}
    
    \vspace{0.5cm}

    \item What makes Pluto’s orbit unique?

    \begin{enumerate}[label=\Alph*.]
        \item It is perfectly circular
        \item It is elliptical and crosses into Neptune’s orbit
        \item It is always closer to the Sun than Neptune
        \item It stays in the same orbit as Neptune
    \end{enumerate}
    
    \vspace{0.5cm}

    \item How many moons does Pluto have?

    \begin{enumerate}[label=\Alph*.]
        \item 3
        \item 4
        \item 5
        \item 6
    \end{enumerate}
    
    \vspace{0.5cm}

    \item What is the largest moon of Pluto?

    \begin{enumerate}[label=\Alph*.]
        \item Charon
        \item Ganymede
        \item Titan
        \item Europa
    \end{enumerate}
    
    \vspace{0.5cm}

    \item What is the Kuiper Belt?

    \begin{enumerate}[label=\Alph*.]
        \item A belt of asteroids around the Sun
        \item A region beyond Neptune filled with icy objects
        \item The region around Earth’s orbit
        \item A group of moons of Jupiter
    \end{enumerate}
    
    \vspace{0.5cm}

    \item What is the main reason for the debate about Pluto’s classification?

    \begin{enumerate}[label=\Alph*.]
        \item Its size
        \item Its orbit and size
        \item Its number of moons
        \item Its distance from the Sun
    \end{enumerate}
    
    \vspace{0.5cm}

    \item Why did the International Astronomical Union reclassify Pluto in 2006?

    \begin{enumerate}[label=\Alph*.]
        \item Because Pluto was too far from the Sun
        \item Because Pluto was too small and did not meet the criteria to be a planet
        \item Because Pluto was discovered to have too many moons
        \item Because Pluto was too close to Neptune
    \end{enumerate}
    
    \vspace{0.5cm}

    \item What is Pluto's size compared to Earth’s moon?

    \begin{enumerate}[label=\Alph*.]
        \item The same size
        \item Half the size
        \item One-sixth the size
        \item Twice the size
    \end{enumerate}
    
    \vspace{0.5cm}

    \item What did NASA's New Horizons mission do?

    \begin{enumerate}[label=\Alph*.]
        \item Landed on Pluto
        \item Took pictures and gathered data about Pluto
        \item Discovered Pluto’s moons
        \item Discovered Pluto’s rings
    \end{enumerate}
    
    \vspace{0.5cm}

    \item What is the main topic of the text?

    \begin{enumerate}[label=\Alph*.]
        \item The discovery of Pluto
        \item The reclassification of Pluto
        \item Pluto’s moons
        \item NASA’s mission to Pluto
    \end{enumerate}
    
    \vspace{0.5cm}

    \item Why do some scientists think Pluto should be considered a moon of Neptune?

    \begin{enumerate}[label=\Alph*.]
        \item Because Pluto is made of gas
        \item Because Pluto’s orbit and size are more like Neptune’s moons
        \item Because Pluto is larger than Neptune’s moons
        \item Because Pluto and Neptune are far apart
    \end{enumerate}
    
    \vspace{0.5cm}

    \item What is the most important reason for the discussion about Pluto's classification?

    \begin{enumerate}[label=\Alph*.]
        \item Its distance from the Sun
        \item Its relationship with Neptune
        \item Its number of moons
        \item Its orbit inside the Kuiper Belt
    \end{enumerate}
    
    \vspace{0.5cm}

    \item What is Pluto's current status?

    \begin{enumerate}[label=\Alph*.]
        \item It is a planet
        \item It is a dwarf planet
        \item It is a moon of Neptune
        \item It is an asteroid
    \end{enumerate}

\end{enumerate}

\end{document}
