\documentclass[12pt]{article}

\usepackage[a4paper, top=0.8in, bottom=0.7in, left=0.7in, right=0.7in]{geometry}
\usepackage{amsmath}
\usepackage{graphicx}
\usepackage{fancyhdr}
\usepackage{tcolorbox}
\usepackage{multicol}
\usepackage{pifont} % For checkboxes
\usepackage[defaultfam,tabular,lining]{montserrat} %% Option 'defaultfam'
\usepackage[T1]{fontenc}
\renewcommand*\oldstylenums[1]{{\fontfamily{Montserrat-TOsF}\selectfont #1}}
\renewcommand{\familydefault}{\sfdefault}
\usepackage{enumitem}
\usepackage{setspace}
\usepackage{parcolumns}
\usepackage{tabularx}

\setlength{\parindent}{0pt}
\hyphenpenalty=10000
\exhyphenpenalty=10000

\pagestyle{fancy}
\fancyhf{}
%\fancyhead[L]{\textbf{5.RI.5: Text Structure and Purpose Practice}}
\fancyhead[R]{\includegraphics[width=1cm]{Round Logo.png}}
\fancyfoot[C]{\footnotesize Study Smart Tutors}

\begin{document}

\subsection*{Analyzing Text Structure and Purpose}
\onehalfspacing

\begin{tcolorbox}[colframe=black!40, colback=gray!0, title=Learning Objective]
\textbf{Objective:} Analyze the structure of a text, including how sections contribute to the development of ideas and the whole.
\end{tcolorbox}

\subsection*{Part 1: Multiple-Choice Questions}

1. \textbf{What is the purpose of the introductory section in the passage below?}\\
"Plastic pollution is one of the most pressing environmental issues of our time. Every year, millions of tons of plastic enter our oceans, harming marine life and ecosystems. Sea turtles often mistake floating plastic bags for jellyfish, swallowing them and blocking their digestive systems. Microplastics, tiny pieces of plastic less than 5mm long, are ingested by fish, which can then enter the human food chain. Additionally, plastic debris damages coral reefs, weakening their ability to support marine \\biodiversity. Plastic pollution does not easily decompose and can remain in the environment for hundreds of years. While efforts like beach cleanups and recycling programs are making a difference, they are not enough. Governments, businesses, and individuals must work together to develop innovative solutions, such as \\biodegradable materials and stricter regulations on plastic production. This \\introduction highlights the global impact of plastic pollution and sets the stage for discussing possible solutions to this growing crisis."\\
\begin{enumerate}[label=\Alph*.]
    \item To summarize a solution to plastic pollution.  
    \item To provide background information on plastic pollution.  
    \item To introduce a specific species affected by pollution.  
    \item To describe how recycling solves plastic pollution.  
\end{enumerate}

\vspace{1cm}

2. \textbf{How does the section on “Renewable Energy Benefits” contribute to the text?}\\
"The world is shifting toward renewable energy sources like solar and wind power. One major advantage of renewable energy is its ability to reduce greenhouse gas emissions, which contribute to climate change. For example, solar panels generate electricity by capturing sunlight, while wind turbines harness wind energy to produce power. Unlike fossil fuels, renewable energy sources do not release harmful pollutants into the air, improving air quality and public health. Additionally, renewable energy is sustainable, meaning it can be replenished naturally without depleting resources. Economically, renewable energy creates jobs in industries like installation, \\maintenance, and research. Communities that invest in renewable energy also reduce their dependence on imported fuels, increasing energy security. However, the \\transition to renewables is not without challenges, such as the high initial cost of technology and reliance on weather conditions. Despite these obstacles, the benefits of renewable energy far outweigh the drawbacks, making it a critical solution for a cleaner, healthier future."\\
\begin{enumerate}[label=\Alph*.]
    \item It explains the environmental impact of fossil fuels.  
    \item It highlights the specific advantages of renewable energy.  
    \item It describes the history of energy production.  
    \item It introduces challenges associated with renewable energy.  
\end{enumerate}

\vspace{1cm}

3. \textbf{What role does the conclusion play in the passage below?}\\
"Bees are vital to the health of our ecosystems and food supply. As pollinators, they facilitate the reproduction of many crops and wild plants, ensuring the production of fruits, vegetables, and seeds. Without bees, food chains would collapse, leading to shortages and biodiversity loss. Unfortunately, bee populations are declining due to habitat destruction, pesticide use, and climate change. Farmers and \\environmentalists are working to address this crisis through sustainable farming practices, planting wildflowers, and banning harmful chemicals. Supporting local beekeepers is another way to protect these essential insects. The conclusion \\emphasizes that protecting bees is not just an environmental issue but a matter of human survival. By taking immediate action to safeguard their habitats and reduce harmful practices, we can ensure a thriving planet for future generations. The \\conclusion reinforces the importance of bees and calls readers to contribute to \\solutions that protect them."\\
\begin{enumerate}[label=\Alph*.]
    \item To introduce new threats to bee populations.  
    \item To explain why bees are unimportant.  
    \item To emphasize the importance of protecting bees.  
    \item To provide detailed statistics about bee populations.  
\end{enumerate}

\newpage

\subsection*{Part 2: Select All That Apply Questions}

4. Select \textbf{all} the ways the introductory paragraph in a text helps the reader:\\
\begin{enumerate}[label=\Alph*.]
    \item It provides background information about the topic.  
    \item It gives detailed solutions to a problem.  
    \item It explains why the topic is important.  
    \item It introduces the main idea of the text.  
\end{enumerate}

\vspace{1cm}

5. What are the purposes of the conclusion in an informational text?\\
\begin{enumerate}[label=\Alph*.]
    \item To summarize the main points.  
    \item To introduce new information unrelated to the topic.  
    \item To restate the importance of the topic.  
    \item To give the reader a call to action.  
\end{enumerate}

\vspace{1cm}

6. How could a section titled “Challenges of Renewable Energy” contribute to a text?\\
\begin{enumerate}[label=\Alph*.]
    \item By describing why renewable energy is unnecessary.  
    \item By highlighting potential obstacles to renewable energy adoption.  
    \item By explaining how renewable energy benefits the environment.  
    \item By offering solutions to overcome challenges in renewable energy.  
\end{enumerate}


\newpage
\subsection*{Part 3: Short Answer Questions}

7. How does the section on “Bee Population Decline” from question 3 contribute to the main idea of the text? Use specific examples.\\
\vspace{4cm}

8. Summarize how the section on “Plastic Pollution Effects” from question 1 helps develop the main idea.\\
\vspace{4cm}

\subsection*{Part 4: Fill in the Blank Questions}

9. The \underline{\hspace{4cm}} of a text provides important background information to set the stage for the main idea.
\vspace{1cm}


10. A  \underline{\hspace{4cm}} often summarizes the main idea and encourages readers to take action.

\vspace{1cm}
% \newpage
% \section*{Answer Key}

% \subsection*{Part 1: Multiple-Choice Questions}

% 1. **B.** To provide background information on plastic pollution.  

% 2. **B.** It highlights the specific advantages of renewable energy.  

% 3. **C.** To emphasize the importance of protecting bees.  

% \subsection*{Part 2: Select All That Apply Questions}

% 4. **A, C, D.**  
%    - It provides background information about the topic.  
%    - It explains why the topic is important.  
%    - It introduces the main idea of the text.  

% 5. **A, C, D.**  
%    - To summarize the main points.  
%    - To restate the importance of the topic.  
%    - To give the reader a call to action.  

% 6. **B, D.**  
%    - By highlighting potential obstacles to renewable energy adoption.  
%    - By offering solutions to overcome challenges in renewable energy.  

% \subsection*{Part 3: Short Answer Questions}

% 7. **Sample Answer:** The section on “Bee Population Decline” explains the critical role bees play in pollination and food production, connecting their survival to human survival. By highlighting threats like habitat destruction and pesticide use, it reinforces the urgency of protecting bees to maintain ecosystems and food security.

% 8. **Sample Answer:** The section on “Plastic Pollution Effects” describes the harmful impacts of plastic on marine life and ecosystems. By providing specific examples, like sea turtles mistaking plastic for food and microplastics entering the food chain, it establishes the importance of addressing plastic pollution as a global crisis.

% \subsection*{Part 4: Fill in the Blank Questions}

% 9. The \underline{introduction} of a text provides important background information to set the stage for the main idea.

% 10. A \underline{conclusion} often summarizes the main idea and encourages readers to take action.

\end{document}
