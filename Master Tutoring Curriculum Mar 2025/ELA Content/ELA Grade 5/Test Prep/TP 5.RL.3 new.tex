\documentclass[12pt]{article}

\usepackage[a4paper, top=0.8in, bottom=0.7in, left=0.7in, right=0.7in]{geometry}
\usepackage{amsmath}
\usepackage{graphicx}
\usepackage{fancyhdr}
\usepackage{tcolorbox}
\usepackage{multicol}
\usepackage{pifont} % For checkboxes
\usepackage[defaultfam,tabular,lining]{montserrat} %% Option 'defaultfam'
\usepackage[T1]{fontenc}
\renewcommand*\oldstylenums[1]{{\fontfamily{Montserrat-TOsF}\selectfont #1}}
\renewcommand{\familydefault}{\sfdefault}
\usepackage{enumitem}
\usepackage{setspace}
\usepackage{parcolumns}
\usepackage{tabularx}

\setlength{\parindent}{0pt}
\hyphenpenalty=10000
\exhyphenpenalty=10000

\pagestyle{fancy}
\fancyhf{}
%\fancyhead[L]{\textbf{5.RL.3: Characters, Settings, and Events Assessment}}
\fancyhead[R]{\includegraphics[width=1cm]{Round Logo.png}}
\fancyfoot[C]{\footnotesize Study Smart Tutors}

\begin{document}

\subsection*{Characters, Settings, and Events Assessment}
\onehalfspacing

\begin{tcolorbox}[colframe=black!40, colback=gray!0, title=Learning Objective]
\textbf{Objective:} Analyze how characters, settings, or events interact and influence the progression of a story.
\end{tcolorbox}

\subsection*{Part 1: Multiple-Choice Questions}

1. How did the setting influence the story?\\
"Ellie lived in a small, isolated town where the winters were long and harsh. Every winter, the town would gather for a festival to celebrate surviving the coldest months. This year, Ellie volunteered to help organize the event. She spent weeks preparing decorations and planning activities. When the day arrived, a sudden snowstorm threatened to cancel the festival. Ellie and her neighbors worked quickly to set up in the town hall instead. The festival brought everyone together, and Ellie realized how important community was in overcoming challenges posed by their environment."\\
\begin{enumerate}[label=\Alph*.]
    \item The setting encouraged the town to move to a warmer location.  
    \item The harsh winter setting brought the community closer.  
    \item The setting was irrelevant to the story’s events.  
    \item Ellie ignored the impact of the setting on her town.  
\end{enumerate}

\vspace{1cm}

2. How did the character’s actions affect the story’s events?\\
"Jamal noticed that his neighborhood park was covered in litter, making it unsafe for children to play. He decided to take action. Jamal organized a community cleanup day and invited his neighbors to help. Many people joined in, and together they cleared the park of trash. Afterward, the children played safely, and the neighbors decided to hold regular cleanup events. Jamal’s leadership inspired others to take care of their environment and brought the community closer."\\
\begin{enumerate}[label=\Alph*.]
    \item Jamal ignored the litter in the park.  
    \item Jamal’s leadership helped improve the park and unite the community.  
    \item The neighbors refused to participate in the cleanup.  
    \item The park remained unsafe for children.  
\end{enumerate}

\vspace{1cm}

3. What lesson did the character learn by the end of the story?\\
"Maya was excited to start a garden in her backyard, but she didn’t realize how much work it would require. She planted seeds but forgot to water them regularly. Her plants struggled to grow, and Maya felt discouraged. After reading a book about gardening, she learned the importance of caring for plants daily. Maya started \\watering her garden and pulling weeds each morning. Over time, her garden \\flourished, and she felt proud of her efforts. Maya discovered that persistence and learning from mistakes were key to achieving her goals."\\
\begin{enumerate}[label=\Alph*.]
    \item Gardening is easy and requires no effort.  
    \item Hard work and consistency lead to success.  
    \item It’s better to give up when faced with challenges.  
    \item Reading books is more important than gardening.  
\end{enumerate}

\vspace{1cm}

\subsection*{Part 2: Select All That Apply Questions}

4. Which details show how Ellie adapted to the snowstorm?\\
\begin{enumerate}[label=\Alph*.]
    \item Ellie canceled the festival entirely.  
    \item Ellie worked with neighbors to move the festival indoors.  
    \item Ellie ignored the snowstorm and continued as planned.  
    \item Ellie realized the importance of community.  
\end{enumerate}

\vspace{1cm}

5. Select \textbf{all} ways Jamal’s leadership influenced the neighborhood:\\
\begin{enumerate}[label=\Alph*.]
    \item The park became safe for children to play.  
    \item The neighbors started regular cleanup events.  
    \item The neighborhood ignored the park’s condition.  
    \item The community became more united.  
\end{enumerate}

\vspace{1cm}
\newpage
6. What actions helped Maya succeed in her gardening efforts?\\
\begin{enumerate}[label=\Alph*.]
    \item Maya started watering her garden daily.  
    \item Maya ignored her garden and let it grow on its own.  
    \item Maya read a book to learn gardening tips.  
    \item Maya gave up when her plants didn’t grow.  
\end{enumerate}



\subsection*{Part 3: Short Answer Questions}

7. \textbf{How did the character’s determination change the outcome of the story?}\\
"Leila had always been afraid of water. As a child, she avoided pools and beaches, preferring to stay dry and safe on land. But when her younger brother Sam wanted to learn how to swim, he begged Leila to join him. Hesitant but wanting to support her brother, Leila signed up for swimming lessons. At first, she struggled, panicking every time she stepped into the pool. However, with the encouragement of her instructor and Sam’s cheers, Leila started to make progress. She learned to float, kick, and eventually swim laps. By the end of the summer, not only was Leila \\swimming confidently, but she had also inspired Sam to overcome his own fears in the water. The siblings celebrated their success with a beach trip, where Leila swam in the ocean for the first time. Leila realized that facing her fears and persisting through challenges had made her stronger and closer to her brother."\\
\vspace{3.5cm}

8. Summarize how Leila’s actions influenced her brother and their relationship.\\
\vspace{4cm}

\subsection*{Part 4: Fill in the Blank Questions}
\vspace{1em}
9. A story’s \underline{\hspace{4cm}} is the moment where something important happens; this may have a major effect on the plot or characters.

\vspace{3cm}

10. Comparing shows the  \underline{\hspace{4cm}} between two things, while contrasting shows the \underline{\hspace{4cm}} between two things.

\vspace{3cm}
% \newpage
% \section*{Answer Key}

% \subsection*{Part 1: Multiple-Choice Questions}

% 1. **B.** The harsh winter setting brought the community closer.  

% 2. **B.** Jamal’s leadership helped improve the park and unite the community.  

% 3. **B.** Hard work and consistency lead to success.  

% \subsection*{Part 2: Select All That Apply Questions}

% 4. **B, D.**  
%    - Ellie worked with neighbors to move the festival indoors.  
%    - Ellie realized the importance of community.  

% 5. **A, B, D.**  
%    - The park became safe for children to play.  
%    - The neighbors started regular cleanup events.  
%    - The community became more united.  

% 6. **A, C.**  
%    - Maya started watering her garden daily.  
%    - Maya read a book to learn gardening tips.  

% \subsection*{Part 3: Short Answer Questions}

% 7. **Sample Answer:** Leila’s determination helped her overcome her fear of water. By facing her fears and persisting through swimming lessons, she became confident in the water. Her success inspired her brother Sam to overcome his fears, strengthening their bond and allowing them to celebrate their achievements together.  

% 8. **Sample Answer:** Leila’s actions inspired her brother to face his own fears of water. By showing persistence and determination, she became a role model for Sam. Their shared success strengthened their relationship and allowed them to celebrate new milestones together.  

% \subsection*{Part 4: Fill in the Blank Questions}

% 9. A story’s \underline{turning point} is the moment where something important happens; this may have a major effect on the plot or characters.

% 10. Comparing shows the \underline{similarities} between two things, while contrasting shows the \underline{differences} between two things.

\end{document}
