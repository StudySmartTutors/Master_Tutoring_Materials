\documentclass[12pt]{article}

\usepackage[a4paper, top=0.8in, bottom=0.7in, left=0.7in, right=0.7in]{geometry}

\usepackage{amsmath}

\usepackage{graphicx}

\usepackage{fancyhdr}

\usepackage{tcolorbox}

\usepackage{multicol}

\usepackage{pifont} % For checkboxes

%\usepackage{tgadventor}

\usepackage[defaultfam,tabular,lining]{montserrat} %% Option 'defaultfam'

\usepackage[T1]{fontenc}

\renewcommand*\oldstylenums[1]{{\fontfamily{Montserrat-TOsF}\selectfont #1}}

\renewcommand{\familydefault}{\sfdefault}

\usepackage{enumitem}

\usepackage{setspace}

\usepackage{parcolumns}

\usepackage{tabularx}

\setlength{\parindent}{0pt}

\hyphenpenalty=10000
\exhyphenpenalty=10000

\pagestyle{fancy}

\fancyhf{}
%\fancyhead[L]{\textbf{6.RL.4: Determining Word Meaning and Figurative Language}}
\fancyhead[R]{\includegraphics[width=1cm]{Round Logo.png}}
\fancyfoot[C]{\footnotesize Study Smart Tutors}

\begin{document}

\onehalfspacing

% Passage
\subsection*{The Mysterious Storm}

\begin{tcolorbox}[colframe=black!40, colback=gray!5]

\begin{spacing}{1.15}

The storm arrived like a beast, fierce and wild, roaring through the sky as the wind howled. Lightning cracked open the sky, and thunder rumbled like a giant’s footsteps. Despite the danger, Sarah stood on the porch, her heart racing with excitement. She had always loved storms, the way they seemed to come alive, unpredictable and full of energy.

As the rain began to pour, Sarah’s older brother, Jake, urged her to come inside. “It’s not safe out here,” he warned, his voice tinged with concern. But Sarah didn’t mind. She felt a thrill as she watched the storm’s power unfold before her eyes. 

The trees bent with the wind, their branches stretching and swaying as if dancing to the storm’s music. Sarah could almost hear the trees singing as the rain drenched them. In the distance, she saw a flash of light. It was the storm’s final roar, signaling its departure.

When the storm passed, the air felt fresh, and the world seemed to hold its breath. Sarah smiled, feeling the excitement of the storm still buzzing in her veins.

\end{spacing}

\end{tcolorbox}

% Worksheet Questions
\subsection*{Questions}

\begin{enumerate}

% 1st Question
\item In the passage, how is the storm described in the phrase “the storm arrived like a beast”?

\begin{enumerate}[label=\Alph*.]
    \item As calm and gentle
    \item As dangerous and fierce
    \item As predictable and calm
    \item As small and unimportant
\end{enumerate}

\vspace{0.5cm}

% 2nd Question
\item What does the phrase “roaring through the sky” suggest about the storm?

\begin{enumerate}[label=\Alph*.]
    \item The storm is quiet and gentle.
    \item The storm is moving quickly and powerfully.
    \item The storm is moving slowly and quietly.
    \item The storm is calm and still.
\end{enumerate}

\vspace{0.5cm}

% 3rd Question
\item How does Sarah feel about the storm, based on the text?

\begin{enumerate}[label=\Alph*.]
    \item She is scared of the storm.
    \item She feels excitement and thrill.
    \item She is indifferent to the storm.
    \item She feels worried and anxious.
\end{enumerate}

\vspace{0.5cm}

% 4th Question
\item What does the phrase “thunder rumbled like a giant’s footsteps” mean?

\begin{enumerate}[label=\Alph*.]
    \item The thunder was silent.
    \item The thunder sounded like a loud drum.
    \item The thunder sounded heavy and powerful, like a giant walking.
    \item The thunder sounded like a soft whisper.
\end{enumerate}

\vspace{0.5cm}

% 5th Question
\item What does Sarah’s reaction to the storm suggest about her character?

\begin{enumerate}[label=\Alph*.]
    \item She is fearful and cautious.
    \item She is bored and uninterested.
    \item She enjoys the excitement and energy of the storm.
    \item She dislikes storms and prefers to stay indoors.
\end{enumerate}

\vspace{0.5cm}

% 6th Question
\item What does the word "drenched" mean in the sentence “the rain drenched them”?

\begin{enumerate}[label=\Alph*.]
    \item Covered in a light mist
    \item Soaked or thoroughly wet
    \item Warm and cozy
    \item Left untouched
\end{enumerate}

\vspace{0.5cm}

% 7th Question
\item How does the author describe the wind in the storm?

\begin{enumerate}[label=\Alph*.]
    \item Gentle and calm
    \item Fierce and wild
    \item Unchanging
    \item Silent and still
\end{enumerate}

\vspace{0.5cm}

% 8th Question
\item What is the meaning of the phrase “the trees bent with the wind” in the passage?

\begin{enumerate}[label=\Alph*.]
    \item The trees broke in half.
    \item The trees remained still in the wind.
    \item The trees moved or leaned in the direction of the wind.
    \item The trees caught fire in the storm.
\end{enumerate}

\vspace{0.5cm}

% 9th Question
\item What does Sarah mean by “the storm’s final roar”?

\begin{enumerate}[label=\Alph*.]
    \item The storm is over.
    \item The storm is starting.
    \item The storm is at its strongest point.
    \item The storm is calm and quiet.
\end{enumerate}

\vspace{0.5cm}

% 10th Question
\item The phrase “the world seemed to hold its breath” suggests that:

\begin{enumerate}[label=\Alph*.]
    \item The world is in motion.
    \item The world is waiting for something important to happen.
    \item The world is completely silent.
    \item The world is getting ready for the storm to return.
\end{enumerate}

\vspace{0.5cm}

% 11th Question
\item What is the meaning of the word “buzzing” in “feeling the excitement of the storm still buzzing in her veins”?

\begin{enumerate}[label=\Alph*.]
    \item Feeling a sense of calmness
    \item Feeling the continued excitement or energy
    \item Feeling sick or unwell
    \item Feeling tired and exhausted
\end{enumerate}

\vspace{0.5cm}

% 12th Question
\item In the passage, what is the effect of the storm’s imagery on the reader?

\begin{enumerate}[label=\Alph*.]
    \item It makes the storm seem peaceful and calm.
    \item It helps the reader understand how Sarah feels about the storm.
    \item It describes the storm as boring and uneventful.
    \item It makes the storm seem far away and unimportant.
\end{enumerate}

\vspace{0.5cm}

% 13th Question
\item Which figurative language technique is used in “the storm arrived like a beast”?

\begin{enumerate}[label=\Alph*.]
    \item Simile
    \item Metaphor
    \item Personification
    \item Onomatopoeia
\end{enumerate}

\vspace{0.5cm}

% 14th Question
\item What is the purpose of the description “the trees...dancing to the storm’s music”?

\begin{enumerate}[label=\Alph*.]
    \item To show that the trees are alive and full of energy
    \item To describe how the trees are destroyed by the storm
    \item To explain that the trees make music during the storm
    \item To suggest that the trees are afraid of the storm
\end{enumerate}

\vspace{0.5cm}

% 15th Question
\item What can be inferred about Sarah’s relationship with nature based on her actions in the story?

\begin{enumerate}[label=\Alph*.]
    \item She fears nature and avoids it.
    \item She loves and appreciates the beauty of nature.
    \item She is indifferent to nature.
    \item She tries to control nature.
\end{enumerate}

\vspace{0.5cm}

\end{enumerate}

\end{document}
