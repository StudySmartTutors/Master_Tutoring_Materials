\documentclass[12pt]{article}

\usepackage[a4paper, top=0.8in, bottom=0.7in, left=0.7in, right=0.7in]{geometry}
\usepackage{amsmath}
\usepackage{graphicx}
\usepackage{fancyhdr}
\usepackage{tcolorbox}
\usepackage[defaultfam,tabular,lining]{montserrat} %% Option 'defaultfam'
\usepackage[T1]{fontenc}
\renewcommand*\oldstylenums[1]{{\fontfamily{Montserrat-TOsF}\selectfont #1}}
\renewcommand{\familydefault}{\sfdefault}
\usepackage{enumitem}
\usepackage{setspace}

\setlength{\parindent}{0pt}
\hyphenpenalty=10000
\exhyphenpenalty=10000

\pagestyle{fancy}
\fancyhf{}
\fancyhead[L]{\textbf{7.RL.1: Textual Evidence and Inference Practice}}
\fancyhead[R]{\includegraphics[width=1cm]{Round Logo.png}}
\fancyfoot[C]{\footnotesize Study Smart Tutors}

\begin{document}

\subsection*{Citing Evidence to Support Analysis and Inferences}
\onehalfspacing

\begin{tcolorbox}[colframe=black!40, colback=gray!0, title=Learning Objective]
\textbf{Objective:} Cite several pieces of textual evidence to support analysis of what the text says explicitly and inferences drawn from it.
\end{tcolorbox}

\subsection*{Part 1: Multiple-Choice Questions}

1. \textbf{What inference can be made about Ethan's feelings during the family dinner?}\\
"Ethan sat at the edge of the table, poking at his food with his fork. His plate was still full, even though everyone else was nearly done eating. His older brother laughed loudly at their father’s jokes, but Ethan barely cracked a smile. He kept his eyes fixed on the table, avoiding eye contact with anyone. Every time someone asked him a question, his answers were brief—'I’m fine,' or 'Sure, whatever.' When his mother asked if everything was okay, Ethan shrugged and muttered, 'Yeah, I’m fine,' though his voice lacked conviction. After a few more moments of sitting silently, Ethan stood up abruptly, muttering, 'I’m not hungry,' and left the table. He walked straight to his room and shut the door behind him. There, he sat on his bed, staring at the floor, feeling a mix of frustration and sadness he couldn’t explain. His family continued chatting at the table, their voices muffled through the walls."  
\begin{enumerate}[label=\Alph*.]
    \item Ethan was excited about the family dinner.  
    \item Ethan felt uncomfortable and withdrawn.  
    \item Ethan wanted to share exciting news.  
    \item Ethan was annoyed by his mother’s question.  
\end{enumerate}

\vspace{1cm}
\newpage
2. \textbf{What evidence supports the inference that Ava was determined to win the science fair?\\}
"Ava had always loved science, but the upcoming school science fair ignited a new level of determination in her. For weeks, she stayed up late, poring over research articles, conducting experiments, and meticulously documenting her findings. Her bedroom turned into a mini-laboratory, with beakers, notebooks, and wires scattered across her desk. Even when her friends invited her out, she politely declined, saying, 'I’ll celebrate after the fair.' Ava rehearsed her presentation daily, standing in front of the mirror to perfect her speech and hand gestures. The night before the fair, she double-checked her calculations and rewrote her conclusion for clarity. On the day of the event, Ava arrived early, carefully setting up her display board and ensuring every detail was perfect. When the judges approached, she smiled confidently and delivered her presentation with passion. After hearing her name announced as the winner, she beamed and whispered to herself, 'All the hard work was worth it.' Her project wasn’t just about winning—it was a testament to her dedication and love for science."  
\begin{enumerate}[label=\Alph*.]
    \item Ava practiced her speech in front of the mirror.  
    \item Ava stayed up late for weeks.  
    \item Ava celebrated with her friends before the event.  
    \item Ava arrived early to set up her display.  
\end{enumerate}

\vspace{1cm}
\newpage
3. \textbf{What explicit evidence shows that the townspeople admired Mr. Harper?\\}
"Mr. Harper was a fixture in the small town, known for his kindness and willingness to lend a hand. Whether someone needed help repairing a fence or organizing a charity event, Mr. Harper was always the first to volunteer. He greeted everyone he met with a warm smile and a wave, and his genuine interest in people’s lives made him beloved by all. At town meetings, his neighbors often said, 'We’re lucky to have someone like Mr. Harper in our community.' When he announced his retirement from his role as a volunteer coordinator, the entire town came together to plan a surprise celebration. On the day of the event, the community hall was filled with people of all ages, eager to thank him. They presented him with a plaque that read, 'In gratitude for your years of kindness and service.' Tears welled up in Mr. Harper’s eyes as he stood at the podium, overwhelmed by the love and support he received. The event was a testament to the admiration and respect the townspeople had for him."  
\begin{enumerate}[label=\Alph*.]
    \item Mr. Harper smiled and waved at everyone he passed.  
    \item The townspeople gave Mr. Harper a plaque to thank him.  
    \item Mr. Harper refused payment for helping neighbors.  
    \item Mr. Harper attended every town meeting.  
\end{enumerate}


\subsection*{Part 2: Select All That Apply Questions}

4. Select \textbf{all} details in the passage from question 1 that show Ethan was uncomfortable during the dinner:  
\begin{enumerate}[label=\Alph*.]
    \item Ethan barely touched his food.  
    \item Ethan’s responses were curt and forced.  
    \item Ethan laughed loudly with his brother.  
    \item Ethan quickly excused himself to go to his room.  
\end{enumerate}

\vspace{1cm}

5. Which details in the passage from question 2 support Ava’s determination to win the science fair?  
\begin{enumerate}[label=\Alph*.]
    \item Ava double-checked her calculations.  
    \item Ava revised her presentation slides.  
    \item Ava rehearsed her speech in front of the mirror.  
    \item Ava spent the night celebrating with her friends before the event.  
\end{enumerate}

\vspace{1cm}

6. Select \textbf{all} details in the passage from question 3 that show the townspeople admired Mr. Harper:  
\begin{enumerate}[label=\Alph*.]
    \item The townspeople gave Mr. Harper a plaque.  
    \item Mr. Harper often helped neighbors with repairs for free.  
    \item The townspeople ignored Mr. Harper at the meeting.  
    \item People said, 'We need more people like Mr. Harper.'  
\end{enumerate}

\vspace{1cm}

\subsection*{Part 3: Short Answer Questions}

7. What does Ethan’s behavior during the dinner suggest about his emotional state? Use evidence from the passage from question 1 to support your answer.  
\vspace{4cm}

8. Based on the story about Ava, explain how her actions reflect her determination. Use specific evidence from the passage from question 2.  
\vspace{4cm}

\subsection*{Part 4: Fill in the Blank Questions}
\vspace{1cm}
9. Textual evidence is used to support both \underline{\hspace{4cm}} statements and \underline{\hspace{4cm}} drawn from the text.  
\vspace{2cm}

10. When using evidence, it's important to include an in-line \underline{\hspace{4cm}} . 
\vspace{2cm}
\newpage
\section*{Answer Key}

\subsection*{Part 1: Multiple-Choice Questions}

1. \textbf{What inference can be made about Ethan's feelings during the family dinner?}  
\textbf{Answer:} B. Ethan felt uncomfortable and withdrawn.  
\textbf{Explanation:} The passage shows Ethan avoiding eye contact, giving brief responses, and retreating to his room, indicating his discomfort and withdrawal.

\vspace{1cm}
2. \textbf{What evidence supports the inference that Ava was determined to win the science fair?}  
\textbf{Answer:} B. Ava stayed up late for weeks.  
\textbf{Explanation:} The passage details how Ava worked diligently for weeks, staying up late to prepare, which demonstrates her determination.

\vspace{1cm}
3. \textbf{What explicit evidence shows that the townspeople admired Mr. Harper?}  
\textbf{Answer:} B. The townspeople gave Mr. Harper a plaque to thank him.  
\textbf{Explanation:} The townspeople's gesture of giving Mr. Harper a plaque and holding a celebration reflects their admiration.

\subsection*{Part 2: Select All That Apply Questions}

4. \textbf{Select all details in the passage from question 1 that show Ethan was uncomfortable during the dinner:}  
\textbf{Answer:} A. Ethan barely touched his food. \\
B. Ethan’s responses were curt and forced. \\
D. Ethan quickly excused himself to go to his room.  
\textbf{Explanation:} These details reflect Ethan's discomfort during the dinner.

\vspace{1cm}
5. \textbf{Which details in the passage from question 2 support Ava’s determination to win the science fair?}  
\textbf{Answer:} A. Ava double-checked her calculations. \\
C. Ava rehearsed her speech in front of the mirror.  
\textbf{Explanation:} Both of these details show Ava's dedication and focus on getting everything perfect for the fair.

\vspace{1cm}
6. \textbf{Select all details in the passage from question 3 that show the townspeople admired Mr. Harper:}  
\textbf{Answer:} A. The townspeople gave Mr. Harper a plaque. \\
B. Mr. Harper often helped neighbors with repairs for free. \\
D. People said, 'We need more people like Mr. Harper.'  
\textbf{Explanation:} These details highlight the townspeople's admiration for Mr. Harper’s kindness and helpfulness.

\subsection*{Part 3: Short Answer Questions}

7. \textbf{What does Ethan’s behavior during the dinner suggest about his emotional state? Use evidence from the passage from question 1 to support your answer.}  
\textbf{Answer:} Ethan’s behavior suggests that he is feeling frustrated and sad. Evidence includes his avoidance of eye contact, lack of interest in the meal, and abrupt exit from the dinner table, all of which suggest he is emotionally withdrawn and upset.

\vspace{1cm}
8. \textbf{Based on the story about Ava, explain how her actions reflect her determination. Use specific evidence from the passage from question 2.}  
\textbf{Answer:} Ava’s actions demonstrate her determination through her tireless work ethic and commitment to preparing for the science fair. She stayed up late conducting research, rehearsed her presentation daily, and even declined social invitations to focus on her goal, showcasing her focus and dedication.

\subsection*{Part 4: Fill in the Blank Questions}

9. Textual evidence is used to support both \underline{explicit} statements and \underline{inferences} drawn from the text.  
\textbf{Answer:} explicit, inferences.

10. When using evidence, it's important to include an in-line \underline{citation}.  
\textbf{Answer:} citation.
\end{document}
