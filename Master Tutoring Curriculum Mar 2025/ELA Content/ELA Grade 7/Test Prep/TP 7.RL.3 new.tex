\documentclass[12pt]{article}

\usepackage[a4paper, top=0.8in, bottom=0.7in, left=0.7in, right=0.7in]{geometry}
\usepackage{amsmath}
\usepackage{graphicx}
\usepackage{fancyhdr}
\usepackage{tcolorbox}
\usepackage[defaultfam,tabular,lining]{montserrat} %% Option 'defaultfam'
\usepackage[T1]{fontenc}
\renewcommand*\oldstylenums[1]{{\fontfamily{Montserrat-TOsF}\selectfont #1}}
\renewcommand{\familydefault}{\sfdefault}
\usepackage{enumitem}
\usepackage{setspace}

\setlength{\parindent}{0pt}
\hyphenpenalty=10000
\exhyphenpenalty=10000

\pagestyle{fancy}
\fancyhf{}
\fancyhead[L]{\textbf{7.RL.3: Interaction of Story Elements Practice}}
\fancyhead[R]{\includegraphics[width=1cm]{Round Logo.png}}
\fancyfoot[C]{\footnotesize Study Smart Tutors}

\begin{document}

\subsection*{Analyzing Story Elements and Their Interactions}
\onehalfspacing

\begin{tcolorbox}[colframe=black!40, colback=gray!0, title=Learning Objective]
\textbf{Objective:} Analyze how particular elements of a story or drama (e.g., characters, setting, or plot) interact and contribute to the story's development.
\end{tcolorbox}

\subsection*{Part 1: Multiple-Choice Questions}

1. \textbf{How does the setting influence the character’s actions in the story?\\}
"Lila lived in a small, quiet town surrounded by forests. She often found inspiration in the tranquility of the woods, spending afternoons sketching animals and plants. One summer morning, a loud siren jolted her awake—a warning of a forest fire \\approaching. Thick smoke filled the sky as residents quickly evacuated. Lila packed essentials, but she hesitated when her sketchbook caught her eye. She grabbed it, determined to save her work. As she fled to the town’s evacuation center, she saw frightened animals darting through the smoky woods. The sight both saddened and inspired her. That evening, safe in the center, she began sketching an image of hope: a forest reborn, filled with life again. Her artwork later became the centerpiece of a campaign to restore the forest. Through this experience, Lila realized the power of art in healing and rebuilding, and she found purpose in turning personal loss into action."  
\begin{enumerate}[label=\Alph*.]
    \item The forest inspired Lila to create artwork.  
    \item The fire forced Lila to leave her home.  
    \item The quiet town made Lila bored and restless.  
    \item The setting had no influence on Lila’s actions.  
\end{enumerate}

\vspace{1cm}
\newpage
2. \textbf{How do the characters' interactions shape the story’s events?\\}
"Two brothers, Aaron and Daniel, had always competed with each other in sports, grades, and now, the school’s science competition. Aaron focused on robotics, building a device to assist people with disabilities, while Daniel designed a solar-powered water purifier to help drought-stricken areas. Both poured hours into their projects, determined to win. However, in the final week, Aaron’s robot stopped functioning, and Daniel’s purifier failed to filter water properly. Frustrated, Aaron overheard Daniel muttering about his struggles. Though hesitant at first, Aaron offered to help. Together, they solved Daniel’s purifier issue and improved its design. At the competition, Daniel won first place. During his speech, he thanked Aaron for his support, surprising the audience with his honesty. The experience brought the brothers closer, teaching them that collaboration often leads to greater success than rivalry. From then on, they worked as a team, combining their skills to tackle even bigger challenges."  
\begin{enumerate}[label=\Alph*.]
    \item The brothers' rivalry helped them improve their projects.  
    \item Their collaboration led to Daniel’s success and a stronger bond.  
    \item Aaron refused to help Daniel with his project.  
    \item The characters’ interactions had no effect on the outcome.  
\end{enumerate}

\vspace{1cm}

3. \textbf{How does the plot unfold in the following story?\\}
"Maya had always dreamed of becoming a musician, but growing up in a small town with limited resources made it difficult. She practiced on a worn-out violin her grandmother gave her, often imagining herself on grand stages. When a famous violinist announced a concert nearby, Maya convinced her parents to take her. The concert was mesmerizing, and she noticed a sign for auditions for the violinist’s youth orchestra. Nervous but hopeful, Maya signed up. On audition day, her hands trembled as she played. Halfway through, she hit a wrong note and froze, embarrassed. However, the violinist encouraged her to finish, seeing potential in her. Despite her mistakes, she was offered a spot in the orchestra. Overjoyed, Maya worked tirelessly to improve. Months later, she performed a solo at a major concert. The journey taught her that failure wasn’t the end—it was just part of the path to success."  
\begin{enumerate}[label=\Alph*.]
    \item Maya’s determination helped her achieve her dreams.  
    \item Maya was too nervous to perform well at her audition.  
    \item Maya’s town provided many musical opportunities.  
    \item Maya gave up after her poor audition.  
\end{enumerate}

\vspace{1cm}

\subsection*{Part 2: Select All That Apply Questions}

4. Select \textbf{all} details that show how the setting influenced Lila’s actions in the story from question 1:  
\begin{enumerate}[label=\Alph*.]
    \item The forest fire forced Lila to evacuate.  
    \item The sight of fleeing animals inspired Lila to draw.  
    \item Lila’s artwork became part of a campaign to restore the forest.  
    \item Lila ignored the fire and stayed in her home.  
\end{enumerate}

\vspace{1cm}

5. Which details highlight how the brothers’ interactions shaped the story from question 2?  
\begin{enumerate}[label=\Alph*.]
    \item Aaron helped Daniel fix his project.  
    \item Daniel refused to acknowledge Aaron’s help.  
    \item Their rivalry initially caused tension.  
    \item The competition strengthened their relationship.  
\end{enumerate}

\vspace{1cm}

6. Select \textbf{all} events that develop Maya’s story in question 3:  
\begin{enumerate}[label=\Alph*.]
    \item Maya attended the concert of a famous violinist.  
    \item Maya signed up for the youth orchestra auditions.  
    \item Maya gave up after stumbling through her audition.  
    \item Maya earned a solo performance through hard work.  
\end{enumerate}

\vspace{1cm}
\newpage
\subsection*{Part 3: Short Answer Questions}

7. How does the forest fire challenge Lila, and how does she respond? Use specific evidence from the passage from question 1 to explain.  
\vspace{4cm}

8. Explain how Aaron and Daniel’s rivalry turned into collaboration. Provide examples from the passage from question 2 to support your answer.  
\vspace{4cm}

\subsection*{Part 4: Fill in the Blank Questions}
\vspace{1cm}
9. The interactions between characters often lead to \underline{\hspace{4cm}} in a story, shaping the plot.  
\vspace{2cm}

10. The setting of a story can influence the \underline{\hspace{4cm}} of the story.  
\vspace{2cm}
\newpage
\section*{Answer Key}

\subsection*{Part 1: Multiple-Choice Questions}

1. \textbf{How does the setting influence the character’s actions in the story?}  
\textbf{Answer:} A. The forest inspired Lila to create artwork.  
\textbf{Explanation:} The peaceful setting of the forest inspired Lila’s artwork, especially after she witnessed the forest fire’s impact and the fleeing animals.

\vspace{1cm}
2. \textbf{How do the characters' interactions shape the story’s events?}  
\textbf{Answer:} B. Their collaboration led to Daniel’s success and a stronger bond.  
\textbf{Explanation:} Aaron and Daniel’s collaboration in fixing the solar-powered water purifier helped Daniel win the competition and strengthened their relationship.

\vspace{1cm}
3. \textbf{How does the plot unfold in the following story?}  
\textbf{Answer:} A. Maya’s determination helped her achieve her dreams.  
\textbf{Explanation:} Maya’s perseverance and determination, even after a difficult audition, ultimately helped her achieve her dream of joining the youth orchestra and performing a solo.

\subsection*{Part 2: Select All That Apply Questions}

4. \textbf{Select all details that show how the setting influenced Lila’s actions in the story from question 1:}  
\textbf{Answer:} A. The forest fire forced Lila to evacuate. \\
B. The sight of fleeing animals inspired Lila to draw. \\
C. Lila’s artwork became part of a campaign to restore the forest.  
\textbf{Explanation:} The setting of the forest, particularly the fire and the fleeing animals, greatly influenced Lila’s emotional response and led to her artwork being part of the restoration campaign.

\vspace{1cm}
5. \textbf{Which details highlight how the brothers’ interactions shaped the story from question 2?}  
\textbf{Answer:} A. Aaron helped Daniel fix his project. \\
C. Their rivalry initially caused tension. \\
D. The competition strengthened their relationship.  
\textbf{Explanation:} Aaron’s assistance and the tension caused by their rivalry eventually led to a stronger bond as they worked together.

\vspace{1cm}
6. \textbf{Select all events that develop Maya’s story in question 3:}  
\textbf{Answer:} A. Maya attended the concert of a famous violinist. \\
B. Maya signed up for the youth orchestra auditions. \\
D. Maya earned a solo performance through hard work.  
\textbf{Explanation:} These events help drive Maya’s growth as a musician, culminating in her hard-earned solo performance.

\subsection*{Part 3: Short Answer Questions}

7. \textbf{How does the forest fire challenge Lila, and how does she respond? Use specific evidence from the passage from question 1 to explain.}  
\textbf{Answer:} The forest fire challenges Lila by forcing her to evacuate her home, but she responds by grabbing her sketchbook and using the tragedy as inspiration. She creates an artwork that reflects hope and later contributes to the restoration of the forest.

\vspace{1cm}
8. \textbf{Explain how Aaron and Daniel’s rivalry turned into collaboration. Provide examples from the passage from question 2 to support your answer.}  
\textbf{Answer:} Initially, Aaron and Daniel’s rivalry caused tension as they competed for the science prize. However, when Aaron overheard Daniel’s struggles, he decided to help. Together, they solved Daniel’s purifier issue, and their collaboration led to Daniel winning. The experience taught them that teamwork often leads to greater success than rivalry.

\subsection*{Part 4: Fill in the Blank Questions}

9. The interactions between characters often lead to \underline{conflict} in a story, shaping the plot.  
\textbf{Answer:} conflict.

10. The setting of a story can influence the \underline{mood} of the story.  
\textbf{Answer:} mood.
\end{document}
