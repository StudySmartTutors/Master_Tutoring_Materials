\documentclass[12pt]{article}

\usepackage[a4paper, top=0.8in, bottom=0.7in, left=0.7in, right=0.7in]{geometry}
\usepackage{amsmath}
\usepackage{graphicx}
\usepackage{fancyhdr}
\usepackage{tcolorbox}
\usepackage[defaultfam,tabular,lining]{montserrat} %% Option 'defaultfam'
\usepackage[T1]{fontenc}
\renewcommand*\oldstylenums[1]{{\fontfamily{Montserrat-TOsF}\selectfont #1}}
\renewcommand{\familydefault}{\sfdefault}
\usepackage{enumitem}
\usepackage{setspace}


\setlength{\parindent}{0pt}
\setlength{\emergencystretch}{1.75em}
\hyphenpenalty=10000
\exhyphenpenalty=10000

\pagestyle{fancy}
\fancyhf{}
\fancyhead[L]{\textbf{7.RI.3: Interactions in Informational Text Practice}}
\fancyhead[R]{\includegraphics[width=1cm]{Round Logo.png}}
\fancyfoot[C]{\footnotesize Study Smart Tutors}

\begin{document}

\subsection*{Understanding Interactions Between Individuals, Events, and Ideas}
\onehalfspacing

\begin{tcolorbox}[colframe=black!40, colback=gray!0, title=Learning Objective]
\textbf{Objective:} Analyze the interactions between individuals, events, and ideas in a text to explain how they contribute to the development of the text.
\end{tcolorbox}

\subsection*{Part 1: Multiple-Choice Questions}

1. \textbf{How does farming sea urchins address environmental and economic \\challenges?}\\
"Sea urchin farming has emerged as a solution to declining wild populations caused by overharvesting and habitat destruction. Wild sea urchins, prized for their roe (uni), are a delicacy in high-end cuisine, but their numbers are dwindling in many coastal regions. Farming sea urchins allows for sustainable harvesting, reducing pressure on wild populations. In Japan and parts of Europe, aquaculture techniques have been developed to cultivate urchins in controlled environments. These methods ensure consistent quality and supply, which benefits both fishermen and consumers. However, some chefs argue that farmed uni lacks the rich flavor and texture of wild urchins due to differences in diet and habitat. Advocates of farming point out that the process is evolving, with researchers working to replicate natural conditions more accurately. Farming also creates economic opportunities for coastal communities while preserving marine ecosystems. By balancing taste, sustainability, and cost, sea urchin farming can meet growing demand without harming the environment."  
\begin{enumerate}[label=\Alph*.]
    \item Farming sea urchins reduces their overall population.  
    \item Farming sea urchins preserves wild populations and creates economic \\opportunities.  
    \item Wild sea urchins are less flavorful than farmed ones.  
    \item Sea urchin farming harms the environment more than wild harvesting.  
\end{enumerate}

\vspace{1cm}
\newpage
2. \textbf{What adaptations make carnivorous plants successful in nutrient-poor environments?\\}
"Carnivorous plants, such as Venus flytraps, pitcher plants, and sundews, have evolved to thrive in habitats with poor soil nutrients. These unique plants capture and digest insects to supplement their nutrient intake. For example, Venus flytraps use specialized leaves that snap shut when triggered by prey, trapping insects for digestion. Pitcher plants have deep, tubular leaves filled with digestive fluids that dissolve trapped insects. Sundews produce sticky, glandular hairs that ensnare prey. These adaptations allow carnivorous plants to obtain essential nitrogen and phosphorus, which are often scarce in their native bogs and wetlands. While their mechanisms for trapping prey are diverse, their reliance on external nutrients highlights the importance of ecological balance. Carnivorous plants are also sensitive to habitat loss and environmental changes, making conservation efforts crucial. These fascinating plants not only illustrate the diversity of life but also demonstrate how organisms adapt to challenging conditions to survive and thrive."  
\begin{enumerate}[label=\Alph*.]
    \item Carnivorous plants capture prey using sticky, snapping, or tubular traps.  
    \item Carnivorous plants require nutrient-rich soil to survive.  
    \item Their adaptations make them vulnerable to environmental changes.  
    \item They rely entirely on insects for their survival.  
\end{enumerate}

\newpage
3. \textbf{How do mushrooms contribute to the balance of ecosystems?\\}
"Mushrooms, a type of fungi, play a crucial role in maintaining healthy ecosystems. They act as decomposers, breaking down organic matter such as fallen leaves, dead wood, and animal remains. This process recycles nutrients back into the soil, enriching it for plants and supporting the entire food web. Some mushrooms, like mycorrhizal fungi, form symbiotic relationships with plants by attaching to their roots. This connection enhances nutrient absorption for plants while providing fungi with sugars produced through photosynthesis. Mushrooms also contribute to forest regeneration by improving soil health and fostering seed growth. Beyond their ecological functions, mushrooms are used in medicine, where compounds from fungi have been developed into antibiotics and anti-cancer treatments. However, fungi are sensitive to environmental changes, such as deforestation and pollution, which threaten their survival. By supporting decomposition, nutrient cycling, and plant growth, mushrooms are indispensable to the ecosystems they inhabit, showcasing their ecological and medicinal significance."  
\begin{enumerate}[label=\Alph*.]
    \item Mushrooms are harmful to plants by breaking down nutrients.  
    \item Mushrooms decompose organic matter and support plant growth. 
    \item Mushrooms play no role in forest regeneration.  
    \item Mushrooms contribute only to medicine, not ecosystems.  
\end{enumerate}



\subsection*{Part 2: Select All That Apply Questions}

4. Select \textbf{all} ways farming sea urchins benefits the environment and economy according to the passage from question 1:  
\begin{enumerate}[label=\Alph*.]
    \item Reduces pressure on wild sea urchin populations.  
    \item Ensures a consistent supply for consumers.  
    \item Creates economic opportunities for coastal communities.  
    \item Destroys natural marine habitats.  
\end{enumerate}

\vspace{1em}

5. Which details in the passage from question 2 highlight the adaptations of carnivorous plants?  
\begin{enumerate}[label=\Alph*.]
    \item Venus flytraps snap shut to trap insects.  
    \item Pitcher plants use tubular leaves filled with digestive fluids.  
    \item Sundews rely on sticky hairs to capture prey.  
    \item Carnivorous plants grow only in nutrient-rich soil.  
\end{enumerate}

\vspace{1cm}

6. Select \textbf{all} roles mushrooms play in ecosystems according to the passage from question 3:  
\begin{enumerate}[label=\Alph*.]
    \item Recycling nutrients by breaking down organic matter.  
    \item Forming symbiotic relationships with plants.  
    \item Enhancing soil health and supporting plant growth.  
    \item Contributing only to human diets, not ecosystems.  
\end{enumerate}



\subsection*{Part 3: Short Answer Questions}

7. Based on the passage, explain how sea urchin farming balances environmental and economic needs. Provide textual evidence from \\the passage from question 1.  
\vspace{4cm}

8. How do carnivorous plants adapt to their environments? Use examples from the passage from question 2 to support your response.  
\vspace{4cm}

\subsection*{Part 4: Fill in the Blank Questions}
\vspace{1em}
9. Interactions between individuals, events, and ideas help \underline{\hspace{4cm}} \\the development of a text’s key points.  
\vspace{2cm}

10. Interactions that matter in terms of the context of the text's message or \\purpose are \underline{\hspace{4cm}}.
\vspace{2cm}
\newpage
\subsection*{Answer Key}

\textbf{Part 1: Multiple-Choice Questions}

1. B. Farming sea urchins preserves wild populations and creates economic opportunities.  
2. A. Carnivorous plants capture prey using sticky, snapping, or tubular traps.  
3. B. Mushrooms decompose organic matter and support plant growth.  

\textbf{Part 2: Select All That Apply Questions}

4. A, B, C.  
5. A, B, C.  
6. A, B, C.  

\textbf{Part 3: Short Answer Questions}

7. Sea urchin farming helps balance environmental and economic needs by reducing pressure on wild sea urchin populations and providing a consistent supply of sea urchins to consumers. The passage mentions that "farming sea urchins allows for sustainable harvesting, reducing pressure on wild populations" and "creates economic opportunities for coastal communities."

8. Carnivorous plants adapt to nutrient-poor environments by developing specialized mechanisms to capture and digest insects. For example, Venus flytraps "snap shut when triggered by prey," pitcher plants have "tubular leaves filled with digestive fluids," and sundews "produce sticky, glandular hairs that ensnare prey."

\textbf{Part 4: Fill in the Blank Questions}

9. support  
10. relevant  

\end{document}

