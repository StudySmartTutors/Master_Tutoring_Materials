\documentclass[12pt]{article}
\usepackage[a4paper, top=0.8in, bottom=0.7in, left=0.8in, right=0.8in]{geometry}
\usepackage{amsmath}
\usepackage{amsfonts}
\usepackage{graphicx}
\usepackage{fancyhdr}
\usepackage{enumitem}
\usepackage{setspace}
\usepackage{tcolorbox}
\usepackage[defaultfam,tabular,lining]{montserrat}

\setlength{\parindent}{0pt}
\pagestyle{fancy}

\setlength{\headheight}{27.11148pt}
\addtolength{\topmargin}{-15.11148pt}

\fancyhf{}
\fancyhead[L]{\textbf{Grading Guide: Opinion Writing Assignment}}
\fancyfoot[C]{\footnotesize \copyright Study Smart Tutors}

\sloppy

\begin{document}

\section*{Grading Guide: Opinion Writing Assignment (Aligned to AZ.ELA.4.W.1)}

\subsection*{Purpose and Structure}
\begin{tcolorbox}[colframe=black!60, colback=white, title=Pass Criteria]
\begin{itemize}
    \item \textbf{Introduction}:
    \begin{itemize}
        \item Includes background information on the topic.
        \item Clearly states the writer's opinion.
    \end{itemize}
    \item \textbf{Body}:
    \begin{itemize}
        \item Includes 2--3 reasons supporting the opinion.
        \item Organizes reasons logically, each reason in a separate paragraph.
        \item Links opinion and reasons using transition phrases (e.g., \textit{for example}, \textit{in addition}).
    \end{itemize}
    \item \textbf{Conclusion}:
    \begin{itemize}
        \item Restates the opinion clearly.
        \item Summarizes the main supporting reasons.
    \end{itemize}
    \item The response stays focused on the assigned topic or prompt.
\end{itemize}
\end{tcolorbox}

\begin{tcolorbox}[colframe=black!60, colback=white, title=Fail Criteria]
\begin{itemize}
    \item No clear introduction, opinion, or conclusion.
    \item Reasons are missing, unclear, or unrelated to the opinion.
    \item Lack of logical organization or connection between paragraphs.
    \item Writing is off-topic.
\end{itemize}
\end{tcolorbox}

\subsection*{Evidence and Elaboration}
\begin{tcolorbox}[colframe=black!60, colback=white, title=Pass Criteria]
\begin{itemize}
    \item \textbf{Use of Evidence}:
    \begin{itemize}
        \item Includes at least 2 details or pieces of evidence from the sources.
        \item Uses examples, facts, or explanations to support each reason.
    \end{itemize}
    \item \textbf{Elaboration}:
    \begin{itemize}
        \item Explains how evidence supports the opinion.
        \item Uses linking words and phrases to connect evidence to reasons (e.g., \textit{because}, \textit{therefore}).
    \end{itemize}
    \item \textbf{Balanced Evidence}:
    \begin{itemize}
        \item Draws on all sources provided.
    \end{itemize}
\end{itemize}
\end{tcolorbox}

\begin{tcolorbox}[colframe=black!60, colback=white, title=Fail Criteria]
\begin{itemize}
    \item Fewer than two supporting details from the sources.
    \item No explanation of evidence or weak connections between evidence and reasons.
    \item Evidence is irrelevant, unclear, or missing.
    \item Does not use linking words.
\end{itemize}
\end{tcolorbox}

\subsection*{Language and Conventions}
\begin{tcolorbox}[colframe=black!60, colback=white, title=Pass Criteria]
\begin{itemize}
    \item Sentences are complete, varied, and clear.
    \item Few or no errors in spelling, punctuation, and grammar.
    \item Uses grade-appropriate vocabulary effectively.
\end{itemize}
\end{tcolorbox}

\begin{tcolorbox}[colframe=black!60, colback=white, title=Fail Criteria]
\begin{itemize}
    \item Frequent grammar, punctuation, or spelling errors that make the writing difficult to understand.
    \item Incomplete or unclear sentences.
    \item Vocabulary is below grade level or used incorrectly.
\end{itemize}
\end{tcolorbox}

\subsection*{Final Determination}
\begin{tcolorbox}[colframe=black!60, colback=white, title=Pass]
\begin{itemize}
    \item Meets the criteria for Purpose and Structure, Evidence and Elaboration, and Language and Conventions.
\end{itemize}
\end{tcolorbox}

\begin{tcolorbox}[colframe=black!60, colback=white, title=Fail]
\begin{itemize}
    \item Does not meet the minimum criteria in one or more categories.
\end{itemize}
\end{tcolorbox}

\subsection*{Teacher Notes}
\begin{itemize}
    \item Provide feedback for borderline responses by identifying specific strengths and areas for improvement (e.g., \textit{Your introduction is clear, but you need to include more evidence from the sources.}).
    \item Encourage students to revise and resubmit if they do not meet the criteria.
\end{itemize}

\end{document}
