\documentclass[12pt]{article}
\usepackage[a4paper, top=0.8in, bottom=0.7in, left=0.8in, right=0.8in]{geometry}
\usepackage{amsmath}
\usepackage{amsfonts}
\usepackage{latexsym}
\usepackage{graphicx}
\usepackage{float}
\usepackage{fancyhdr}
\usepackage{enumitem}
\usepackage{setspace}
\usepackage{tcolorbox}
\usepackage[defaultfam,tabular,lining]{montserrat}

\setlength{\parindent}{0pt}
\pagestyle{fancy}

\setlength{\headheight}{27.11148pt}
\addtolength{\topmargin}{-15.11148pt}

\fancyhf{}
\fancyhead[L]{\textbf{Scoring Guide: Opinion Writing (Score out of 10)}}
\fancyfoot[C]{\footnotesize \copyright Study Smart Tutors}

\sloppy

\begin{document}

\section*{Scoring Guide: Opinion Writing Assignment (Score Out of 10)}

\subsection*{Overview}
This scoring guide outlines how to evaluate an opinion essay based on the following criteria:\ Purpose, Focus, and Organization; Evidence and Elaboration; and Language and Conventions. Each category is scored out of a total of \textbf{10 points}. The final score is the sum of all category scores.

\subsection*{Criteria and Scoring Rubric}

\begin{tcolorbox}[colframe=black!60, colback=white, title=Purpose, Focus, and Organization (0-4 Points)]
\textbf{Key Elements:}
\begin{itemize}
    \item Clearly states an opinion in the introduction. \hfill (1 Point)
    \item Includes a clear structure with an introduction, body (reasons and evidence), and conclusion. \hfill (1 Point)
    \item Maintains focus on the assigned topic or prompt throughout the essay. \hfill (1 Point)
    \item Effectively uses linking words and phrases to connect ideas. \hfill (1 Point)
\end{itemize}
\textbf{Score:} \underline{\hspace{2cm}} / 4 Points
\end{tcolorbox}

\begin{tcolorbox}[colframe=black!60, colback=white, title=Evidence and Elaboration (0-4 Points)]
\textbf{Key Elements:}
\begin{itemize}
    \item Provides at least two reasons that support the opinion. \hfill (1 Point)
    \item Uses evidence from at least one source to support the reasons. \hfill (1 Point)
    \item Clearly explains how the evidence connects to the opinion. \hfill (1 Point)
    \item Includes relevant details, examples, or elaboration that enhance the argument. \hfill (1 Point)
\end{itemize}
\textbf{Score:} \underline{\hspace{2cm}} / 4 Points
\end{tcolorbox}

\begin{tcolorbox}[colframe=black!60, colback=white, title=Language and Conventions (0-2 Points)]
\textbf{Key Elements:}
\begin{itemize}
    \item Demonstrates proper sentence structure, capitalization, punctuation, and spelling. \hfill (1 Point)
    \item Uses vocabulary appropriate for the grade level and writing purpose. \hfill (1 Point)
\end{itemize}
\textbf{Score:} \underline{\hspace{2cm}} / 2 Points
\end{tcolorbox}

\subsection*{Final Score}
\begin{tcolorbox}[colframe=black!60, colback=white, title=Calculating the Final Score]
Add the scores from each category to determine the final score out of 10 points.
\begin{itemize}
    \item \textbf{Purpose, Focus, and Organization:} \underline{\hspace{3cm}} / 4
    \item \textbf{Evidence and Elaboration:} \underline{\hspace{3cm}} / 4
    \item \textbf{Language and Conventions:} \underline{\hspace{3cm}} / 2
\end{itemize}
\textbf{Total Score:} \underline{\hspace{3cm}} / 10
\end{tcolorbox}

\subsection*{Teacher Notes}
\begin{itemize}
    \item Encourage students to review their writing to ensure it meets all rubric requirements.
    \item Provide specific feedback for areas where points were deducted to help students improve.
    \item Use the rubric to guide discussions about how to strengthen opinion writing.
\end{itemize}

\end{document}
