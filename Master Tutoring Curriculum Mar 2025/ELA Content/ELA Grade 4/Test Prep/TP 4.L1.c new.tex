\documentclass[12pt]{article}

\usepackage[a4paper, top=0.8in, bottom=0.7in, left=0.7in, right=0.7in]{geometry}
\usepackage{amsmath}
\usepackage{graphicx}
\usepackage{fancyhdr}
\usepackage{tcolorbox}
\usepackage{multicol}
\usepackage{pifont} % For checkboxes
\usepackage[defaultfam,tabular,lining]{montserrat} %% Option 'defaultfam'
\usepackage[T1]{fontenc}
\renewcommand*\oldstylenums[1]{{\fontfamily{Montserrat-TOsF}\selectfont #1}}
\renewcommand{\familydefault}{\sfdefault}
\usepackage{enumitem}
\usepackage{setspace}
\usepackage{parcolumns}
\usepackage{tabularx}

\setlength{\parindent}{0pt}
\hyphenpenalty=10000
\exhyphenpenalty=10000

\pagestyle{fancy}
\fancyhf{}
%\fancyhead[L]{\textbf{4.L.1c: Grammar and Usage Practice}}
\fancyhead[R]{\includegraphics[width=1cm]{Round Logo.png}}
\fancyfoot[C]{\footnotesize Study Smart Tutors}

\begin{document}

\subsection*{Understanding Sentence Structure and Grammar}
\onehalfspacing

\begin{tcolorbox}[colframe=black!40, colback=gray!0, title=Learning Objective]
\textbf{Objective:} Demonstrate command of standard English grammar, focusing on the correct use of modal auxiliaries to express conditions.
\end{tcolorbox}

\subsection*{Part 1: Multiple-Choice Questions}

1. Which sentence uses a modal auxiliary correctly?\\
\begin{enumerate}[label=\Alph*.]
    \item You must finish your homework before playing outside.  
    \item She can sings beautifully.  
    \item They might arrives late to the party.  
    \item We should to go to the library.  
\end{enumerate}

\vspace{1cm}

2. Which modal auxiliary best completes the sentence?\\
"If it rains tomorrow, we \underline{\hspace{2cm}} stay indoors."\\
\begin{enumerate}[label=\Alph*.]
    \item must  
    \item should  
    \item could  
    \item can  
\end{enumerate}

\vspace{1cm}

3. Which sentence expresses a condition correctly?\\
\begin{enumerate}[label=\Alph*.]
    \item If you will study, you can pass the test.  
    \item If you studied, you can pass the test.  
    \item If you study, you will pass the test.  
    \item If you studies, you will pass the test.  
\end{enumerate}

\vspace{1cm}
\newpage

4. Which modal auxiliary shows possibility?\\
\begin{enumerate}[label=\Alph*.]
    \item She must go to the store.  
    \item She might go to the store.  
    \item She can go to the store.  
    \item She should go to the store.  
\end{enumerate}

\vspace{1cm}


\subsection*{Part 2: Short Answer Questions}

5. Rewrite the following sentence using a modal auxiliary to show obligation: \\
"You have to complete your project by Friday."\\
\vspace{3cm}

6. Combine the sentences below using a modal auxiliary to show possibility: \\
"We are thinking about going to the park. It might rain later."\\
\vspace{3cm}

\subsection*{Part 3: Select All That Apply}

7. Select \textbf{all} sentences that use modal auxiliaries correctly:\\
\begin{enumerate}[label=\Alph*.]
    \item You should call your friend to apologize.  
    \item We can to go to the movies after dinner.  
    \item They might be late because of traffic.  
    \item He must finishes his homework before playing.  
\end{enumerate}

\vspace{1cm}

8. Which sentences show conditions correctly?\\
\begin{enumerate}[label=\Alph*.]
    \item If you study, you can do well on the test.  
    \item If it rains, we might cancel the picnic.  
    \item If she practices, she could improve her skills.  
    \item If he work harder, he will get a promotion.  
\end{enumerate}

\vspace{1cm}

\subsection*{Part 4: Fill in the Blank}
\vspace{1cm}
9. Modal auxiliaries, such as \underline{\hspace{4cm}} and \underline{\hspace{4cm}}, are used to express conditions or possibilities.

\vspace{3cm}

10. A sentence with a condition often begins with the word \underline{\hspace{4cm}}.

\vspace{3cm}
\newpage
% \subsection*{Answer Key}

% 1. \textbf{A} - You must finish your homework before playing outside.\\
% 2. \textbf{B} - should\\
% 3. \textbf{C} - If you study, you will pass the test.\\
% 4. \textbf{B} - She might go to the store.\\
% 5. \textbf{Correct answer:} "You must complete your project by Friday."\\
% 6. \textbf{Correct answer:} "We are thinking about going to the park, but it might rain later."\\
% 7. \textbf{Correct answers:} A, C\\
% 8. \textbf{Correct answers:} A, B, C\\
% 9. \textbf{must, should} \\
% 10. \textbf{If} \\

\end{document}

