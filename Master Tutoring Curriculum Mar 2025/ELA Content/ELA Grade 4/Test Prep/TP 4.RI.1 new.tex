\documentclass[12pt]{article}

\usepackage[a4paper, top=0.8in, bottom=0.7in, left=0.7in, right=0.7in]{geometry}
\usepackage{amsmath}
\usepackage{graphicx}
\usepackage{fancyhdr}
\usepackage{tcolorbox}
\usepackage{multicol}
\usepackage{pifont} % For checkboxes
\usepackage[defaultfam,tabular,lining]{montserrat} %% Option 'defaultfam'
\usepackage[T1]{fontenc}
\renewcommand*\oldstylenums[1]{{\fontfamily{Montserrat-TOsF}\selectfont #1}}
\renewcommand{\familydefault}{\sfdefault}
\usepackage{enumitem}
\usepackage{setspace}
\usepackage{parcolumns}
\usepackage{tabularx}

\setlength{\parindent}{0pt}
\hyphenpenalty=10000
\exhyphenpenalty=10000

\pagestyle{fancy}
\fancyhf{}
%\fancyhead[L]{\textbf{4.RI.1: Key Details and Evidence Practice}}
\fancyhead[R]{\includegraphics[width=1cm]{Round Logo.png}}
\fancyfoot[C]{\footnotesize Study Smart Tutors}

\begin{document}

\subsection*{Understanding Key Details and Supporting Evidence}
\onehalfspacing

\begin{tcolorbox}[colframe=black!40, colback=gray!0, title=Learning Objective]
\textbf{Objective:} Refer to details and examples in a text to explain what the text says explicitly and make inferences.
\end{tcolorbox}

\subsection*{Part 1: Multiple-Choice Questions}

1. What is the main idea of the passage?\\
"Rainforests are essential for the planet. They produce oxygen, regulate the climate, and are home to countless plants and animals. Unfortunately, rainforests are \\disappearing due to deforestation. This destruction results in loss of biodiversity, which affects ecosystems and human populations globally. Rainforests also play a significant role in the water cycle, helping to maintain rainfall patterns. Without them, areas around the world could face droughts and severe weather conditions. Protecting these vital ecosystems is critical not only for plants and animals but also for human survival. Governments, environmental groups, and individuals must work together to conserve rainforests through reforestation, sustainable farming practices, and stricter regulations."\\
\begin{enumerate}[label=\Alph*.]
    \item Rainforests are hard to visit.  
    \item Rainforests are home to many plants and animals.  
    \item Rainforests are important for the Earth and need protection.  
    \item Deforestation is not harmful to the planet.  
\end{enumerate}

\vspace{1cm}
\newpage
2. Which detail supports the main idea?\\
"Bees are important for pollination. They help plants grow by transferring pollen from one flower to another. Without bees, many plants would not produce fruits or seeds. This process is essential for food production and the survival of ecosystems. For example, crops like apples, almonds, and blueberries depend on bees for\\ pollination. Additionally, bees contribute to biodiversity by supporting the growth of flowering plants. However, bee populations are declining due to habitat loss, pesticides, and climate change. Protecting bees ensures the stability of food supplies and the health of natural environments. Farmers and gardeners can help by planting wildflowers and reducing pesticide use."\\
\begin{enumerate}[label=\Alph*.]
    \item Bees live in hives.  
    \item Bees are black and yellow.  
    \item Bees help plants grow by pollinating flowers.  
    \item Bees make honey.  
\end{enumerate}

\vspace{1cm}

3. What inference can you make from this text?\\
"Recycling helps reduce waste in landfills. It also conserves natural resources by reusing materials like paper, plastic, and glass. By recycling, people can help protect the environment for future generations. For example, recycling aluminum saves 95 percent of the energy required to produce it from raw materials. Additionally, recycling reduces greenhouse gas emissions, helping to combat climate change. Communities that prioritize recycling programs create cleaner, healthier\\ environments. By reusing resources and reducing pollution, recycling contributes to a more sustainable future. Encouraging more people to recycle can significantly decrease the negative impact of waste on the planet."\\
\begin{enumerate}[label=\Alph*.]
    \item Recycling is not necessary for environmental protection.  
    \item Recycling helps reduce waste and protect natural resources.  
    \item Only paper and plastic can be recycled.  
    \item Recycling is more important than other environmental efforts.  
\end{enumerate}

\vspace{1cm}
\newpage

\subsection*{Part 2: Short Answer Questions}

4. \textbf{Based on the passage below, why are trees important for the environment?\\}
"Trees absorb carbon dioxide and produce oxygen, helping to clean the air. They also provide shelter and food for animals. Additionally, trees prevent soil erosion by holding the ground in place with their roots. Forests play a key role in the water cycle by releasing water vapor into the atmosphere, which contributes to rainfall. Moreover, trees help reduce the urban heat island effect by providing shade and cooling the air. Without trees, the balance of ecosystems would be disrupted, leading to negative consequences for both wildlife and humans. Reforestation and \\conservation efforts are crucial to preserving these benefits."\\
\vspace{4cm}

5. \textbf{What evidence from the text supports the idea that recycling is good for the environment?\\}
"Recycling reduces the need to extract and process raw materials, which decreases pollution. It also saves energy and reduces greenhouse gas emissions. For example, recycling one ton of paper can save 17 trees and over 7,000 gallons of water. \\Communities that recycle create less waste and preserve valuable resources for future use. By minimizing the environmental impact of production and consumption, recycling contributes to a healthier planet. Encouraging businesses to use recycled materials can amplify these benefits."\\
\vspace{4cm}

6. Make an inference about why rainforests are often called "the lungs of the Earth."\\
\vspace{4cm}

\subsection*{Part 3: Select All That Apply}

7. Select \textbf{all} details that support the idea that rainforests are important: \\
\begin{enumerate}[label=\Alph*.]
    \item Rainforests produce oxygen.  
    \item Rainforests are difficult to access.  
    \item Rainforests regulate the Earth’s climate.  
    \item Rainforests are home to many plants and animals.  
\end{enumerate}

\vspace{1cm}

8. Which details explain how bees help plants?\\
\begin{enumerate}[label=\Alph*.]
    \item Bees transfer pollen between flowers.  
    \item Bees live in large hives.  
    \item Bees help plants grow fruits and seeds.  
    \item Bees produce honey.  
\end{enumerate}

\vspace{1cm}

\subsection*{Part 4: Fill in the Blank}

9. Evidence from a text can be used to support \underline{\hspace{4cm}} and inferences.

\vspace{3cm}

10. When reading a text, \underline{\hspace{4cm}} helps explain the meaning of key details.

\vspace{3cm}
% \newpage
% \subsection*{Answer Key}

% 1. \textbf{C} - Rainforests are important for the Earth and need protection.\\
% 2. \textbf{C} - Bees help plants grow by pollinating flowers.\\
% 3. \textbf{B} - Recycling helps reduce waste and protect natural resources.\\
% 4. \textbf{Sample Answer:} Trees are important because they absorb carbon dioxide, produce oxygen, help prevent soil erosion, provide shelter for animals, and contribute to the water cycle. They also reduce the urban heat island effect and help balance ecosystems.\\
% 5. \textbf{Sample Answer:} Recycling reduces pollution, saves energy, and conserves natural resources. For example, recycling one ton of paper saves 17 trees and thousands of gallons of water.\\
% 6. \textbf{Sample Answer:} Rainforests are called "the lungs of the Earth" because they produce oxygen and help regulate the climate, just like lungs help oxygenate the body.\\
% 7. \textbf{A, C, D} - Rainforests produce oxygen, regulate the Earth’s climate, and are home to many plants and animals.\\
% 8. \textbf{A, C} - Bees transfer pollen between flowers and help plants grow fruits and seeds.\\
% 9. \textbf{claims} \\
% 10. \textbf{evidence} \\

\end{document}

