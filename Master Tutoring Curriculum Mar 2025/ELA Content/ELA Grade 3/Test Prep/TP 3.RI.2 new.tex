\documentclass[12pt]{article}

\usepackage[a4paper, top=0.8in, bottom=0.7in, left=0.7in, right=0.7in]{geometry}
\usepackage{amsmath}
\usepackage{graphicx}
\usepackage{fancyhdr}
\usepackage{tcolorbox}
\usepackage{multicol}
\usepackage{pifont} % For checkboxes
%
\usepackage[defaultfam,tabular,lining]{montserrat} %% Option 'defaultfam'
\usepackage[T1]{fontenc}
\renewcommand*\oldstylenums[1]{{\fontfamily{Montserrat-TOsF}\selectfont #1}}
\renewcommand{\familydefault}{\sfdefault}
\usepackage{enumitem}
\usepackage{setspace}
\usepackage{parcolumns}
\usepackage{tabularx}

\setlength{\parindent}{0pt}
\hyphenpenalty=10000
\exhyphenpenalty=10000

\pagestyle{fancy}
\fancyhf{}
%\fancyhead[L]{\textbf{3.RI.2: Determining Main Idea and Supporting Details Practice}}
\fancyhead[R]{\includegraphics[width=1cm]{Round Logo.png}}
\fancyfoot[C]{\footnotesize Study Smart Tutors}

\begin{document}

\subsection*{Understanding the Main Idea and Supporting Details}
\onehalfspacing

\begin{tcolorbox}[colframe=black!40, colback=gray!0, title=Learning Objective]
\textbf{Objective:} Develop the ability to identify the main idea and supporting details in informational texts.
\end{tcolorbox}

\subsection*{Part 1: Multiple-Choice Questions}

1. What is the main idea of the following passage? \\
"Tigers are powerful hunters. They have sharp teeth and claws, which help them catch their prey. Tigers also have stripes that help them blend into their surroundings. These stripes provide camouflage in the tall grass, making it easier for tigers to sneak up on their prey. Additionally, tigers are solitary animals that rely on their strength and stealth to survive in the wild."\\
\begin{enumerate}[label=\Alph*.]
    \item Tigers are good swimmers.
    \item Tigers have stripes that make them unique.
    \item Tigers are skilled hunters with adaptations for survival.
    \item Tigers only eat plants.
\end{enumerate}

\vspace{1cm}

2. Which detail supports the main idea of the passage below? \\
"Penguins are birds that cannot fly. Instead, they are excellent swimmers and use their wings to glide through the water. Penguins live in cold climates and have a layer of fat to keep them warm. They also huddle together in large groups to stay warm and protect themselves from the harsh weather conditions. These adaptations help penguins survive in extreme environments."\\
\begin{enumerate}[label=\Alph*.]
    \item Penguins are black and white.
    \item Penguins live in zoos.
    \item Penguins have a layer of fat to keep them warm.
    \item Penguins are excellent at building nests.
\end{enumerate}

\vspace{1cm}

\newpage
3. What is the purpose of the details in this passage?\\
"Rainforests are important ecosystems. They provide oxygen, are home to many animals, and contain valuable plants used for medicine. Rainforests also regulate the Earth’s climate by absorbing carbon dioxide and releasing oxygen. Without rainforests, the planet’s biodiversity and environmental balance would be at risk."\\
\begin{enumerate}[label=\Alph*.]
    \item To describe animals in the rainforest.
    \item To explain why rainforests are important.
    \item To list the types of plants found in rainforests.
    \item To show how rainforests are being destroyed.
\end{enumerate}

\vspace{1cm}


\subsection*{Part 2: Multi-Select Questions}

4. Select \textbf{all} supporting details from the passage:  \\
"Dolphins are highly intelligent marine animals. They communicate using clicks and whistles. Dolphins are also known for their playful behavior and ability to learn tricks. Furthermore, dolphins live in social groups called pods, which help them work together to find food and protect one another from predators."\\
\begin{enumerate}[label=\Alph*.]
    \item Dolphins are mammals.
    \item Dolphins use clicks and whistles to communicate.
    \item Dolphins are highly intelligent.
    \item Dolphins have playful behavior.
\end{enumerate}

\vspace{1cm}

5. Which of the following questions can help you identify supporting details in a text? (Select \textbf{all} that apply.)  
\begin{enumerate}[label=\Alph*.]
    \item What is the text mostly about?  
    \item What examples are given?  
    \item What does the author think about the topic?  
    \item What facts or information support the main idea?  
\end{enumerate}

\vspace{1cm}

\subsection*{Part 3: Short Answer Questions}

6. What is the main idea of the following passage?\\
"Bees are essential for pollination. They carry pollen from one flower to another, helping plants grow fruits and seeds. Without bees, many plants would not survive. Bees also produce honey, which is a valuable food source for humans and animals. Their role in maintaining ecosystems makes bees one of the most important insects on Earth."

\vspace{3cm}

7. List two supporting details from the passage above.

\vspace{3cm}

8. Why are supporting details important when explaining a main idea?

\vspace{3cm}

\subsection*{Part 4: Fill in the Blank}

9. The main idea tells what the text is mostly about, while the \underline{\hspace{4cm}} explain the main idea.

\vspace{3cm}

10. The \underline{\hspace{4cm}} is the most important point the author wants you to understand from the text.

% \vspace{3cm}
% \newpage
% \subsection*{Answer Key}

% \textbf{Part 1: Multiple-Choice Questions}  
% 1. C. Tigers are skilled hunters with adaptations for survival.  
% 2. C. Penguins have a layer of fat to keep them warm.  
% 3. B. To explain why rainforests are important.  

% \textbf{Part 2: Multi-Select Questions}  
% 4. B. Dolphins use clicks and whistles to communicate.  
%    C. Dolphins are highly intelligent.  
%    D. Dolphins have playful behavior.  
% 5. B. What examples are given?  
%    D. What facts or information support the main idea?  

% \textbf{Part 3: Short Answer Questions}  
% 6. Main Idea: Bees are essential for pollination and maintaining ecosystems.  
% 7. Supporting Details:  
%    - Bees carry pollen, helping plants grow fruits and seeds.  
%    - Bees produce honey, which is valuable to humans and animals.  
% 8. Supporting details provide evidence and examples that clarify and strengthen the main idea.  

% \textbf{Part 4: Fill in the Blank}  
% 9. Supporting details  
% 10. Main idea  
\end{document}
