\documentclass[12pt]{article}

\usepackage[a4paper, top=0.8in, bottom=0.7in, left=0.7in, right=0.7in]{geometry}
\usepackage{amsmath}
\usepackage{graphicx}
\usepackage{fancyhdr}
\usepackage{tcolorbox}
\usepackage{multicol}
\usepackage{pifont} % For checkboxes
\usepackage[defaultfam,tabular,lining]{montserrat} %% Option 'defaultfam'
\usepackage[T1]{fontenc}
\renewcommand*\oldstylenums[1]{{\fontfamily{Montserrat-TOsF}\selectfont #1}}
\renewcommand{\familydefault}{\sfdefault}
\usepackage{enumitem}
\usepackage{setspace}
\usepackage{parcolumns}
\usepackage{tabularx}

\setlength{\parindent}{0pt}
\hyphenpenalty=10000
\exhyphenpenalty=10000

\pagestyle{fancy}
\fancyhf{}
%\fancyhead[L]{\textbf{3.W.2: Informative Writing}}
\fancyhead[R]{\includegraphics[width=1cm]{Round Logo.png}}
\fancyfoot[C]{\footnotesize Study Smart Tutors}

\begin{document}

\onehalfspacing

% Informational Text - Silkworms
\vspace{5cm}
\subsection*{Informational Text 1: Silkworms}

\begin{tcolorbox}[colframe=black!40, colback=gray!5]

\begin{spacing}{1.15}
    Silkworms are the larvae of the silk moth, a small insect that is famous for producing silk. Silkworms are native to China and have been raised for thousands of years. These caterpillars are often farmed in large groups on special farms called sericulture farms. 

    Silkworms start their lives as tiny eggs. After they hatch, they become larvae and start eating. Their diet mostly consists of mulberry leaves, which help them grow. As they grow bigger, silkworms shed their skin several times, a process known as molting. This continues until they are large enough to start spinning their silk cocoon.

    When silkworms are ready to spin their cocoon, they stop eating. They begin to release a sticky fluid from special glands near their mouth. This fluid hardens into a strong, continuous thread of silk. Silkworms spin this thread around themselves to form a protective cocoon. This spinning process can take two to three days, and the silkworm never stops until its cocoon is complete.

    Inside the cocoon, the silkworm enters a stage called pupation, where it changes into a moth. However, before the moth can break out of the cocoon, farmers usually kill the silkworm by boiling or steaming the cocoon. This stops the moth from emerging and damaging the silk thread.

    Once the cocoon is ready, the long thread of silk is carefully unwound. The thread is very strong and smooth, which is why it is used to make silk fabric. Silk is prized for its softness and shiny appearance, and it is used to make clothing, bedding, and other items.

\end{spacing}

\end{tcolorbox}

\vspace{1cm}

% Informational Text - Sea Silk

\subsection*{Informational Text 2: Sea Silk}

\begin{tcolorbox}[colframe=black!40, colback=gray!5]

\begin{spacing}{1.15}
    Sea silk is a rare and special type of fabric made from the fibers of a sea creature called the Mediterranean mussel. This mussel lives along the coasts of Italy, Spain, and other Mediterranean countries. Unlike regular silk, which comes from silkworms, sea silk is made from the byssus, a strong, golden thread that mussels use to attach themselves to rocks and other surfaces underwater.

    To collect sea silk, people have to dive into the sea and carefully remove the byssus from the mussels. This process is very delicate and time-consuming. The threads are very fine, and they must be harvested by hand. Once collected, the threads are cleaned and spun into a fine yarn. Sea silk is much rarer than silkworm silk and can be more expensive to make.

    The yarn made from sea silk is incredibly strong, yet it is soft and shiny. It has a golden color, which is why it is sometimes called “golden silk.” The fabric made from sea silk is lightweight and has a luxurious feel. In ancient times, sea silk was used to make beautiful clothing and other items for the wealthy, and it was highly prized by royalty.

    Today, sea silk is still made by a few artisans, although it is not as common as it once was. The delicate nature of the process and the rarity of the mussels make sea silk very special. It is used to create unique textiles that are both beautiful and rare.

\end{spacing}

\end{tcolorbox}

\vspace{1cm}

% Writing Prompt

\subsection*{Writing Prompt}

\begin{spacing}{1.15}
    After reading the two texts, write a multi-paragraph essay explaining how silk is made in different ways. Use information from both passages to support your answer. Be sure to:
    \begin{itemize}
        \item Describe the process of making silk from silkworms.
        \item Describe how sea silk is made from mussels.
        \item Compare the two processes and explain how they are different.
    \end{itemize}

    Manage your time carefully so you can do the following:
    \begin{itemize}
        \item Read the texts carefully.
        \item Plan your response.
        \item Write your response.
        \item Revise and edit your response.
    \end{itemize}

    Be sure to include the following tasks:
    \begin{itemize}
        \item An introduction
        \item Information from the sources as support
        \item A conclusion that is related to the information presented
    \end{itemize}
\end{spacing}

\end{document}
