\documentclass[12pt]{article}
\usepackage[a4paper, top=0.8in, bottom=0.7in, left=0.8in, right=0.8in]{geometry}
\usepackage{amsmath}
\usepackage{amsfonts}
\usepackage{latexsym}
\usepackage{graphicx}
\usepackage{fancyhdr}
\usepackage{enumitem}
\usepackage{setspace}
\usepackage{tcolorbox}
\usepackage{textcomp}
\usepackage[defaultfam,tabular,lining]{montserrat} % Font settings for Montserrat

% ChatGPT Directions:
% ----------------------------------------------------------------------
% This template is designed for creating guided lessons that align strictly with specific standards.
% Key points to ensure proper usage:
% 
% 1. **Key Concepts and Vocabulary**:
%    - Include only the concepts necessary for meeting the standards.
%    - Each Key Concept section must align explicitly with the standards being addressed.
% 2. **Examples**:
%    - Provide concrete worked examples to illustrate the Key Concepts.
%    - These should directly tie back to the Key Concepts presented earlier.
% 3. **Guided Practice**:
%    - Problems should reinforce Key Concepts and Examples.
%    - Allow for ample spacing between problems to give students room for work.
% 4. **Independent Practice**:
%    - Provide problems for students to practice Key Concepts individually.
% 5. **Exit Ticket**:
%    - Include a reflective or assessment-based question to evaluate student understanding.
% ----------------------------------------------------------------------

\setlength{\parindent}{0pt}
\pagestyle{fancy}

\setlength{\headheight}{27.11148pt}
\addtolength{\topmargin}{-15.11148pt}

\fancyhf{}
%\fancyhead[L]{\textbf{6.RP.A.3: Ratio and Rate Reasoning}}
\fancyhead[R]{\includegraphics[width=0.8cm]{Round Logo.png}} % Placeholder for logo
\fancyfoot[C]{\footnotesize © Study Smart Tutors}

\sloppy

\title{}
\date{}
\hyphenpenalty=10000
\exhyphenpenalty=10000

\begin{document}

\subsection*{Guided Lesson: Solving Real-World Problems with Ratio and Rate Reasoning}
\onehalfspacing

% Learning Objective Box
\begin{tcolorbox}[colframe=black!40, colback=gray!5, 
coltitle=black, colbacktitle=black!20, fonttitle=\bfseries\Large, 
title=Learning Objective, halign title=center, left=5pt, right=5pt, top=5pt, bottom=15pt]
\textbf{Objective:} Solve real-world problems using ratio and rate reasoning, and represent solutions using tables, tape diagrams, and double number lines.
\end{tcolorbox}

\vspace{1em}

% Key Concepts and Vocabulary
\begin{tcolorbox}[colframe=black!60, colback=white, 
coltitle=black, colbacktitle=black!15, fonttitle=\bfseries\Large, 
title=Key Concepts and Vocabulary, halign title=center, left=10pt, right=10pt, top=10pt, bottom=15pt]
\textbf{Key Concepts:}
\begin{itemize}
    \item A \textbf{ratio} is a comparison of two quantities, written as \( a:b \), \( \frac{a}{b} \), or "a to b."
    \item A \textbf{rate} is a special ratio that compares two quantities with different units (e.g., miles per hour).
    \item A \textbf{unit rate} describes how much of one quantity corresponds to one unit of another (e.g., cost per item).
    \item \textbf{Proportions} are equations stating that two ratios are equivalent.
    \item \textbf{Representation Tools:}
    \begin{itemize}
        \item \textbf{Tables:} Used to list equivalent ratios.
        \item \textbf{Tape Diagrams:} Used to visually compare quantities.
        \item \textbf{Double Number Lines:} Used to compare quantities with consistent spacing.
    \end{itemize}
\end{itemize}
\end{tcolorbox}

\vspace{1em}

% Examples Box
\begin{tcolorbox}[colframe=black!60, colback=white, 
coltitle=black, colbacktitle=black!15, fonttitle=\bfseries\Large, 
title=Examples, halign title=center, left=10pt, right=10pt, top=10pt, bottom=15pt]
\textbf{Example 1: Finding Ratios}
\begin{itemize}
    \item Problem: Write the ratio of apples to oranges if there are 8 apples and 12 oranges.
    \item {Solution: Write the ratio as \( 8:12 \). Simplify by dividing both parts by their GCF, which is 4. \( 8 \div 4 : 12 \div 4 = 2:3 \). The simplified ratio is \( 2:3 \).}
\end{itemize}

\textbf{Example 2: Solving a Proportion}
\begin{itemize}
    \item Problem: Solve \( \frac{3}{4} = \frac{x}{12} \).
    \item {Solution: Use cross-multiplication: \( 3 \cdot 12 = 4 \cdot x \). This gives \( 36 = 4x \). Divide both sides by 4: \( x = \frac{36}{4} = 9 \). The solution is \( x = 9 \).}
\end{itemize}


\end{tcolorbox}

\vspace{1em}

% Guided Practice
\begin{tcolorbox}[colframe=black!60, colback=white, 
coltitle=black, colbacktitle=black!15, fonttitle=\bfseries\Large, 
title=Guided Practice, halign title=center, left=10pt, right=10pt, top=10pt, bottom=45pt]
\textbf{Solve the following problems with teacher support:}
\begin{enumerate}[itemsep=3em]
    \item A recipe calls for \( 2 \) cups of sugar for every \( 3 \) cups of flour. How much sugar is needed for \( 12 \) cups of flour? (Hint: Use a proportion.)
    \item A student earns \$40 for 5 hours of work. How much will they earn in 8 hours?
    \item A store sells 4 apples for \$2. How much will 10 apples cost?
\end{enumerate}
\end{tcolorbox}

\vspace{1em}

% Independent Practice
\begin{tcolorbox}[colframe=black!60, colback=white, 
coltitle=black, colbacktitle=black!15, fonttitle=\bfseries\Large, 
title=Independent Practice, halign title=center, left=10pt, right=10pt, top=10pt, bottom=45pt]
\textbf{Solve the following problems independently:}
\begin{enumerate}[itemsep=3em]
    \item A map has a scale of 1 inch = 50 miles. What is the actual distance represented by \( 2.5 \) inches on the map?
    \item A car travels 200 miles in 4 hours. How far can it travel in 7 hours?
    \item Complete the table for a rate of 3 stickers per dollar:
    \[
    \begin{array}{|c|c|c|c|}
    \hline
    Dollars & 1 & 3 & 5 \\
    \hline
    Stickers & \_\_\_ & \_\_\_ & \_\_\_ \\
    \hline
    \end{array}
    \]
\end{enumerate}
\end{tcolorbox}

\vspace{1em}

% Exit Ticket
\begin{tcolorbox}[colframe=black!60, colback=white, 
coltitle=black, colbacktitle=black!15, fonttitle=\bfseries\Large, 
title=Exit Ticket, halign title=center, left=10pt, right=10pt, top=10pt, bottom=110pt]
\textbf{Answer the following question:}
\begin{itemize}
    \item A car travels 300 miles using 10 gallons of fuel. What is the unit rate in miles per gallon?
\end{itemize}
\end{tcolorbox}

\end{document}
