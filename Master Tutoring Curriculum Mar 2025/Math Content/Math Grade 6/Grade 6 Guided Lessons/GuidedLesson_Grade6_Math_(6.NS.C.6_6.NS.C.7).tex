\documentclass[12pt]{article}
\usepackage[a4paper, top=0.8in, bottom=0.7in, left=0.8in, right=0.8in]{geometry}
\usepackage{amsmath}
\usepackage{amsfonts}
\usepackage{latexsym}
\usepackage{graphicx}
\usepackage{fancyhdr}
\usepackage{enumitem}
\usepackage{setspace}
\usepackage{tcolorbox}
\usepackage{textcomp}
\usepackage[defaultfam,tabular,lining]{montserrat} % Font settings for Montserrat
\usepackage{pgfplots}

\setlength{\parindent}{0pt}
\pagestyle{fancy}

\setlength{\headheight}{27.11148pt}
\addtolength{\topmargin}{-15.11148pt}

\fancyhf{}
%\fancyhead[L]{\textbf{6.NS.C.6, 6.NS.C.7: Number Lines, Inequalities, and Absolute Value}} % Header with standards and topic title
\fancyhead[R]{\includegraphics[width=0.8cm]{Round Logo.png}} % Placeholder for logo
\fancyfoot[C]{\footnotesize © Study Smart Tutors}

\sloppy

\title{}
\date{}
\hyphenpenalty=10000
\exhyphenpenalty=10000

\pgfplotsset{compat=1.18}

\begin{document}

\subsection*{Guided Lesson: Number Lines, Inequalities, and Absolute Value}
\onehalfspacing

% Learning Objective Box
\begin{tcolorbox}[colframe=black!40, colback=gray!5, 
coltitle=black, colbacktitle=black!20, fonttitle=\bfseries\Large, 
title=Learning Objective, halign title=center, left=5pt, right=5pt, top=5pt, bottom=15pt]
\textbf{Objective:} Develop an understanding of how to locate numbers on a number line, compare rational numbers, interpret inequalities, and use absolute value to solve real-world problems.
\end{tcolorbox}

% Key Concepts and Vocabulary
\begin{tcolorbox}[colframe=black!60, colback=white, 
coltitle=black, colbacktitle=black!15, fonttitle=\bfseries\Large, 
title=Key Concepts and Vocabulary, halign title=center, left=10pt, right=10pt, top=10pt, bottom=15pt]
\textbf{Key Concepts:}
\begin{itemize}
    \item \textbf{Number Line:} A visual representation of numbers where positive numbers are to the right of zero and negative numbers are to the left.
    \item \textbf{Inequalities:} Statements about the relative size of two numbers, using \( <, \leq, >, \geq \).
    \item \textbf{Absolute Value:} The distance of a number from zero on a number line, always non-negative.
    \item \textbf{Coordinate Plane:} A two-dimensional system where points are defined by \( (x, y) \), representing horizontal and vertical positions.
\end{itemize}
\end{tcolorbox}

% Examples Box
\begin{tcolorbox}[colframe=black!60, colback=white, 
coltitle=black, colbacktitle=black!15, fonttitle=\bfseries\Large, 
title=Examples, halign title=center, left=10pt, right=10pt, top=10pt, bottom=15pt]
\textbf{Example 1: Representing Numbers on a Number Line}
\begin{itemize}
    \item Problem: Plot \( -3, 0, 2.5, -1.5 \) on a number line.
    \item Solution: Mark each number at its corresponding position.
    \begin{center}
        \begin{tikzpicture}
            \draw[thick, <->] (-5.5,0) -- (5.5,0); % Number line
            \foreach \x in {-5,-4,-3,-2,-1,0,1,2,3,4,5} {
                \draw (\x,0.1) -- (\x,-0.1) node[below] {\x};
            }
        \end{tikzpicture}
    \end{center}
\end{itemize}

\textbf{Example 2: Solving and Graphing Inequalities}
\begin{itemize}
    \item Problem: Solve \( x + 2 \leq 5 \) and graph the solution.
    \item Solution: Subtract 2 from both sides: \( x \leq 3 \). Graph \( x \leq 3 \) on a number line.
    \begin{center}
        \begin{tikzpicture}
            \draw[thick, <->] (-3,0) -- (6,0); % Number line
            \foreach \x in {-2,-1,0,1,2,3,4,5} {
                \draw (\x,0.1) -- (\x,-0.1) node[below] {\x};
            }
        \end{tikzpicture}
    \end{center}
\end{itemize}

\textbf{Example 3: Using Absolute Value}
\begin{itemize}
    \item Problem: Find the distance between \( -4 \) and \( 2 \) on a number line.
    \item Solution: The distance is \( |-4 - 2| = | -6 | = 6 \).
\end{itemize}
\end{tcolorbox}

% Guided Practice Box
\begin{tcolorbox}[colframe=black!60, colback=white, 
coltitle=black, colbacktitle=black!15, fonttitle=\bfseries\Large, 
title=Guided Practice, halign title=center, left=10pt, right=10pt, top=10pt, bottom=45pt]
\textbf{Solve the following problems with teacher support:}
\begin{enumerate}[itemsep=3em]
    \item Plot \( -5, 0, 3.2, -1.8 \) on a number line.
    \item Solve and graph: \( y - 1 \geq -2 \).
    \item Find the absolute value of \( -7 \).
    \item Write an inequality to represent: "The number of participants must be greater than 10 and less than or equal to 30."
\end{enumerate}
\end{tcolorbox}

% Independent Practice Box
\begin{tcolorbox}[colframe=black!60, colback=white, 
coltitle=black, colbacktitle=black!15, fonttitle=\bfseries\Large, 
title=Independent Practice, halign title=center, left=10pt, right=10pt, top=10pt, bottom=45pt]
\textbf{Solve the following problems independently:}
\begin{enumerate}[itemsep=3em]
    \item Order \( -3, 0, 1.5, -1.2 \) from least to greatest.
    \item Solve \( x + 4 > 7 \) and graph the solution.
    \item Compare \( |3.2| \) and \( |-3.5| \). Which is greater?
    \item A ship is at \( -25 \) feet below sea level. It rises \( 15 \) feet. Where is it now?
\end{enumerate}
\end{tcolorbox}

% Exit Ticket Box
\begin{tcolorbox}[colframe=black!60, colback=white, 
coltitle=black, colbacktitle=black!15, fonttitle=\bfseries\Large, 
title=Exit Ticket, halign title=center, left=10pt, right=10pt, top=10pt, bottom=110pt]
\textbf{Answer the following question:}
\begin{itemize}
    \item How does understanding number lines and absolute value help in solving real-world problems? Provide an example.
\end{itemize}
\end{tcolorbox}

\end{document}
