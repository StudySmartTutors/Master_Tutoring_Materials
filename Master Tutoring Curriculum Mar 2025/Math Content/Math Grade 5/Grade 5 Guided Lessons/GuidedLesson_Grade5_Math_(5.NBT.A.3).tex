\documentclass[12pt]{article}
\usepackage[a4paper, top=0.8in, bottom=0.7in, left=0.8in, right=0.8in]{geometry}
\usepackage{amsmath}
\usepackage{amsfonts}
\usepackage{latexsym}
\usepackage{graphicx}
\usepackage{fancyhdr}
\usepackage{tcolorbox}
\usepackage{enumitem}
\usepackage{setspace}
\usepackage[defaultfam,tabular,lining]{montserrat} % Font settings for Montserrat

% General Comment: Template for creating problem sets in a structured format with headers, titles, and sections.
% This document uses Montserrat font and consistent styles for exercises, problems, and performance tasks.

\setlength{\parindent}{0pt}
\pagestyle{fancy}

\setlength{\headheight}{27.11148pt}
\addtolength{\topmargin}{-15.11148pt}

\fancyhf{}
%\fancyhead[L]{\textbf{5.NBT.A.3: Reading, Writing, and Comparing Decimals}}
\fancyhead[R]{\includegraphics[width=0.8cm]{Round Logo.png}} % Placeholder for logo
\fancyfoot[C]{\footnotesize © Study Smart Tutors}

\sloppy

\title{}
\date{}
\hyphenpenalty=10000
\exhyphenpenalty=10000

\begin{document}

\subsection*{Guided Lesson: Reading, Writing, and Comparing Decimals}
\onehalfspacing

% Learning Objective Box
\begin{tcolorbox}[colframe=black!40, colback=gray!5, 
coltitle=black, colbacktitle=black!20, fonttitle=\bfseries\Large, 
title=Learning Objective, halign title=center, left=5pt, right=5pt, top=5pt, bottom=15pt]
\textbf{Objective:} Read, write, and compare decimals to the thousandths place. Perform operations with decimals and apply rounding to solve real-world problems.
\end{tcolorbox}

\vspace{1em}

% Key Concepts and Vocabulary
\begin{tcolorbox}[colframe=black!60, colback=white, 
coltitle=black, colbacktitle=black!15, fonttitle=\bfseries\Large, 
title=Key Concepts and Vocabulary, halign title=center, left=10pt, right=10pt, top=10pt, bottom=15pt]
\textbf{Key Concepts:}
\begin{itemize}
    \item \textbf{Place Value:} Decimals represent parts of a whole. The place values include tenths (\(0.1\)), hundredths (\(0.01\)), and thousandths (\(0.001\)).
    \item \textbf{Reading and Writing Decimals:} 
    \begin{itemize}
        \item \(0.456\): "Four hundred fifty-six thousandths"
        \item Write "three and seventy-two thousandths" as \(3.072\).
    \end{itemize}
    \item \textbf{Expanded Form:} Represent decimals as a sum of place values. For example:
    \[ 0.456 = 0.4 + 0.05 + 0.006 \]
    \item \textbf{Comparing Decimals:} Align decimals by place value and compare digits from left to right. For example:
    \[ 0.45 > 0.405 \text{ (tenths place is greater)}. \]
    \item \textbf{Rounding:} To round a decimal, identify the place value to round to and look at the digit to its right:
    \begin{itemize}
        \item Round \(8.456\) to the nearest tenth: \(8.5\).
        \item Round \(8.456\) to the nearest hundredth: \(8.46\).
    \end{itemize}
\end{itemize}
\end{tcolorbox}

\vspace{1em}

% Examples Box
\begin{tcolorbox}[colframe=black!60, colback=white, 
coltitle=black, colbacktitle=black!15, fonttitle=\bfseries\Large, 
title=Examples, halign title=center, left=10pt, right=10pt, top=10pt, bottom=15pt]
\textbf{Example 1: Writing Decimals in Expanded Form}
\begin{itemize}
    \item Decimal: \( 2.305 \)
    \item Expanded Form: \( 2 + 0.3 + 0.005 \)
\end{itemize}

\textbf{Example 2: Comparing Decimals}
\begin{itemize}
    \item Compare \( 0.75 \) and \( 0.750 \):
    \item Solution: The decimals are equivalent because adding trailing zeros does not change the value of a decimal.
\end{itemize}

\textbf{Example 3: Rounding Decimals}
\begin{itemize}
    \item Round \( 4.356 \) to the nearest tenth:
    \[ \text{Look at the hundredths place: } 5 \geq 5, \text{ so round up.} \]
    \[ \text{Answer: } 4.4 \]
\end{itemize}

\textbf{Example 4: Adding and Subtracting Decimals}
\begin{itemize}
    \item Problem: \( 5.671 - 2.345 \)
    \item Solution: Align the decimals and subtract:
    \[ 5.671 - 2.345 = 3.326 \]
\end{itemize}

\textbf{Example 5: Real-World Application}
\begin{itemize}
    \item A baker uses \( 0.75 \) pounds of flour per cake. If they bake 4 cakes, how much flour do they use?
    \item Solution:
    \[ 0.75 \times 4 = 3.00 \text{ pounds.} \]
\end{itemize}
\end{tcolorbox}

\vspace{1em}

% Guided Practice
\begin{tcolorbox}[colframe=black!60, colback=white, 
coltitle=black, colbacktitle=black!15, fonttitle=\bfseries\Large, 
title=Guided Practice, halign title=center, left=10pt, right=10pt, top=10pt, bottom=15pt]
\textbf{Work with a partner or teacher to solve the following problems:}
\begin{enumerate}[itemsep=3em]
    \item Write \( 3.067 \) in expanded form.
    \item Compare \( 0.872 \) and \( 0.87 \). Use \( >, <, \) or \( = \).
    \item Round \( 9.564 \) to the nearest tenth and hundredth.
    \item Solve: \( 4.305 + 2.75 \).
    \item A runner jogged \( 5.4 \) miles on Monday and \( 4.356 \) miles on Wednesday. What is their total distance? Round to the nearest tenth.
    \item Write a multiplication equation for the following: A bag of sugar weighs \( 2.5 \) pounds. If a baker uses \( 3 \) bags, how much sugar is used in total?
\end{enumerate}
\end{tcolorbox}

\vspace{1em}

% Additional Notes
\begin{tcolorbox}[colframe=black!40, colback=gray!5, 
coltitle=black, colbacktitle=black!20, fonttitle=\bfseries\Large, 
title=Additional Notes, halign title=center, left=5pt, right=5pt, top=5pt, bottom=15pt]
\textbf{Notes:}
\begin{itemize}
    \item When ordering decimals, use a place value chart to compare digits.
    \item When adding and subtracting decimals, always align the decimal points.
    \item Multiplying decimals can be visualized as repeated addition.
\end{itemize}
\end{tcolorbox}

\vspace{1em}

% Independent Practice
\begin{tcolorbox}[colframe=black!60, colback=white, 
coltitle=black, colbacktitle=black!15, fonttitle=\bfseries\Large, 
title=Independent Practice, halign title=center, left=10pt, right=10pt, top=10pt, bottom=15pt]
\textbf{Solve the following problems on your own:}
\begin{enumerate}[itemsep=3em]
    \item Write \( 6.204 \) in expanded form.
    \item Compare \( 3.45 \) and \( 3.450 \). Use \( >, <, \) or \( = \).
    \item Round \( 7.894 \) to the nearest tenth and hundredth.
    \item Subtract \( 12.75 - 3.254 \).
    \item A store sells notebooks for \$2.65 each. If a customer buys \( 6 \) notebooks, how much will they spend in total?
    \item Write a division equation for the following: A cake weighs \( 3.75 \) pounds and is cut into \( 5 \) equal pieces. What is the weight of each piece?
\end{enumerate}
\end{tcolorbox}

\vspace{1em}

% Exit Ticket
\begin{tcolorbox}[colframe=black!60, colback=white, 
coltitle=black, colbacktitle=black!15, fonttitle=\bfseries\Large, 
title=Exit Ticket, halign title=center, left=10pt, right=10pt, top=10pt, bottom=15pt]
\textbf{Answer the following question:}
\begin{itemize}
    \item How can you use rounding and place value to simplify real-world problems involving decimals? Provide an example.
\end{itemize}
\end{tcolorbox}

\end{document}
