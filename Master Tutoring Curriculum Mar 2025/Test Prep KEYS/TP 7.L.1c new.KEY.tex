\documentclass[12pt]{article}

\usepackage[a4paper, top=0.8in, bottom=0.7in, left=0.7in, right=0.7in]{geometry}
\usepackage{amsmath}
\usepackage{graphicx}
\usepackage{fancyhdr}
\usepackage{tcolorbox}
\usepackage[defaultfam,tabular,lining]{montserrat} %% Option 'defaultfam'
\usepackage[T1]{fontenc}
\renewcommand*\oldstylenums[1]{{\fontfamily{Montserrat-TOsF}\selectfont #1}}
\renewcommand{\familydefault}{\sfdefault}
\usepackage{enumitem}
\usepackage{setspace}

\setlength{\parindent}{0pt}
\hyphenpenalty=10000
\exhyphenpenalty=10000

\pagestyle{fancy}
\fancyhf{}
\fancyhead[L]{\textbf{7.L.1c: Misplaced and Dangling Modifiers Practice}}
\fancyhead[R]{\includegraphics[width=1cm]{Round Logo.png}}
\fancyfoot[C]{\footnotesize Study Smart Tutors}

\begin{document}

\subsection*{Recognizing and Correcting Misplaced and Dangling Modifiers}
\onehalfspacing

\begin{tcolorbox}[colframe=black!40, colback=gray!0, title=Learning Objective]
\textbf{Objective:} Recognize and correct misplaced and dangling modifiers in sentences to ensure clarity and proper syntax.
\end{tcolorbox}


\subsection*{Answer Key}

\textbf{Part 1: Multiple-Choice Questions}

1. \textbf{B} – Running through the park, the dog barked at everyone. (This is a misplaced modifier because "running through the park" seems to describe the dog, but it should describe the person running through the park.)

2. \textbf{B} – While walking to school, I felt the backpack slip off my shoulder. (This corrects the dangling modifier by clearly identifying the subject performing the action.)

3. \textbf{A} – She read the article in a magazine about climate change. (This corrects the misplaced modifier by placing "in a magazine" next to the noun it modifies, "article.")

\textbf{Part 2: Select All That Apply Questions}

4. \textbf{A, B} – The teacher handed out assignments to the students written in cursive. (Misplaced modifier: "written in cursive" should describe the assignments, not the students.)  
Exhausted from the hike, the tent was set up quickly. (Misplaced modifier: "Exhausted from the hike" should describe the person setting up the tent, not the tent.)

5. \textbf{B, D} – Admired by everyone, the painting hung on the wall. (Corrected for clarity by placing the modifier next to what it describes.)  
Everyone admired the painting hung on the wall. (Corrected for clarity by simplifying the sentence structure.)

6. \textbf{A, B, D} – The dishes were left on the table after we finished the meal. (Corrected: the subject performing the action is clarified.)  
Having finished the meal, we left the dishes on the table. (Corrected: the subject performing the action is clarified.)  
We finished the meal, leaving the dishes on the table. (Corrected: the subject performing the action is clarified.)

\textbf{Part 4: Fill in the Blank Questions}

9. Attached  
10. Unclear  





\end{document}

