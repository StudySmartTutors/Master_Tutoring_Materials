\documentclass[12pt]{article}

\usepackage[a4paper, top=0.8in, bottom=0.7in, left=0.7in, right=0.7in]{geometry}
\usepackage{amsmath}
\usepackage{graphicx}
\usepackage{fancyhdr}
\usepackage{tcolorbox}
\usepackage{multicol}
\usepackage{pifont} % For checkboxes
\usepackage[defaultfam,tabular,lining]{montserrat} %% Option 'defaultfam'
\usepackage[T1]{fontenc}
\renewcommand*\oldstylenums[1]{{\fontfamily{Montserrat-TOsF}\selectfont #1}}
\renewcommand{\familydefault}{\sfdefault}
\usepackage{enumitem}
\usepackage{setspace}
\usepackage{parcolumns}
\usepackage{tabularx}

\setlength{\parindent}{0pt}
\hyphenpenalty=10000
\exhyphenpenalty=10000

\pagestyle{fancy}
\fancyhf{}
%\fancyhead[L]{\textbf{3.W.2: Informative Writing}}
\fancyhead[R]{\includegraphics[width=1cm]{Round Logo.png}}
\fancyfoot[C]{\footnotesize Study Smart Tutors}

\begin{document}

\onehalfspacing

% Passage 1 - Pros of Building Public Parks

\subsection*{Informational Text 1: The Pros of Building Public Parks}

\begin{tcolorbox}[colframe=black!40, colback=gray!5]

\begin{spacing}{1.15}
    Public parks offer numerous advantages that benefit both individuals and communities as a whole. One of the most important benefits is that they provide spaces for physical activity. People can go to parks to jog, walk, bike, or take part in various sports. This promotes healthier lifestyles and can reduce the risk of health problems such as obesity, heart disease, and stress. Children benefit from having a safe environment to play and exercise, and families can spend time together in an enjoyable and relaxing setting. 

    In addition to promoting physical activity, parks also offer a place for people to connect with nature. Many parks feature beautiful landscapes, gardens, and bodies of water, providing an escape from the hustle and bustle of everyday life. Time spent in nature has been shown to reduce stress, improve mood, and enhance mental well-being. This is especially important in urban areas, where green spaces are often limited. 

    Public parks can also strengthen community ties. They serve as gathering spaces where people from diverse backgrounds can come together for social events, festivals, and activities. These events foster a sense of belonging and create opportunities for people to interact and build relationships. Furthermore, parks have been shown to increase the value of nearby homes, which benefits homeowners and the overall local economy. 

    Lastly, public parks are a resource for environmental conservation. By preserving natural habitats and wildlife, parks help protect local ecosystems. They provide homes for various species of plants and animals, contributing to biodiversity. In many cases, parks also help to improve air quality and reduce pollution, making them a vital asset to urban areas.

\end{spacing}

\end{tcolorbox}

\vspace{1cm}

% Passage 2 - Cons of Building Public Parks

\subsection*{Informational Text 2: The Cons of Building Public Parks}

\begin{tcolorbox}[colframe=black!40, colback=gray!5]

\begin{spacing}{1.15}
    While public parks can greatly benefit a community, there are some drawbacks that need to be considered before building one. One of the major concerns is the cost. Developing and maintaining a public park can be expensive. The land itself must be purchased, and construction costs for playgrounds, walking paths, sports courts, and other amenities can add up quickly. In addition, ongoing maintenance is required to keep the park clean, safe, and well-maintained. This includes the costs of landscaping, repairs, security, and waste disposal. Many local governments struggle to allocate sufficient funds for park upkeep, which can lead to neglected or deteriorating facilities.

    Another challenge is the potential for crime and vandalism. Public parks, especially those that are not well-maintained or lack adequate security, can become hotspots for illegal activities. This can range from minor offenses like graffiti and littering to more serious crimes such as theft or drug use. As a result, parks may become unsafe places for families, children, and other visitors. The fear of crime can discourage people from using parks, which defeats the purpose of providing a public space for recreation and relaxation.

    Noise and overcrowding can also be a problem, particularly in larger cities. On weekends or during special events, parks can become very noisy, especially if large groups gather. This can disturb nearby residents and reduce the quality of life for those who live close by. Additionally, overcrowded parks may not provide the peaceful retreat that many people expect from these public spaces.

    Finally, some people argue that land used for parks could be better utilized for other purposes, such as building schools, hospitals, or businesses. In areas with limited space, choosing between a park and other community needs can be a difficult decision. Critics of park development often suggest that resources should be invested in infrastructure that provides more immediate and essential services to the community.

\end{spacing}

\end{tcolorbox}

\vspace{1cm}

% Writing Prompt

\subsection*{Writing Prompt}

\begin{spacing}{1.15}
    Your city is considering how to make the community better. Should your city build more public parks?

    Write a multi-paragraph essay expressing your opinion about whether building public parks is a good use of the city's money. Explain why your choice is better than the other. Use information from the sources in your essay.

    Manage your time carefully so you can do the following actions:
    \begin{itemize}
        \item Read the sources.
        \item Plan your response.
        \item Write your response.
        \item Revise and edit your response
    \end{itemize}

Be sure to include the following tasks:
\begin{itemize}
    \item an introduction
    \item support for your opinion using information from the sources
    \item a conclusion that is related to your opinion
\end{itemize}


\end{spacing}

\end{document}

