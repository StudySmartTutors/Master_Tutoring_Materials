\documentclass[12pt]{article}

\usepackage[a4paper, top=0.8in, bottom=0.7in, left=0.7in, right=0.7in]{geometry}
\usepackage{amsmath}
\usepackage{graphicx}
\usepackage{fancyhdr}
\usepackage{tcolorbox}
\usepackage[defaultfam,tabular,lining]{montserrat} %% Option 'defaultfam'
\usepackage[T1]{fontenc}
\renewcommand*\oldstylenums[1]{{\fontfamily{Montserrat-TOsF}\selectfont #1}}
\renewcommand{\familydefault}{\sfdefault}
\usepackage{enumitem}
\usepackage{setspace}

\setlength{\parindent}{0pt}
\hyphenpenalty=10000
\exhyphenpenalty=10000

\pagestyle{fancy}
\fancyhf{}
%\fancyhead[L]{\textbf{6.RI.2: Central Idea Practice}}
\fancyhead[R]{\includegraphics[width=1cm]{Round Logo.png}}
\fancyfoot[C]{\footnotesize Study Smart Tutors}

\begin{document}

\subsection*{Determining the Central Idea of Informational Texts}
\onehalfspacing

\begin{tcolorbox}[colframe=black!40, colback=gray!0, title=Learning Objective]
\textbf{Objective:} Determine the central idea of a text and explain how it is conveyed through supporting details.
\end{tcolorbox}

\subsection*{Answer Key}

\begin{itemize}
    \item \textbf{Question 1:} B. Deforestation contributes to climate change and harms ecosystems.
    \item \textbf{Question 2:} B. Plastic pollution harms marine life and human health, but solutions exist.
    \item \textbf{Question 3:} C. Renewable energy offers sustainable solutions for climate change and energy security.
    \item \textbf{Question 4:} A, B, D.
    \item \textbf{Question 5:} A, C.
    \item \textbf{Question 6:} A, C, D.
    \item \textbf{Question 7:} Forests absorb carbon dioxide and release oxygen (benefit to climate); forests prevent soil erosion and support biodiversity.
    \item \textbf{Question 8:} Renewable energy reduces greenhouse gas emissions and creates jobs in manufacturing, installation, and maintenance.
    \item \textbf{Question 9:} Supporting details.
    \item \textbf{Question 10:} Bias.
\end{itemize}

\end{document}

