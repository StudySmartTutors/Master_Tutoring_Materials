\documentclass[12pt]{article}

\usepackage[a4paper, top=0.8in, bottom=0.7in, left=0.7in, right=0.7in]{geometry}
\usepackage{amsmath}
\usepackage{graphicx}
\usepackage{fancyhdr}
\usepackage{tcolorbox}
\usepackage{multicol}
\usepackage{pifont} % For checkboxes
\usepackage[defaultfam,tabular,lining]{montserrat} %% Option 'defaultfam'
\usepackage[T1]{fontenc}
\renewcommand*\oldstylenums[1]{{\fontfamily{Montserrat-TOsF}\selectfont #1}}
\renewcommand{\familydefault}{\sfdefault}
\usepackage{enumitem}
\usepackage{setspace}
\usepackage{parcolumns}
\usepackage{tabularx}

\setlength{\parindent}{0pt}
\hyphenpenalty=10000
\exhyphenpenalty=10000
\setlength{\parindent}{0pt}
\pagestyle{fancy}

\setlength{\headheight}{32.80098pt}


% \addtolength{\topmargin}{5.1pt}
% \fancyhf{}
%\fancyhead[L]{\textbf{3.RL.3: Character Actions and Events Practice}}
\fancyhead[R]{\includegraphics[width=1cm]{Round Logo.png}}
\fancyfoot[C]{\footnotesize \textcopyright Study Smart Tutors}

\begin{document}

\subsection*{Analyzing Character Actions and Events}
\onehalfspacing

\begin{tcolorbox}[colframe=black!40, colback=gray!0, title=Learning Objective]
\textbf{Objective:} Understand how characters’ actions and events influence a story’s development.
\end{tcolorbox}

\subsection*{Part 1: Multiple-Choice Questions}

1. How did the lion cub’s actions influence the story? \\
"A young lion cub always relied on his mother to hunt for food. One day, the mother lion fell ill and could no longer hunt. The cub realized he needed to learn how to find food himself. After many attempts and failures, he succeeded and became independent. Over time, the cub grew stronger and was able to provide not only for himself but also for his mother. His persistence and hard work helped them survive."\\
\begin{enumerate}[label=\Alph*.]
    \item He ignored his mother’s situation.
    \item He gave up and relied on others.
    \item He learned to hunt and became independent.
    \item He waited for someone to help him.
\end{enumerate}

\vspace{1cm}

2. What role did the turtle’s steady pace play in the story?\\
"A turtle and a rabbit decided to race. The rabbit ran quickly but became overconfident and took a nap during the race. The turtle, though slow, kept moving steadily toward the finish line. When the rabbit woke up, he realized the turtle was almost at the finish line. Despite his best efforts, the rabbit could not catch up. The turtle won the race because of his persistence and steady effort."\\
\begin{enumerate}[label=\Alph*.]
    \item It made the turtle lose the race.
    \item It helped the turtle win by being consistent.
    \item It caused the rabbit to speed up.
    \item It distracted the rabbit from winning.
\end{enumerate}

\vspace{1cm}




\subsection*{Character Actions and Their Influence on Events}
\onehalfspacing

\begin{tcolorbox}[colframe=black!40, colback=gray!0, title=Learning Objective]
\textbf{Objective:} Analyze how a character's actions contribute to the sequence of events in a story.
\end{tcolorbox}

\subsection*{Part 1: Multiple-Choice Questions}

1. How did the hare’s decision to nap influence the race? \\
"A hare and a tortoise decided to race. The hare was much faster and raced ahead. However, feeling overconfident, the hare decided to take a nap. Meanwhile, the tortoise kept moving at a slow and steady pace. By the time the hare woke up, the tortoise was near the finish line and won the race."\\
\begin{enumerate}[label=\Alph*.]
    \item It showed the hare's confidence helped him win.
    \item It gave the tortoise time to catch up and win.
    \item It allowed the hare to rest and finish first.
    \item It didn’t influence the race at all.
\end{enumerate}

\vspace{1cm}

2. How did the boy’s repeated lies in "The Boy Who Cried Wolf" affect the villagers?\\
"A boy thought it was funny to yell "Wolf!" even when no wolf was near. The villagers ran to help him each time, only to find it was a trick. When a real wolf appeared, the boy cried for help, but the villagers didn’t believe him. The wolf ate the sheep, teaching the boy a hard lesson about honesty."\\
\begin{enumerate}[label=\Alph*.]
    \item The villagers ignored the boy, causing the sheep to be lost.
    \item The villagers trusted the boy more after his lies.
    \item The villagers helped the boy every time he cried.
    \item The villagers captured the wolf to protect the sheep.
\end{enumerate}

\vspace{1cm}

3. What did the ants’ teamwork achieve?\\
"A group of ants worked together to move a large piece of food to their colony. Alone, none of them could have carried it, but by working as a team, they succeeded. Their cooperation ensured the entire colony had enough food for the season."\\
\begin{enumerate}[label=\Alph*.]
    \item They moved the food quickly by working as a team.
    \item They left the food behind because it was too heavy.
    \item They took too long and lost the food.
    \item They didn’t work together and failed.
\end{enumerate}

\vspace{1cm}


\subsection*{Part 2: Short Answer Questions}

4. Describe how the turtle’s steady pace in "The Tortoise and the Hare" influenced the outcome of the race.\\
\vspace{4cm}

5. Retell a story where a character’s decision directly changed the story’s ending. What lesson can be learned from their actions?\\
\vspace{4cm}

6. Explain how teamwork in "The Ants and the Grasshopper" helped the ants prepare for winter.\\
\vspace{4cm}

\subsection*{Part 3: Select All That Apply}

7. Select \textbf{all} ways the hare’s overconfidence affected the race: \\
\begin{enumerate}[label=\Alph*.]
    \item It made the hare take a nap.  
    \item It slowed the hare down.  
    \item It helped the hare stay ahead.  
    \item It allowed the tortoise to win.  
\end{enumerate}

\vspace{1cm}

8. Which actions show the ants' cooperation?\\
\begin{enumerate}[label=\Alph*.]
    \item Sharing the task of carrying food.  
    \item Working together to move heavy objects.  
    \item Leaving food behind because it was too hard to move.  
    \item Ensuring enough food was stored for winter.  
\end{enumerate}

\vspace{1cm}

\subsection*{Part 4: Fill in the Blank}

9. A character’s \underline{\hspace{4cm}} can change the direction of a story’s events.

\vspace{3cm}

10. Understanding how characters respond to \underline{\hspace{4cm}} helps explain the story’s development.

\vspace{3cm}


% 3. How did the ants’ teamwork influence the outcome of their story?\\
% "A group of ants worked together to carry food to their colony. Each ant carried a small piece, but together, they moved a large amount. Their cooperation ensured the entire colony had enough to eat. When a heavy object blocked their path, the ants worked as a team to move it out of the way. Their combined effort showed that teamwork can accomplish tasks that seem impossible for an individual."\\
% \begin{enumerate}[label=\Alph*.]
%     \item It showed teamwork is unnecessary.
%     \item It allowed them to collect food successfully.
%     \item It caused them to lose their food.
%     \item It made the task harder than working alone.
% \end{enumerate}

% \vspace{1cm}


% \subsection*{Part 2: Short Answer Questions}

% 4. How did the boy’s repeated lies in "The Boy Who Cried Wolf" affect the events of the story?\\
% \vspace{4cm}

% 5. How did the hare’s overconfidence change the outcome of "The Tortoise and the Hare"?\\
% \vspace{4cm}

% 6. Describe a story where a character’s actions directly caused an event to happen. How did their actions shape the story?\\
% \vspace{4cm}

% \subsection*{Part 3: Select All That Apply}

% 7. Select \textbf{all} character traits that influenced the turtle’s success in "The Tortoise and the Hare":\\
% \begin{enumerate}[label=\Alph*.]
%     \item Patience  
%     \item Determination  
%     \item Overconfidence  
%     \item Perseverance  
% \end{enumerate}

% \vspace{1cm}

% 8. Which \textbf{events} are key to understanding the ants’ teamwork?\\
% \begin{enumerate}[label=\Alph*.]
%     \item Moving the heavy object together.  
%     \item Collecting food as a group.  
%     \item Fighting with each other.  
%     \item Sharing responsibilities to succeed.  
% \end{enumerate}

% \vspace{1cm}

% \subsection*{Part 4: Fill in the Blank}

% 9. A character’s \underline{\hspace{4cm}} can directly influence the events in a story.

% \vspace{3cm}

% 10. Understanding \underline{\hspace{4cm}} helps explain how the story develops.

% \vspace{3cm}


%??????? 







\end{document}
