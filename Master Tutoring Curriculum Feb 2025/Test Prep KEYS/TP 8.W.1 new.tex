\documentclass[12pt]{article}

\usepackage[a4paper, top=0.8in, bottom=0.7in, left=0.7in, right=0.7in]{geometry}
\usepackage{amsmath}
\usepackage{graphicx}
\usepackage{fancyhdr}
\usepackage{tcolorbox}
\usepackage[defaultfam,tabular,lining]{montserrat} %% Option 'defaultfam'
\usepackage[T1]{fontenc}
\renewcommand*\oldstylenums[1]{{\fontfamily{Montserrat-TOsF}\selectfont #1}}
\renewcommand{\familydefault}{\sfdefault}
\usepackage{enumitem}
\usepackage{setspace}

\setlength{\parindent}{0pt}
\hyphenpenalty=10000
\exhyphenpenalty=10000

\pagestyle{fancy}
\fancyhf{}
\fancyhead[L]{\textbf{8.W.1: Argumentative Writing Practice}}
\fancyhead[R]{\includegraphics[width=1cm]{Round Logo.png}}
\fancyfoot[C]{\footnotesize Study Smart Tutors}

\begin{document}

\subsection*{Argumentative Writing: Exploring Perspectives on Social Media}
\onehalfspacing

\begin{tcolorbox}[colframe=black!40, colback=gray!0, title=Learning Objective]
\textbf{Objective:} Write an argumentative essay that introduces and supports claims with clear reasons and relevant evidence, acknowledging counterclaims.
\end{tcolorbox}

\subsection*{Prompt}

After reading the passages below, write an argumentative essay responding to the question:  
"Does social media have a positive or negative impact on teenagers?" 
Use evidence from the texts to support your position, address counterclaims, and provide a strong conclusion.

\subsection*{Passage 1: The Benefits of Social Media for Teenagers}

Social media provides teenagers with opportunities to connect, learn, and grow. Platforms like Instagram, TikTok, and YouTube allow young people to express \\themselves creatively, sharing art, videos, and personal stories with a global audience. Social media also helps teenagers stay connected with friends and family, especially those who live far away. Beyond social connections, social media can be a powerful learning tool. Educational accounts share tutorials, historical facts, and study tips, while discussion groups allow teenagers to collaborate on school projects. \\Additionally, social media can raise awareness about important social issues. Teens often use platforms to support causes, share information, and organize events, giving them a voice in their communities. However, the key to gaining these benefits is balance and responsible use. When teenagers use social media in moderation and with purpose, it can be a tool for creativity, connection, and learning.
\newpage
\subsection*{Passage 2: The Negative Effects of Social Media on Teenagers}

While social media offers benefits, it can also have negative effects on teenagers’ mental health and well-being. Studies show that excessive time on social media can lead to feelings of loneliness, anxiety, and depression. Comparing themselves to the seemingly perfect lives portrayed online, teenagers may develop low self-esteem or unrealistic expectations. Cyberbullying is another major concern. Social media platforms make it easy for bullies to target others, sometimes anonymously, causing significant emotional harm. In addition to mental health risks, social media can be a major distraction. Many teenagers struggle to focus on schoolwork or spend quality time with family because they are glued to their screens. Sleep deprivation is another issue, as late-night scrolling often interrupts healthy sleep patterns. While social media has its advantages, its potential harms should not be overlooked. Parents and educators must guide teenagers toward healthier habits to minimize these risks.

\subsection*{Passage 3: Striking a Balance with Social Media Use}

Social media is neither entirely good nor entirely bad—it depends on how it is used. Striking a balance is essential for teenagers to gain its benefits while avoiding its downsides. Experts recommend setting time limits for daily social media use to prevent overexposure. For example, spending 30–60 minutes a day on social media allows teenagers to stay connected without becoming overly reliant on their screens. Another strategy is to prioritize meaningful interactions. Following educational \\accounts, participating in positive communities, and avoiding toxic environments can make social media a healthier experience. Schools and parents can play an important role by teaching digital literacy skills, helping teenagers recognize\\ misinformation and handle cyberbullying effectively. Additionally, promoting offline activities—like sports, arts, and spending time in nature—can help teenagers maintain a well-rounded lifestyle. By using social media thoughtfully and responsibly, teenagers can enjoy its advantages while minimizing its drawbacks.
\newpage
\subsection*{Instructions for Students}

\begin{enumerate}
    \item **Choose a side.** Decide whether you believe social media has a positive impact, a negative impact, or that it depends on how it is used.
    \item **Plan your essay.** Organize your ideas and include:
    \begin{itemize}
        \item A clear claim that states your position.
        \item Reasons and evidence from the texts to support your argument.
        \item Acknowledgment and refutation of counterclaims.
        \item A strong conclusion that reinforces your position.
    \end{itemize}
    \item **Write your essay.** Use formal language and logical reasoning to present your argument.
    \item **Revise and edit.** Check your essay for grammar, clarity, and organization.
\end{enumerate}

\subsection*{Scoring Guide}

Your essay will be evaluated on the following criteria:
\begin{enumerate}
    \item \textbf{Content and Ideas}: Strength of argument, use of evidence, and \\acknowledgment of counterclaims.
    \item \textbf{Organization}: Clear introduction, logical transitions, and structured \\paragraphs.
    \item \textbf{Style and Tone}: Formal style, precise language, and strong voice.
    \item \textbf{Conventions}: Proper grammar, punctuation, and spelling.
\end{enumerate}

\end{document}
