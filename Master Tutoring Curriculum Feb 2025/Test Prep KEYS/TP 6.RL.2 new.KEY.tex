\documentclass[12pt]{article}

\usepackage[a4paper, top=0.8in, bottom=0.7in, left=0.7in, right=0.7in]{geometry}
\usepackage{amsmath}
\usepackage{graphicx}
\usepackage{fancyhdr}
\usepackage{tcolorbox}
\usepackage[defaultfam,tabular,lining]{montserrat} %% Option 'defaultfam'
\usepackage[T1]{fontenc}
\renewcommand*\oldstylenums[1]{{\fontfamily{Montserrat-TOsF}\selectfont #1}}
\renewcommand{\familydefault}{\sfdefault}
\usepackage{enumitem}
\usepackage{setspace}

\setlength{\parindent}{0pt}
\hyphenpenalty=10000
\exhyphenpenalty=10000

\pagestyle{fancy}
\fancyhf{}
%\fancyhead[L]{\textbf{6.RL.2: Determining Central Ideas Practice}}
\fancyhead[R]{\includegraphics[width=1cm]{Round Logo.png}}
\fancyfoot[C]{\footnotesize Study Smart Tutors}

\begin{document}

\subsection*{Understanding the Central Idea of Literary Texts}
\onehalfspacing

\begin{tcolorbox}[colframe=black!40, colback=gray!0, title=Learning Objective]
\textbf{Objective:} Determine a central idea of a text and explain how it is conveyed through supporting details.
\end{tcolorbox}


\subsection*{Answer Key}

\textbf{Part 1: Multiple-Choice Questions}
\begin{enumerate}[label=\arabic*.]
    \item B. Jacob learned the importance of persistence.  
    \item A. Baking brought Anna and her grandmother closer together.  
    \item B. The teacher used the event to teach a lesson on kindness and patience.  
\end{enumerate}

\textbf{Part 2: Select All That Apply Questions}
\begin{enumerate}[label=\arabic*.]
    \item A, B, D.  
    \item A, C.  
    \item A, B, D.  
\end{enumerate}

\textbf{Part 3: Short Answer Questions}
\begin{itemize}
    \item (7) Jacob’s persistence is illustrated by his repeated practice and determination. Despite initial difficulties and a near fall, he continued trying and eventually succeeded in climbing the oak tree.  
    \item (8) Anna’s baking experience helped her connect with her family traditions by learning recipes passed down through generations and bonding with her grandmother over stories of the past.  
\end{itemize}

\textbf{Part 4: Fill in the Blank Questions}
\begin{itemize}
    \item (9) message; experiences  
    \item (10) citation  
\end{itemize}

\end{document}

