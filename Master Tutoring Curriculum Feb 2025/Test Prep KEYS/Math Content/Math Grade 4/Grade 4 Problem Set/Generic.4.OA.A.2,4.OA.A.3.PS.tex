% ChatGPT Directions 0 : 
% This is a Tbox Problem set for the following standards 4.OA.A.2, 4.OA.A.3
%--------------------------------------------------
\documentclass[12pt]{article}
\usepackage[a4paper, top=0.8in, bottom=0.7in, left=0.8in, right=0.8in]{geometry}
\usepackage{amsmath}
\usepackage{amsfonts}
\usepackage{latexsym}
\usepackage{graphicx}
\usepackage{fancyhdr}
\usepackage{tcolorbox}
\usepackage{enumitem}
\usepackage{setspace}
\usepackage[defaultfam,tabular,lining]{montserrat} % Font settings for Montserrat

% General Comment: Template for creating problem sets in a structured format with headers, titles, and sections.
% This document uses Montserrat font and consistent styles for exercises, problems, and performance tasks.

% -------------------------------------------------------------------
% ChatGPT Directions:
% 1. Always include a header with standards and topic title: \fancyhead[L]{\textbf{<Standards>: <Topic Title>}}.
% 2. Subsection titles should always start with "Problem Set:" followed by the topic title.
% 3. Use tcolorbox for distinct sections: Learning Objective, Exercises, Problems, Performance Task, and Reflection.
% 4. Style guidelines:
%    - Frame color: black or dark gray (colframe=black!60).
%    - Background color: light gray or white (colback=gray!5 or colback=white).
%    - Title background: slightly darker gray (colbacktitle=black!15).
%    - Font style: Bold for titles (fonttitle=\bfseries\Large).
% -------------------------------------------------------------------

\setlength{\parindent}{0pt}
\pagestyle{fancy}

\setlength{\headheight}{27.11148pt}
\addtolength{\topmargin}{-15.11148pt}

\fancyhf{}
%\fancyhead[L]{\textbf{4.OA.A.2, 4.OA.A.3: Multi-Step Word Problems}}
\fancyhead[R]{\includegraphics[width=0.8cm]{RoundLogo.png}} % Placeholder for logo
\fancyfoot[C]{\footnotesize © Study Smart Tutors}

\sloppy

\title{}
\date{}
\hyphenpenalty=10000
\exhyphenpenalty=10000

\begin{document}

\subsection*{Problem Set: Multi-Step Word Problems Using the Four Operations}
\onehalfspacing

% Learning Objective Box
\begin{tcolorbox}[colframe=black!40, colback=gray!5, 
coltitle=black, colbacktitle=black!20, fonttitle=\bfseries\Large, 
title=Learning Objective, halign title=center, left=5pt, right=5pt, top=5pt, bottom=15pt]
\textbf{Objective:} Solve multi-step word problems using addition, subtraction, multiplication, and division, and represent solutions using equations with variables.
\end{tcolorbox}

% Exercises Box
\begin{tcolorbox}[colframe=black!60, colback=white, 
coltitle=black, colbacktitle=black!15, fonttitle=\bfseries\Large, 
title=Exercises, halign title=center, left=10pt, right=10pt, top=10pt, bottom=30pt]
\begin{enumerate}[itemsep=3em]
    \item Add: \( 357 + 489 \).
    \item Subtract: \( 1,254 - 678 \).
    \item Multiply: \( 42 \times 6 \).
    \item Divide: \( 840 \div 7 \).
    \item A student runs 3 times farther than another. If the second student runs 5 miles, how far does the first student run?
    \item Divide \( 25 \div 4 \). Write a sentence explaining the remainder in the context of sharing apples among friends.
    \item Write the equation and solve: "The total cost of 6 pencils is \$24. Find the cost of one pencil."
    \item Solve: \( 500 - (12 \times 4) \).
\end{enumerate}
\end{tcolorbox}

\vspace{1em}

% Problems Box
\begin{tcolorbox}[colframe=black!60, colback=white, 
coltitle=black, colbacktitle=black!15, fonttitle=\bfseries\Large, 
title=Problems, halign title=center, left=10pt, right=10pt, top=10pt, bottom=80pt]
\begin{enumerate}[start=9, itemsep=5em]
    \item A school raised \$480 for a field trip. The students plan to divide the money equally among 8 buses. How much money will each bus get? Write and solve the equation.
    \item Maria bought 4 shirts for \$25 each and a pair of pants for \$40. How much money did Maria spend in total?
    \item Estimate \( 548 \div 6 \). Then calculate the exact value and explain how the remainder is represented.
    \item A teacher says \( 8 \times 4 = 32 \) and \( 32 \div 4 = 8 \). Why does this relationship hold? Explain using a model or equation.
    \item A farmer has 12 rows of crops, with 15 plants in each row. 20 plants are damaged and need to be removed. Write and solve an equation to find how many healthy plants are left.
    \item A library has 3,000 books. 40\% of them are borrowed by students. How many books are still in the library?
\end{enumerate}
\end{tcolorbox}

\vspace{1em}

% Performance Task Box
\begin{tcolorbox}[colframe=black!60, colback=white, 
coltitle=black, colbacktitle=black!15, fonttitle=\bfseries\Large, 
title=Performance Task: Organizing a Charity Event, halign title=center, left=10pt, right=10pt, top=10pt, bottom=50pt]
Your class is organizing a charity event:
\begin{itemize}
    \item You need to rent chairs and tables. Each chair costs \$2 and each table costs \$15.
    \item The class has \$200 to spend.
    \item You want to rent 20 chairs.
\end{itemize}
\textbf{Task:}
\begin{enumerate}[itemsep=3em]
    \item Write and solve an equation to determine how many tables you can rent with the remaining budget.
    \item Before solving exactly, estimate how many tables you can rent with \$200. Compare your estimate to the exact value.
    \item Create a budget plan showing your calculations.
\end{enumerate}
\end{tcolorbox}

\vspace{1em}

% Reflection Box
\begin{tcolorbox}[colframe=black!60, colback=white, 
coltitle=black, colbacktitle=black!15, fonttitle=\bfseries\Large, 
title=Reflection, halign title=center, left=10pt, right=10pt, top=10pt, bottom=100pt]
Reflect on the strategies you used to solve multi-step problems. How did you decide when to use multiplication, division, addition, or subtraction?  Share any patterns or shortcuts you noticed while solving the problems.
\end{tcolorbox}

\end{document}
