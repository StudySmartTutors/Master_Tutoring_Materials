\documentclass[11pt]{article}
\usepackage[a4paper, top=0.8in, bottom=0.7in, left=0.8in, right=0.8in]{geometry}
\usepackage{amsmath}
\usepackage{amsfonts}
\usepackage{latexsym}
\usepackage{graphicx}
\usepackage{fancyhdr}
\usepackage{tcolorbox}
\usepackage{enumitem}
\usepackage{setspace}
\usepackage[defaultfam,tabular,lining]{montserrat}
\usepackage{xcolor}

% General Comment: Answer Key for problem set with detailed step-by-step solutions in red.
% -------------------------------------------------------------------

\setlength{\parindent}{0pt}
\pagestyle{fancy}

\setlength{\headheight}{27.11148pt}
\addtolength{\topmargin}{-15.11148pt}

\fancyhf{}
%\fancyhead[L]{\textbf{5.NBT.B.7: Adding, Subtracting, Multiplying, and Dividing Decimals - Answer Key}}
\fancyhead[R]{\includegraphics[width=0.8cm]{Round Logo.png}} % Placeholder for logo
\fancyfoot[C]{\footnotesize © Study Smart Tutors}

\sloppy

\title{}
\date{}
\hyphenpenalty=10000
\exhyphenpenalty=10000

\begin{document}

\subsection*{Problem Set: Operations with Decimals - Answer Key}
\onehalfspacing

% Learning Objective Box
\begin{tcolorbox}[colframe=black!40, colback=gray!5, 
coltitle=black, colbacktitle=black!20, fonttitle=\bfseries\Large, 
title=Learning Objective, halign title=center, left=5pt, right=5pt, top=5pt, bottom=15pt]
\textbf{Objective:} Perform addition, subtraction, multiplication, and division of decimals to hundredths. Solve real-world problems using equations with variables.
\end{tcolorbox}

% Exercises Box
\begin{tcolorbox}[colframe=black!60, colback=white, 
coltitle=black, colbacktitle=black!15, fonttitle=\bfseries\Large, 
title=Exercises, halign title=center, left=10pt, right=10pt, top=10pt, bottom=60pt]
\begin{enumerate}[itemsep=3.5em]
    \item Add: \( 12.45 + 8.37 \).\\
    \textcolor{red}{\textbf{Solution:} \( 12.45 + 8.37 = 20.82 \).}

    \item Subtract: \( 25.8 - 13.47 \).\\
    \textcolor{red}{\textbf{Solution:} \( 25.8 - 13.47 = 12.33 \).}

    \item Multiply: \( 3.6 \times 4.2 \).\\
    \textcolor{red}{\textbf{Solution:} \( 3.6 \times 4.2 = 15.12 \).}

    \item Divide: \( 18.75 \div 3.5 \).\\
    \textcolor{red}{\textbf{Solution:} \( 18.75 \div 3.5 = 5.36 \).}

    \item Solve: \( (7.25 \times 3) - 5.4 \).\\
    \textcolor{red}{\textbf{Solution:} \( 7.25 \times 3 = 21.75 \), then \( 21.75 - 5.4 = 16.35 \).}

    \item A bag of rice weighs 2.5 kg. If there are 4 bags, what is the total weight?\\
    \textcolor{red}{\textbf{Solution:} \( 2.5 \times 4 = 10 \) kg.}

    \item Solve for \( x \): \( 5.4x = 27 \).\\
    \textcolor{red}{\textbf{Solution:} \( x = 27 \div 5.4 = 5 \).}

    \item Divide \( 48.6 \div 6 \), then add 12.5 to the quotient.\\
    \textcolor{red}{\textbf{Solution:} \( 48.6 \div 6 = 8.1 \), then \( 8.1 + 12.5 = 20.6 \).}
\end{enumerate}
\end{tcolorbox}

\vspace{1em}

% Problems Box
\begin{tcolorbox}[colframe=black!60, colback=white, 
coltitle=black, colbacktitle=black!15, fonttitle=\bfseries\Large, 
title=Problems, halign title=center, left=10pt, right=10pt, top=10pt, bottom=100pt]
\begin{enumerate}[start=9, itemsep=7em]
    \item A store sells apples for \$2.75 per kg. If you buy 3.2 kg of apples and a \$1.50 bag of oranges, how much do you spend in total?\\
    \textcolor{red}{\textbf{Solution:} \( 2.75 \times 3.2 = 8.8 \), then \( 8.8 + 1.5 = 10.3 \). Total: \$10.30.}

    \item A car travels 15.5 miles on 1 gallon of fuel. If the tank holds 12.4 gallons, how far can the car travel on a full tank?\\
    \textcolor{red}{\textbf{Solution:} \( 15.5 \times 12.4 = 192.2 \) miles.}

    \item A runner jogs \( 3.8 \) miles daily for \( 5 \) days. What is the total distance jogged? Write and solve an equation.\\
    \textcolor{red}{\textbf{Solution:} Equation: \( 3.8 \times 5 = x \). Solve: \( x = 19 \). Total: \( 19 \) miles.}

    \item A chef uses \( 2.8 \) kg of sugar and \( 1.35 \) kg of flour. If the sugar costs \$3.50 per kg and the flour costs \$2.40 per kg, what is the total cost?\\
    \textcolor{red}{\textbf{Solution:} \( 2.8 \times 3.5 = 9.8 \), \( 1.35 \times 2.4 = 3.24 \), then \( 9.8 + 3.24 = 13.04 \). Total: \$13.04.}

    \item A bakery uses \( 6.25 \) cups of flour for 5 loaves of bread. How many cups of flour are used per loaf?\\
    \textcolor{red}{\textbf{Solution:} \( 6.25 \div 5 = 1.25 \). Each loaf uses \( 1.25 \) cups of flour.}
\end{enumerate}
\end{tcolorbox}

\vspace{1em}

% Performance Task Box
\begin{tcolorbox}[colframe=black!60, colback=white, 
coltitle=black, colbacktitle=black!15, fonttitle=\bfseries\Large, 
title=Performance Task: Monitoring Plant Growth - Answer Key, halign title=center, left=10pt, right=10pt, top=10pt, bottom=50pt]
\begin{enumerate}[itemsep=3em]
    \item Calculate the total growth of Plant A after \( 12 \) days.\\
    \textcolor{red}{\textbf{Solution:} \( 1.25 \times 12 = 15 \) cm.}

    \item Calculate the total growth of Plant B after \( 12 \) days.\\
    \textcolor{red}{\textbf{Solution:} \( 0.95 \times 12 = 11.4 \) cm.}

    \item Find the difference in growth between the two plants.\\
    \textcolor{red}{\textbf{Solution:} \( 15 - 11.4 = 3.6 \) cm.}

    \item Predict how many days it will take for Plant A to grow \( 25 \) centimeters. Write an equation to represent your prediction.\\
    \textcolor{red}{\textbf{Solution:} Equation: \( 1.25x = 25 \). Solve: \( x = 25 \div 1.25 = 20 \). It will take \( 20 \) days.}
\end{enumerate}
\end{tcolorbox}

\vspace{1em}

% Reflection Box
\begin{tcolorbox}[colframe=black!60, colback=white, 
coltitle=black, colbacktitle=black!15, fonttitle=\bfseries\Large, 
title=Reflection, halign title=center, left=10pt, right=10pt, top=10pt, bottom=100pt]
What strategies did you use to solve problems with decimals? How did estimating help you verify your solutions? Reflect on the usefulness of decimal calculations in real-world scenarios.
\end{tcolorbox}

\end{document}
