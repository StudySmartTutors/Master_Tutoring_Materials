\documentclass[12pt]{article}

\usepackage[a4paper, top=0.8in, bottom=0.7in, left=0.7in, right=0.7in]{geometry}
\usepackage{amsmath}
\usepackage{graphicx}
\usepackage{fancyhdr}
\usepackage{tcolorbox}
\usepackage{multicol}
\usepackage{pifont} % For checkboxes
\usepackage[defaultfam,tabular,lining]{montserrat} %% Option 'defaultfam'
\usepackage[T1]{fontenc}
\renewcommand*\oldstylenums[1]{{\fontfamily{Montserrat-TOsF}\selectfont #1}}
\renewcommand{\familydefault}{\sfdefault}
\usepackage{enumitem}
\usepackage{setspace}
\usepackage{parcolumns}
\usepackage{tabularx}

\setlength{\parindent}{0pt}
\hyphenpenalty=10000
\exhyphenpenalty=10000

\pagestyle{fancy}
\fancyhf{}
%\fancyhead[L]{\textbf{3.RI.1: Reading Informational Text Practice}}
\fancyhead[R]{\includegraphics[width=1cm]{Round Logo.png}}
\fancyfoot[C]{\footnotesize Study Smart Tutors}

\begin{document}

\subsection*{Understanding and Analyzing Informational Text}
\onehalfspacing

\begin{tcolorbox}[colframe=black!40, colback=gray!0, title=Learning Objective]
\textbf{Objective:} Develop the ability to ask and answer questions about key details in a text using informational reading passages.
\end{tcolorbox}

\subsection*{Part 1: Multiple-Choice Questions}

1. What is the main idea of the following passage? \\
"Polar bears live in the Arctic and are excellent swimmers. They have thick fur and a layer of fat to keep them warm. Polar bears mainly eat seals, but they are also known to scavenge for food when necessary. Their white fur helps them blend into the snowy environment, making it easier to hunt. As the Arctic environment changes, polar bears face new challenges, including melting sea ice and reduced access to their natural prey. Despite these challenges, they remain one of the most iconic species of the Arctic ecosystem."\\
\begin{enumerate}[label=\Alph*.]
    \item Polar bears eat fish.
    \item Polar bears are good swimmers and live in cold areas.
    \item Polar bears are found in zoos.
    \item Polar bears only eat plants.
\end{enumerate}

\vspace{1cm}

2. Which detail supports the main idea of the passage below? \\
"Bees are important for pollination. They carry pollen from one flower to another, helping plants grow fruits and seeds. Bees also produce honey, which is used by humans as a sweetener. Without bees, many plants, including crops, would struggle to grow. In addition to their role in agriculture, bees are vital for maintaining \\biodiversity. By pollinating a wide variety of plants, they help sustain ecosystems. However, bee populations are declining due to habitat loss, pesticides, and climate change, making their conservation more critical than ever."\\
\begin{enumerate}[label=\Alph*.]
    \item Bees live in hives.
    \item Bees help plants by pollinating flowers.
    \item Bees make honey.
    \item Bees can sting.
\end{enumerate}

\vspace{1cm}

3. What question can you ask to better understand this passage?\\
"Rainforests are home to many animals and plants. They are found near the equator where it is warm and wet. Rainforests provide essential resources, including oxygen and medicinal plants. These ecosystems support a wide variety of species, making them one of the most biodiverse areas on Earth. Unfortunately, rainforests are disappearing rapidly due to deforestation and climate change. Protecting these vital ecosystems is necessary to ensure a healthy planet for future generations."\\
\begin{enumerate}[label=\Alph*.]
    \item What is the weather like in rainforests?
    \item How many trees are in a rainforest?
    \item How tall is the tallest tree?
    \item Why do some animals live in the desert?
\end{enumerate}




\subsection*{Part 2: Multi-Select Questions}

4. Select \textbf{all} key details from the passage:  \\
"Elephants are the largest land animals. They have long trunks that they use to eat, drink, and pick up objects. Elephants also have large ears that help keep them cool. They are herbivores, feeding on grass, leaves, and fruits. Elephants live in social groups called herds, which are led by a matriarch. These animals play a crucial role in their ecosystems by spreading seeds and creating clearings in forests, which help other plants and animals thrive."\\
\begin{enumerate}[label=\Alph*.]
    \item Elephants live in water.  
    \item Elephants use their trunks for many tasks.  
    \item Elephants are the largest land animals.  
    \item Elephants have large ears that keep them cool.  
\end{enumerate}

\vspace{1cm}
\newpage
5. Which of the following are questions you could ask about a nonfiction text? (Select \textbf{all} that apply.)  
\begin{enumerate}[label=\Alph*.]
    \item What is the main idea of the text?  
    \item What is the author’s favorite color?  
    \item What are the key details?  
    \item Why is this topic important?  
\end{enumerate}

\vspace{1cm}

\subsection*{Part 3: Short Answer Questions}

6. What is the main idea of the following passage?\\
"Frogs live near water and are great jumpers. They have long legs that help them leap far distances. Frogs eat insects and are important for keeping the insect \\population in balance. Many species of frogs are known for their vibrant colors, which can serve as a warning to predators. Frogs play an essential role in freshwater ecosystems and are considered indicators of environmental health. However, many frog populations are declining due to pollution, habitat destruction, and climate change."\\

\vspace{3cm}

7. Write one question you could ask to learn more about frogs after reading the passage above.

\vspace{3cm}

8. How do frogs’ long legs help them survive?

\vspace{3cm}

\subsection*{Part 4: Fill in the Blank}

9. When we use our own words to explain what happened in a text, we are \\ \underline{\hspace{4cm}} the text.

\vspace{3cm}

10. An \underline{\hspace{4cm}} is a smart guess you make by using clues from what you know or see and combining them with what you already know.

% \vspace{3cm}
% \newpage
% \subsection*{Answer Key}

% \textbf{Part 1: Multiple-Choice Questions}  
% 1. \textbf{B}  
% 2. \textbf{B}  
% 3. \textbf{A}  

% \textbf{Part 2: Multi-Select Questions}  
% 4. \textbf{B, C, D}  
% 5. \textbf{A, C, D}  

% \textbf{Part 3: Short Answer Questions}  
% 6. "Frogs play an essential role in freshwater ecosystems and are indicators of environmental health."  
% 7. Example: "What do frogs eat other than insects?"  
% 8. Frogs’ long legs help them leap far distances, allowing them to escape predators and catch food efficiently.  

% \textbf{Part 4: Fill in the Blank}  
% 9. \textbf{summarizing}  
% 10. \textbf{inference}  

\end{document}

