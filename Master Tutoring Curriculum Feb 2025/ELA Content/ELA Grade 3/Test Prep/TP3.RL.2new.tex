\documentclass[12pt]{article}

\usepackage[a4paper, top=0.8in, bottom=0.7in, left=0.7in, right=0.7in]{geometry}
\usepackage{amsmath}
\usepackage{graphicx}
\usepackage{fancyhdr}
\usepackage{tcolorbox}
\usepackage{multicol}
\usepackage{pifont} % For checkboxes
\usepackage[defaultfam,tabular,lining]{montserrat} %% Option 'defaultfam'
\usepackage[T1]{fontenc}
\renewcommand*\oldstylenums[1]{{\fontfamily{Montserrat-TOsF}\selectfont #1}}
\renewcommand{\familydefault}{\sfdefault}
\usepackage{enumitem}
\usepackage{setspace}
\usepackage{parcolumns}
\usepackage{tabularx}

\setlength{\parindent}{0pt}
\hyphenpenalty=10000
\exhyphenpenalty=10000

\pagestyle{fancy}
\fancyhf{}
%\fancyhead[L]{\textbf{3.RL.2: Recounting Stories Practice}}
\fancyhead[R]{\includegraphics[width=1cm]{Round Logo.png}}
\fancyfoot[C]{\footnotesize Study Smart Tutors}

\begin{document}

\subsection*{Recounting Stories and Determining the Central Message}
\onehalfspacing

\begin{tcolorbox}[colframe=black!40, colback=gray!0, title=Learning Objective]
\textbf{Objective:} Understand how to recount stories and determine their central message, lesson, or moral.
\end{tcolorbox}

\subsection*{Part 1: Multiple-Choice Questions}

1. What is the central message of the following story? \\
"A young lion cub always relied on his mother to hunt for food. One day, the mother lion fell ill and could no longer hunt. The cub realized he needed to learn how to find food himself. After many attempts and failures, he succeeded and became independent. Over time, the cub grew stronger and was able to provide not only for himself but also for his mother. The experience taught the lion cub that challenges can make us stronger and that independence is an important skill to develop."\\
\begin{enumerate}[label=\Alph*.]
    \item Always listen to your parents.
    \item Learning to be independent is important.
    \item Never leave your family.
    \item Being strong is more important than being smart.
\end{enumerate}

\vspace{1cm}

2. What lesson can be learned from this passage?\\
"A turtle and a rabbit decided to race. The rabbit ran quickly but became overconfident and decided to take a nap during the race. The turtle, though slow, kept moving steadily toward the finish line. When the rabbit woke up, he realized the turtle was almost at the finish line. Despite his best efforts, the rabbit could not catch up. The turtle won the race because of his persistence and steady effort. This story reminds us that overconfidence can lead to failure, and consistent effort often leads to success."\\
\begin{enumerate}[label=\Alph*.]
    \item Slow and steady wins the race.
    \item It is important to be fast in life.
    \item Taking naps is helpful during a race.
    \item Always challenge those who are slower.
\end{enumerate}

\vspace{1cm}

3. What does this story teach about teamwork?\\
"A group of ants worked together to carry food to their colony. Each ant carried a small piece of food, but together, they moved a large amount. Their cooperation ensured that the entire colony had enough to eat. When a heavy object blocked their path, the ants worked as a team to move it out of the way. The combined effort of the ants showed how teamwork can accomplish tasks that seem impossible for an individual. The ants’ success highlights the importance of collaboration and sharing responsibilities to achieve common goals."\\
\begin{enumerate}[label=\Alph*.]
    \item Sharing is not necessary.
    \item Cooperation helps achieve big goals.
    \item It is better to work alone.
    \item Always find someone to do the work for you.
\end{enumerate}

\vspace{1cm}


\subsection*{Part 2: Short Answer Questions}

4. Retell the story of "The Boy Who Cried Wolf" in your own words. What is the lesson of the story?\\
\vspace{4cm}

5. How does the central message of "The Tortoise and the Hare" apply to your life?\\
\vspace{4cm}

\newpage
6. Think of a story where a character learns a lesson. Briefly recount the story and explain the lesson learned.\\
\vspace{3cm}

\subsection*{Part 3: Select All That Apply}

7. Select \textbf{all} reasons why stories teach important lessons: \\
\begin{enumerate}[label=\Alph*.]
    \item They entertain readers.  
    \item They provide examples of how to solve problems.  
    \item They help readers understand emotions and challenges.  
    \item They give advice on what to avoid in life.  
\end{enumerate}

\vspace{1cm}

8. Which \textbf{details} are important when recounting a story?\\
\begin{enumerate}[label=\Alph*.]
    \item Including the main characters.  
    \item Describing the setting.  
    \item Explaining every small detail.  
    \item Summarizing the main events.  
\end{enumerate}

\vspace{1cm}
\subsection*{Part 4: Fill in the Blank}

9. Stories often teach a \underline{\hspace{4cm}}, which helps readers understand a deeper meaning or lesson.

\vspace{1cm}

10. When retelling a story, it is important to include the \underline{\hspace{4cm}} characters and events.

% \vspace{3cm}
% % Add the Answer Key Section
% \newpage
% \subsection*{Answer Key}

% \textbf{Part 1: Multiple-Choice Questions}  
% 1. B. Learning to be independent is important.  
% 2. A. Slow and steady wins the race.  
% 3. B. Cooperation helps achieve big goals.  

% \textbf{Part 2: Short Answer Questions}  
% 4. Retelling of "The Boy Who Cried Wolf": A boy repeatedly tricked villagers by shouting that a wolf was attacking his sheep. When a wolf actually came, no one believed him, and the sheep were lost. Lesson: Always tell the truth; lying can lead to a loss of trust.  
% 5. "The Tortoise and the Hare" shows that steady, consistent effort can overcome challenges and achieve success, even when faced with overconfidence or quicker opponents.  
% 6. Example: In "The Lion and the Mouse," the mouse helps the lion by freeing it from a hunter's net. Lesson: Even the smallest individuals can help in big ways, showing the value of kindness and teamwork.  

% \textbf{Part 3: Select All That Apply}  
% 7. A. They entertain readers.  
%    B. They provide examples of how to solve problems.  
%    C. They help readers understand emotions and challenges.  
%    D. They give advice on what to avoid in life.  
% 8. A. Including the main characters.  
%    B. Describing the setting.  
%    D. Summarizing the main events.  

% \textbf{Part 4: Fill in the Blank}  
% 9. Moral  
% 10. Important  
\end{document}
