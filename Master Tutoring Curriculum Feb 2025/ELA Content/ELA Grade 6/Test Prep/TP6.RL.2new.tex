\documentclass[12pt]{article}

\usepackage[a4paper, top=0.8in, bottom=0.7in, left=0.7in, right=0.7in]{geometry}
\usepackage{amsmath}
\usepackage{graphicx}
\usepackage{fancyhdr}
\usepackage{tcolorbox}
\usepackage[defaultfam,tabular,lining]{montserrat} %% Option 'defaultfam'
\usepackage[T1]{fontenc}
\renewcommand*\oldstylenums[1]{{\fontfamily{Montserrat-TOsF}\selectfont #1}}
\renewcommand{\familydefault}{\sfdefault}
\usepackage{enumitem}
\usepackage{setspace}

\setlength{\parindent}{0pt}
\hyphenpenalty=10000
\exhyphenpenalty=10000

\pagestyle{fancy}
\fancyhf{}
%\fancyhead[L]{\textbf{6.RL.2: Determining Central Ideas Practice}}
\fancyhead[R]{\includegraphics[width=1cm]{Round Logo.png}}
\fancyfoot[C]{\footnotesize Study Smart Tutors}

\begin{document}

\subsection*{Understanding the Central Idea of Literary Texts}
\onehalfspacing

\begin{tcolorbox}[colframe=black!40, colback=gray!0, title=Learning Objective]
\textbf{Objective:} Determine a central idea of a text and explain how it is conveyed through supporting details.
\end{tcolorbox}

\subsection*{Part 1: Multiple-Choice Questions}

1. \textbf{What is the central idea of the passage?\\}
"Jacob always wanted to climb the tallest tree in the park, a giant oak that seemed to touch the clouds. Every day after school, he practiced climbing smaller trees, learning how to place his hands and feet securely. At first, his arms ached, and he often slipped, but he didn’t give up. Over time, he became more confident and stronger. Finally, after weeks of preparation, he stood at the base of the oak tree. He took a deep breath and began his ascent, moving slowly but steadily. Halfway up, his foot slipped, and his heart raced, but he clung tightly to a branch and regained his balance. Determined, he continued climbing. When he reached the top, Jacob was breathless, but a sense of accomplishment washed over him. From the top, he could see the entire park and even his school in the distance. The experience taught Jacob that persistence and preparation could help him achieve any goal, no matter how daunting it seemed at first."  
\begin{enumerate}[label=\Alph*.]
    \item Jacob enjoyed spending time at the park.  
    \item Jacob learned the importance of persistence.  
    \item Climbing trees can be dangerous.  
    \item Jacob was scared of falling.  
\end{enumerate}

\vspace{1cm}
\newpage
2. \textbf{What is the central idea of the passage?\\}
"Every Saturday morning, Anna and her grandmother would spend hours in the kitchen. As they kneaded dough and baked cookies, her grandmother shared vivid stories of her childhood. She talked about growing up in a small village, where the entire community gathered to bake bread in a communal oven. Anna listened intently, imagining the scenes her grandmother described. Over time, Anna became skilled at baking, mastering family recipes that had been passed down for generations. One Saturday, her grandmother, too frail to bake, sat nearby and watched as Anna took over the kitchen. Anna carefully followed the recipes, remembering every tip her grandmother had taught her. As the aroma of freshly baked bread filled the kitchen, Anna realized that she was keeping her family’s traditions alive. Baking had become more than a hobby; it was a way to connect with her heritage and preserve her grandmother’s legacy."  
\begin{enumerate}[label=\Alph*.]
    \item Baking brought Anna and her grandmother closer together.  
    \item Baking is a challenging skill to learn.  
    \item Anna wanted to open a bakery.  
    \item Saturdays were Anna’s favorite day of the week.  
\end{enumerate}

\vspace{0.5cm}

3. \textbf{What is the central idea of the passage?\\}
"On a cloudy afternoon, a small bird flew into the classroom through an open window, causing a stir among the students. The bird fluttered around the room, its tiny wings beating frantically as it searched for a way out. Some students shrieked, while others laughed nervously. The teacher, however, remained calm and guided the students to sit quietly at their desks. She explained that the bird was likely confused and scared, much like anyone would be in an unfamiliar place. As they watched, the teacher described how birds navigate using instincts and the position of the sun. She opened several windows wide, and together, the class waited patiently. Eventually, the bird perched on the windowsill, paused for a moment, and flew out into the open air. The students cheered as it disappeared into the sky. The teacher then used the experience to discuss the importance of kindness, patience, and \\respecting animals in their natural habitats. The unexpected visitor left the class with a lesson they wouldn’t soon forget."  
\begin{enumerate}[label=\Alph*.]
    \item Birds often get lost indoors.  
    \item The teacher used the event to teach a lesson on kindness and patience.  
    \item Students enjoy unexpected events in class.  
    \item Animals should not enter classrooms.  
\end{enumerate}

\vspace{1cm}

\subsection*{Part 2: Select All That Apply Questions}

4. Select \textbf{all} supporting details that convey the central idea in the passage about Jacob from question 1:  
\begin{enumerate}[label=\Alph*.]
    \item Jacob practiced climbing smaller trees every day.  
    \item Jacob felt accomplished when he reached the top of the oak tree.  
    \item Jacob gave up after his foot slipped.  
    \item Jacob learned that persistence helped him achieve his goal.  
\end{enumerate}

\vspace{1cm}

5. Which details support the central idea of Anna’s story from question 2?  
\begin{enumerate}[label=\Alph*.]
    \item Anna’s grandmother shared stories while baking.  
    \item Anna’s grandmother grew up in a small village.  
    \item Anna realized baking connected her to her heritage.  
    \item Anna’s recipes became popular at the local bakery.  
\end{enumerate}

\vspace{1cm}

6. Select \textbf{all} supporting details that convey the central idea of the bird story from question 3:  
\begin{enumerate}[label=\Alph*.]
    \item The students worked together to help the bird find its way out.  
    \item The teacher explained how birds navigate using instincts.  
    \item The students were scared of the bird.  
    \item The teacher used the situation to teach a lesson in kindness and patience.  
\end{enumerate}

\vspace{1cm}
\newpage
\subsection*{Part 3: Short Answer Questions}

7. How does the story about Jacob from question 1 illustrate the importance of persistence? Use specific examples from the text.  
\vspace{4cm}

8. In what ways did Anna’s experience baking with her grandmother help her connect with her family’s traditions in the passage from question 2? Use evidence from the text.  
\vspace{4cm}

\subsection*{Part 4: Fill in the Blank Questions}
\vspace{1cm}
9. A theme is a general \underline{\hspace{4cm}} , people, or \underline{\hspace{4cm}} , not a statement about the specific text.
\vspace{2cm}

10. When using a quote for evidence, you should include an in-line \underline{\hspace{4cm}} .
\vspace{2cm}
% \newpage
% \subsection*{Answer Key}

% \textbf{Part 1: Multiple-Choice Questions}
% \begin{enumerate}[label=\arabic*.]
%     \item B. Jacob learned the importance of persistence.  
%     \item A. Baking brought Anna and her grandmother closer together.  
%     \item B. The teacher used the event to teach a lesson on kindness and patience.  
% \end{enumerate}

% \textbf{Part 2: Select All That Apply Questions}
% \begin{enumerate}[label=\arabic*.]
%     \item A, B, D.  
%     \item A, C.  
%     \item A, B, D.  
% \end{enumerate}

% \textbf{Part 3: Short Answer Questions}
% \begin{itemize}
%     \item (7) Jacob’s persistence is illustrated by his repeated practice and determination. Despite initial difficulties and a near fall, he continued trying and eventually succeeded in climbing the oak tree.  
%     \item (8) Anna’s baking experience helped her connect with her family traditions by learning recipes passed down through generations and bonding with her grandmother over stories of the past.  
% \end{itemize}

% \textbf{Part 4: Fill in the Blank Questions}
% \begin{itemize}
%     \item (9) message; experiences  
%     \item (10) citation  
% \end{itemize}

\end{document}

