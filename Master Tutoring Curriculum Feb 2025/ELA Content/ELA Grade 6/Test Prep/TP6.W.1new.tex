\documentclass[12pt]{article}

\usepackage[a4paper, top=0.8in, bottom=0.7in, left=0.7in, right=0.7in]{geometry}
\usepackage{amsmath}
\usepackage{graphicx}
\usepackage{fancyhdr}
\usepackage{tcolorbox}
\usepackage[defaultfam,tabular,lining]{montserrat} %% Option 'defaultfam'
\usepackage[T1]{fontenc}
\renewcommand*\oldstylenums[1]{{\fontfamily{Montserrat-TOsF}\selectfont #1}}
\renewcommand{\familydefault}{\sfdefault}
\usepackage{enumitem}
\usepackage{setspace}

\setlength{\parindent}{0pt}
\hyphenpenalty=10000
\exhyphenpenalty=10000

\pagestyle{fancy}
\fancyhf{}
%\fancyhead[L]{\textbf{6.W.1: Argumentative Writing Practice}}
\fancyhead[R]{\includegraphics[width=1cm]{Round Logo.png}}
\fancyfoot[C]{\footnotesize Study Smart Tutors}

\begin{document}

\subsection*{Argumentative Writing: Exploring the Use of Technology in Schools}
\onehalfspacing

\begin{tcolorbox}[colframe=black!40, colback=gray!0, title=Learning Objective]
\textbf{Objective:} Write an argumentative essay that introduces and supports claims with clear reasons and relevant evidence, acknowledging counterclaims.
\end{tcolorbox}

\subsection*{Prompt}

After reading the passages below, write an argumentative essay responding to the question:  
"Should technology, such as laptops and tablets, be used more in \\classrooms?"  
Use evidence from the texts to support your position, address \\counterclaims, and provide a strong conclusion.

\subsection*{Passage 1: The Benefits of Technology in Education}

Technology in classrooms offers many advantages for both students and teachers. Laptops and tablets give students access to a world of information instantly, allowing them to research topics and learn at their own pace. Interactive apps and videos can make learning more engaging and fun, especially for subjects like math and science. Teachers benefit as well, using technology to track student progress, grade assignments, and create lessons that suit different learning styles. Technology also prepares students for the future by teaching skills like typing, using spreadsheets, and managing digital tools. While some worry that technology can be distracting, setting clear rules and using education-specific apps can help ensure it is used \\responsibly. When balanced with traditional methods, technology can improve \\learning and make education more exciting for students.

\subsection*{Passage 2: The Drawbacks of Technology in Classrooms}

While technology has benefits, it also comes with drawbacks. Many teachers report that students are easily distracted by games, videos, or social media when using laptops or tablets. Additionally, some students rely too much on technology, losing important skills like handwriting and mental math. Not all families can afford the latest devices, which can create inequality in the classroom. Another concern is that screen time can lead to eye strain, headaches, and less physical activity. Teachers may also need extra training to use new tools effectively, which takes time and money. While technology has potential, these challenges show that schools should be cautious about relying too much on it for learning.

\subsection*{Passage 3: Finding Balance with Technology Use}

Many experts believe that the key to using technology in schools is balance. \\Technology should not replace traditional learning methods, such as reading books, writing by hand, and participating in hands-on projects. Instead, it can be used as a helpful tool alongside these activities. For example, students could use tablets for research but complete essays on paper. Teachers can integrate technology into lessons but still emphasize face-to-face discussions and teamwork. Schools can also set limits on screen time to prevent overuse and encourage breaks for physical activity. By balancing technology with traditional methods, classrooms can give students the benefits of digital tools while avoiding distractions and other problems. This balanced approach can help students succeed in both the classroom and the real world.

\subsection*{Instructions for Students}

\begin{enumerate}
    \item **Choose a side.** Decide whether you think schools should use more \\technology, less technology, or find a balanced approach.
    \item **Plan your essay.** Organize your ideas and include:
    \begin{itemize}
        \item A clear claim that states your position.
        \item Reasons and evidence from the texts to support your argument.
        \item Acknowledgment and refutation of counterclaims.
        \item A strong conclusion that reinforces your position.
    \end{itemize}
    \item **Write your essay.** Use formal language and logical reasoning to present your \\argument.
    \item **Revise and edit.** Check your essay for grammar, clarity, and organization.
\end{enumerate}

\subsection*{Scoring Guide}

Your essay will be evaluated on the following criteria:
\begin{enumerate}
    \item \textbf{Content and Ideas}: Strength of argument, use of evidence, and acknowledgment of counterclaims.
    \item \textbf{Organization}: Clear introduction, logical transitions, and structured paragraphs.
    \item \textbf{Style and Tone}: Formal style, precise language, and strong voice.
    \item \textbf{Conventions}: Proper grammar, punctuation, and spelling.
\end{enumerate}

\end{document}
