\documentclass[12pt]{article}
\usepackage[a4paper, top=0.8in, bottom=0.7in, left=0.7in, right=0.7in]{geometry}
\usepackage{amsmath}
\usepackage{graphicx}
\usepackage{fancyhdr}
\usepackage{tcolorbox}
\usepackage{multicol}
\usepackage{pifont} % For checkboxes
\usepackage[defaultfam,tabular,lining]{montserrat} %% Option 'defaultfam'
\usepackage[T1]{fontenc}
\renewcommand*\oldstylenums[1]{{\fontfamily{Montserrat-TOsF}\selectfont #1}}
\renewcommand{\familydefault}{\sfdefault}
\usepackage{enumitem}
\usepackage{setspace}
\usepackage{parcolumns}
\usepackage{tabularx}

\setlength{\parindent}{0pt}

\hyphenpenalty=10000
\exhyphenpenalty=10000

\pagestyle{fancy}
\fancyhf{}
%\fancyhead[L]{\textbf{6.RI.1, 6.RI.2: Citing Evidence and Determining Central Idea}}
\fancyhead[R]{\includegraphics[width=1cm]{Round Logo.png}}
\fancyfoot[C]{\footnotesize Study Smart Tutors}

\begin{document}

\onehalfspacing

% Passage
\subsection*{The Impact of Plastic Pollution on Oceans}
\begin{tcolorbox}[colframe=black!40, colback=gray!5]
\begin{spacing}{1.15}
    Plastic pollution is a major environmental problem affecting oceans worldwide. Every year, millions of tons of plastic waste end up in the ocean, harming marine life and ecosystems. Marine animals such as sea turtles, dolphins, and fish often mistake plastic for food, leading to injury or death. In addition, the plastic waste breaks down into smaller particles called microplastics, which are ingested by even the smallest ocean creatures.

    The impact of plastic pollution goes beyond harming individual animals. It affects entire ecosystems, disrupting food chains and threatening the balance of ocean environments. When plastic debris accumulates on beaches or in the water, it not only damages the habitat of many species but also harms the tourism industry and local economies that rely on clean beaches and healthy oceans.

    Efforts to reduce plastic pollution include banning single-use plastics, organizing beach cleanups, and encouraging recycling. Public education and awareness campaigns have also helped to spread the message that individuals can play a role in reducing plastic waste by making more environmentally conscious decisions, such as using reusable bags and bottles.

    In conclusion, plastic pollution is a serious problem that requires immediate action. By making small changes in our daily lives, we can all contribute to reducing plastic waste and protecting ocean ecosystems for future generations.
\end{spacing}
\end{tcolorbox}

% Worksheet Questions
\subsection*{Questions}
\begin{enumerate}

    % Multiple Choice Questions
    \item What is the central idea of the passage about plastic pollution?
    \begin{enumerate}[label=\Alph*.]
        \item Plastic pollution causes harm to ocean ecosystems and marine life.
        \item Plastic pollution is not a major environmental issue.
        \item Plastic pollution only affects beaches.
        \item Plastic pollution only harms the tourism industry.
    \end{enumerate}

    \vspace{0.5cm}

    \item Which of the following is a key detail that supports the central idea of the passage?
    \begin{enumerate}[label=\Alph*.]
        \item Plastic is biodegradable and harmless to animals.
        \item Marine animals mistake plastic for food, which harms them.
        \item The tourism industry benefits from plastic pollution.
        \item Microplastics do not affect marine life.
    \end{enumerate}

    \vspace{0.5cm}

    \item How does plastic pollution affect marine life, according to the passage?
    \begin{enumerate}[label=\Alph*.]
        \item Marine animals use plastic to build nests.
        \item Marine animals mistake plastic for food and often die from it.
        \item Plastic helps marine life grow stronger.
        \item Plastic does not affect marine life.
    \end{enumerate}

    \vspace{0.5cm}

    \item Which of the following is NOT mentioned as a consequence of plastic pollution in the passage?
    \begin{enumerate}[label=\Alph*.]
        \item Harm to marine animals
        \item Disruption of food chains in ocean ecosystems
        \item Damage to beaches and local economies
        \item Plastic helping to clean up ocean waste
    \end{enumerate}

    \vspace{0.5cm}

    \item How do microplastics affect marine creatures?
    \begin{enumerate}[label=\Alph*.]
        \item Microplastics make marine creatures stronger.
        \item Microplastics are ingested by small ocean creatures.
        \item Microplastics provide food for ocean creatures.
        \item Microplastics are harmless and float on the surface.
    \end{enumerate}

    \vspace{0.5cm}

    \item What are some ways people are trying to reduce plastic pollution?
    \begin{enumerate}[label=\Alph*.]
        \item Banning single-use plastics and organizing beach cleanups.
        \item Throwing more plastic waste into the ocean.
        \item Encouraging the use of plastic bags only.
        \item Educating the public to use more disposable plastics.
    \end{enumerate}

    \vspace{0.5cm}

    \item What role does public education play in reducing plastic pollution?
    \begin{enumerate}[label=\Alph*.]
        \item It teaches people to use more plastic.
        \item It encourages people to make environmentally conscious decisions.
        \item It promotes the use of single-use plastics.
        \item It focuses only on reducing food waste.
    \end{enumerate}

    \vspace{0.5cm}

    \item Which of the following is a solution to plastic pollution mentioned in the passage?
    \begin{enumerate}[label=\Alph*.]
        \item Using plastic bottles to clean the ocean
        \item Banning single-use plastics
        \item Encouraging the use of plastic waste
        \item Reducing the amount of seaweed in the ocean
    \end{enumerate}

    \vspace{0.5cm}

    \item What impact does plastic pollution have on tourism, according to the passage?
    \begin{enumerate}[label=\Alph*.]
        \item It helps the tourism industry grow.
        \item It has no impact on tourism.
        \item It damages beaches and local economies.
        \item It creates new tourism opportunities in polluted areas.
    \end{enumerate}

    \vspace{0.5cm}

    \item What is the main purpose of the passage?
    \begin{enumerate}[label=\Alph*.]
        \item To explain how plastic helps ocean ecosystems
        \item To inform readers about the effects of plastic pollution and how to reduce it
        \item To encourage people to use more plastic products
        \item To describe how plastic is used to clean the ocean
    \end{enumerate}

    \vspace{0.5cm}

    \item According to the passage, why is plastic pollution a problem for marine life?
    \begin{enumerate}[label=\Alph*.]
        \item It increases the population of marine animals.
        \item It makes the ocean cleaner for marine life.
        \item Marine animals mistake plastic for food, causing harm.
        \item It helps marine life grow stronger.
    \end{enumerate}

    \vspace{0.5cm}

    \item What is the role of microplastics in the ocean ecosystem?
    \begin{enumerate}[label=\Alph*.]
        \item Microplastics improve the ocean environment.
        \item Microplastics provide food for marine life.
        \item Microplastics are harmful and consumed by small marine creatures.
        \item Microplastics help clean the ocean.
    \end{enumerate}

    \vspace{0.5cm}

    \item Which of the following statements is supported by the passage?
    \begin{enumerate}[label=\Alph*.]
        \item Plastic pollution is easily solved by banning plastic.
        \item Plastic pollution has no effect on marine ecosystems.
        \item Reducing plastic waste requires collective efforts from individuals and communities.
        \item Plastic pollution only affects tropical oceans.
    \end{enumerate}

    \vspace{0.5cm}

    \item Which of the following is mentioned as a harmful effect of plastic on the ocean?
    \begin{enumerate}[label=\Alph*.]
        \item Plastic provides food for sea animals.
        \item Plastic helps regulate the temperature of the ocean.
        \item Plastic disrupts food chains and damages ecosystems.
        \item Plastic promotes ocean biodiversity.
    \end{enumerate}

    \vspace{0.5cm}

    \item What happens when plastic debris accumulates on beaches?
    \begin{enumerate}[label=\Alph*.]
        \item It improves the environment.
        \item It damages animal habitats and harms local economies.
        \item It helps marine creatures build homes.
        \item It makes the beach more attractive to tourists.
    \end{enumerate}

    \vspace{0.5cm}

    \item Which action can help reduce plastic pollution, as suggested in the passage?
    \begin{enumerate}[label=\Alph*.]
        \item Continuing to use single-use plastic products
        \item Using reusable bags and bottles
        \item Disposing of plastic waste in the ocean
        \item Ignoring recycling programs
    \end{enumerate}

    \vspace{0.5cm}

    \item What is the tone of the passage about plastic pollution?
    \begin{enumerate}[label=\Alph*.]
        \item Optimistic and light-hearted
        \item Informative and concerned
        \item Angry and critical
        \item Indifferent and neutral
    \end{enumerate}

\end{enumerate}

\end{document}
