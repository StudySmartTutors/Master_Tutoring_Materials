\documentclass[12pt]{article}

\usepackage[a4paper, top=0.8in, bottom=0.7in, left=0.7in, right=0.7in]{geometry}
\usepackage{amsmath}
\usepackage{graphicx}
\usepackage{fancyhdr}
\usepackage{tcolorbox}
\usepackage[defaultfam,tabular,lining]{montserrat} %% Option 'defaultfam'
\usepackage[T1]{fontenc}
\renewcommand*\oldstylenums[1]{{\fontfamily{Montserrat-TOsF}\selectfont #1}}
\renewcommand{\familydefault}{\sfdefault}
\usepackage{enumitem}
\usepackage{setspace}

\setlength{\parindent}{0pt}
\hyphenpenalty=10000
\exhyphenpenalty=10000

\pagestyle{fancy}
\fancyhf{}
\fancyhead[L]{\textbf{7.L.1c: Misplaced and Dangling Modifiers Practice}}
\fancyhead[R]{\includegraphics[width=1cm]{Round Logo.png}}
\fancyfoot[C]{\footnotesize Study Smart Tutors}

\begin{document}

\subsection*{Recognizing and Correcting Misplaced and Dangling Modifiers}
\onehalfspacing

\begin{tcolorbox}[colframe=black!40, colback=gray!0, title=Learning Objective]
\textbf{Objective:} Recognize and correct misplaced and dangling modifiers in sentences to ensure clarity and proper syntax.
\end{tcolorbox}

\subsection*{Part 1: Multiple-Choice Questions}

1. Which sentence contains a misplaced modifier?  
\begin{enumerate}[label=\Alph*.]
    \item The chef served a dish that was cooked to perfection.  
    \item Running through the park, the dog barked at everyone.  
    \item The woman walked her dog wearing a red jacket.  
    \item The book on the table belongs to Jake.  
\end{enumerate}

\vspace{1cm}

2. How can the dangling modifier in this sentence be corrected? \\  
"While walking to school, the backpack slipped off my shoulder."  
\begin{enumerate}[label=\Alph*.]
    \item The backpack slipped off my shoulder while walking to school.  
    \item While walking to school, I felt the backpack slip off my shoulder.  
    \item The backpack slipped while I walked to school.  
    \item While walking, the backpack was heavy on my shoulder.  
\end{enumerate}

\vspace{1cm}

3. Which revision corrects the misplaced modifier? \\  
"She read the article about climate change in a magazine."  
\begin{enumerate}[label=\Alph*.]
    \item She read the article in a magazine about climate change.  
    \item In a magazine, she read the article about climate change.  
    \item About climate change, she read the article in a magazine.  
    \item She read the article about climate change that was in a magazine.  
\end{enumerate}

\vspace{1cm}

\subsection*{Part 2: Select All That Apply Questions}

4. Select \textbf{all} sentences that contain a misplaced modifier:  
\begin{enumerate}[label=\Alph*.]
    \item The teacher handed out assignments to the students written in cursive.  
    \item Exhausted from the hike, the tent was set up quickly.  
    \item The bird flew over the lake gliding gracefully.  
    \item The student studying for the test scored well.  
\end{enumerate}

\vspace{1cm}

5. Which of the following sentences are corrected for clarity?  
\begin{enumerate}[label=\Alph*.]
    \item The painting hung on the wall was admired by everyone in the room.  
    \item Admired by everyone, the painting hung on the wall.  
    \item The wall admired everyone with a painting hanging on it.  
    \item Everyone admired the painting hung on the wall.  
\end{enumerate}

\vspace{1cm}

6. Select \textbf{all} ways to correct the dangling modifier: \\  
"Having finished the meal, the dishes were left on the table."  
\begin{enumerate}[label=\Alph*.]
    \item The dishes were left on the table after we finished the meal.  
    \item Having finished the meal, we left the dishes on the table.  
    \item After the meal, the dishes were left on the table.  
    \item We finished the meal, leaving the dishes on the table.  
\end{enumerate}

\vspace{1cm}
\newpage
\subsection*{Part 3: Short Answer Questions}

7. Rewrite the following sentence to correct the misplaced modifier: \\  
"The musician played a song for the audience with a guitar."  
\vspace{3cm}

8. Why is it important to correct misplaced and dangling modifiers in writing? Use examples to explain.  
\vspace{3cm}

\subsection*{Part 4: Fill in the Blank Questions}
\vspace{1cm}
9. A misplaced modifier is a word, phrase, or clause that is not \underline{\hspace{4cm}} to the word it modifies.  
\vspace{2cm}

10. A dangling modifier occurs when the word being modified is \underline{\hspace{4cm}} or missing from the sentence.  
\vspace{2cm}
\newpage
\subsection*{Answer Key}

\textbf{Part 1: Multiple-Choice Questions}

1. \textbf{B} – Running through the park, the dog barked at everyone. (This is a misplaced modifier because "running through the park" seems to describe the dog, but it should describe the person running through the park.)

2. \textbf{B} – While walking to school, I felt the backpack slip off my shoulder. (This corrects the dangling modifier by clearly identifying the subject performing the action.)

3. \textbf{A} – She read the article in a magazine about climate change. (This corrects the misplaced modifier by placing "in a magazine" next to the noun it modifies, "article.")

\textbf{Part 2: Select All That Apply Questions}

4. \textbf{A, B} – The teacher handed out assignments to the students written in cursive. (Misplaced modifier: "written in cursive" should describe the assignments, not the students.)  
Exhausted from the hike, the tent was set up quickly. (Misplaced modifier: "Exhausted from the hike" should describe the person setting up the tent, not the tent.)

5. \textbf{B, D} – Admired by everyone, the painting hung on the wall. (Corrected for clarity by placing the modifier next to what it describes.)  
Everyone admired the painting hung on the wall. (Corrected for clarity by simplifying the sentence structure.)

6. \textbf{A, B, D} – The dishes were left on the table after we finished the meal. (Corrected: the subject performing the action is clarified.)  
Having finished the meal, we left the dishes on the table. (Corrected: the subject performing the action is clarified.)  
We finished the meal, leaving the dishes on the table. (Corrected: the subject performing the action is clarified.)

\textbf{Part 4: Fill in the Blank Questions}

9. Attached  
10. Unclear  

\end{document}

