\documentclass[12pt]{article}

\usepackage[a4paper, top=0.8in, bottom=0.7in, left=0.7in, right=0.7in]{geometry}
\usepackage{amsmath}
\usepackage{graphicx}
\usepackage{fancyhdr}
\usepackage{tcolorbox}
\usepackage[defaultfam,tabular,lining]{montserrat} %% Option 'defaultfam'
\usepackage[T1]{fontenc}
\renewcommand*\oldstylenums[1]{{\fontfamily{Montserrat-TOsF}\selectfont #1}}
\renewcommand{\familydefault}{\sfdefault}
\usepackage{enumitem}
\usepackage{setspace}

\setlength{\parindent}{0pt}
\hyphenpenalty=10000
\exhyphenpenalty=10000

\pagestyle{fancy}
\fancyhf{}
\fancyhead[L]{\textbf{7.RL.2: Determining Themes and Central Ideas Practice}}
\fancyhead[R]{\includegraphics[width=1cm]{Round Logo.png}}
\fancyfoot[C]{\footnotesize Study Smart Tutors}

\begin{document}

\subsection*{Understanding Themes and Central Ideas}
\onehalfspacing

\begin{tcolorbox}[colframe=black!40, colback=gray!0, title=Learning Objective]
\textbf{Objective:} Determine the theme or central idea of a text and analyze its development over the course of the text.
\end{tcolorbox}

\subsection*{Part 1: Multiple-Choice Questions}

1. \textbf{What is the central idea of the passage?\\}
"Emma had always been a talented swimmer, winning local competitions effortlessly. However, when she advanced to a regional championship, she found herself \\surrounded by competitors just as skilled as she was. In the first round, she placed fourth—not enough to qualify for the finals. Disappointed but determined, Emma watched the replay of her race, noting her mistakes and areas for improvement. Over the next week, she trained tirelessly, waking up early to perfect her technique and pushing herself harder than ever before. By the day of the next race, Emma felt ready. As she dove into the water, she kept her focus, remembering everything she had practiced. She touched the wall just milliseconds ahead of her opponents, securing her place in the finals. Although she didn’t win the championship, she felt a greater sense of pride knowing that her determination and resilience had pushed her to improve. Emma realized that growth mattered more than victory."  
\begin{enumerate}[label=\Alph*.]
    \item Winning is the ultimate goal.  
    \item Growth and improvement are more important than victory.  
    \item Watching replays is the key to success.  
    \item Competing against skilled opponents is discouraging.  
\end{enumerate}

\vspace{1cm}
\newpage
2. \textbf{What theme is revealed in this passage?\\}
"In a small town, a terrible storm caused the river to overflow, flooding homes and destroying crops. The townspeople, initially overwhelmed by the devastation, decided to come together to rebuild. Families opened their homes to those who had lost everything, and volunteers worked tirelessly to repair homes and restore farmland. A local carpenter, Mr. Hughes, organized a fundraiser to provide tools and supplies for rebuilding efforts. People donated what they could, even if it was just a few dollars. Over the next several months, the town transformed. Neighbors who had barely spoken before became close friends, bonded by their shared experiences. By the time the town was restored, the community felt stronger than ever. They realized that in times of hardship, unity and kindness could overcome even the greatest challenges."  
\begin{enumerate}[label=\Alph*.]
    \item Natural disasters bring people together.  
    \item Storms are unavoidable.  
    \item Helping others leads to strong communities.  
    \item Hardship creates lasting divisions.  
\end{enumerate}

\vspace{1cm}

3. \textbf{What central idea emerges in the story?\\}
"Alex always loved tinkering with gadgets in his father’s workshop. One day, he decided to build a robot to help with household chores. His first attempt was a disaster—the robot bumped into walls and spilled drinks instead of cleaning them up. Frustrated but determined, Alex studied books on robotics and watched online tutorials. Each evening, after finishing his homework, he worked on improving the robot. Weeks turned into months, but Alex never gave up. Eventually, the robot worked perfectly, sweeping floors and even folding laundry. When Alex demonstrated it at the school science fair, the judges were amazed. His robot won first prize, but Alex was most proud of the journey it took to get there. He realized that perseverance and learning from failure were as valuable as the finished product."  
\begin{enumerate}[label=\Alph*.]
    \item Robots are hard to build.  
    \item Perseverance and learning from failure lead to success.  
    \item Winning a science fair is the ultimate achievement.  
    \item Hard work is not always worth the effort.  
\end{enumerate}

\vspace{1cm}

\subsection*{Part 2: Select All That Apply Questions}

4. Select \textbf{all} details that support the central idea in Emma’s story from question 1:  
\begin{enumerate}[label=\Alph*.]
    \item Emma studied her mistakes and improved her technique.  
    \item Emma gave up after her first loss.  
    \item Emma realized growth was more important than winning.  
    \item Emma trained harder for her next race.  
\end{enumerate}

\vspace{1cm}

5. Which details reveal the theme in the story from question 2 about the town recovering from a flood?  
\begin{enumerate}[label=\Alph*.]
    \item Families opened their homes to displaced neighbors.  
    \item Volunteers worked tirelessly to rebuild.  
    \item Mr. Hughes organized a fundraiser for supplies.  
    \item The townspeople ignored the flood and moved away.  
\end{enumerate}

\vspace{1cm}

6. Select \textbf{all} details that convey the central idea in Alex’s story from question 3:  
\begin{enumerate}[label=\Alph*.]
    \item Alex never gave up despite early failures.  
    \item Alex built a perfect robot on his first try.  
    \item Alex studied robotics to improve his design.  
    \item Alex learned that perseverance was valuable.  
\end{enumerate}

\vspace{1cm}
\newpage

\subsection*{Part 3: Short Answer Questions}

7. How does the story of Emma demonstrate the value of growth over victory? Use evidence in the passage from question 1 to support your answer.  
\vspace{4cm}

8. Based on the story about Alex, explain how failure contributed to his ultimate success. Provide textual evidence in your response from the passage from question 3.  
\vspace{4cm}

\subsection*{Part 4: Fill in the Blank Questions}

9. The theme of a story is a general statment about life, people, or society and should not be about a specific \underline{\hspace{4cm}} .
\vspace{2cm}

10. A summary of a fictional text should be\underline{\hspace{4cm}} and not biased.
\vspace{2cm}
\newpage
\section*{Answer Key}

\subsection*{Part 1: Multiple-Choice Questions}

1. \textbf{What is the central idea of the passage?}  
\textbf{Answer:} B. Growth and improvement are more important than victory.  
\textbf{Explanation:} The passage emphasizes Emma’s realization that growth mattered more than winning, as she was proud of her improvement and determination.

\vspace{1cm}
2. \textbf{What theme is revealed in this passage?}  
\textbf{Answer:} A. Natural disasters bring people together.  
\textbf{Explanation:} The passage shows how the townspeople came together to rebuild after the storm, demonstrating unity in the face of adversity.

\vspace{1cm}
3. \textbf{What central idea emerges in the story?}  
\textbf{Answer:} B. Perseverance and learning from failure lead to success.  
\textbf{Explanation:} The central idea is that Alex’s persistence and ability to learn from his mistakes allowed him to eventually succeed with his robot.

\subsection*{Part 2: Select All That Apply Questions}

4. \textbf{Select all details that support the central idea in Emma’s story from question 1:}  
\textbf{Answer:} A. Emma studied her mistakes and improved her technique. \\
C. Emma realized growth was more important than winning. \\
D. Emma trained harder for her next race.  
\textbf{Explanation:} These details demonstrate Emma’s dedication to improvement and her realization that personal growth mattered more than victory.

\vspace{1cm}
5. \textbf{Which details reveal the theme in the story from question 2 about the town recovering from a flood?}  
\textbf{Answer:} A. Families opened their homes to displaced neighbors. \\
B. Volunteers worked tirelessly to rebuild. \\
C. Mr. Hughes organized a fundraiser for supplies.  
\textbf{Explanation:} These actions highlight the theme of unity and collaboration in times of hardship.

\vspace{1cm}
6. \textbf{Select all details that convey the central idea in Alex’s story from question 3:}  
\textbf{Answer:} A. Alex never gave up despite early failures. \\
C. Alex studied robotics to improve his design. \\
D. Alex learned that perseverance was valuable.  
\textbf{Explanation:} These details show how Alex's perseverance, learning from mistakes, and continual effort led to success.

\subsection*{Part 3: Short Answer Questions}

7. \textbf{How does the story of Emma demonstrate the value of growth over victory? Use evidence in the passage from question 1 to support your answer.}  
\textbf{Answer:} Emma demonstrates the value of growth over victory through her determination to improve after placing fourth. She studied her mistakes, trained harder, and realized that personal growth was more meaningful than winning. She felt pride in her progress, not just in her race results.

\vspace{1cm}
8. \textbf{Based on the story about Alex, explain how failure contributed to his ultimate success. Provide textual evidence in your response from the passage from question 3.}  
\textbf{Answer:} Alex’s failures were crucial in helping him improve his design. After his robot’s initial failure, he studied robotics and worked tirelessly to fix it. His perseverance and ability to learn from his mistakes ultimately led him to success, as his robot was perfect by the time of the science fair.

\subsection*{Part 4: Fill in the Blank Questions}

9. The theme of a story is a general statement about life, people, or society and should not be about a specific \underline{event}.  
\textbf{Answer:} event.

10. A summary of a fictional text should be \underline{objective} and not biased.  
\textbf{Answer:} objective.
\end{document}
