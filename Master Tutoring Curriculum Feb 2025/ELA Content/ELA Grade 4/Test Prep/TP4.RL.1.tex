\documentclass[12pt]{article}

\usepackage[a4paper, top=0.8in, bottom=0.7in, left=0.7in, right=0.7in]{geometry}
\usepackage{amsmath}
\usepackage{graphicx}
\usepackage{fancyhdr}
\usepackage{tcolorbox}
\usepackage{multicol}
\usepackage{pifont} % For checkboxes
\usepackage[defaultfam,tabular,lining]{montserrat} %% Option 'defaultfam'
\usepackage[T1]{fontenc}
\renewcommand*\oldstylenums[1]{{\fontfamily{Montserrat-TOsF}\selectfont #1}}
\renewcommand{\familydefault}{\sfdefault}
\usepackage{enumitem}
\usepackage{setspace}
\usepackage{parcolumns}
\usepackage{tabularx}

\setlength{\parindent}{0pt}
\hyphenpenalty=10000
\exhyphenpenalty=10000

\pagestyle{fancy}
\fancyhf{}
%\fancyhead[L]{\textbf{4.RL.1: Making Inferences and Citing Evidence Practice}}
\fancyhead[R]{\includegraphics[width=1cm]{Round Logo.png}}
\fancyfoot[C]{\footnotesize Study Smart Tutors}

\begin{document}

\subsection*{Making Inferences and Citing Evidence}
\onehalfspacing

\begin{tcolorbox}[colframe=black!40, colback=gray!0, title=Learning Objective]
\textbf{Objective:} Refer explicitly to the text when explaining what it says and when making inferences.
\end{tcolorbox}

\subsection*{Part 1: Multiple-Choice Questions}

1. What can you infer about the character in the passage?\\
"Lila always carried a book with her. At lunch, she would sit under the big oak tree, lost in a world of stories. Even when her friends asked her to join their games, she often preferred reading. Lila loved exploring new worlds through the pages of her books."\\
\begin{enumerate}[label=\Alph*.]
    \item Lila is very athletic.  
    \item Lila enjoys reading more than playing games.  
    \item Lila does not have any friends.  
    \item Lila dislikes being outdoors.  
\end{enumerate}

\vspace{1cm}

2. Why did the villagers in the story feel grateful?\\
"The drought had lasted for months, and the crops were failing. Then one day, a stranger arrived with a solution. He showed the villagers how to build a system to collect rainwater and store it for dry times. When the rains finally came, the villagers were prepared, and their crops flourished again."\\
\begin{enumerate}[label=\Alph*.]
    \item The stranger gave them food.  
    \item The stranger taught them how to collect and store rainwater.  
    \item The stranger brought them rain.  
    \item The stranger built their homes.  
\end{enumerate}

\vspace{1cm}
\newpage
3. What evidence supports the idea that the bird was resourceful?\\
"A thirsty crow found a jar with a small amount of water at the bottom. Unable to reach the water with its beak, the crow thought of a plan. It began dropping small stones into the jar, raising the water level. Eventually, the water rose high enough for the crow to drink. The crow’s clever thinking saved its life."\\
\begin{enumerate}[label=\Alph*.]
    \item The crow gave up when it couldn’t reach the water.  
    \item The crow used stones to raise the water level.  
    \item The crow drank from a nearby stream instead.  
    \item The crow knocked the jar over to spill the water.  
\end{enumerate}

\vspace{1cm}

4. What can be inferred about the setting of the story?\\
"The sun blazed down on the dry, cracked earth. In the distance, a few scraggly trees stood against the horizon. The air was still, and not a single cloud offered relief from the heat."\\
\begin{enumerate}[label=\Alph*.]
    \item The story takes place in a forest.  
    \item The story is set in a desert or dry area.  
    \item The story happens during winter.  
    \item The story is set in a bustling city.  
\end{enumerate}


\subsection*{Part 2: Select All That Apply Questions}

5. Which details suggest that the character is kind and helpful?\\
"Emma noticed her neighbor struggling with heavy grocery bags. She immediately offered to carry the bags and walked with her neighbor to her door. Emma often volunteered at the local animal shelter and helped her classmates with their \\homework."\\
\begin{enumerate}[label=\Alph*.]
    \item Emma helped her neighbor with grocery bags.  
    \item Emma helped her classmates with homework.  
    \item Emma volunteered at an animal shelter.  
    \item Emma ignored her neighbor’s struggle.  
\end{enumerate}

\vspace{1cm}

6. Based on the following passage, answer the next questions:\\
"The forest was unusually quiet. Mia tiptoed along the trail, her eyes scanning the trees for movement. She had heard stories of a rare golden bird that lived in these woods, and she was determined to find it. Suddenly, a flash of gold caught her eye. Mia froze and watched as the bird perched on a low branch, its feathers shimmering in the sunlight. She reached for her camera, but the bird quickly darted away, \\disappearing into the trees."\\
\textbf{Which emotions might Mia have felt during her search? Select all that apply.}\\
\begin{enumerate}[label=\Alph*.]
    \item Excitement  
    \item Frustration  
    \item Determination  
    \item Boredom  
\end{enumerate}

\vspace{1cm}

7. \textbf{Select all reasons why the golden bird is significant in the story.}\\
\begin{enumerate}[label=\Alph*.]
    \item It is rare and beautiful.  
    \item It represents Mia’s goal and determination.  
    \item It teaches Mia about patience.  
    \item It is a common bird found everywhere.  
\end{enumerate}

\vspace{1cm}
\newpage
\subsection*{Part 3: Short Answer Questions}

8. What does Mia’s reaction to seeing the golden bird reveal about her character?\\
\vspace{4cm}



9. How does the author use the setting to create a sense of mystery in the story?\\
\vspace{4cm}

\subsection*{Part 4: Fill in the Blank Questions}
\vspace{1em}
10. The setting of a story helps the reader understand where and\\ \underline{\hspace{4cm}} the events take place.

\vspace{3cm}
% \newpage
% \section*{Answer Key}

% \subsection*{Part 1: Multiple-Choice Questions}

% B. Lila enjoys reading more than playing games.

% B. The stranger taught them how to collect and store rainwater.

% B. The crow used stones to raise the water level.

% B. The story is set in a desert or dry area.

% \subsection*{Part 2: Select All That Apply Questions}

% A, B, C.

% Emma helped her neighbor with grocery bags.
% Emma helped her classmates with homework.
% Emma volunteered at an animal shelter.
% A, C.

% Excitement
% Determination
% A, B, C.

% It is rare and beautiful.
% It represents Mia’s goal and determination.
% It teaches Mia about patience.
% \subsection*{Part 3: Short Answer Questions}

% Answer: Mia’s reaction to seeing the golden bird reveals her patience, determination, and focus. Her desire to capture the bird with her camera shows her dedication to her goal, and her ability to remain still when she spotted the bird indicates her attentiveness and carefulness.

% Answer: The author uses the setting to create a sense of mystery by describing the quiet, eerie atmosphere of the forest. The unusual silence, along with Mia tiptoeing along the trail, builds suspense and anticipation. The description of the golden bird as rare and fleeting adds to the mystery and intrigue of the setting.

% \subsection*{Part 4: Fill in the Blank Questions}

% The setting of a story helps the reader understand where and \underline{when} the events take place.
\end{document}
