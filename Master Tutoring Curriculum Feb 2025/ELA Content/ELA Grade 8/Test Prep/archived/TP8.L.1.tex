\documentclass[12pt]{article}

\usepackage[a4paper, top=0.8in, bottom=0.7in, left=0.7in, right=0.7in]{geometry}

\usepackage{amsmath}
\usepackage{graphicx}
\usepackage{fancyhdr}
\usepackage{tcolorbox}
\usepackage{multicol}
\usepackage{pifont} % For checkboxes
\usepackage[defaultfam,tabular,lining]{montserrat} %% Option 'defaultfam'
\usepackage[T1]{fontenc}
\renewcommand*\oldstylenums[1]{{\fontfamily{Montserrat-TOsF}\selectfont #1}}
\renewcommand{\familydefault}{\sfdefault}
\usepackage{enumitem}
\usepackage{setspace}
\usepackage{parcolumns}
\usepackage{tabularx}

\setlength{\parindent}{0pt}
\hyphenpenalty=10000
\exhyphenpenalty=10000

\pagestyle{fancy}
\fancyhf{}
\fancyhead[L]{\textbf{8.L.1: Language and Grammar}}
\fancyhead[R]{\includegraphics[width=1cm]{Round Logo.png}}
\fancyfoot[C]{\footnotesize Study Smart Tutors}

\begin{document}

\onehalfspacing

% Passage
\subsection*{25 Multiple Choice Questions: Language and Grammar}

\begin{enumerate}

    \item Which of the following sentences is grammatically correct?
    \begin{enumerate}[label=\Alph*.]
        \item He don't like to play soccer.
        \item He doesn't like to play soccer.
        \item He don't likes to play soccer.
        \item He doesn't likes to play soccer.
    \end{enumerate}
    \vspace{0.5cm}

    \item Choose the correct form of the verb: She \underline{\hspace{2cm}} (go) to the store yesterday.
    \begin{enumerate}[label=\Alph*.]
        \item goes
        \item going
        \item gone
        \item went
    \end{enumerate}
    \vspace{0.5cm}

    \item Which word is a synonym for "happy"?
    \begin{enumerate}[label=\Alph*.]
        \item Sad
        \item Angry
        \item Joyful
        \item Nervous
    \end{enumerate}
    \vspace{0.5cm}

    \item Identify the correctly punctuated sentence:
    \begin{enumerate}[label=\Alph*.]
        \item I am going to the store, I need some milk.
        \item I am going to the store; I need some milk.
        \item I am going to the store I need some milk.
        \item I am going to the store: I need some milk.
    \end{enumerate}
    \vspace{0.5cm}

    \item Which of the following is a compound sentence?
    \begin{enumerate}[label=\Alph*.]
        \item I wanted pizza for dinner.
        \item I wanted pizza, but my brother wanted burgers.
        \item The sun is shining.
        \item I went to the store and bought milk.
    \end{enumerate}
    \vspace{0.5cm}

    \item Choose the correctly written sentence:
    \begin{enumerate}[label=\Alph*.]
        \item He are my best friend.
        \item He is my best friend.
        \item He am my best friend.
        \item He be my best friend.
    \end{enumerate}
    \vspace{0.5cm}

    \item Select the sentence that uses correct subject-verb agreement:
    \begin{enumerate}[label=\Alph*.]
        \item The dogs run fast.
        \item The dog run fast.
        \item The dogs runs fast.
        \item The dog runs fast.
    \end{enumerate}
    \vspace{0.5cm}

    \item Choose the correct form of the adjective: This test is \underline{\hspace{2cm}} (easy) than the last one.
    \begin{enumerate}[label=\Alph*.]
        \item easyer
        \item more easier
        \item easier
        \item easiest
    \end{enumerate}
    \vspace{0.5cm}

    \item Which of the following is a preposition?
    \begin{enumerate}[label=\Alph*.]
        \item Under
        \item Laugh
        \item Run
        \item Happy
    \end{enumerate}
    \vspace{0.5cm}

    \item What is the plural form of the word "child"?
    \begin{enumerate}[label=\Alph*.]
        \item Children
        \item Childs
        \item Childes
        \item Childrens
    \end{enumerate}
    \vspace{0.5cm}

    \item Which of the following sentences uses an apostrophe correctly?
    \begin{enumerate}[label=\Alph*.]
        \item The dogs bone is buried in the backyard.
        \item The dog’s bone is buried in the backyard.
        \item The dogs' bone is buried in the backyard.
        \item The dog bone’s is buried in the backyard.
    \end{enumerate}
    \vspace{0.5cm}

    \item Choose the correct form of the verb: He \underline{\hspace{2cm}} (write) an email right now.
    \begin{enumerate}[label=\Alph*.]
        \item is writing
        \item writes
        \item wrote
        \item write
    \end{enumerate}
    \vspace{0.5cm}

    \item Which of the following sentences contains a pronoun?
    \begin{enumerate}[label=\Alph*.]
        \item John went to the store.
        \item I went to the store.
        \item The store is big.
        \item He is my friend.
    \end{enumerate}
    \vspace{0.5cm}

    \item Select the correct verb tense: I \underline{\hspace{2cm}} (study) for my exam right now.
    \begin{enumerate}[label=\Alph*.]
        \item study
        \item am studying
        \item studied
        \item will study
    \end{enumerate}
    \vspace{0.5cm}

    \item Which sentence is correct?
    \begin{enumerate}[label=\Alph*.]
        \item There are many books on the table.
        \item Their are many books on the table.
        \item They’re are many books on the table.
        \item There is many books on the table.
    \end{enumerate}
    \vspace{0.5cm}

    \item What is the plural form of the word "foot"?
    \begin{enumerate}[label=\Alph*.]
        \item Foots
        \item Feets
        \item Footes
        \item Feet
    \end{enumerate}
    \vspace{0.5cm}

    \item Which of the following sentences is a question?
    \begin{enumerate}[label=\Alph*.]
        \item I am going to the store.
        \item You should go to the store.
        \item Are you going to the store?
        \item Going to the store.
    \end{enumerate}
    \vspace{0.5cm}

    \item Which of the following is a conjunction?
    \begin{enumerate}[label=\Alph*.]
        \item Because
        \item Happy
        \item Fast
        \item Quickly
    \end{enumerate}
    \vspace{0.5cm}

    \item Select the correct form of the adjective: This is the \underline{\hspace{2cm}} (fast) runner in the race.
    \begin{enumerate}[label=\Alph*.]
        \item faster
        \item fastest
        \item more fast
        \item more faster
    \end{enumerate}
    \vspace{0.5cm}

    \item Which word is an antonym for "strong"?
    \begin{enumerate}[label=\Alph*.]
        \item Weak
        \item Tough
        \item Sturdy
        \item Powerful
    \end{enumerate}
    \vspace{0.5cm}

    \item Choose the correct form of the verb: They \underline{\hspace{2cm}} (run) in the race last week.
    \begin{enumerate}[label=\Alph*.]
        \item run
        \item ran
        \item running
        \item runned
    \end{enumerate}
    \vspace{0.5cm}

    \item Which of the following is a compound-complex sentence?
    \begin{enumerate}[label=\Alph*.]
        \item I wanted to go to the store, but it was closed.
        \item I went to the store, and I bought milk because I needed some.
        \item She was happy because she got a gift.
        \item He likes playing soccer, and he plays every weekend.
    \end{enumerate}
    \vspace{0.5cm}

    \item Choose the correct spelling of the word:
    \begin{enumerate}[label=\Alph*.]
        \item Recieve
        \item Receive
        \item Recive
        \item Reciive
    \end{enumerate}
    \vspace{0.5cm}

    \item Identify the sentence with a misplaced modifier:
    \begin{enumerate}[label=\Alph*.]
        \item Walking to school, I saw a beautiful bird.
        \item I saw a beautiful bird walking to school.
        \item I saw a bird that was beautiful while walking to school.
        \item While walking to school, I saw a beautiful bird.
    \end{enumerate}
    \vspace{0.5cm}

    \item Select the correct article: \underline{\hspace{2cm}} apple a day keeps the doctor away.
    \begin{enumerate}[label=\Alph*.]
        \item A
        \item An
        \item The
        \item No article needed
    \end{enumerate}
    \vspace{0.5cm}

\end{enumerate}

\end{document}
