\documentclass[12pt]{article}

\usepackage[a4paper, top=0.8in, bottom=0.7in, left=0.7in, right=0.7in]{geometry}
\usepackage{amsmath}
\usepackage{graphicx}
\usepackage{fancyhdr}
\usepackage{tcolorbox}
\usepackage[defaultfam,tabular,lining]{montserrat} %% Option 'defaultfam'
\usepackage[T1]{fontenc}
\renewcommand*\oldstylenums[1]{{\fontfamily{Montserrat-TOsF}\selectfont #1}}
\renewcommand{\familydefault}{\sfdefault}
\usepackage{enumitem}
\usepackage{setspace}

\setlength{\parindent}{0pt}
\hyphenpenalty=10000
\exhyphenpenalty=10000

\pagestyle{fancy}
\fancyhf{}
\fancyhead[L]{\textbf{8.RI.1: Textual Evidence and Inferences Practice}}
\fancyhead[R]{\includegraphics[width=1cm]{Round Logo.png}}
\fancyfoot[C]{\footnotesize Study Smart Tutors}

\begin{document}

\subsection*{Citing Evidence and Drawing Inferences}
\onehalfspacing

\begin{tcolorbox}[colframe=black!40, colback=gray!0, title=Learning Objective]
\textbf{Objective:} Cite strong textual evidence to support analysis of what the text says explicitly as well as inferences drawn from the text.
\end{tcolorbox}

\subsection*{Part 1: Multiple-Choice Questions}

1. \textbf{What is the explicit message of the following passage? \\}  
"Invasive species pose significant threats to ecosystems worldwide. These non-native plants, animals, and microbes often outcompete native species for resources, leading to declines in biodiversity. For example, the zebra mussel, introduced to North \\America’s Great Lakes, filters vast amounts of plankton, depriving native aquatic life of food. Similarly, kudzu, a fast-growing vine native to Asia, smothers trees and shrubs across the southern United States. The impacts of invasive species are not limited to the environment; they can also harm agriculture, costing billions in lost crops and control efforts annually. Preventing the spread of invasive species requires public awareness, stricter biosecurity measures, and early detection programs. While some efforts, like biological controls, show promise, the challenge of eradicating established invasive populations remains immense. Cooperation between nations is essential to address the global scope of this issue. Ultimately, protecting native ecosystems requires a balance of innovation, vigilance, and public involvement."  
\begin{enumerate}[label=\Alph*.]
    \item Invasive species only harm the economy, not ecosystems.  
    \item Invasive species threaten biodiversity and require coordinated efforts to control.  
    \item The zebra mussel and kudzu are native species that benefit their environments.  
    \item Efforts to stop invasive species have already eradicated the problem.  
\end{enumerate}

\vspace{1cm}
\newpage
2. \textbf{What inference can be made from the following text? }\\  
"Mars exploration has captured humanity’s imagination for decades. As robotic rovers like Perseverance explore its surface, scientists uncover evidence of ancient rivers and lakes, suggesting that Mars may once have supported life. Despite these \\discoveries, challenges remain. Mars’ thin atmosphere provides little protection from cosmic radiation, and its surface temperatures are often far below freezing. Efforts to send humans to Mars are underway, with private companies and space agencies alike working to design habitats, transportation systems, and life-support \\technologies. Beyond scientific interest, colonizing Mars raises ethical and logistical questions, such as how to avoid contaminating the planet with Earth-based \\organisms. Some argue that resources should be prioritized to solve problems on Earth before investing billions in interplanetary exploration. Others see Mars \\exploration as essential to ensuring humanity’s survival by providing a backup planet. Regardless of these debates, Mars remains a symbol of exploration, resilience, and the desire to push boundaries."  
\begin{enumerate}[label=\Alph*.]
    \item Mars exploration has no scientific value.  
    \item Mars exploration raises ethical and practical concerns alongside its scientific goals.  
    \item Mars has a thick atmosphere that makes colonization easy.  
    \item All scientists agree that exploring Mars should take priority over Earth’s issues.  
\end{enumerate}

\vspace{1cm}
\newpage
3. \textbf{Which piece of evidence best supports the idea that the Wild West was a time of both opportunity and danger? }\\  
"The Wild West era in American history, spanning the late 19th century, is often romanticized as a time of adventure, freedom, and opportunity. Settlers moved westward, lured by the promise of land under the Homestead Act or fortunes in gold and silver. However, the reality was often harsh. Towns sprang up rapidly, with little law enforcement, leading to violence and lawlessness. Famous outlaws like Billy the Kid and Jesse James became legends, while lawmen like Wyatt Earp tried to maintain order. Native American tribes faced devastating displacement as settlers encroached on their lands, leading to brutal conflicts. Despite the challenges, the era saw remarkable innovation, including the construction of the transcontinental railroad, which connected the country and facilitated trade. The Wild West remains an enduring symbol of resilience, ambition, and the complicated legacy of expansion in America’s history."  
\begin{enumerate}[label=\Alph*.]
    \item The Wild West was a peaceful time for all involved.  
    \item The transcontinental railroad provided economic opportunities.  
    \item Violence and lawlessness were common in Wild West towns.  
    \item Both opportunity and conflict defined the Wild West.  
\end{enumerate}




\subsection*{Part 2: Select All That Apply Questions}

4. Which details from the passage from question 1 explain the impact of \\invasive species?  
\begin{enumerate}[label=\Alph*.]
    \item Zebra mussels deplete plankton, reducing food for native aquatic life.  
    \item Kudzu smothers trees and shrubs, disrupting ecosystems.  
    \item Invasive species only affect local ecosystems, not global ones.  
    \item Deforestation is a primary cause of invasive species spread.  
\end{enumerate}

\vspace{1cm}

5. What evidence from the passage from question 2 supports the inference \\that Mars exploration is both exciting and challenging?  
\begin{enumerate}[label=\Alph*.]
    \item Robotic rovers have found evidence of ancient rivers and lakes.  
    \item Mars’ thin atmosphere provides little protection from cosmic radiation.  
    \item Colonizing Mars raises ethical and logistical concerns.  
    \item Scientists debate whether resources should prioritize Earth or Mars exploration.  
\end{enumerate}



\vspace{1cm}

6. Which details from the passage from question 3 support the idea that the Wild West was a time of both opportunity and danger?  
\begin{enumerate}[label=\Alph*.]
    \item Settlers moved westward for land and riches.  
    \item Violence and lawlessness were common in towns.  
    \item Native American tribes were displaced from their lands.  
    \item The transcontinental railroad facilitated trade and expansion.  
\end{enumerate}





\subsection*{Part 3: Short Answer Questions}

7. Based on the passage about invasive species from question 1, what are \\some solutions to combat their spread and protect ecosystems? Use evidence \\from the text to support your response.  
\vspace{4cm}

8. Based on the passage about Mars exploration from question 2, what are \\some of the challenges and ethical concerns involved in sending humans to \\Mars? Use evidence from the passage from question 2 to support your response.  

\vspace{4cm}

\subsection*{Part 4: Fill in the Blank Questions}
\vspace{1cm}
9. Textual evidence is used to support both \underline{\hspace{4cm}} and\\ \underline{\hspace{4cm}} drawn from the text.  
\vspace{1cm}

10. Inferences are conclusions drawn from textual evidence and information that\\ is \underline{\hspace{4cm}} stated in the text.  
\vspace{2cm}
\newpage
\section*{Answer Key}

\subsection*{Part 1: Multiple-Choice Questions}

1. \textbf{What is the explicit message of the following passage?}  
\textbf{Answer:} B. Invasive species threaten biodiversity and require coordinated efforts to control.  
\textbf{Explanation:} The passage discusses the significant threat posed by invasive species to ecosystems, biodiversity, and agriculture, and the need for coordinated efforts to address this issue.

\vspace{1cm}
2. \textbf{What inference can be made from the following text?}  
\textbf{Answer:} B. Mars exploration raises ethical and practical concerns alongside its scientific goals.  
\textbf{Explanation:} The passage mentions the ethical and logistical issues related to Mars exploration, such as the potential contamination of Mars and debates over prioritizing resources for Earth vs. Mars exploration.

\vspace{1cm}
3. \textbf{Which piece of evidence best supports the idea that the Wild West was a time of both opportunity and danger?}  
\textbf{Answer:} D. Both opportunity and conflict defined the Wild West.  
\textbf{Explanation:} The passage mentions both the opportunities (land and riches) and the dangers (violence, lawlessness, and displacement of Native Americans) that characterized the Wild West era.

\subsection*{Part 2: Select All That Apply Questions}

4. \textbf{Which details from the passage from question 1 explain the impact of invasive species?}  
\textbf{Answer:} A. Zebra mussels deplete plankton, reducing food for native aquatic life. \\
B. Kudzu smothers trees and shrubs, disrupting ecosystems.  
\textbf{Explanation:} These details directly explain how invasive species impact ecosystems by outcompeting native species for resources and disrupting natural environments.

\vspace{1cm}
5. \textbf{What evidence from the passage from question 2 supports the inference that Mars exploration is both exciting and challenging?}  
\textbf{Answer:} A. Robotic rovers have found evidence of ancient rivers and lakes. \\
B. Mars’ thin atmosphere provides little protection from cosmic radiation. \\
C. Colonizing Mars raises ethical and logistical concerns.  
\textbf{Explanation:} The evidence shows the scientific excitement of discovering potential signs of past life on Mars, while also highlighting the challenges posed by Mars' environment and the ethical concerns of colonization.

\vspace{1cm}
6. \textbf{Which details from the passage from question 3 support the idea that the Wild West was a time of both opportunity and danger?}  
\textbf{Answer:} A. Settlers moved westward for land and riches. \\
B. Violence and lawlessness were common in towns. \\
C. Native American tribes were displaced from their lands. \\
D. The transcontinental railroad facilitated trade and expansion.  
\textbf{Explanation:} These details illustrate both the opportunities (land, riches, economic expansion) and the dangers (violence, lawlessness, displacement of Native Americans) of the Wild West.

\subsection*{Part 3: Short Answer Questions}

7. \textbf{Based on the passage about invasive species from question 1, what are some solutions to combat their spread and protect ecosystems? Use evidence from the text to support your response.}  
\textbf{Answer:} The text suggests that solutions to combat the spread of invasive species include public awareness, stricter biosecurity measures, and early detection programs. Additionally, biological controls show promise in managing invasive species.  
\textbf{Explanation:} The passage explicitly mentions these measures as necessary to prevent the spread of invasive species and protect native ecosystems.

\vspace{1cm}
8. \textbf{Based on the passage about Mars exploration from question 2, what are some of the challenges and ethical concerns involved in sending humans to Mars? Use evidence from the passage from question 2 to support your response.}  
\textbf{Answer:} The challenges of sending humans to Mars include the planet’s thin atmosphere, which provides little protection from cosmic radiation, and its freezing surface temperatures. Ethical concerns include the potential contamination of Mars with Earth-based organisms and the debate over whether resources should be allocated to Mars exploration or solving problems on Earth.  
\textbf{Explanation:} The passage discusses these challenges and concerns explicitly in relation to the ongoing efforts to explore Mars.

\subsection*{Part 4: Fill in the Blank Questions}

9. Textual evidence is used to support both \underline{explicit statements} and \underline{inferences} drawn from the text.  
\textbf{Answer:} explicit statements, inferences.  
\textbf{Explanation:} Textual evidence is used both to support direct information (explicit) and to make conclusions based on that evidence (inferences).

10. Inferences are conclusions drawn from textual evidence and information that is \underline{implicitly} stated in the text.  
\textbf{Answer:} implicitly.  
\textbf{Explanation:} Inferences are conclusions that go beyond what is directly stated (explicitly), using implicit information to form a logical conclusion.
\end{document}
