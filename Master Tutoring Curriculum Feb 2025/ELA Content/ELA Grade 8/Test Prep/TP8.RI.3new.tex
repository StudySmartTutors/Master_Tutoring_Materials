\documentclass[12pt]{article}

\usepackage[a4paper, top=0.8in, bottom=0.7in, left=0.7in, right=0.7in]{geometry}
\usepackage{amsmath}
\usepackage{graphicx}
\usepackage{fancyhdr}
\usepackage{tcolorbox}
\usepackage[defaultfam,tabular,lining]{montserrat} %% Option 'defaultfam'
\usepackage[T1]{fontenc}
\renewcommand*\oldstylenums[1]{{\fontfamily{Montserrat-TOsF}\selectfont #1}}
\renewcommand{\familydefault}{\sfdefault}
\usepackage{enumitem}
\usepackage{setspace}

\setlength{\parindent}{0pt}
\hyphenpenalty=10000
\exhyphenpenalty=10000

\pagestyle{fancy}
\fancyhf{}
\fancyhead[L]{\textbf{8.RI.3: Analyzing Connections and Relationships Practice}}
\fancyhead[R]{\includegraphics[width=1cm]{Round Logo.png}}
\fancyfoot[C]{\footnotesize Study Smart Tutors}

\begin{document}

\subsection*{Understanding Connections and Relationships}
\onehalfspacing

\begin{tcolorbox}[colframe=black!40, colback=gray!0, title=Learning Objective]
\textbf{Objective:} Analyze how a text makes connections among and distinctions between individuals, ideas, or events.
\end{tcolorbox}

\subsection*{Part 1: Multiple-Choice Questions}

1. \textbf{How did advancements in weaponry change medieval warfare?\\}
"During the medieval period, technological advancements drastically transformed how battles were fought. In the early Middle Ages, knights on horseback wielding swords, lances, and shields dominated the battlefield. However, the invention of the longbow changed the balance of power. Archers could strike from great distances, making heavy armor less effective. Armies began relying more on skilled archers and less on heavily armored knights. As the medieval period progressed, the \\introduction of gunpowder and weapons like cannons and early firearms further revolutionized warfare. Gunpowder rendered castle walls vulnerable to siege \\weapons, ending the era of heavily fortified strongholds. Cannons could break \\through walls that once seemed impenetrable, forcing armies to develop new \\strategies, such as open field battles. Gunpowder also allowed foot soldiers armed with firearms to stand against knights, shifting the power dynamics in war. These innovations not only changed how wars were fought but also shaped medieval \\society, weakening the power of feudal lords and giving rise to centralized \\governments."\\
\begin{enumerate}[label=\Alph*.]
    \item Knights became more important in battles.  
    \item The longbow made shields and armor less effective.  
    \item Castles became more powerful over time.  
    \item Armies abandoned the use of gunpowder.  
\end{enumerate}

\vspace{1cm}
\newpage
2. \textbf{What is the primary connection between science and cheesemaking?\\}
"Cheesemaking is both an ancient tradition and a modern science. The process begins with milk, which contains proteins, fats, and sugars. By adding enzymes or acids, cheesemakers separate milk into curds (solid) and whey (liquid). This step forms the base for cheese. Fermentation, driven by bacteria or mold, transforms the curds into cheese by developing flavor and texture. The type of bacteria or mold, as well as the conditions, like temperature and humidity, determine the final product. For example, blue cheeses like Roquefort develop their distinct flavor from mold introduced during fermentation. The aging process, known as affinage, is another critical step. Cheesemakers monitor conditions carefully to ensure the cheese ripens correctly. Different aging times create a variety of textures, from soft brie to hard parmesan. Modern science enhances cheesemaking by ensuring food safety,\\ consistency, and innovation, such as creating lactose-free cheese. By combining tradition with scientific techniques, cheesemaking continues to evolve as both an art and a science."\\
\begin{enumerate}[label=\Alph*.]
    \item Cheesemaking depends on fermentation and food science.  
    \item Aging techniques are no longer used in modern cheesemaking.  
    \item Bacteria make milk unsafe for cheese production.  
    \item Cheesemaking relies only on tradition, not science.  
\end{enumerate}

\vspace{1cm}
\newpage
3. \textbf{How did the invention of gunpowder affect exploration and warfare?\\}
"The invention of gunpowder, originating in China, had a profound impact on \\exploration and warfare. Initially used for fireworks, gunpowder’s explosive power soon led to the creation of weapons like cannons and firearms. In warfare, gunpowder revolutionized how battles were fought. Castles that once offered protection against invaders became vulnerable to cannon fire, leading to changes in military architecture and strategy. Armies equipped with firearms and cannons gained an advantage over those relying on traditional weapons like swords and bows. Beyond the battlefield, gunpowder played a crucial role in exploration. Cannons mounted on ships allowed explorers to defend themselves and assert control over new territories. Gunpowder also fueled global trade, as countries sought materials like saltpeter to produce it. This demand connected regions and influenced political alliances. While it brought destructive power, gunpowder also symbolized progress, shaping the course of history by connecting cultures and advancing technology."\\
\begin{enumerate}[label=\Alph*.]
    \item Gunpowder limited the ability to explore distant lands.  
    \item Firearms and cannons made exploration more dangerous.  
    \item Gunpowder improved trade, exploration, and warfare.  
    \item Gunpowder was primarily used for scientific experiments.  
\end{enumerate}


\vspace{1cm}

\newpage
\subsection*{Part 2: Select All That Apply Questions}

4. Select \textbf{all} reasons why the longbow changed medieval warfare:\\
\begin{enumerate}[label=\Alph*.]
    \item It allowed soldiers to attack from a safe distance.  
    \item It replaced cannons as the main weapon in battles.  
    \item It made heavy armor less effective.  
    \item It required fewer soldiers to defend a position.  
\end{enumerate}

\vspace{1cm}

5. What factors influence the flavor of cheese? (Select \textbf{all} that apply.)\\
\begin{enumerate}[label=\Alph*.]
    \item The type of milk used.  
    \item The aging time.  
    \item The amount of saltpeter added.  
    \item The fermentation process.  
\end{enumerate}

\vspace{1cm}

6. What were some effects of gunpowder on global interactions?\\
\begin{enumerate}[label=\Alph*.]
    \item It connected regions through trade.  
    \item It ended all forms of warfare.  
    \item It allowed explorers to defend themselves.  
    \item It changed military tactics worldwide.  
\end{enumerate}

\vspace{1cm}

\subsection*{Part 3: Short Answer Questions}

7. Explain how the invention of the longbow affected medieval battles.\\
\vspace{3cm}

8. Describe how science is used in cheesemaking to create different types of cheese.\\
\vspace{3cm}

\subsection*{Part 4: Fill in the Blank Questions}

9. A \underline{\hspace{4cm}} is a group of things that are similar in some way and \\based on shared traits.\\
\vspace{2cm}

10.A \underline{\hspace{4cm}} is when you look at two or more things and explain \\how they are different.\\
\vspace{2cm}
\newpage
\section*{Answer Key}

\subsection*{Part 1: Multiple-Choice Questions}

1. \textbf{How did advancements in weaponry change medieval warfare?}
\begin{enumerate}[label=\Alph*.]
    \item \textbf{B} The longbow made shields and armor less effective.
\end{enumerate}

2. \textbf{What is the primary connection between science and cheesemaking?}
\begin{enumerate}[label=\Alph*.]
    \item \textbf{A} Cheesemaking depends on fermentation and food science.
\end{enumerate}

3. \textbf{How did the invention of gunpowder affect exploration and warfare?}
\begin{enumerate}[label=\Alph*.]
    \item \textbf{C} Gunpowder improved trade, exploration, and warfare.
\end{enumerate}

\subsection*{Part 2: Select All That Apply Questions}

4. Select \textbf{all} reasons why the longbow changed medieval warfare:
\begin{enumerate}[label=\Alph*.]
    \item \textbf{A} It allowed soldiers to attack from a safe distance.
    \item \textbf{C} It made heavy armor less effective.
\end{enumerate}

5. What factors influence the flavor of cheese? (Select \textbf{all} that apply.)
\begin{enumerate}[label=\Alph*.]
    \item \textbf{A} The type of milk used.
    \item \textbf{B} The aging time.
    \item \textbf{D} The fermentation process.
\end{enumerate}

6. What were some effects of gunpowder on global interactions?
\begin{enumerate}[label=\Alph*.]
    \item \textbf{A} It connected regions through trade.
    \item \textbf{C} It allowed explorers to defend themselves.
    \item \textbf{D} It changed military tactics worldwide.
\end{enumerate}

\subsection*{Part 3: Short Answer Questions}

7. \textbf{Explain how the invention of the longbow affected medieval battles.}
\textbf{Sample Answer:} The longbow allowed archers to attack from a distance, making knights’ heavy armor less effective. This change in weaponry reduced the dominance of knights on horseback and shifted the balance of power in medieval warfare.

8. \textbf{Describe how science is used in cheesemaking to create different types of cheese.}
\textbf{Sample Answer:} Science is used in cheesemaking through the fermentation process, where bacteria or mold is introduced to curds to develop different flavors and textures. Temperature, humidity, and aging time all affect the final product, allowing cheesemakers to produce varieties such as soft brie or hard parmesan.

\subsection*{Part 4: Fill in the Blank Questions}

9. A \underline{category} is a group of things that are similar in some way and based on shared traits.

10. A \underline{distinction} is when you look at two or more things and explain how they are different.

\end{document}


