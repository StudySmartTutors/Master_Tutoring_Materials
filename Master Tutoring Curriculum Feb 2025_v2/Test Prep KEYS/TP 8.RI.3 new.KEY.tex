\documentclass[12pt]{article}

\usepackage[a4paper, top=0.8in, bottom=0.7in, left=0.7in, right=0.7in]{geometry}
\usepackage{amsmath}
\usepackage{graphicx}
\usepackage{fancyhdr}
\usepackage{tcolorbox}
\usepackage[defaultfam,tabular,lining]{montserrat} %% Option 'defaultfam'
\usepackage[T1]{fontenc}
\renewcommand*\oldstylenums[1]{{\fontfamily{Montserrat-TOsF}\selectfont #1}}
\renewcommand{\familydefault}{\sfdefault}
\usepackage{enumitem}
\usepackage{setspace}

\setlength{\parindent}{0pt}
\hyphenpenalty=10000
\exhyphenpenalty=10000

\pagestyle{fancy}
\fancyhf{}
\fancyhead[L]{\textbf{8.RI.3: Analyzing Connections and Relationships Practice}}
\fancyhead[R]{\includegraphics[width=1cm]{Round Logo.png}}
\fancyfoot[C]{\footnotesize Study Smart Tutors}

\begin{document}

\subsection*{Understanding Connections and Relationships}
\onehalfspacing

\begin{tcolorbox}[colframe=black!40, colback=gray!0, title=Learning Objective]
\textbf{Objective:} Analyze how a text makes connections among and distinctions between individuals, ideas, or events.
\end{tcolorbox}


\section*{Answer Key}

\subsection*{Part 1: Multiple-Choice Questions}

1. \textbf{How did advancements in weaponry change medieval warfare?}
\begin{enumerate}[label=\Alph*.]
    \item \textbf{B} The longbow made shields and armor less effective.
\end{enumerate}

2. \textbf{What is the primary connection between science and cheesemaking?}
\begin{enumerate}[label=\Alph*.]
    \item \textbf{A} Cheesemaking depends on fermentation and food science.
\end{enumerate}

3. \textbf{How did the invention of gunpowder affect exploration and warfare?}
\begin{enumerate}[label=\Alph*.]
    \item \textbf{C} Gunpowder improved trade, exploration, and warfare.
\end{enumerate}

\subsection*{Part 2: Select All That Apply Questions}

4. Select \textbf{all} reasons why the longbow changed medieval warfare:
\begin{enumerate}[label=\Alph*.]
    \item \textbf{A} It allowed soldiers to attack from a safe distance.
    \item \textbf{C} It made heavy armor less effective.
\end{enumerate}

5. What factors influence the flavor of cheese? (Select \textbf{all} that apply.)
\begin{enumerate}[label=\Alph*.]
    \item \textbf{A} The type of milk used.
    \item \textbf{B} The aging time.
    \item \textbf{D} The fermentation process.
\end{enumerate}

6. What were some effects of gunpowder on global interactions?
\begin{enumerate}[label=\Alph*.]
    \item \textbf{A} It connected regions through trade.
    \item \textbf{C} It allowed explorers to defend themselves.
    \item \textbf{D} It changed military tactics worldwide.
\end{enumerate}

\subsection*{Part 3: Short Answer Questions}

7. \textbf{Explain how the invention of the longbow affected medieval battles.}
\textbf{Sample Answer:} The longbow allowed archers to attack from a distance, making knights’ heavy armor less effective. This change in weaponry reduced the dominance of knights on horseback and shifted the balance of power in medieval warfare.

8. \textbf{Describe how science is used in cheesemaking to create different types of cheese.}
\textbf{Sample Answer:} Science is used in cheesemaking through the fermentation process, where bacteria or mold is introduced to curds to develop different flavors and textures. Temperature, humidity, and aging time all affect the final product, allowing cheesemakers to produce varieties such as soft brie or hard parmesan.

\subsection*{Part 4: Fill in the Blank Questions}

9. A \underline{category} is a group of things that are similar in some way and based on shared traits.

10. A \underline{distinction} is when you look at two or more things and explain how they are different.





\end{document}


