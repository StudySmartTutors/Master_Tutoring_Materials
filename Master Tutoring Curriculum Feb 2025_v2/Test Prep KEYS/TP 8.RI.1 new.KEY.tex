\documentclass[12pt]{article}

\usepackage[a4paper, top=0.8in, bottom=0.7in, left=0.7in, right=0.7in]{geometry}
\usepackage{amsmath}
\usepackage{graphicx}
\usepackage{fancyhdr}
\usepackage{tcolorbox}
\usepackage[defaultfam,tabular,lining]{montserrat} %% Option 'defaultfam'
\usepackage[T1]{fontenc}
\renewcommand*\oldstylenums[1]{{\fontfamily{Montserrat-TOsF}\selectfont #1}}
\renewcommand{\familydefault}{\sfdefault}
\usepackage{enumitem}
\usepackage{setspace}

\setlength{\parindent}{0pt}
\hyphenpenalty=10000
\exhyphenpenalty=10000

\pagestyle{fancy}
\fancyhf{}
\fancyhead[L]{\textbf{8.RI.1: Textual Evidence and Inferences Practice}}
\fancyhead[R]{\includegraphics[width=1cm]{Round Logo.png}}
\fancyfoot[C]{\footnotesize Study Smart Tutors}

\begin{document}

\subsection*{Citing Evidence and Drawing Inferences}
\onehalfspacing

\begin{tcolorbox}[colframe=black!40, colback=gray!0, title=Learning Objective]
\textbf{Objective:} Cite strong textual evidence to support analysis of what the text says explicitly as well as inferences drawn from the text.
\end{tcolorbox}

% \newpage
\section*{Answer Key}

\subsection*{Part 1: Multiple-Choice Questions}

1. \textbf{What is the explicit message of the following passage?}  
\textbf{Answer:} B. Invasive species threaten biodiversity and require coordinated efforts to control.  
\textbf{Explanation:} The passage discusses the significant threat posed by invasive species to ecosystems, biodiversity, and agriculture, and the need for coordinated efforts to address this issue.

\vspace{1cm}
2. \textbf{What inference can be made from the following text?}  
\textbf{Answer:} B. Mars exploration raises ethical and practical concerns alongside its scientific goals.  
\textbf{Explanation:} The passage mentions the ethical and logistical issues related to Mars exploration, such as the potential contamination of Mars and debates over prioritizing resources for Earth vs. Mars exploration.

\vspace{1cm}
3. \textbf{Which piece of evidence best supports the idea that the Wild West was a time of both opportunity and danger?}  
\textbf{Answer:} D. Both opportunity and conflict defined the Wild West.  
\textbf{Explanation:} The passage mentions both the opportunities (land and riches) and the dangers (violence, lawlessness, and displacement of Native Americans) that characterized the Wild West era.

\subsection*{Part 2: Select All That Apply Questions}

4. \textbf{Which details from the passage from question 1 explain the impact of invasive species?}  
\textbf{Answer:} A. Zebra mussels deplete plankton, reducing food for native aquatic life. \\
B. Kudzu smothers trees and shrubs, disrupting ecosystems.  
\textbf{Explanation:} These details directly explain how invasive species impact ecosystems by outcompeting native species for resources and disrupting natural environments.

\vspace{1cm}
5. \textbf{What evidence from the passage from question 2 supports the inference that Mars exploration is both exciting and challenging?}  
\textbf{Answer:} A. Robotic rovers have found evidence of ancient rivers and lakes. \\
B. Mars’ thin atmosphere provides little protection from cosmic radiation. \\
C. Colonizing Mars raises ethical and logistical concerns.  
\textbf{Explanation:} The evidence shows the scientific excitement of discovering potential signs of past life on Mars, while also highlighting the challenges posed by Mars' environment and the ethical concerns of colonization.

\vspace{1cm}
6. \textbf{Which details from the passage from question 3 support the idea that the Wild West was a time of both opportunity and danger?}  
\textbf{Answer:} A. Settlers moved westward for land and riches. \\
B. Violence and lawlessness were common in towns. \\
C. Native American tribes were displaced from their lands. \\
D. The transcontinental railroad facilitated trade and expansion.  
\textbf{Explanation:} These details illustrate both the opportunities (land, riches, economic expansion) and the dangers (violence, lawlessness, displacement of Native Americans) of the Wild West.

\subsection*{Part 3: Short Answer Questions}

7. \textbf{Based on the passage about invasive species from question 1, what are some solutions to combat their spread and protect ecosystems? Use evidence from the text to support your response.}  
\textbf{Answer:} The text suggests that solutions to combat the spread of invasive species include public awareness, stricter biosecurity measures, and early detection programs. Additionally, biological controls show promise in managing invasive species.  
\textbf{Explanation:} The passage explicitly mentions these measures as necessary to prevent the spread of invasive species and protect native ecosystems.

\vspace{1cm}
8. \textbf{Based on the passage about Mars exploration from question 2, what are some of the challenges and ethical concerns involved in sending humans to Mars? Use evidence from the passage from question 2 to support your response.}  
\textbf{Answer:} The challenges of sending humans to Mars include the planet’s thin atmosphere, which provides little protection from cosmic radiation, and its freezing surface temperatures. Ethical concerns include the potential contamination of Mars with Earth-based organisms and the debate over whether resources should be allocated to Mars exploration or solving problems on Earth.  
\textbf{Explanation:} The passage discusses these challenges and concerns explicitly in relation to the ongoing efforts to explore Mars.

\subsection*{Part 4: Fill in the Blank Questions}

9. Textual evidence is used to support both \underline{explicit statements} and \underline{inferences} drawn from the text.  
\textbf{Answer:} explicit statements, inferences.  
\textbf{Explanation:} Textual evidence is used both to support direct information (explicit) and to make conclusions based on that evidence (inferences).

10. Inferences are conclusions drawn from textual evidence and information that is \underline{implicitly} stated in the text.  
\textbf{Answer:} implicitly.  
\textbf{Explanation:} Inferences are conclusions that go beyond what is directly stated (explicitly), using implicit information to form a logical conclusion.






\end{document}
