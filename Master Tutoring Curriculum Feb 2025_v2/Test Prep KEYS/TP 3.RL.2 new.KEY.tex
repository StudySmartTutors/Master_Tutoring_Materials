\documentclass[12pt]{article}

\usepackage[a4paper, top=0.8in, bottom=0.7in, left=0.7in, right=0.7in]{geometry}
\usepackage{amsmath}
\usepackage{graphicx}
\usepackage{fancyhdr}
\usepackage{tcolorbox}
\usepackage{multicol}
\usepackage{pifont} % For checkboxes
\usepackage[defaultfam,tabular,lining]{montserrat} %% Option 'defaultfam'
\usepackage[T1]{fontenc}
\renewcommand*\oldstylenums[1]{{\fontfamily{Montserrat-TOsF}\selectfont #1}}
\renewcommand{\familydefault}{\sfdefault}
\usepackage{enumitem}
\usepackage{setspace}
\usepackage{parcolumns}
\usepackage{tabularx}

\setlength{\parindent}{0pt}
\hyphenpenalty=10000
\exhyphenpenalty=10000

\pagestyle{fancy}
\fancyhf{}
%\fancyhead[L]{\textbf{3.RL.2: Recounting Stories Practice}}
\fancyhead[R]{\includegraphics[width=1cm]{Round Logo.png}}
\fancyfoot[C]{\footnotesize Study Smart Tutors}

\begin{document}

\subsection*{Recounting Stories and Determining the Central Message}
\onehalfspacing

\begin{tcolorbox}[colframe=black!40, colback=gray!0, title=Learning Objective]
\textbf{Objective:} Understand how to recount stories and determine their central message, lesson, or moral.
\end{tcolorbox}


\subsection*{Answer Key}

\textbf{Part 1: Multiple-Choice Questions}  
1. B. Learning to be independent is important.  
2. A. Slow and steady wins the race.  
3. B. Cooperation helps achieve big goals.  

\textbf{Part 2: Short Answer Questions}  
4. Retelling of "The Boy Who Cried Wolf": A boy repeatedly tricked villagers by shouting that a wolf was attacking his sheep. When a wolf actually came, no one believed him, and the sheep were lost. Lesson: Always tell the truth; lying can lead to a loss of trust.  
5. "The Tortoise and the Hare" shows that steady, consistent effort can overcome challenges and achieve success, even when faced with overconfidence or quicker opponents.  
6. Example: In "The Lion and the Mouse," the mouse helps the lion by freeing it from a hunter's net. Lesson: Even the smallest individuals can help in big ways, showing the value of kindness and teamwork.  

\textbf{Part 3: Select All That Apply}  
7. A. They entertain readers.  
   B. They provide examples of how to solve problems.  
   C. They help readers understand emotions and challenges.  
   D. They give advice on what to avoid in life.  
8. A. Including the main characters.  
   B. Describing the setting.  
   D. Summarizing the main events.  

\textbf{Part 4: Fill in the Blank}  
9. Moral  
10. Important  
\end{document}
