\documentclass[12pt]{article}
\usepackage[a4paper, top=0.8in, bottom=0.7in, left=0.8in, right=0.8in]{geometry}
\usepackage{amsmath}
\usepackage{amsfonts}
\usepackage{latexsym}
\usepackage{graphicx}
\usepackage{fancyhdr}
\usepackage{tcolorbox}
\usepackage{enumitem}
\usepackage{setspace}
\usepackage[defaultfam,tabular,lining]{montserrat}
\usepackage{xcolor}

% General Comment: Answer Key for problem set with detailed step-by-step solutions in red.
% -------------------------------------------------------------------

\setlength{\parindent}{0pt}
\pagestyle{fancy}

\setlength{\headheight}{27.11148pt}
\addtolength{\topmargin}{-15.11148pt}

\fancyhf{}
%\fancyhead[L]{\textbf{5.NBT.A.5: Multiplication and Division Problem Solving - Answer Key}} % Header with standards and topic title
\fancyhead[R]{\includegraphics[width=0.8cm]{Round Logo.png}} % Placeholder for logo
\fancyfoot[C]{\footnotesize © Study Smart Tutors}

\sloppy

\title{}
\date{}
\hyphenpenalty=10000
\exhyphenpenalty=10000

\begin{document}

\subsection*{Problem Set: Solving Two-Step Word Problems Using Multiplication and Division - Answer Key}
\onehalfspacing

% Learning Objective Box
\begin{tcolorbox}[colframe=black!40, colback=gray!5, 
coltitle=black, colbacktitle=black!20, fonttitle=\bfseries\Large, 
title=Learning Objective, halign title=center, left=5pt, right=5pt, top=5pt, bottom=15pt]
\textbf{Objective:} Solve two-step word problems involving multiplication and division, representing solutions using equations with a variable.
\end{tcolorbox}

% Exercises Box
\begin{tcolorbox}[colframe=black!60, colback=white, 
coltitle=black, colbacktitle=black!15, fonttitle=\bfseries\Large, 
title=Exercises, halign title=center, left=10pt, right=10pt, top=10pt, bottom=60pt]
\begin{enumerate}[itemsep=3em]
    \item Find the product: \( 23 \times 45 \).\\
    \textcolor{red}{\textbf{Solution:} \( 23 \times 45 = 1,035 \).}

    \item Divide and find the quotient: \( 750 \div 25 \).\\
    \textcolor{red}{\textbf{Solution:} \( 750 \div 25 = 30 \).}

    \item Multiply: \( 356 \times 12 \).\\
    \textcolor{red}{\textbf{Solution:} \( 356 \times 12 = 4,272 \).}

    \item Solve: \( 48 \div 6 \times 4 \).\\
    \textcolor{red}{\textbf{Solution:} \( 48 \div 6 = 8 \), then \( 8 \times 4 = 32 \). Final answer: \( 32 \).}

    \item Write the equation and solve: "A box contains 125 pencils, and there are 8 boxes. How many pencils are there in total?"\\
    \textcolor{red}{\textbf{Solution:} 
    Equation: \( 125 \times 8 = x \). 
    Solve: \( x = 1,000 \). Total pencils: \( 1,000 \).}

    \item Simplify: \( (240 \div 8) + 42 \).\\
    \textcolor{red}{\textbf{Solution:} \( 240 \div 8 = 30 \), then \( 30 + 42 = 72 \). Final answer: \( 72 \).}

    \item Solve for \( x \): \( 3x = 450 \div 10 \).\\
    \textcolor{red}{\textbf{Solution:} 
    \( 450 \div 10 = 45 \), then \( 3x = 45 \). 
    Divide by 3: \( x = 15 \).}

    \item Write and solve: "A group of 5 friends collects 350 marbles in total. If they divide the marbles equally, how many marbles does each friend get?"\\
    \textcolor{red}{\textbf{Solution:} 
    Equation: \( 350 \div 5 = x \). 
    Solve: \( x = 70 \). Each friend gets \( 70 \) marbles.}
\end{enumerate}
\end{tcolorbox}

\vspace{1em}

% Problems Box
\begin{tcolorbox}[colframe=black!60, colback=white, 
coltitle=black, colbacktitle=black!15, fonttitle=\bfseries\Large, 
title=Problems, halign title=center, left=10pt, right=10pt, top=10pt, bottom=100pt]
\begin{enumerate}[start=9, itemsep=5em]
    \item Maria bought 6 boxes of markers, and each box costs \$15. She also purchased a notebook for \$12. Write and solve an equation to find the total amount Maria spent.\\
    \textcolor{red}{\textbf{Solution:} 
    Equation: \( 6 \times 15 + 12 = x \). 
    Solve: \( 90 + 12 = 102 \). Total: \$102.}

    \item A bakery bakes 120 cupcakes in the morning and 140 in the afternoon. If each box holds 10 cupcakes, how many boxes do they use in total?\\
    \textcolor{red}{\textbf{Solution:} 
    Total cupcakes: \( 120 + 140 = 260 \). 
    Boxes needed: \( 260 \div 10 = 26 \).}

    \item A farmer harvests 1,250 apples in a week. If he packs them into baskets holding 25 apples each, how many baskets are needed?\\
    \textcolor{red}{\textbf{Solution:} 
    \( 1,250 \div 25 = 50 \). Total baskets: \( 50 \).}

    \item Find the missing factor: \( 45 \times x = 2,250 \). Solve for \( x \) and explain your reasoning.\\
    \textcolor{red}{\textbf{Solution:} 
    Divide: \( 2,250 \div 45 = x \). 
    \( x = 50 \).}

    \item A teacher bought 300 pencils for her class. If each student gets 12 pencils, how many students can receive pencils?\\
    \textcolor{red}{\textbf{Solution:} 
    \( 300 \div 12 = 25 \). 
    Total students: \( 25 \).}
\end{enumerate}
\end{tcolorbox}

\vspace{1em}

% Performance Task Box
\begin{tcolorbox}[colframe=black!60, colback=white, 
coltitle=black, colbacktitle=black!15, fonttitle=\bfseries\Large, 
title=Performance Task: Designing a Water Conservation Experiment - Answer Key, halign title=center, left=10pt, right=10pt, top=10pt, bottom=50pt]
\begin{enumerate}[itemsep=3em]
    \item Calculate the total amount of water needed for all experiments before evaporation. Show your work.\\
    \textcolor{red}{\textbf{Solution:} 
    Total containers: \( 18 \times 5 = 90 \). 
    Total water: \( 90 \times 1.25 = 112.5 \) liters.}

    \item Determine how much water remains in all the containers after 3 hours of evaporation.\\
    \textcolor{red}{\textbf{Solution:} 
    Water remaining: \( 112.5 \div 2 = 56.25 \) liters.}

    \item Compare: How does the total water before evaporation compare to the water remaining after evaporation? Write a sentence describing the difference as a fraction or percentage.\\
    \textcolor{red}{\textbf{Solution:} 
    Half the water remains, so \( 50\% \) of the water is left.}
\end{enumerate}
\end{tcolorbox}

\vspace{1em}

% Reflection Box
\begin{tcolorbox}[colframe=black!60, colback=white, 
coltitle=black, colbacktitle=black!15, fonttitle=\bfseries\Large, 
title=Reflection, halign title=center, left=10pt, right=10pt, top=10pt, bottom=90pt]
What challenges did you encounter when solving multi-step problems? Reflect on the importance of using equations with variables to represent unknown quantities.
\end{tcolorbox}

\end{document}
