\documentclass[12pt]{article}
\usepackage[a4paper, top=0.8in, bottom=0.7in, left=0.8in, right=0.8in]{geometry}
\usepackage{amsmath}
\usepackage{amsfonts}
\usepackage{latexsym}
\usepackage{graphicx}
\usepackage{fancyhdr}
\usepackage{tcolorbox}
\usepackage{enumitem}
\usepackage{setspace}
\usepackage[defaultfam,tabular,lining]{montserrat} % Font settings for Montserrat

% General Comment: Template for creating problem sets in a structured format with headers, titles, and sections.
% This document uses Montserrat font and consistent styles for exercises, problems, and performance tasks.

% -------------------------------------------------------------------
% Directions for LaTeX Styling and Content
% 1. Include a header with standards and topic title: \fancyhead[L]{\textbf{<Standards>: <Topic Title>}}.
% 2. Section Breakdown:
%    - Learning Objective: Concise goal statement.
%    - Exercises: Procedural fluency tasks.
%    - Problems: Moderately complex scenarios.
%    - Performance Task: Real-world, multi-step tasks.
%    - Reflection: Prompt to reflect on strategies and learning.
% -------------------------------------------------------------------

\setlength{\parindent}{0pt}
\pagestyle{fancy}

\setlength{\headheight}{27.11148pt}
\addtolength{\topmargin}{-15.11148pt}

\fancyhf{}
%\fancyhead[L]{\textbf{5.NBT.B.7: Adding, Subtracting, Multiplying, and Dividing Decimals}}
\fancyhead[R]{\includegraphics[width=0.8cm]{Round Logo.png}} % Placeholder for logo
\fancyfoot[C]{\footnotesize © Study Smart Tutors}

\sloppy

\title{}
\date{}
\hyphenpenalty=10000
\exhyphenpenalty=10000

\begin{document}

\subsection*{Problem Set: Operations with Decimals}
\onehalfspacing

% Learning Objective Box
\begin{tcolorbox}[colframe=black!40, colback=gray!5, 
coltitle=black, colbacktitle=black!20, fonttitle=\bfseries\Large, 
title=Learning Objective, halign title=center, left=5pt, right=5pt, top=5pt, bottom=15pt]
\textbf{Objective:} Perform addition, subtraction, multiplication, and division of decimals to hundredths. Solve real-world problems using equations with variables.
\end{tcolorbox}

% Exercises Box
\begin{tcolorbox}[colframe=black!60, colback=white, 
coltitle=black, colbacktitle=black!15, fonttitle=\bfseries\Large, 
title=Exercises, halign title=center, left=10pt, right=10pt, top=10pt, bottom=60pt]
\begin{enumerate}[itemsep=3.5em]
    \item Add: \( 12.45 + 8.37 \).
    \item Subtract: \( 25.8 - 13.47 \).
    \item Multiply: \( 3.6 \times 4.2 \).
    \item Divide: \( 18.75 \div 3.5 \).
    \item Solve: \( (7.25 \times 3) - 5.4 \).
    \item A bag of rice weighs 2.5 kg. If there are 4 bags, what is the total weight?
    \item Solve for \( x \): \( 5.4x = 27 \).
    \item Divide \( 48.6 \div 6 \), then add 12.5 to the quotient.
\end{enumerate}
\end{tcolorbox}

\vspace{1em}

% Problems Box
\begin{tcolorbox}[colframe=black!60, colback=white, 
coltitle=black, colbacktitle=black!15, fonttitle=\bfseries\Large, 
title=Problems, halign title=center, left=10pt, right=10pt, top=10pt, bottom=100pt]
\begin{enumerate}[start=9, itemsep=7em]
    \item A store sells apples for \$2.75 per kg. If you buy 3.2 kg of apples and a \$1.50 bag of oranges, how much do you spend in total?
    \item A car travels 15.5 miles on 1 gallon of fuel. If the tank holds 12.4 gallons, how far can the car travel on a full tank?
    \item A runner jogs \( 3.8 \) miles daily for \( 5 \) days. What is the total distance jogged? Write and solve an equation.
    \item A chef uses \( 2.8 \) kg of sugar and \( 1.35 \) kg of flour. If the sugar costs \$3.50 per kg and the flour costs \$2.40 per kg, what is the total cost?
    \item A bakery uses \( 6.25 \) cups of flour for 5 loaves of bread. How many cups of flour are used per loaf?
\end{enumerate}
\end{tcolorbox}

\vspace{1em}

% Performance Task Box
\begin{tcolorbox}[colframe=black!60, colback=white, 
coltitle=black, colbacktitle=black!15, fonttitle=\bfseries\Large, 
title=Performance Task: Monitoring Plant Growth, halign title=center, left=10pt, right=10pt, top=10pt, bottom=50pt]
You are monitoring the growth of plants in a science experiment:
\begin{itemize}
    \item Plant A grows \( 1.25 \) centimeters per day.
    \item Plant B grows \( 0.95 \) centimeters per day.
    \item Both plants are measured for \( 12 \) days.
\end{itemize}
\textbf{Task:}
\begin{enumerate}[itemsep=3em]
    \item Calculate the total growth of Plant A after \( 12 \) days.
    \item Calculate the total growth of Plant B after \( 12 \) days.
    \item Find the difference in growth between the two plants.
    \item Predict how many days it will take for Plant A to grow \( 25 \) centimeters. Write an equation to represent your prediction.
    \vspace{1cm}
\end{enumerate}
\end{tcolorbox}


\vspace{1em}

% Reflection Box
\begin{tcolorbox}[colframe=black!60, colback=white, 
coltitle=black, colbacktitle=black!15, fonttitle=\bfseries\Large, 
title=Reflection, halign title=center, left=10pt, right=10pt, top=10pt, bottom=100pt]
What strategies did you use to solve problems with decimals? How did estimating help you verify your solutions? Reflect on the usefulness of decimal calculations in real-world scenarios.
\end{tcolorbox}

\end{document}
