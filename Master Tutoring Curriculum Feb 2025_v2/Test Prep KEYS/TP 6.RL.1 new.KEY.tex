\documentclass[12pt]{article}

\usepackage[a4paper, top=0.8in, bottom=0.7in, left=0.7in, right=0.7in]{geometry}
\usepackage{amsmath}
\usepackage{graphicx}
\usepackage{fancyhdr}
\usepackage{tcolorbox}
\usepackage[defaultfam,tabular,lining]{montserrat} %% Option 'defaultfam'
\usepackage[T1]{fontenc}
\renewcommand*\oldstylenums[1]{{\fontfamily{Montserrat-TOsF}\selectfont #1}}
\renewcommand{\familydefault}{\sfdefault}
\usepackage{enumitem}
\usepackage{setspace}

\setlength{\parindent}{0pt}
\hyphenpenalty=10000
\exhyphenpenalty=10000

\pagestyle{fancy}
\fancyhf{}
%\fancyhead[L]{\textbf{6.RL.1: Textual Evidence in Literary Texts Practice}}
\fancyhead[R]{\includegraphics[width=1cm]{Round Logo.png}}
\fancyfoot[C]{\footnotesize Study Smart Tutors}

\begin{document}

\subsection*{Analyzing Literary Texts and Supporting with Evidence}
\onehalfspacing

\begin{tcolorbox}[colframe=black!40, colback=gray!0, title=Learning Objective]
\textbf{Objective:} Cite textual evidence to support analysis of what the text says explicitly and inferences drawn from the text.
\end{tcolorbox}


\subsection*{Answer Key}

\textbf{Part 1: Multiple-Choice Questions}
\begin{enumerate}[label=\arabic*.]
    \item B. Jenna is quiet but kind.  
    \item D. A loud thud startled Emma.  
    \item B. They have a strained relationship with underlying tension.  
\end{enumerate}

\textbf{Part 2: Select All That Apply Questions}
\begin{enumerate}[label=\arabic*.]
    \item A, B, D.  
    \item A, B, D.  
    \item A, B, D.  
\end{enumerate}

\textbf{Part 3: Short Answer Questions}
\begin{itemize}
    \item (7) The author creates suspense by describing the storm's effect on the house, such as the creaking wood, flickering candlelight, and a loud thud from the attic. These details, combined with Emma’s nervous reactions, heighten the tension.  
    \item (8) Jenna is inferred to be a quiet and thoughtful individual. She avoids eye contact and searches for solitude but shows kindness by helping a younger student, demonstrating her caring nature.  
\end{itemize}

\textbf{Part 4: Fill in the Blank Questions}
\begin{itemize}
    \item (9) inferences  
    \item (10) actions, thoughts  
\end{itemize}

\end{document}

