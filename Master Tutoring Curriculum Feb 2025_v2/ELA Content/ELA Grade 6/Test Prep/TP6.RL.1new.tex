\documentclass[12pt]{article}

\usepackage[a4paper, top=0.8in, bottom=0.7in, left=0.7in, right=0.7in]{geometry}
\usepackage{amsmath}
\usepackage{graphicx}
\usepackage{fancyhdr}
\usepackage{tcolorbox}
\usepackage[defaultfam,tabular,lining]{montserrat} %% Option 'defaultfam'
\usepackage[T1]{fontenc}
\renewcommand*\oldstylenums[1]{{\fontfamily{Montserrat-TOsF}\selectfont #1}}
\renewcommand{\familydefault}{\sfdefault}
\usepackage{enumitem}
\usepackage{setspace}

\setlength{\parindent}{0pt}
\hyphenpenalty=10000
\exhyphenpenalty=10000

\pagestyle{fancy}
\fancyhf{}
%\fancyhead[L]{\textbf{6.RL.1: Textual Evidence in Literary Texts Practice}}
\fancyhead[R]{\includegraphics[width=1cm]{Round Logo.png}}
\fancyfoot[C]{\footnotesize Study Smart Tutors}

\begin{document}

\subsection*{Analyzing Literary Texts and Supporting with Evidence}
\onehalfspacing

\begin{tcolorbox}[colframe=black!40, colback=gray!0, title=Learning Objective]
\textbf{Objective:} Cite textual evidence to support analysis of what the text says explicitly and inferences drawn from the text.
\end{tcolorbox}

\subsection*{Part 1: Multiple-Choice Questions}

1. \textbf{What inference can be made about the main character's personality?}\\
"Jenna walked past the bustling hallway, her shoulders hunched and her eyes fixed on the floor. She gripped her notebook tightly, muttering under her breath as she scanned for a quiet corner. The noise of laughter and chatter seemed overwhelming, but she pressed on. A few students brushed past her, barely noticing her presence. Suddenly, she noticed a younger student drop their books, scattering papers everywhere. Jenna hesitated for a moment, glancing nervously at the crowd. Then, with a deep breath, she crouched down to help. 'Here,' she said softly, gathering the papers and handing them back with a shy smile. The younger student looked up and said, 'Thanks!' Jenna quickly nodded and continued on her way, her heart racing. As she turned the corner, she spotted the library and sighed with relief. She found an empty seat near the back, opened her notebook, and began writing furiously, blocking out the world around her. Jenna’s kindness in helping the student contrasted with her clear discomfort in crowded spaces, suggesting a quiet but thoughtful personality."  
\begin{enumerate}[label=\Alph*.]
    \item Jenna is confident and outgoing.  
    \item Jenna is quiet but kind.  
    \item Jenna dislikes helping others.  
    \item Jenna enjoys being the center of attention.  
\end{enumerate}

\vspace{1cm}
\newpage
2. \textbf{What explicit evidence supports the idea that the setting is tense?\\}
"The storm roared outside, rattling the windows and shaking the old house to its core. The wind howled through the cracks, and every gust seemed to bring a new groan from the wooden beams. Emma lit the last remaining candle, its flickering flame casting long, shifting shadows on the peeling wallpaper. She wrapped the blanket tighter around her shoulders, her knuckles white as she gripped it. Her eyes darted toward the attic door, which had been creaking ominously all evening. Each sound seemed louder than the last, from the relentless drumming of rain on the roof to the sharp whistle of the wind. Suddenly, a loud thud echoed from the attic above her head. Emma froze, her breath caught in her throat. She strained to listen, her ears catching faint scratching noises. Her mind raced with possibilities—was it just the wind, or was someone—or something—up there? The candle sputtered, threatening to go out, and Emma found herself inching toward the staircase, unsure whether to investigate or stay put. The tense atmosphere was thick, every sound amplifying her sense of dread."  
\begin{enumerate}[label=\Alph*.]
    \item Emma lit the last candle.  
    \item The storm caused the house to creak.  
    \item Shadows flickered on the walls.  
    \item A loud thud startled Emma.  
\end{enumerate}

\vspace{1em}

3. \textbf{How does the dialogue reveal the characters' relationship?\\}
"‘You’re late again,’ Mark said, his arms crossed and his brow furrowed.  
‘Sorry, traffic was terrible,’ Alice replied, her eyes darting to the floor as she set her bag down.  
‘You always have an excuse,’ Mark muttered, shaking his head and turning away.  
Alice sighed, running a hand through her hair. ‘Look, I said I’m sorry, okay? It’s not like I planned for this.’  
‘You’re always like this,’ Mark snapped, his tone sharp.  
‘And you’re always so quick to judge,’ Alice shot back, her voice rising. She took a deep breath, lowering her tone. ‘Let’s just focus on the project, alright?’  
Mark glanced at her, his expression softening slightly, though he didn’t respond.  
The room fell silent as they both opened their laptops, the tension lingering like a heavy cloud. Despite their words, there was an undercurrent of familiarity, as if this argument had played out many times before. The dialogue and their body language suggest a strained relationship, with hints of frustration and unresolved issues."  
\begin{enumerate}[label=\Alph*.]
    \item They are best friends working together easily.  
    \item They have a strained relationship with underlying tension.  
    \item They are family members who understand each other well.  
    \item They are strangers meeting for the first time.  
\end{enumerate}

\vspace{1cm}

\subsection*{Part 2: Select All That Apply Questions}

4. Select \textbf{all} explicit details that create a suspenseful tone in the passage about the storm from question 2:  
\begin{enumerate}[label=\Alph*.]
    \item The house creaked with every gust of wind.  
    \item Emma lit a candle that flickered on the walls.  
    \item Emma relaxed under her blanket.  
    \item A loud thud echoed from the attic.  
\end{enumerate}

\vspace{1cm}

5. Which details reveal Jenna’s personality in the passage from question 1?  
\begin{enumerate}[label=\Alph*.]
    \item She avoids eye contact with others.  
    \item She helps a younger student who drops their books.  
    \item She enjoys sitting in crowded places.  
    \item She mutters under her breath as she looks for a quiet spot.  
\end{enumerate}

\vspace{1cm}

6. Select \textbf{all} statements that show tension in the dialogue between Mark and Alice in the passage from question 3:  
\begin{enumerate}[label=\Alph*.]
    \item Mark crossed his arms and commented on Alice’s lateness.  
    \item Alice apologized but avoided looking at Mark.  
    \item Alice smiled and joked about the traffic.  
    \item Mark muttered that Alice always had an excuse.  
\end{enumerate}

\vspace{1cm}
\newpage
\subsection*{Part 3: Short Answer Questions}

7. How does the author create suspense in the storm passage from question 2? Use specific evidence from the text.  
\vspace{4cm}

8. Based on Jenna’s actions in the passage from question 1, what can you infer about her character? Provide textual evidence to support your response.  
\vspace{4cm}

\subsection*{Part 4: Fill in the Blank Questions}
\vspace{1cm}
9. Textual evidence can support explicit statements and \underline{\hspace{4cm}} drawn from the text.  
\vspace{2cm}

10. To infer a character’s personality, readers should consider their \underline{\hspace{4cm}}, \underline{\hspace{4cm}}, and dialogue.  
\vspace{2cm}
% \newpage
% \subsection*{Answer Key}

% \textbf{Part 1: Multiple-Choice Questions}
% \begin{enumerate}[label=\arabic*.]
%     \item B. Jenna is quiet but kind.  
%     \item D. A loud thud startled Emma.  
%     \item B. They have a strained relationship with underlying tension.  
% \end{enumerate}

% \textbf{Part 2: Select All That Apply Questions}
% \begin{enumerate}[label=\arabic*.]
%     \item A, B, D.  
%     \item A, B, D.  
%     \item A, B, D.  
% \end{enumerate}

% \textbf{Part 3: Short Answer Questions}
% \begin{itemize}
%     \item (7) The author creates suspense by describing the storm's effect on the house, such as the creaking wood, flickering candlelight, and a loud thud from the attic. These details, combined with Emma’s nervous reactions, heighten the tension.  
%     \item (8) Jenna is inferred to be a quiet and thoughtful individual. She avoids eye contact and searches for solitude but shows kindness by helping a younger student, demonstrating her caring nature.  
% \end{itemize}

% \textbf{Part 4: Fill in the Blank Questions}
% \begin{itemize}
%     \item (9) inferences  
%     \item (10) actions, thoughts  
% \end{itemize}

\end{document}

