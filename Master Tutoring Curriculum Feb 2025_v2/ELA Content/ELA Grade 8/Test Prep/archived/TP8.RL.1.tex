\documentclass[12pt]{article}

\usepackage[a4paper, top=0.8in, bottom=0.7in, left=0.7in, right=0.7in]{geometry}

\usepackage{amsmath}
\usepackage{graphicx}
\usepackage{fancyhdr}
\usepackage{tcolorbox}
\usepackage{multicol}
\usepackage{pifont} % For checkboxes
\usepackage[defaultfam,tabular,lining]{montserrat} %% Option 'defaultfam'
\usepackage[T1]{fontenc}
\renewcommand*\oldstylenums[1]{{\fontfamily{Montserrat-TOsF}\selectfont #1}}
\renewcommand{\familydefault}{\sfdefault}
\usepackage{enumitem}
\usepackage{setspace}
\usepackage{parcolumns}
\usepackage{tabularx}

\setlength{\parindent}{0pt}
\hyphenpenalty=10000
\exhyphenpenalty=10000

\pagestyle{fancy}
\fancyhf{}
\fancyhead[L]{\textbf{8.RL.1: Citing Evidence from Text}}
\fancyhead[R]{\includegraphics[width=1cm]{Round Logo.png}}
\fancyfoot[C]{\footnotesize Study Smart Tutors}

\begin{document}

\onehalfspacing

% Passage
\subsection*{Passage: The Argument}

\begin{tcolorbox}[colframe=black!40, colback=gray!5]
\begin{spacing}{1.15}
    One sunny afternoon, Liam and Ava sat in the park, enjoying their usual weekend routine of playing frisbee. However, today was different. After a few minutes of tossing the frisbee back and forth, a disagreement suddenly arose. Liam, who was typically easygoing, thought that the game had been getting a little boring lately. "I think we should change things up and play something else," he suggested.

    Ava, on the other hand, was deeply invested in their frisbee games. "But we’ve been playing frisbee for years! Why would we stop now?" she said, her voice rising slightly.

    "I’m just saying that it’s getting repetitive. We always do the same thing," Liam responded, growing a bit frustrated. He didn’t want to hurt Ava’s feelings, but he also didn’t want to keep doing something that no longer felt fun.

    Ava crossed her arms and shook her head. "I don’t see why we can’t just have fun like we always do," she replied stubbornly.

    The argument escalated, and soon the two friends were sitting on the grass, not speaking to each other. Liam felt guilty for bringing up the issue, but Ava was too upset to respond. For a long time, neither of them moved.

    Eventually, Liam took a deep breath. "Hey, I’m sorry. I didn’t mean to make you upset," he said quietly.

    Ava looked at him, her arms uncrossing as she softened. "I’m sorry too," she replied, her voice much gentler now. "I just didn’t want to stop playing something we’ve always enjoyed."

    The two friends smiled awkwardly at each other, and without another word, they picked up the frisbee again. This time, they played with a new sense of understanding and a little more appreciation for each other’s feelings.
\end{spacing}
\end{tcolorbox}

% Worksheet Questions
\subsection*{Questions}

\begin{enumerate}

    \item What is the main conflict in the story?
    \begin{enumerate}[label=\Alph*.]
        \item Liam and Ava disagree about playing frisbee.
        \item Liam and Ava disagree about what game to play.
        \item Liam doesn’t want to play frisbee anymore.
        \item Ava doesn’t like playing frisbee.
    \end{enumerate}
    \vspace{0.5cm}

    \item How does Liam feel about the game of frisbee?
    \begin{enumerate}[label=\Alph*.]
        \item He enjoys playing frisbee every time.
        \item He is bored of playing frisbee and wants to try something new.
        \item He wants to stop playing games altogether.
        \item He is focused on winning.
    \end{enumerate}
    \vspace{0.5cm}

    \item What is Ava’s reaction when Liam suggests changing the game?
    \begin{enumerate}[label=\Alph*.]
        \item She agrees to try something new.
        \item She is angry and refuses to listen to Liam.
        \item She is upset because she doesn’t want to stop playing frisbee.
        \item She suggests another game to play.
    \end{enumerate}
    \vspace{0.5cm}

    \item Why does Liam grow frustrated during the argument?
    \begin{enumerate}[label=\Alph*.]
        \item He feels that Ava is not understanding his feelings.
        \item He wants to end the game quickly.
        \item He is too tired to play.
        \item He is annoyed by the way Ava plays frisbee.
    \end{enumerate}
    \vspace{0.5cm}

    \item What does Ava do when Liam apologizes?
    \begin{enumerate}[label=\Alph*.]
        \item She ignores him.
        \item She becomes more upset.
        \item She apologizes as well and accepts his apology.
        \item She says she no longer wants to be friends.
    \end{enumerate}
    \vspace{0.5cm}

    \item What does the author imply about the friendship between Liam and Ava by the end of the passage?
    \begin{enumerate}[label=\Alph*.]
        \item Their argument has ruined their friendship.
        \item They both learned to understand each other better and are ready to play again.
        \item They no longer want to play games together.
        \item They are not willing to forgive each other.
    \end{enumerate}
    \vspace{0.5cm}

    \item What does the phrase "Liam felt guilty for bringing up the issue" suggest about Liam’s character?
    \begin{enumerate}[label=\Alph*.]
        \item He is quick to get angry.
        \item He is thoughtful and cares about how Ava feels.
        \item He doesn’t care about the argument.
        \item He doesn’t like playing games at all.
    \end{enumerate}
    \vspace{0.5cm}

    \item How do the friends resolve the conflict?
    \begin{enumerate}[label=\Alph*.]
        \item They stop playing games altogether.
        \item They agree never to play frisbee again.
        \item They apologize and continue playing frisbee.
        \item They agree to play a new game.
    \end{enumerate}
    \vspace{0.5cm}

    \item What does Ava do after Liam apologizes?
    \begin{enumerate}[label=\Alph*.]
        \item She forgives him and agrees to try something new.
        \item She accepts his apology and is willing to continue playing frisbee.
        \item She refuses to forgive him.
        \item She continues to argue.
    \end{enumerate}
    \vspace{0.5cm}

    \item Which statement best describes the mood of the story after the argument?
    \begin{enumerate}[label=\Alph*.]
        \item The mood is tense and uncomfortable.
        \item The mood is happy and lighthearted.
        \item The mood is angry and hostile.
        \item The mood is sad and regretful.
    \end{enumerate}
    \vspace{0.5cm}

    \item How does the author show that Liam and Ava are able to understand each other better?
    \begin{enumerate}[label=\Alph*.]
        \item They begin to argue more.
        \item They both stop playing frisbee.
        \item They apologize to each other and continue playing frisbee.
        \item They talk about why they like playing frisbee.
    \end{enumerate}
    \vspace{0.5cm}

    \item What does the interaction between Liam and Ava teach us about friendships?
    \begin{enumerate}[label=\Alph*.]
        \item That arguments can sometimes lead to a better understanding between friends.
        \item That friends should never disagree.
        \item That friends should always agree on everything.
        \item That arguments ruin friendships.
    \end{enumerate}
    \vspace{0.5cm}

    \item Why is Liam’s apology important in the story?
    \begin{enumerate}[label=\Alph*.]
        \item It shows that he values their friendship and wants to fix the situation.
        \item It shows that he doesn’t care about the argument.
        \item It shows that he wants to end the game.
        \item It shows that he wants to change the rules of the game.
    \end{enumerate}
    \vspace{0.5cm}

    \item What is the primary reason Ava gets upset with Liam’s suggestion to change the game?
    \begin{enumerate}[label=\Alph*.]
        \item She wants to keep playing frisbee because she enjoys it.
        \item She doesn’t want to play any games anymore.
        \item She is not interested in playing any new games.
        \item She is angry at Liam for being bored.
    \end{enumerate}
    \vspace{0.5cm}

    \item What lesson can we learn from the resolution of the argument between Liam and Ava?
    \begin{enumerate}[label=\Alph*.]
        \item That it’s important to be honest with your friends, even if it causes an argument.
        \item That friends should always agree with each other.
        \item That playing new games is not as fun as playing the same game.
        \item That apologies should never be made during an argument.
    \end{enumerate}

\end{enumerate}

\end{document}
