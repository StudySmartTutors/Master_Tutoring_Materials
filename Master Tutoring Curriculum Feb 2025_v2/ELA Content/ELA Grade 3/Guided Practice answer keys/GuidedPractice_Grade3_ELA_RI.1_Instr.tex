\documentclass[12pt]{article}
\usepackage[a4paper, top=0.8in, bottom=0.7in, left=0.8in, right=0.8in]{geometry}
\usepackage{amsmath}
\usepackage{amsfonts}
\usepackage{graphicx}
\usepackage{float}
\usepackage{fancyhdr}
\usepackage{enumitem}
\usepackage{setspace}
\usepackage{tcolorbox}
\usepackage[defaultfam,tabular,lining]{montserrat}

\setlength{\parindent}{0pt}
\pagestyle{fancy}

\fancyhf{}
\fancyhead[L]{\textbf{Answer Key: Using Quotes to Summarize and Make Inferences}}
\fancyfoot[C]{\footnotesize © Study Smart Tutors}

\begin{document}

\section*{Answer Key: Using Quotes to Summarize and Make Inferences}

\subsection*{Guided Practice: Supporting the Main Idea with Quotations}
\textbf{The main idea of \textit{Why Eat Yogurt?} is that yogurt is a delicious and healthy food. Supporting quotes:}
\begin{enumerate}
    \item “Yogurt is not only delicious but also full of nutrients that are great for your body.”
    \item “Calcium helps keep your bones and teeth strong, which is especially important as you grow.”
    \item “The healthy bacteria in yogurt, called probiotics, are good for your stomach and help with digestion.”
\end{enumerate}

\subsection*{Guided Practice: Writing Inferences}
\textbf{Write down an inference you might be able to make based on the information in the following sentences:}
\begin{enumerate}
    \item The dog wagged its tail wildly as Robert walked through the door.  
    \textbf{Inference:} The dog is excited and happy to see Robert.
    \item Sophie looked in her backpack for her umbrella, but she had left it at home.  
    \textbf{Inference:} Sophie is worried because it might be raining, and she doesn’t have her umbrella.
    \item Dominic got a really high score on the spelling test.  
    \textbf{Inference:} Dominic studied hard for the test or is very good at spelling.
    \item Mary spent all day sleeping.  
    \textbf{Inference:} Mary might be feeling sick, tired, or recovering from a long day.
\end{enumerate}

\subsection*{Independent Practice: Making Inferences and Supporting Evidence}
\begin{enumerate}
    \item "Liam trembled as he looked out the window and saw dark clouds covering the sky. The wind howled, and raindrops began to hit the glass."  
    \textbf{Inference:} Liam might be scared because a storm is starting.  
    \textbf{Evidence:} “Liam trembled,” “dark clouds,” and “wind howled.”
    
    \item "Mia packed her backpack with snacks, water, and a map. She told her mom, 'I’ll be back before dinner,' and headed out the door."  
    \textbf{Inference:} Mia is going on a hike or an adventure.  
    \textbf{Evidence:} “Snacks, water, and a map” and “headed out the door.”

    \item "The little girl held the balloon tightly with both hands."  
    \textbf{Inference:} The little girl is trying not to lose the balloon.  
    \textbf{Evidence:} “Held the balloon tightly.”
\end{enumerate}

\subsection*{Exit Ticket}
\textbf{Sample Answer:}  
Draw a picture of a dog wearing a party hat, sitting at a table with a birthday cake. The setting and accessories should suggest it’s a birthday party without explicitly stating it.

\end{document}
