\documentclass[12pt]{article}

\usepackage[a4paper, top=0.8in, bottom=0.7in, left=0.7in, right=0.7in]{geometry}
\usepackage{amsmath}
\usepackage{graphicx}
\usepackage{fancyhdr}
\usepackage{tcolorbox}
\usepackage{multicol}
\usepackage{pifont} % For checkboxes
\usepackage[defaultfam,tabular,lining]{montserrat} %% Option 'defaultfam'
\usepackage[T1]{fontenc}
\renewcommand*\oldstylenums[1]{{\fontfamily{Montserrat-TOsF}\selectfont #1}}
\renewcommand{\familydefault}{\sfdefault}
\usepackage{enumitem}
\usepackage{setspace}
\usepackage{parcolumns}
\usepackage{tabularx}

\setlength{\parindent}{0pt}
\hyphenpenalty=10000
\exhyphenpenalty=10000

\pagestyle{fancy}
\fancyhf{}
%\fancyhead[L]{\textbf{5.RI.1: Key Details and Inference Practice}}
\fancyhead[R]{\includegraphics[width=1cm]{Round Logo.png}}
\fancyfoot[C]{\footnotesize Study Smart Tutors}

\begin{document}

\subsection*{Key Details and Inference Assessment}
\onehalfspacing

\begin{tcolorbox}[colframe=black!40, colback=gray!0, title=Learning Objective]
\textbf{Objective:} Refer to details and examples in a text to explain what the text says explicitly and make inferences.
\end{tcolorbox}

\subsection*{Part 1: Multiple-Choice Questions}

1. What is the main idea of the passage?\\
"Mountains play a critical role in the water cycle, providing a source of fresh water for millions of people around the world. Snow from mountain peaks melts during warmer months, feeding rivers and streams that flow into valleys and plains. These rivers provide drinking water for nearby communities, irrigation for agriculture, and habitats for wildlife. However, climate change is affecting this delicate balance. Rising temperatures are causing snow to melt earlier in the year, disrupting the natural timing of water availability. This early melting can result in floods during spring and water shortages in summer, when demand is highest. Additionally, the loss of snowpack reduces the long-term storage of water that many ecosystems rely on. Protecting mountain ecosystems is essential not only for maintaining a stable water supply but also for preserving biodiversity. Efforts such as reforestation and reducing greenhouse gas emissions can help mitigate the effects of climate change on these vital environments."\\
\begin{enumerate}[label=\Alph*.]
    \item Mountains only affect local weather patterns.  
    \item Mountains are important for providing water and supporting ecosystems.  
    \item Climate change has no impact on mountain ecosystems.  
    \item Wildlife is unaffected by changes in the water cycle.  
\end{enumerate}

\vspace{1cm}
\newpage
2. What inference can you make from this text?\\
"A local library has been at the center of the community for decades, offering books, study spaces, and a quiet environment for learning. However, as technology advances, fewer people visit the library to check out physical books. Instead, they use e-books and other digital resources from home. To adapt to these changes, the library now offers free Wi-Fi, computer stations, and digital literacy classes to help residents navigate new technology. These efforts have not only attracted a younger audience but have also encouraged more community members to use the library in new ways. For example, parents bring their children to coding workshops, and seniors attend sessions on how to use smartphones and tablets. The library has become a hub for lifelong learning and innovation, demonstrating its ability to evolve with the times while remaining a vital part of the community."\\
\begin{enumerate}[label=\Alph*.]
    \item The library is no longer needed in the community.  
    \item Adapting to technology has helped the library remain relevant.  
    \item People only use the library for digital resources.  
    \item Libraries should focus solely on physical books.  
\end{enumerate}

\vspace{1em}

3. Why is it important to protect mountain ecosystems?\\
"Mountain ecosystems are home to unique plants and animals that cannot survive elsewhere, making them hotspots for biodiversity. These ecosystems also act as natural water reservoirs, storing snow in the winter and releasing it as water in the summer. This process supports rivers and streams that provide water for drinking, farming, and energy production. Unfortunately, mountain ecosystems are under threat from human activities such as deforestation, mining, and urban development, as well as climate change. When these ecosystems are damaged, the entire water cycle is disrupted, leading to droughts, loss of biodiversity, and challenges for farming communities. Protecting mountain ecosystems helps maintain the balance of nature, ensuring clean water for future generations and preserving habitats for countless species. Initiatives like reforestation, sustainable tourism, and stricter regulations on mining can mitigate these threats and safeguard the critical role mountains play in supporting life on Earth."\\
\begin{enumerate}[label=\Alph*.]
    \item Protecting mountain ecosystems is unnecessary.  
    \item Mountain ecosystems have no connection to the water cycle.  
    \item Protecting mountain ecosystems benefits both nature and humans.  
    \item Mountain ecosystems are not affected by human activity.  
\end{enumerate}

\vspace{1cm}

\subsection*{Part 2: Select All That Apply Questions}

4. Select \textbf{all} reasons why mountains are important in the water cycle, according to the passage from question 1:\\
\begin{enumerate}[label=\Alph*.]
    \item They store snow that melts into rivers.  
    \item They have no role in providing water.  
    \item They act as natural water reservoirs.  
    \item They provide habitats for plants and animals.  
\end{enumerate}

\vspace{1cm}

5. What actions can help libraries adapt to changing technology, according to the passage from question 2?\\
\begin{enumerate}[label=\Alph*.]
    \item Offering e-books and digital literacy classes.  
    \item Removing all physical books.  
    \item Providing free Wi-Fi and computer access.  
    \item Ignoring the needs of the community.  
\end{enumerate}

\vspace{1cm}

6. What efforts can protect mountain ecosystems, according to the passage from question 3?\\
\begin{enumerate}[label=\Alph*.]
    \item Planting trees to prevent deforestation.  
    \item Encouraging responsible mining practices.  
    \item Ignoring climate change impacts.  
    \item Supporting conservation initiatives.  
\end{enumerate}

\vspace{1cm}

\subsection*{Part 3: Short Answer Questions}

7. How does the melting of snow on mountains impact communities? Use evidence from the passage from question 3\\
\vspace{4cm}

8. Why is it important for libraries to adapt to new technology? Use evidence from the passage from question 2\\
\vspace{4cm}

\subsection*{Part 4: Fill in the Blank Questions}
\vspace{1cm}
9. A paragraph's main idea can usually be found in the  \underline{\hspace{4cm}} sentence.

\vspace{3cm}

10. A summary should be \underline{\hspace{4cm}} in length than the original text.

\vspace{3cm}
% \newpage
% \section*{Answer Key}

% \subsection*{Part 1: Multiple-Choice Questions}

% 1. **B.** Mountains are important for providing water and supporting ecosystems.

% 2. **B.** Adapting to technology has helped the library remain relevant.

% 3. **C.** Protecting mountain ecosystems benefits both nature and humans.

% \subsection*{Part 2: Select All That Apply Questions}

% 4. **A, C, D.**  
%    - They store snow that melts into rivers.  
%    - They act as natural water reservoirs.  
%    - They provide habitats for plants and animals.

% 5. **A, C.**  
%    - Offering e-books and digital literacy classes.  
%    - Providing free Wi-Fi and computer access.

% 6. **A, B, D.**  
%    - Planting trees to prevent deforestation.  
%    - Encouraging responsible mining practices.  
%    - Supporting conservation initiatives.

% \subsection*{Part 3: Short Answer Questions}

% 7. **Sample Answer:** The melting of snow on mountains provides water to rivers and streams, which supports drinking water, farming, and energy production. However, early melting caused by climate change can lead to spring floods and summer water shortages, disrupting communities' access to water.

% 8. **Sample Answer:** Libraries must adapt to new technology to remain relevant in their communities. By offering e-books, free Wi-Fi, computer access, and digital literacy classes, libraries attract a diverse audience and provide valuable resources that meet modern needs.

% \subsection*{Part 4: Fill in the Blank Questions}

% 9. A paragraph's main idea can usually be found in the \underline{topic} sentence.

% 10. A summary should be \underline{shorter} in length than the original text.


\end{document}
