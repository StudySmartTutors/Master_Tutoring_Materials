\documentclass[12pt]{article}

\usepackage[a4paper, top=0.8in, bottom=0.7in, left=0.7in, right=0.7in]{geometry}
\usepackage{amsmath}
\usepackage{graphicx}
\usepackage{fancyhdr}
\usepackage{tcolorbox}
\usepackage{multicol}
\usepackage{pifont} % For checkboxes
\usepackage[defaultfam,tabular,lining]{montserrat} %% Option 'defaultfam'
\usepackage[T1]{fontenc}
\renewcommand*\oldstylenums[1]{{\fontfamily{Montserrat-TOsF}\selectfont #1}}
\renewcommand{\familydefault}{\sfdefault}
\usepackage{enumitem}
\usepackage{setspace}
\usepackage{parcolumns}
\usepackage{tabularx}

\setlength{\parindent}{0pt}
\hyphenpenalty=10000
\exhyphenpenalty=10000

\pagestyle{fancy}
\fancyhf{}
\fancyhead[L]{\textbf{7.RI.2: Informational Text Analysis}}
\fancyhead[R]{\includegraphics[width=1cm]{Round Logo.png}}
\fancyfoot[C]{\footnotesize Study Smart Tutors}

\begin{document}

\onehalfspacing

\section*{Informational Text: Sublimation}

\textbf{Read the passage below and answer the questions that follow.}

Sublimation is a process in which a solid changes directly into a gas without first becoming a liquid. This phenomenon occurs when a substance gains enough energy to overcome the forces that hold its particles in a solid state. The most common example of sublimation is the transition of dry ice (solid carbon dioxide) directly into carbon dioxide gas. When dry ice is exposed to room temperature, it skips the liquid phase and immediately turns into a gas, creating a fog-like effect.

Sublimation is also responsible for the gradual shrinking of snow and ice in cold conditions, even though the temperature may still be below freezing. This process is common in areas with low humidity, where ice or snow can disappear without ever melting into liquid water. Sublimation is a key process in the water cycle, where snow and ice can sublimate directly into water vapor in the atmosphere.

In addition to dry ice, other substances, such as iodine and camphor, also undergo sublimation under specific conditions. The process requires the substance to absorb enough heat energy to overcome the bonds between molecules. Sublimation occurs most readily at low pressure, which is why it is often observed at high altitudes or in a vacuum.

Sublimation has many applications. For instance, it is used in the process of freeze-drying, where water is removed from food or other materials by sublimating the water molecules in a vacuum. This technique is used to preserve food and is a common method for creating freeze-dried meals that are lightweight and easy to store. Sublimation is also used in the creation of certain types of inks, dyes, and chemicals.

\newpage

\section*{Multiple Choice Questions}

\begin{enumerate}

\item What is the main idea of the passage?
\begin{enumerate}[label=\Alph*.]
    \item Sublimation is the process by which a solid changes directly into a liquid.
    \item Sublimation is the process in which a solid changes directly into a gas without becoming a liquid.
    \item Sublimation only occurs in high-pressure environments.
    \item Sublimation only occurs with dry ice.
\end{enumerate}

\vspace{0.5cm}

\item Which of the following is an example of sublimation?
\begin{enumerate}[label=\Alph*.]
    \item Water boiling to become steam
    \item Dry ice turning into carbon dioxide gas
    \item Ice melting into liquid water
    \item Salt dissolving in water
\end{enumerate}

\vspace{0.5cm}

\item What happens when a substance undergoes sublimation?
\begin{enumerate}[label=\Alph*.]
    \item It changes from a gas to a liquid.
    \item It changes from a solid to a gas without becoming a liquid.
    \item It melts into a liquid and then evaporates into a gas.
    \item It evaporates directly into a liquid.
\end{enumerate}

\vspace{0.5cm}

\item What is one example of a substance that undergoes sublimation?
\begin{enumerate}[label=\Alph*.]
    \item Water
    \item Carbon dioxide (dry ice)
    \item Salt
    \item Oil
\end{enumerate}

\vspace{0.5cm}

\item Which of the following is true about sublimation?
\begin{enumerate}[label=\Alph*.]
    \item It requires the substance to lose heat energy.
    \item It occurs only when the substance is at a high pressure.
    \item It occurs most readily at low pressure.
    \item It only occurs in the presence of liquid water.
\end{enumerate}

\vspace{0.5cm}

\item What is one effect of sublimation in cold conditions?
\begin{enumerate}[label=\Alph*.]
    \item Snow and ice melt into liquid water.
    \item Snow and ice disappear without melting into liquid water.
    \item Snow and ice turn into fog.
    \item Snow and ice freeze into solid ice.
\end{enumerate}

\vspace{0.5cm}

\item What role does sublimation play in the water cycle?
\begin{enumerate}[label=\Alph*.]
    \item It helps ice and snow directly turn into water vapor in the atmosphere.
    \item It causes liquid water to freeze into ice.
    \item It helps water from the ocean evaporate into clouds.
    \item It causes clouds to form rain.
\end{enumerate}

\vspace{0.5cm}

\item What is the most common example of sublimation mentioned in the passage?
\begin{enumerate}[label=\Alph*.]
    \item Ice turning into water vapor
    \item Dry ice turning into gas
    \item Snow melting into liquid water
    \item Water evaporating into steam
\end{enumerate}

\vspace{0.5cm}

\item What is needed for sublimation to occur?
\begin{enumerate}[label=\Alph*.]
    \item The substance must lose heat energy.
    \item The substance must absorb enough heat energy to overcome the bonds between molecules.
    \item The substance must cool down to below freezing.
    \item The substance must be exposed to sunlight.
\end{enumerate}

\vspace{0.5cm}

\item Which of the following conditions is necessary for sublimation to occur most readily?
\begin{enumerate}[label=\Alph*.]
    \item High humidity
    \item Low pressure
    \item High temperature
    \item High pressure
\end{enumerate}

\vspace{0.5cm}

\item Which of these substances can undergo sublimation?
\begin{enumerate}[label=\Alph*.]
    \item Ice
    \item Carbon dioxide (dry ice)
    \item Salt
    \item Water
\end{enumerate}

\vspace{0.5cm}

\item What is freeze-drying?
\begin{enumerate}[label=\Alph*.]
    \item A process where food is frozen and then heated to remove water.
    \item A process where food is preserved by turning the water in it into gas.
    \item A process where food is exposed to high pressure to remove water.
    \item A process where water is sublimated from food in a vacuum.
\end{enumerate}

\vspace{0.5cm}

\item How does sublimation help preserve food?
\begin{enumerate}[label=\Alph*.]
    \item It freezes the food to prevent bacterial growth.
    \item It removes the water from the food by turning it directly into vapor.
    \item It makes the food more nutritious.
    \item It changes the food into a dry powder.
\end{enumerate}

\vspace{0.5cm}

\item What does sublimation directly result in for substances like dry ice?
\begin{enumerate}[label=\Alph*.]
    \item The substance changes into a gas.
    \item The substance changes into a liquid.
    \item The substance freezes into a solid.
    \item The substance evaporates into water vapor.
\end{enumerate}

\vspace{0.5cm}

\item In what kind of environment is sublimation most likely to occur?
\begin{enumerate}[label=\Alph*.]
    \item A high-pressure environment
    \item A humid environment
    \item A low-pressure environment
    \item A high-temperature environment
\end{enumerate}

\vspace{0.5cm}

\item What happens when dry ice is exposed to room temperature?
\begin{enumerate}[label=\Alph*.]
    \item It melts into a liquid.
    \item It sublimates directly into a gas.
    \item It freezes into a solid.
    \item It evaporates into water vapor.
\end{enumerate}

\vspace{0.5cm}

\item Which of the following best describes sublimation in simple terms?
\begin{enumerate}[label=\Alph*.]
    \item A process where solid substances melt into liquids.
    \item A process where solid substances change directly into gas.
    \item A process where liquids turn into solids.
    \item A process where gases condense into liquids.
\end{enumerate}

\end{enumerate}

\end{document}
