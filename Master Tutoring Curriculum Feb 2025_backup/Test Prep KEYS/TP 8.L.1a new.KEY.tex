\documentclass[12pt]{article}

\usepackage[a4paper, top=0.8in, bottom=0.7in, left=0.7in, right=0.7in]{geometry}
\usepackage{amsmath}
\usepackage{graphicx}
\usepackage{fancyhdr}
\usepackage{tcolorbox}
\usepackage[defaultfam,tabular,lining]{montserrat} %% Option 'defaultfam'
\usepackage[T1]{fontenc}
\renewcommand*\oldstylenums[1]{{\fontfamily{Montserrat-TOsF}\selectfont #1}}
\renewcommand{\familydefault}{\sfdefault}
\usepackage{enumitem}
\usepackage{setspace}

\setlength{\parindent}{0pt}
\hyphenpenalty=10000
\exhyphenpenalty=10000

\pagestyle{fancy}
\fancyhf{}
\fancyhead[L]{\textbf{8.L.1a: Verbals in Sentences Practice}}
\fancyhead[R]{\includegraphics[width=1cm]{Round Logo.png}}
\fancyfoot[C]{\footnotesize Study Smart Tutors}

\begin{document}

\subsection*{Mastering Verbals: Gerunds, Participles, and Infinitives}
\onehalfspacing

\begin{tcolorbox}[colframe=black!40, colback=gray!0, title=Learning Objective]
\textbf{Objective:} Demonstrate command of standard English grammar by identifying and using verbals (gerunds, participles, and infinitives) effectively in sentences.
\end{tcolorbox}

% \newpage
% \section*{Answer Key}

\subsection*{Part 1: Multiple-Choice Questions}

1. \textbf{Which sentence contains a **gerund**?}  
\textbf{Answer:} A. Running is my favorite way to exercise.  
\textbf{Explanation:} "Running" is a gerund, which is a verb form ending in -ing that functions as a noun.

\vspace{1cm}
2. \textbf{Which sentence contains a **participle**?}  
\textbf{Answer:} A. The crying baby needed a bottle.  
\textbf{Explanation:} "Crying" is a participle, modifying the noun "baby" and functioning as an adjective.

\vspace{1cm}
3. \textbf{Which sentence uses an **infinitive** correctly?}  
\textbf{Answer:} A. To win the game, we must focus on teamwork.  
\textbf{Explanation:} "To win" is an infinitive, which is a verb form preceded by "to" that can function as a noun.

\subsection*{Part 2: Select All That Apply Questions}

4. \textbf{Select all sentences containing gerunds:}  
\textbf{Answer:} A. Swimming is a great way to stay fit. \\
C. He enjoys swimming in the lake.  
\textbf{Explanation:} Both "Swimming" in sentences A and C are gerunds functioning as nouns.

\vspace{1cm}
5. \textbf{Which sentences contain participles?}  
\textbf{Answer:} A. The broken vase lay on the floor. \\
C. The excited students cheered loudly.  
\textbf{Explanation:} "Broken" in sentence A and "excited" in sentence C are participles, modifying the nouns "vase" and "students," respectively.

\vspace{1cm}
6. \textbf{Select all sentences using infinitives:}  
\textbf{Answer:} A. She likes to read novels in her free time. \\
B. To climb a mountain takes determination. \\
D. I love to explore new hiking trails.  
\textbf{Explanation:} "To read," "To climb," and "to explore" are infinitives in sentences A, B, and D, respectively, functioning as nouns.

\subsection*{Part 3: Short Answer Questions}

7. \textbf{Rewrite the sentence to include a participle:}  
\textbf{Answer:} The boy, catching the ball, smiled with excitement.  
\textbf{Explanation:} "Catching" is a participle modifying "the boy."

\vspace{1cm}
8. \textbf{Write a sentence using a gerund, a participle, and an infinitive. Label each verbal in your sentence.}  
\textbf{Answer:} Swimming (gerund) in the pool, I noticed a man running (participle) to join me. To swim (infinitive) every day is my goal.  
\textbf{Explanation:} "Swimming" is a gerund, "running" is a participle, and "To swim" is an infinitive.

\subsection*{Part 4: Fill in the Blank Questions}

9. A **gerund** is a verbal that ends in \underline{–ing} and functions as a \underline{noun} in the sentence.  
\textbf{Answer:} –ing, noun.

10. An **infinitive** is a verbal that begins with \underline{to} and can function as a noun, adjective, or adverb.  
\textbf{Answer:} to.





\end{document}
