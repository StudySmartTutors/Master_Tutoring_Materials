\documentclass[12pt]{article}

\usepackage[a4paper, top=0.8in, bottom=0.7in, left=0.7in, right=0.7in]{geometry}
\usepackage{amsmath}
\usepackage{graphicx}
\usepackage{fancyhdr}
\usepackage{tcolorbox}
\usepackage[defaultfam,tabular,lining]{montserrat} %% Option 'defaultfam'
\usepackage[T1]{fontenc}
\renewcommand*\oldstylenums[1]{{\fontfamily{Montserrat-TOsF}\selectfont #1}}
\renewcommand{\familydefault}{\sfdefault}
\usepackage{enumitem}
\usepackage{setspace}

\setlength{\parindent}{0pt}
\hyphenpenalty=10000
\exhyphenpenalty=10000

\pagestyle{fancy}
\fancyhf{}
\fancyhead[L]{\textbf{8.RI.2: Connections and Distinctions Practice}}
\fancyhead[R]{\includegraphics[width=1cm]{Round Logo.png}}
\fancyfoot[C]{\footnotesize Study Smart Tutors}

\begin{document}

\subsection*{Analyzing Connections and Distinctions Between Ideas, Individuals, and Events}
\onehalfspacing

\begin{tcolorbox}[colframe=black!40, colback=gray!0, title=Learning Objective]
\textbf{Objective:} Analyze how a text makes connections among and distinctions between individuals, ideas, or events.
\end{tcolorbox}

\section*{Answer Key}

\subsection*{Part 1: Multiple-Choice Questions}

1. B) By describing the strategies and countermeasures used by both sides.  
2. B) By tracing the historical origins and cultural significance of cheesemaking.  
3. A) It was initially used for fireworks but later adapted for warfare.

\subsection*{Part 2: Select All That Apply Questions}

4. A) Blockades aimed to starve defenders into submission.  
   B) Siege engines like trebuchets were used to breach walls.  
   C) Defenders poured boiling oil on attackers.  
   D) Disease was a constant threat during sieges.  

5. A) The type of milk affects flavor and texture.  
   B) Aging conditions, like humidity, shape the final product.  
   D) Regional practices led to unique cheeses like Roquefort.

6. A) Gunpowder led to the development of fire arrows and cannons.  
   B) Fireworks became a symbol of Chinese cultural celebrations.  
   C) Smaller armies could challenge larger, well-fortified forces.

\subsection*{Part 3: Short Answer Questions}

7. Medieval siege warfare drove innovations in both offensive and defensive strategies. Attackers developed siege engines like trebuchets, battering rams, and tunnels to breach walls, while defenders used boiling oil, arrows, and other weapons to resist. The development of concentric castles made defenses even stronger. This constant back-and-forth innovation shaped both offense and defense.

8. The cheesemaking process demonstrates ancient ingenuity through the discovery of fermentation and its adaptation into a preservation method. Modern adaptations include technological advancements in mass production, yet traditional methods are still used by artisans, showing the blend of ancient and modern practices in cheesemaking.

\subsection*{Part 4: Fill in the Blank Questions}

9. opinions  
10. different




\end{document}

