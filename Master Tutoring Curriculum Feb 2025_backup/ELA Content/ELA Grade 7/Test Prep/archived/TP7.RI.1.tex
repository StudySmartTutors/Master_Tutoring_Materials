\documentclass[12pt]{article}

\usepackage[a4paper, top=0.8in, bottom=0.7in, left=0.7in, right=0.7in]{geometry}
\usepackage{amsmath}
\usepackage{graphicx}
\usepackage{fancyhdr}
\usepackage{tcolorbox}
\usepackage{multicol}
\usepackage{pifont} % For checkboxes
\usepackage[defaultfam,tabular,lining]{montserrat} %% Option 'defaultfam'
\usepackage[T1]{fontenc}
\renewcommand*\oldstylenums[1]{{\fontfamily{Montserrat-TOsF}\selectfont #1}}
\renewcommand{\familydefault}{\sfdefault}
\usepackage{enumitem}
\usepackage{setspace}
\usepackage{parcolumns}
\usepackage{tabularx}

\setlength{\parindent}{0pt}
\hyphenpenalty=10000
\exhyphenpenalty=10000

\pagestyle{fancy}
\fancyhf{}
\fancyhead[L]{\textbf{7.RI.1: Informational Text Analysis}}
\fancyhead[R]{\includegraphics[width=1cm]{Round Logo.png}}
\fancyfoot[C]{\footnotesize Study Smart Tutors}

\begin{document}

\onehalfspacing

\section*{Informational Text: The Mayan Culture}

\textbf{Read the passage below and answer the questions that follow.}

The ancient Mayan civilization flourished in the region now known as Central America, which includes present-day Mexico, Guatemala, Belize, Honduras, and El Salvador. The Mayans were known for their advanced knowledge in astronomy, mathematics, and architecture. Their civilization reached its peak between the 6th and 9th centuries, although their influence continued until the Spanish arrived in the 16th century.

The Mayans developed a complex writing system that combined hieroglyphs and phonetic symbols. This system allowed them to record important events, such as the lives of their rulers and the details of their religious ceremonies. The Mayan script was often carved into stone monuments, and many of these inscriptions have survived for centuries.

Mayan society was highly structured, with a ruling class of kings and priests at the top. Below them were skilled workers, farmers, and laborers. The Mayans built impressive cities, such as Tikal, Chichen Itza, and Copan, which were centers of political, religious, and economic life. These cities were home to large pyramids, temples, and observatories.

The Mayans were also skilled in agriculture and developed advanced farming techniques. They built terraced fields and used irrigation systems to grow crops such as maize, beans, and squash. Their knowledge of the environment allowed them to produce abundant food, which supported a growing population.

In addition to their achievements in science and agriculture, the Mayans were known for their art and architecture. They created beautiful pottery, jewelry, and textiles, often using bright colors and intricate designs. Their architecture included grand pyramids, palaces, and ball courts, which were used for ceremonial and athletic events.

Although the Mayan civilization declined around the 9th century, many of their cities continued to be inhabited for several centuries after. Today, the descendants of the ancient Mayans still live in Central America, preserving their language, traditions, and cultural practices.

\newpage

\section*{Multiple Choice Questions}

\begin{enumerate}

\item What is the main topic of the passage?
\begin{enumerate}[label=\Alph*.]
    \item The history of Mayan writing
    \item The achievements of the Mayan civilization
    \item The art and architecture of the Mayans
    \item The decline of the Mayan civilization
\end{enumerate}

\vspace{0.5cm}

\item Where was the ancient Mayan civilization located?
\begin{enumerate}[label=\Alph*.]
    \item In North America
    \item In Europe
    \item In Central America
    \item In the Caribbean
\end{enumerate}

\vspace{0.5cm}

\item When did the Mayan civilization reach its peak?
\begin{enumerate}[label=\Alph*.]
    \item Between the 3rd and 5th centuries
    \item Between the 6th and 9th centuries
    \item Between the 10th and 12th centuries
    \item In the 16th century
\end{enumerate}

\vspace{0.5cm}

\item What did the Mayans develop to record important events?
\begin{enumerate}[label=\Alph*.]
    \item A written language using only pictures
    \item A complex writing system with hieroglyphs and phonetic symbols
    \item A system of oral storytelling
    \item A series of stone tablets
\end{enumerate}

\vspace{0.5cm}

\item Where were Mayan inscriptions often carved?
\begin{enumerate}[label=\Alph*.]
    \item On pottery
    \item On wood
    \item On stone monuments
    \item On textiles
\end{enumerate}

\vspace{0.5cm}

\item Who was at the top of the Mayan social hierarchy?
\begin{enumerate}[label=\Alph*.]
    \item Kings and priests
    \item Farmers and laborers
    \item Skilled workers
    \item Merchants and traders
\end{enumerate}

\vspace{0.5cm}

\item Which of the following Mayan cities is mentioned in the passage?
\begin{enumerate}[label=\Alph*.]
    \item Machu Picchu
    \item Teotihuacan
    \item Tikal
    \item Cuzco
\end{enumerate}

\vspace{0.5cm}

\item What was one of the Mayans' major accomplishments in agriculture?
\begin{enumerate}[label=\Alph*.]
    \item They created the first plow
    \item They developed advanced farming techniques, including terraced fields and irrigation
    \item They discovered new types of crops
    \item They built irrigation systems for irrigation-based cities
\end{enumerate}

\vspace{0.5cm}

\item Which of the following crops did the Mayans grow?
\begin{enumerate}[label=\Alph*.]
    \item Maize, beans, and squash
    \item Rice, wheat, and potatoes
    \item Corn, potatoes, and tomatoes
    \item Wheat, tomatoes, and barley
\end{enumerate}

\vspace{0.5cm}

\item What were some of the materials used by the Mayans to create art?
\begin{enumerate}[label=\Alph*.]
    \item Bronze and marble
    \item Bright colors and intricate designs in pottery, jewelry, and textiles
    \item Metal and glass
    \item Wood and ivory
\end{enumerate}

\vspace{0.5cm}

\item What was one of the main uses of the large pyramids built by the Mayans?
\begin{enumerate}[label=\Alph*.]
    \item For ceremonial and religious purposes
    \item For storage of crops
    \item For housing the population
    \item For trade and commerce
\end{enumerate}

\vspace{0.5cm}

\item What were ball courts used for in Mayan cities?
\begin{enumerate}[label=\Alph*.]
    \item For storing water
    \item For athletic and ceremonial events
    \item For political meetings
    \item For growing crops
\end{enumerate}

\vspace{0.5cm}

\item What happened to the Mayan civilization around the 9th century?
\begin{enumerate}[label=\Alph*.]
    \item It was conquered by the Aztecs
    \item It declined but many cities continued to be inhabited for centuries
    \item It moved to another continent
    \item It was taken over by the Spanish immediately
\end{enumerate}

\vspace{0.5cm}

\item How do Mayan descendants preserve their culture today?
\begin{enumerate}[label=\Alph*.]
    \item By practicing the same religion as their ancestors
    \item By learning ancient Mayan languages and cultural practices
    \item By building large pyramids
    \item By adopting modern practices
\end{enumerate}

\vspace{0.5cm}

\item What type of writing system did the Mayans use?
\begin{enumerate}[label=\Alph*.]
    \item A symbol-based system
    \item A phonetic and alphabet-based system
    \item A combination of hieroglyphs and phonetic symbols
    \item An oral-only system
\end{enumerate}

\vspace{0.5cm}

\item Why is the Mayan civilization significant in history?
\begin{enumerate}[label=\Alph*.]
    \item They invented writing
    \item They were known for their advanced knowledge in science and architecture
    \item They created the first wheel
    \item They were the first to discover the Americas
\end{enumerate}

\vspace{0.5cm}

\item What is one characteristic of Mayan art mentioned in the passage?
\begin{enumerate}[label=\Alph*.]
    \item It was mostly abstract
    \item It was created using dark colors and simple designs
    \item It featured intricate designs and bright colors
    \item It used only natural materials
\end{enumerate}

\end{enumerate}

\end{document}
