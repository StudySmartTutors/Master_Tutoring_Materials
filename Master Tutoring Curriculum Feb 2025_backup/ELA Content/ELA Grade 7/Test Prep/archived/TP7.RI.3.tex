\documentclass[12pt]{article}

\usepackage[a4paper, top=0.8in, bottom=0.7in, left=0.7in, right=0.7in]{geometry}
\usepackage{amsmath}
\usepackage{graphicx}
\usepackage{fancyhdr}
\usepackage{tcolorbox}
\usepackage{multicol}
\usepackage{pifont} % For checkboxes
\usepackage[defaultfam,tabular,lining]{montserrat} %% Option 'defaultfam'
\usepackage[T1]{fontenc}
\renewcommand*\oldstylenums[1]{{\fontfamily{Montserrat-TOsF}\selectfont #1}}
\renewcommand{\familydefault}{\sfdefault}
\usepackage{enumitem}
\usepackage{setspace}
\usepackage{parcolumns}
\usepackage{tabularx}

\setlength{\parindent}{0pt}
\hyphenpenalty=10000
\exhyphenpenalty=10000

\pagestyle{fancy}
\fancyhf{}
\fancyhead[L]{\textbf{7.RI.3: Informational Text Analysis}}
\fancyhead[R]{\includegraphics[width=1cm]{Round Logo.png}}
\fancyfoot[C]{\footnotesize Study Smart Tutors}

\begin{document}

\onehalfspacing

\section*{Informational Text: The History of Jello}

\textbf{Read the passage below and answer the questions that follow.}

Jello, the famous gelatin dessert, has a long and interesting history. The story of Jello begins in the late 19th century. In 1897, a man named Pearle Bixby Wait, a carpenter from LeRoy, New York, discovered a way to make gelatin powder easier to use. He created a flavored gelatin mix that could be dissolved in hot water and then cooled to form a solid, jiggly dessert. Bixby Wait sold his invention, and it became an instant hit.

However, it wasn't until 1902 that the Jello brand name became widely known. A man named Frank Woodward, who worked for the Genesee Pure Food Company, bought the rights to the gelatin recipe. Woodward and his company improved the recipe, made the packaging more attractive, and launched an advertising campaign to promote the product. This marketing helped make Jello a household name, and sales skyrocketed.

In the 1920s, Jello became even more popular as it was featured in cookbooks and advertised on the radio. Its easy preparation and variety of flavors made it a favorite dessert for families across America. During the Great Depression in the 1930s, Jello was inexpensive, easy to make, and offered a fun treat for families on a budget. The company used creative marketing to ensure that people continued to buy Jello even during tough economic times.

The brand continued to evolve over the years, introducing new flavors and even becoming a staple of American culture. Jello has been used in countless recipes, from salads to desserts, and it remains a beloved treat in homes across the country today.

Jello’s success can be attributed to its simplicity, creativity, and adaptability. Its transformation from a local invention into a national sensation showcases how a small idea can become a lasting part of American culture.

\newpage

\section*{Multiple Choice Questions}

\begin{enumerate}

\item What is the main idea of the passage?
\begin{enumerate}[label=\Alph*.]
    \item Jello is a complex product that requires special skills to make.
    \item Jello has evolved from a local invention to a widely known dessert.
    \item The history of Jello is primarily about its marketing strategies.
    \item Jello is a modern dessert with no historical significance.
\end{enumerate}

\vspace{0.5cm}

\item Who was the first person to create a flavored gelatin powder?
\begin{enumerate}[label=\Alph*.]
    \item Frank Woodward
    \item Pearle Bixby Wait
    \item Jell-O Factory
    \item Albert Einstein
\end{enumerate}

\vspace{0.5cm}

\item Which company helped promote Jello to a national brand?
\begin{enumerate}[label=\Alph*.]
    \item Genesee Pure Food Company
    \item Kellogg's
    \item The Jello Factory
    \item General Electric
\end{enumerate}

\vspace{0.5cm}

\item How did Jello’s marketing strategy contribute to its success?
\begin{enumerate}[label=\Alph*.]
    \item The company offered free samples.
    \item Jello was advertised on the radio and in cookbooks.
    \item Jello was made in different shapes and colors.
    \item Jello was sold at a low price for a short time.
\end{enumerate}

\vspace{0.5cm}

\item What made Jello popular during the Great Depression?
\begin{enumerate}[label=\Alph*.]
    \item It was cheap, easy to make, and offered a fun treat for families.
    \item It was a luxury item that only wealthy families could afford.
    \item It was sold in large quantities to feed entire families.
    \item It was the only dessert available at the time.
\end{enumerate}

\vspace{0.5cm}

\item Who bought the rights to the Jello recipe from Pearle Bixby Wait?
\begin{enumerate}[label=\Alph*.]
    \item Frank Woodward
    \item The Genesee Pure Food Company
    \item Albert Wait
    \item John Kellogg
\end{enumerate}

\vspace{0.5cm}

\item In what decade did Jello become more popular due to radio advertisements and cookbooks?
\begin{enumerate}[label=\Alph*.]
    \item 1890s
    \item 1920s
    \item 1930s
    \item 1950s
\end{enumerate}

\vspace{0.5cm}

\item What role did Jello play in American culture in the 20th century?
\begin{enumerate}[label=\Alph*.]
    \item It became a symbol of luxury and wealth.
    \item It became a staple of American culture and was used in many recipes.
    \item It was used as a medical treatment for illnesses.
    \item It was mostly popular in restaurants, not homes.
\end{enumerate}

\vspace{0.5cm}

\item What is one reason Jello was easy to make?
\begin{enumerate}[label=\Alph*.]
    \item It required no cooking or special equipment.
    \item It was made from only two ingredients.
    \item It could be eaten immediately after being mixed.
    \item It was the first packaged dessert.
\end{enumerate}

\vspace{0.5cm}

\item What was Pearle Bixby Wait’s profession?
\begin{enumerate}[label=\Alph*.]
    \item Carpenter
    \item Chef
    \item Radio announcer
    \item Marketing expert
\end{enumerate}

\vspace{0.5cm}

\item How did Jello adapt over time?
\begin{enumerate}[label=\Alph*.]
    \item It became a more expensive dessert.
    \item It introduced new flavors and continued to be a beloved treat.
    \item It stopped being sold in stores.
    \item It was used only in recipes and not sold as a dessert.
\end{enumerate}

\vspace{0.5cm}

\item How did Jello’s affordability during the Great Depression impact its sales?
\begin{enumerate}[label=\Alph*.]
    \item Sales decreased because people couldn’t afford it.
    \item Sales remained the same because it was too expensive.
    \item Sales increased as people sought inexpensive desserts.
    \item Sales decreased because it was hard to find in stores.
\end{enumerate}

\vspace{0.5cm}

\item What was one of the improvements made by Frank Woodward to the Jello recipe?
\begin{enumerate}[label=\Alph*.]
    \item The recipe was made with no sugar.
    \item The recipe was improved, and packaging became more attractive.
    \item The recipe was made with only one flavor.
    \item The recipe was sold in smaller packages.
\end{enumerate}

\vspace{0.5cm}

\item Which of the following contributed to Jello’s national recognition?
\begin{enumerate}[label=\Alph*.]
    \item Its high price and scarcity
    \item Its radio advertisements and improved packaging
    \item Its exclusive use in fine dining
    \item Its involvement in political campaigns
\end{enumerate}

\vspace{0.5cm}

\item How did Jello evolve as a product?
\begin{enumerate}[label=\Alph*.]
    \item It changed from a luxury item to a budget-friendly treat.
    \item It became a dessert that could only be made by professionals.
    \item It remained a simple product with no changes to its ingredients.
    \item It became more difficult to make over time.
\end{enumerate}

\vspace{0.5cm}

\item Why did Jello continue to sell well even during tough economic times?
\begin{enumerate}[label=\Alph*.]
    \item It was expensive, making it a symbol of wealth.
    \item It was marketed as a luxury item.
    \item It was cheap, easy to make, and marketed creatively.
    \item It was the only dessert available to buy.
\end{enumerate}

\vspace{0.5cm}

\item What was Jello primarily used for in American homes in the 20th century?
\begin{enumerate}[label=\Alph*.]
    \item It was served as a soup.
    \item It was used in desserts and recipes.
    \item It was consumed only during holidays.
    \item It was used for medical treatments.
\end{enumerate}

\end{enumerate}

\end{document}
