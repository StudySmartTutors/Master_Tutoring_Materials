\documentclass[12pt]{article}

\usepackage[a4paper, top=0.8in, bottom=0.7in, left=0.7in, right=0.7in]{geometry}
\usepackage{amsmath}
\usepackage{graphicx}
\usepackage{fancyhdr}
\usepackage{tcolorbox}
\usepackage[defaultfam,tabular,lining]{montserrat} %% Option 'defaultfam'
\usepackage[T1]{fontenc}
\renewcommand*\oldstylenums[1]{{\fontfamily{Montserrat-TOsF}\selectfont #1}}
\renewcommand{\familydefault}{\sfdefault}
\usepackage{enumitem}
\usepackage{setspace}

\setlength{\parindent}{0pt}
\hyphenpenalty=10000
\exhyphenpenalty=10000

\pagestyle{fancy}
\fancyhf{}
\fancyhead[L]{\textbf{7.W.1: Argumentative Writing Practice}}
\fancyhead[R]{\includegraphics[width=1cm]{Round Logo.png}}
\fancyfoot[C]{\footnotesize Study Smart Tutors}

\begin{document}

\subsection*{Argumentative Writing: Exploring Multiple Perspectives}
\onehalfspacing

\begin{tcolorbox}[colframe=black!40, colback=gray!0, title=Learning Objective]
\textbf{Objective:} Write an argumentative essay that introduces and supports claims with clear reasons and relevant evidence, acknowledging counterclaims.
\end{tcolorbox}
\subsection*{Prompt}

After reading the passages below, write an argumentative essay responding to the question:  
"Should schools require students to wear uniforms?"  
Use evidence from the texts to support your position, address counterclaims, and provide a strong \\conclusion.
\subsection*{Passage 1: The Benefits of School Uniforms}

Supporters of school uniforms argue that they promote equality among students. By requiring everyone to wear the same attire, uniforms reduce the pressure to compete over clothing brands or styles, creating a level playing field. This helps students focus on academics rather than their appearance. Additionally, uniforms improve school safety by making it easier to identify intruders on campus. Teachers and staff can quickly spot someone who doesn’t belong, enhancing security. Uniforms also save families money because parents do not need to buy trendy or expensive clothes for school. Finally, many educators believe that uniforms instill a sense of discipline and professionalism, preparing students for future environments, like \\workplaces, where dress codes are common. While some students may resist the idea of uniforms, the benefits—equality, safety, cost savings, and discipline—make them a valuable policy for schools.
\newpage
\subsection*{Passage 2: The Case Against School Uniforms}

Critics of school uniforms argue that they limit students’ self-expression. Clothing is a way for young people to showcase their individuality, creativity, and cultural heritage. When students are forced to wear uniforms, they lose an important avenue for expressing who they are. Additionally, uniforms can be uncomfortable and may not fit all body types equally, leading to frustration or embarrassment for some\\ students. While proponents claim that uniforms save money, families often must purchase uniforms in addition to regular clothing, which can become expensive. Furthermore, there is little evidence to suggest that uniforms improve academic performance or behavior. Schools with strict uniform policies often continue to face the same challenges as those without them. For these reasons, many believe that schools should focus on other strategies to promote equality and discipline rather than requiring uniforms.

\subsection*{Passage 3: Finding a Middle Ground}

Some educators and parents propose a compromise when it comes to school uniforms. Instead of strict uniforms, schools could implement a dress code that allows for flexibility while maintaining standards for modesty and appropriateness. For example, students could be required to wear plain clothing without logos or offensive graphics, allowing them to express themselves within guidelines. This approach addresses concerns about self-expression while still preventing excessive focus on fashion or expensive clothing. Dress codes can also be adjusted to accommodate cultural or religious attire, ensuring inclusivity. Another solution is to offer uniform options that are more affordable and comfortable, giving families and students choices. Supporters of this middle ground believe that it combines the benefits of both sides: reducing distractions and fostering equality while respecting individuality. Schools that adopt flexible dress codes often find that students are happier and more willing to comply with policies, creating a positive school environment.
\newpage
\subsection*{Instructions for Students}

\begin{enumerate}
    \item **Choose a side.** Decide whether you support requiring school uniforms, oppose them, or agree with a compromise.
    \item **Plan your essay.** Organize your ideas and include:
    \begin{itemize}
        \item A clear claim that states your position.
        \item Reasons and evidence from the texts to support your argument.
        \item Acknowledgment and refutation of counterclaims.
        \item A strong conclusion that reinforces your position.
    \end{itemize}
    \item **Write your essay.** Use formal language and logical reasoning to present your argument.
    \item **Revise and edit.** Check your essay for grammar, clarity, and organization.
\end{enumerate}

\subsection*{Scoring Guide}

Your essay will be evaluated on the following criteria:
\begin{enumerate}
    \item \textbf{Content and Ideas}: Strength of argument, use of evidence, and acknowledgment of counterclaims.
    \item \textbf{Organization}: Clear introduction, logical transitions, and structured paragraphs.
    \item \textbf{Style and Tone}: Formal style, precise language, and strong voice.
    \item \textbf{Conventions}: Proper grammar, punctuation, and spelling.
\end{enumerate}

\end{document}
