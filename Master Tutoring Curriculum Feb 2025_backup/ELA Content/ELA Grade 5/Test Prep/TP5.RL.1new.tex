\documentclass[12pt]{article}

\usepackage[a4paper, top=0.8in, bottom=0.7in, left=0.7in, right=0.7in]{geometry}
\usepackage{amsmath}
\usepackage{graphicx}
\usepackage{fancyhdr}
\usepackage{tcolorbox}
\usepackage{multicol}
\usepackage{pifont} % For checkboxes
\usepackage[defaultfam,tabular,lining]{montserrat} %% Option 'defaultfam'
\usepackage[T1]{fontenc}
\renewcommand*\oldstylenums[1]{{\fontfamily{Montserrat-TOsF}\selectfont #1}}
\renewcommand{\familydefault}{\sfdefault}
\usepackage{enumitem}
\usepackage{setspace}
\usepackage{parcolumns}
\usepackage{tabularx}

\setlength{\parindent}{0pt}
\hyphenpenalty=10000
\exhyphenpenalty=10000

\pagestyle{fancy}
\fancyhf{}
%\fancyhead[L]{\textbf{5.RL.1: Key Details and Inference Practice}}
\fancyhead[R]{\includegraphics[width=1cm]{Round Logo.png}}
\fancyfoot[C]{\footnotesize Study Smart Tutors}

\begin{document}

\subsection*{Key Details and Inference Assessment}
\onehalfspacing

\begin{tcolorbox}[colframe=black!40, colback=gray!0, title=Learning Objective]
\textbf{Objective:} Refer to details and examples in a text to explain what the text says explicitly and make inferences.
\end{tcolorbox}

\subsection*{Part 1: Multiple-Choice Questions}

1. \textbf{What can be inferred about the relationship between Jack and his father?\\}
"Jack loved building model airplanes with his father. They spent hours in the garage, carefully assembling each piece, discussing how planes fly and how they could improve their designs. One day, while rushing to paint a wing, Jack accidentally knocked it off the table, breaking it in half. He hesitated to show his father, afraid of disappointing him. Instead of getting upset, his father smiled and said, 'Mistakes are how we learn, Jack. Let’s fix it together.' They repaired the wing, painted it, and attached it back to the plane. The moment taught Jack about problem-solving and patience. Later, when Jack entered a model plane competition, he used the skills his father had taught him. Jack didn’t win first place, but his father reminded him, 'What matters is that you tried and learned something new.' Their bond grew stronger, showing how shared experiences can build trust and understanding."\\
\begin{enumerate}[label=\Alph*.]
    \item Jack and his father do not enjoy spending time together.  
    \item Jack is afraid to ask his father for help.  
    \item Jack and his father share a bond strengthened through challenges.  
    \item Jack’s father dislikes mistakes.  
\end{enumerate}

\vspace{1cm}
\newpage
2. \textbf{Why does the author describe the storm in such detail?\\}
"A fierce storm swept through the coastal town with howling winds and torrential rain. The winds bent trees to the ground, and waves crashed against the shore, flooding streets and homes. Families scrambled to secure windows and doors as the power flickered. Despite the chaos, neighbors helped one another, sharing supplies and offering shelter. When the storm finally subsided, the damage was evident: broken fences, scattered debris, and flooded basements. However, the community came together to clean up, repair the damage, and rebuild stronger than before. Children drew chalk rainbows on sidewalks to bring cheer, and a local artist painted a mural celebrating the town’s resilience. This story demonstrates not only the storm’s strength but also the unity and determination of the town’s residents in the face of adversity."\\
\begin{enumerate}[label=\Alph*.]
    \item To emphasize the power of the town’s residents.  
    \item To show the storm’s strength and its impact.  
    \item To suggest the town was unprepared for the storm.  
    \item To explain why the town experienced frequent storms.  
\end{enumerate}


\vspace{1em}
3. \textbf{What is the main lesson from the story?}\\
"Lila, a young artist, spent hours creating colorful drawings but often doubted her talent. Her room was filled with unfinished sketches she thought weren’t good enough. One day, her teacher announced a school art competition, encouraging everyone to participate. Reluctantly, Lila submitted one of her drawings. When the winners were announced, she didn’t win, and disappointment made her consider quitting. However, her teacher praised her effort and suggested small improvements. Inspired, Lila began practicing daily, focusing on shading and detail. Months later, she entered another competition and won first place. Her classmates and family celebrated her achievement, but what mattered most to Lila was her growth. She learned that persistence and learning from failure were key to success, and her \\confidence grew with each piece she completed. The story highlights the \\importance of resilience and embracing challenges as opportunities to improve."\\
\begin{enumerate}[label=\Alph*.]
    \item Winning is more important than learning.  
    \item Practice and persistence lead to improvement.  
    \item Art competitions are stressful for young artists.  
    \item It’s better to avoid challenges than face failure.  
\end{enumerate}

\vspace{1cm}

\subsection*{Part 2: Select All That Apply Questions}

4. Which details show how Jack’s father helped him in the story from question 1?\\
\begin{enumerate}[label=\Alph*.]
    \item He scolded Jack for breaking the wing.  
    \item He repaired the broken wing with Jack.  
    \item He told Jack that mistakes are part of learning.  
    \item He taught Jack to take better care of his work.  
\end{enumerate}

\vspace{1cm}

5. Select \textbf{all} actions the residents took after the storm in the story from question 2:\\
\begin{enumerate}[label=\Alph*.]
    \item They ignored the storm’s aftermath.  
    \item They worked together to clean up debris.  
    \item They demonstrated resilience and unity.  
    \item They repaired fences and broken homes.  
\end{enumerate}

\vspace{1cm}

6. What efforts helped Lila grow as an artist in the story from question 3?\\
\begin{enumerate}[label=\Alph*.]
    \item She stopped drawing after losing the competition.  
    \item She kept practicing despite her doubts.  
    \item Her teacher encouraged her to continue drawing.  
    \item She tried again and eventually won first place.  
\end{enumerate}

\vspace{1cm}
\newpage
\subsection*{Part 3: Short Answer Questions}

7. Based on the passage about the storm, how did the town show resilience?\\
\vspace{4cm}

8. How did Lila’s persistence help her achieve her goal in the story from question 3?\\
\vspace{4cm}

\subsection*{Part 4: Fill in the Blank Questions}
\vspace{1em}
9. A character’s \underline{\hspace{4cm}} can influence how they face challenges in a story.

\vspace{3cm}

10. The \underline{\hspace{4cm}} in a story often provides clues about the characters’ actions and decisions.

\vspace{3cm}
% \newpage
% \section*{Answer Key}

% \subsection*{Part 1: Multiple-Choice Questions}

% 1. **C.** Jack and his father share a bond strengthened through challenges.  

% 2. **B.** To show the storm’s strength and its impact.  

% 3. **B.** Practice and persistence lead to improvement.  

% \subsection*{Part 2: Select All That Apply Questions}

% 4. **B, C.**  
%    - He repaired the broken wing with Jack.  
%    - He told Jack that mistakes are part of learning.  

% 5. **B, C, D.**  
%    - They worked together to clean up debris.  
%    - They demonstrated resilience and unity.  
%    - They repaired fences and broken homes.  

% 6. **B, C, D.**  
%    - She kept practicing despite her doubts.  
%    - Her teacher encouraged her to continue drawing.  
%    - She tried again and eventually won first place.  

% \subsection*{Part 3: Short Answer Questions}

% 7. **Sample Answer:** The town showed resilience by coming together to clean up debris, repair fences and homes, and support each other. Acts like painting a mural and children drawing rainbows reflected the community’s determination to rebuild and bring hope after the storm.  

% 8. **Sample Answer:** Lila’s persistence helped her refine her skills and build confidence. Even after losing the first competition, she practiced shading and detail daily, eventually achieving success in another competition and gaining recognition for her hard work.

% \subsection*{Part 4: Fill in the Blank Questions}

% 9. A character’s \underline{attitude} can influence how they face challenges in a story.

% 10. The \underline{context} in a story often provides clues about the characters’ actions and decisions.

\end{document}

