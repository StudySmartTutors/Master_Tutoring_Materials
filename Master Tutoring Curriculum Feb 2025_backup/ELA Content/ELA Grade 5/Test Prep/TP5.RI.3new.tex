\documentclass[12pt]{article}

\usepackage[a4paper, top=0.8in, bottom=0.7in, left=0.7in, right=0.7in]{geometry}
\usepackage{amsmath}
\usepackage{graphicx}
\usepackage{fancyhdr}
\usepackage{tcolorbox}
\usepackage{multicol}
\usepackage{pifont} % For checkboxes
\usepackage[defaultfam,tabular,lining]{montserrat} %% Option 'defaultfam'
\usepackage[T1]{fontenc}
\renewcommand*\oldstylenums[1]{{\fontfamily{Montserrat-TOsF}\selectfont #1}}
\renewcommand{\familydefault}{\sfdefault}
\usepackage{enumitem}
\usepackage{setspace}
\usepackage{parcolumns}
\usepackage{tabularx}

\setlength{\parindent}{0pt}
\hyphenpenalty=10000
\exhyphenpenalty=10000

\pagestyle{fancy}
\fancyhf{}
%\fancyhead[L]{\textbf{5.RI.3: Relationships Between Individuals, Events, and Ideas Practice}}
\fancyhead[R]{\includegraphics[width=1cm]{Round Logo.png}}
\fancyfoot[C]{\footnotesize Study Smart Tutors}

\begin{document}

\subsection*{Relationships Between Individuals, Events, and Ideas}
\onehalfspacing

\begin{tcolorbox}[colframe=black!40, colback=gray!0, title=Learning Objective]
\textbf{Objective:} Explain relationships or interactions between two or more individuals, events, or ideas in a text based on specific information.
\end{tcolorbox}

\subsection*{Part 1: Multiple-Choice Questions}

1. \textbf{How did the events influence the creation of the National Park System?\\}
"In the late 1800s, America’s wilderness faced significant threats from logging, \\mining, and urban expansion. Naturalist John Muir was a vocal advocate for \\preserving these lands. His writings about the beauty of Yosemite Valley inspired public support for conservation. President Theodore Roosevelt, influenced by Muir’s ideas, prioritized the protection of natural landscapes. In 1906, Roosevelt signed the Antiquities Act, enabling the president to designate national monuments. This marked the beginning of systematic efforts to preserve wilderness areas. Over the next decades, the National Park Service was established to manage these protected lands, ensuring that future generations could enjoy their beauty and resources. This effort demonstrated how collaboration between visionaries like Muir and \\policymakers like Roosevelt could lead to transformative changes for the \\environment."\\
\begin{enumerate}[label=\Alph*.]
    \item The events showed that wilderness areas were unimportant.  
    \item Advocacy and government action led to the preservation of natural areas.  
    \item The National Park System was created solely to support tourism.  
    \item Logging and mining expanded due to Roosevelt’s policies.  
\end{enumerate}

\vspace{1cm}
\newpage
2. \textbf{How did the invention of the printing press influence society?\\}
"Before the printing press, books were copied by hand, a slow and expensive process that limited access to information. In 1440, Johannes Gutenberg invented the \\movable-type printing press, revolutionizing the way knowledge was shared. \\Suddenly, books could be produced quickly and cheaply, making them available to a broader audience. This invention fueled the Renaissance by spreading new ideas in science, art, and literature. It also played a crucial role in the Protestant Reformation by enabling the distribution of religious texts. Additionally, the printing press encouraged literacy and education, empowering individuals to question \\traditional authority. By making information accessible, Gutenberg’s invention \\transformed societies and laid the foundation for the modern age of information."\\
\begin{enumerate}[label=\Alph*.]
    \item The printing press limited access to books and knowledge.  
    \item The printing press helped spread knowledge and empower individuals.  
    \item Literacy rates declined after the printing press was invented.  
    \item The printing press had no effect on the Renaissance.  
\end{enumerate}

\vspace{1cm}

3. \textbf{What relationship existed between Rosa Parks and the Civil Rights Movement?\\}
"On December 1, 1955, Rosa Parks refused to give up her seat to a white passenger on a segregated bus in Montgomery, Alabama. Her arrest sparked outrage in the Black community, leading to the Montgomery Bus Boycott, a pivotal event in the Civil Rights Movement. For over a year, African Americans refused to use city buses, causing financial strain on the transit system. Parks’ act of defiance and the success of the boycott highlighted the power of peaceful protest. Her bravery inspired leaders like Martin Luther King Jr., who emerged as a prominent figure during the boycott. Parks’ actions not only challenged unjust laws but also galvanized the fight for racial equality, demonstrating how one individual’s courage can ignite change in society."\\
\begin{enumerate}[label=\Alph*.]
    \item Rosa Parks’ actions had no impact on the Civil Rights Movement.  
    \item Rosa Parks inspired peaceful protests that advanced racial equality.  
    \item The boycott failed to make any significant changes.  
    \item Rosa Parks avoided involvement in the Civil Rights Movement.  
\end{enumerate}

\vspace{1cm}

\subsection*{Part 2: Select All That Apply Questions}

4. Select \textbf{all} contributions made by John Muir to the creation of national parks based on the passage from question 1:\\
\begin{enumerate}[label=\Alph*.]
    \item Writing about the beauty of nature to inspire public support.  
    \item Partnering with Theodore Roosevelt to promote conservation.  
    \item Supporting mining in wilderness areas.  
    \item Advocating for laws to protect natural landscapes.  
\end{enumerate}

\vspace{1cm}

5. What changes did the printing press bring to society, according to the passage from question 2?\\
\begin{enumerate}[label=\Alph*.]
    \item It increased the availability of books.  
    \item It made education accessible to more people.  
    \item It discouraged the spread of new ideas.  
    \item It supported the Renaissance and Reformation.  
\end{enumerate}

\vspace{1cm}

6. How did Rosa Parks’ actions influence the Civil Rights Movement, according to the passage from question 3?\\
\begin{enumerate}[label=\Alph*.]
    \item They inspired the Montgomery Bus Boycott.  
    \item They encouraged peaceful protests for equality.  
    \item They led to no significant changes in segregation laws.  
    \item They brought Martin Luther King Jr. to national prominence.  
\end{enumerate}

\vspace{1cm}

\subsection*{Part 3: Short Answer Questions}

7. Explain how John Muir’s advocacy and Theodore Roosevelt’s policies worked \\together to protect wilderness areas. Use evidence from the passage from question 1.\\
\vspace{4cm}

8. Summarize the impact of the printing press on knowledge and society. \\Use evidence from the passage from question 2.\\
\vspace{4cm}

\subsection*{Part 4: Fill in the Blank Questions}
\vspace{1cm}
9.  \underline{\hspace{4cm}} order organizes information in the order the events or \\steps occurred.

\vspace{3cm}

10. Words like "because," "as a result," "which caused," or "due to" often signal \\a \underline{\hspace{4cm}} relationship.

\vspace{3cm}
\newpage
% \section*{Answer Key}

% \subsection*{Part 1: Multiple-Choice Questions}

% 1. **B.** Advocacy and government action led to the preservation of natural areas.

% 2. **B.** The printing press helped spread knowledge and empower individuals.

% 3. **B.** Rosa Parks inspired peaceful protests that advanced racial equality.

% \subsection*{Part 2: Select All That Apply Questions}

% 4. **A, B, D.**  
%    - Writing about the beauty of nature to inspire public support.  
%    - Partnering with Theodore Roosevelt to promote conservation.  
%    - Advocating for laws to protect natural landscapes.  

% 5. **A, B, D.**  
%    - It increased the availability of books.  
%    - It made education accessible to more people.  
%    - It supported the Renaissance and Reformation.  

% 6. **A, B, D.**  
%    - They inspired the Montgomery Bus Boycott.  
%    - They encouraged peaceful protests for equality.  
%    - They brought Martin Luther King Jr. to national prominence.  

% \subsection*{Part 3: Short Answer Questions}

% 7. **Sample Answer:** John Muir’s writings highlighted the beauty and importance of nature, inspiring public support for conservation. His partnership with Theodore Roosevelt helped secure policies like the Antiquities Act, which preserved wilderness areas. Together, their efforts established the foundation for the National Park System.

% 8. **Sample Answer:** The printing press revolutionized knowledge sharing by making books more affordable and accessible. It fueled the Renaissance by spreading ideas in science, art, and literature and supported the Protestant Reformation by distributing religious texts. It also increased literacy and education, empowering individuals and transforming society.

% \subsection*{Part 4: Fill in the Blank Questions}

% 9. \underline{Chronological} order organizes information in the order the events or steps occurred.

% 10. Words like "because," "as a result," "which caused," or "due to" often signal a \underline{cause-and-effect} relationship.

\end{document}
