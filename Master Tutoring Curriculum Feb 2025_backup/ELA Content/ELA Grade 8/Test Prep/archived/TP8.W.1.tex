\documentclass[12pt]{article}

\usepackage[a4paper, top=0.8in, bottom=0.7in, left=0.7in, right=0.7in]{geometry}

\usepackage{amsmath}
\usepackage{graphicx}
\usepackage{fancyhdr}
\usepackage{tcolorbox}
\usepackage{multicol}
\usepackage{pifont} % For checkboxes
\usepackage[defaultfam,tabular,lining]{montserrat} %% Option 'defaultfam'
\usepackage[T1]{fontenc}
\renewcommand*\oldstylenums[1]{{\fontfamily{Montserrat-TOsF}\selectfont #1}}
\renewcommand{\familydefault}{\sfdefault}
\usepackage{enumitem}
\usepackage{setspace}
\usepackage{parcolumns}
\usepackage{tabularx}

\setlength{\parindent}{0pt}
\hyphenpenalty=10000
\exhyphenpenalty=10000

\pagestyle{fancy}
\fancyhf{}
\fancyhead[L]{\textbf{8.W.1: Argumentative Writing Prompt}}
\fancyhead[R]{\includegraphics[width=1cm]{Round Logo.png}}
\fancyfoot[C]{\footnotesize Study Smart Tutors}

\begin{document}

\onehalfspacing

% Section Title
\subsection*{Writing Prompt: The Use of Tax Money for Public Transportation}

Many cities face a difficult decision about how to use the tax money they collect from residents. One area where this decision is often debated is whether tax money should be spent on improving public transportation systems. Below are two informational texts that describe the pros and cons of investing in public transportation. After reading the texts, write an argumentative essay that takes a clear position on whether you think tax money should be spent on improving public transportation. Use evidence from both texts to support your argument.

% Text 1: Pros of Spending Tax Money on Improving Public Transportation
\subsection*{Text 1: Pros of Spending Tax Money on Improving Public Transportation}

\begin{tcolorbox}[colframe=black!40, colback=gray!5]
\begin{spacing}{1.15}
    One of the biggest challenges facing growing cities is traffic congestion. Expanding public transportation systems like buses, subways, and trains can help reduce the number of cars on the road. Fewer cars mean less traffic, shorter commute times, and lower levels of pollution. Additionally, public transportation systems are often more energy-efficient than private vehicles. With less reliance on cars, cities can reduce their carbon footprint, which helps the environment.

    Public transportation also provides a more affordable way for people to get around. Not everyone can afford a car, and public transportation offers an affordable alternative, especially for low-income individuals. By investing in public transit, cities provide an essential service that helps people access employment, education, and healthcare, improving the quality of life for all residents.

    Furthermore, improving public transportation can create jobs. Building new rail lines, expanding bus routes, and maintaining transportation systems require workers. As cities invest in these projects, they generate employment opportunities and stimulate local economies. Cities that improve their public transit systems often see an increase in business activity as well because people can easily access shops, restaurants, and services.

    Overall, investing in public transportation is a smart decision that reduces traffic, improves air quality, supports low-income residents, and stimulates economic growth.
\end{spacing}
\end{tcolorbox}

\vspace{1cm}

% Text 2: Cons of Spending Tax Money on Improving Public Transportation
\subsection*{Text 2: Cons of Spending Tax Money on Improving Public Transportation}

\begin{tcolorbox}[colframe=black!40, colback=gray!5]
\begin{spacing}{1.15}
    While public transportation has benefits, there are also significant drawbacks to using tax dollars to improve these systems. One of the main concerns is cost. Constructing new public transit infrastructure, such as subways or bus lines, is expensive. Many cities do not have the funds to pay for such projects, and taxpayers may end up footing the bill through higher taxes. Critics argue that there are other areas where tax money could be better spent, such as improving schools, healthcare, or public safety.

    Another concern is that public transportation systems are often underused. In some cities, the demand for buses or trains is low, and running these systems can be a waste of resources. For example, in suburban or rural areas, there may not be enough people to justify the expense of maintaining public transportation. In these areas, people are more likely to drive cars, and public transit can be inefficient and cost-ineffective.

    Finally, public transportation can sometimes be unreliable. Buses and trains may run late or be overcrowded, which makes them less appealing to many people. If public transportation is inconvenient or uncomfortable, people may prefer to drive their own cars instead, defeating the purpose of investing tax money in the system.

    Because of these concerns, some believe that tax money could be better spent on other priorities that have a more immediate and widespread impact on the community.
\end{spacing}
\end{tcolorbox}

\vspace{1cm}

% Writing Prompt
\subsection*{Writing Prompt}

Write an argumentative essay in which you take a position on whether tax money should be spent on improving public transportation. Use evidence from the two texts provided to support your argument. In your essay, be sure to:

\begin{itemize}
    \item Clearly state your position on whether tax money should be spent on improving public transportation.
    \item Use evidence from both texts to support your position.
    \item Address and refute counterarguments from the second text to strengthen your position.
    \item Organize your essay with an introduction, body paragraphs, and a conclusion.
\end{itemize}

Be sure to proofread your essay for clarity, coherence, and grammar. Your response should be well-organized and use specific examples from the texts to support your point of view.

\end{document}
