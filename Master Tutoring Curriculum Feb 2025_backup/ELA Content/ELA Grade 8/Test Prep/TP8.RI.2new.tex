\documentclass[12pt]{article}

\usepackage[a4paper, top=0.8in, bottom=0.7in, left=0.7in, right=0.7in]{geometry}
\usepackage{amsmath}
\usepackage{graphicx}
\usepackage{fancyhdr}
\usepackage{tcolorbox}
\usepackage[defaultfam,tabular,lining]{montserrat} %% Option 'defaultfam'
\usepackage[T1]{fontenc}
\renewcommand*\oldstylenums[1]{{\fontfamily{Montserrat-TOsF}\selectfont #1}}
\renewcommand{\familydefault}{\sfdefault}
\usepackage{enumitem}
\usepackage{setspace}

\setlength{\parindent}{0pt}
\hyphenpenalty=10000
\exhyphenpenalty=10000

\pagestyle{fancy}
\fancyhf{}
\fancyhead[L]{\textbf{8.RI.2: Connections and Distinctions Practice}}
\fancyhead[R]{\includegraphics[width=1cm]{Round Logo.png}}
\fancyfoot[C]{\footnotesize Study Smart Tutors}

\begin{document}

\subsection*{Analyzing Connections and Distinctions Between Ideas, Individuals, and Events}
\onehalfspacing

\begin{tcolorbox}[colframe=black!40, colback=gray!0, title=Learning Objective]
\textbf{Objective:} Analyze how a text makes connections among and distinctions between individuals, ideas, or events.
\end{tcolorbox}

\subsection*{Part 1: Multiple-Choice Questions}

1. \textbf{How does the passage highlight the challenges of medieval siege warfare?}\\
"Medieval siege warfare was a defining feature of conflict during the Middle Ages, demanding strategic thinking and technological ingenuity. Castles and walled cities were built to withstand long sieges, equipped with thick stone walls, watchtowers, and defensive moats. Attackers, determined to conquer these strongholds, used advanced siege technologies, including trebuchets to hurl heavy stones, battering rams to break gates, and ladders to scale walls. They also dug tunnels beneath walls to weaken their foundations.

Blockades were a common strategy to cut off supplies, forcing defenders to endure hunger and thirst. Those inside the walls faced disease due to overcrowding and lack of sanitation. Meanwhile, defenders fought back fiercely, using arrows, boiling oil, and flaming projectiles to repel the invaders. Night raids were often launched to destroy siege equipment and disrupt the attackers’ plans.

This constant battle between offense and defense led to innovations in both \\fortifications and siege tactics. For example, the development of concentric \\castles—fortresses with multiple layers of walls—made breaching defenses even harder. Siege warfare not only reshaped military tactics but also revealed the \\resilience of those defending their homes. Despite its brutality, this era of warfare significantly influenced the architectural and technological advancements of the time."\\  
\begin{enumerate}[label=\Alph*.]
    \item By focusing only on the hardships faced by attackers.  
    \item By describing the strategies and countermeasures used by both sides.  
    \item By explaining the political outcomes of medieval battles.  
    \item By emphasizing the role of disease during sieges.  
\end{enumerate}

\newpage

2. \textbf{How is the central idea developed in the cheesemaking passage?}\\  
"Cheesemaking is one of the oldest forms of food preservation, dating back \\thousands of years. Ancient people likely discovered the process by accident when milk stored in animal stomachs curdled due to the enzyme rennet. Over time, humans turned this chance occurrence into a deliberate method of transforming milk into cheese—a nutritious and long-lasting food.

The process begins with milk, which is treated with bacteria and enzymes to create curds and whey. The curds are then shaped, salted, and aged to develop distinct flavors and textures. Environmental factors such as humidity and temperature during the aging process play a crucial role. For instance, the blue mold in Roquefort cheese comes from the caves where it is aged.

Cheese became essential for survival in early civilizations, offering a way to store nutrients and sustain communities through harsh seasons. Regional specialties like Parmesan from Italy and cheddar from England reflect the cultural significance of cheesemaking. Today, modern technology has made it possible to produce cheese on a large scale, yet many artisans continue to follow traditional methods.

Cheese is more than a food—it is a testament to human ingenuity, blending ancient practices with modern innovations. Its enduring popularity across cultures and \\centuries shows its importance not just for survival, but also as a cherished part of global cuisine."\\  
\begin{enumerate}[label=\Alph*.]
    \item By explaining how cheese is made using only modern methods.  
    \item By tracing the historical origins and cultural significance of cheesemaking.  
    \item By highlighting the environmental challenges of cheesemaking.  
    \item By describing the differences between artisanal and mass-produced cheese.  
\end{enumerate}

\newpage

3. \textbf{What distinction does the passage make about gunpowder’s uses in early China?}\\  
"Gunpowder, one of the most influential inventions in history, was discovered in China during the Tang Dynasty (618–907 CE). Initially created by alchemists \\searching for a magical elixir, it was first used in fireworks and religious ceremonies. These early uses showcased its ability to create spectacular displays of light and sound.

By the Song Dynasty (960–1279 CE), gunpowder had become a critical military tool. The Chinese developed weapons such as fire arrows, explosive bombs, and rudimentary cannons, revolutionizing warfare. These innovations gave smaller \\armies a significant advantage, allowing them to penetrate heavily fortified defenses that had once been nearly impenetrable.

Gunpowder's impact extended beyond the battlefield. It became a symbol of \\celebration and technological achievement, used in festivals and ceremonies across China. Through trade and conquest, knowledge of gunpowder spread to the Middle East and Europe, where it inspired further advancements in weaponry. While initially a product of curiosity and experimentation, gunpowder transformed societies by changing the way wars were fought and how people celebrated milestones. Its dual role as a weapon and a cultural artifact underscores its complexity and importance in history."\\  
\begin{enumerate}[label=\Alph*.]
    \item It was initially used for fireworks but later adapted for warfare.  
    \item Gunpowder was only used for military purposes in ancient China.  
    \item Alchemists used gunpowder to invent cannons during the Tang Dynasty.  
    \item Its primary use remained ceremonial throughout Chinese history.  
\end{enumerate}

\newpage
\subsection*{Part 2: Select All That Apply Questions}

4. Select \textbf{all} details that describe challenges and innovations in medieval siege \\warfare according to the passage from question 1:\\  
\begin{enumerate}[label=\Alph*.]
    \item Blockades aimed to starve defenders into submission.  
    \item Siege engines like trebuchets were used to breach walls.  
    \item Defenders poured boiling oil on attackers.  
    \item Disease was a constant threat during sieges.  
\end{enumerate}

\vspace{1cm}

5. Which details from the passage from question 2 explain the factors \\influencing cheesemaking?\\  
\begin{enumerate}[label=\Alph*.]
    \item The type of milk affects flavor and texture.  
    \item Aging conditions, like humidity, shape the final product.  
    \item Cheesemaking was discovered in the 20th century.  
    \item Regional practices led to unique cheeses like Roquefort.  
\end{enumerate}

\vspace{1cm}

6. Select \textbf{all} details that explain gunpowder’s impact in early China according \\to the passage from question 3: \\  
\begin{enumerate}[label=\Alph*.]
    \item Gunpowder led to the development of fire arrows and cannons.  
    \item Fireworks became a symbol of Chinese cultural celebrations.  
    \item Smaller armies could challenge larger, well-fortified forces.  
    \item Gunpowder remained confined to China until the 20th century.  
\end{enumerate}

\vspace{1cm}

\subsection*{Part 3: Short Answer Questions}

7. How did medieval siege warfare drive innovations in both offensive and defensive strategies? Cite evidence from the text to support your answer.  
\vspace{4cm}

8. Explain how the cheesemaking process demonstrates both ancient ingenuity and modern adaptation. Use evidence from the passage to support your response.  
\vspace{4cm}

\subsection*{Part 4: Fill in the Blank Questions}
\vspace{1cm}
9. A text may be biased if the author includes their own \underline{\hspace{4cm}} instead of remaining objectively truthful.  
\vspace{2cm}

10. Distinctions in a text explain how individuals, ideas, or events are\\ \underline{\hspace{4cm}} from one another.  
\vspace{2cm}
\newpage
\section*{Answer Key}

\subsection*{Part 1: Multiple-Choice Questions}

1. B) By describing the strategies and countermeasures used by both sides.  
2. B) By tracing the historical origins and cultural significance of cheesemaking.  
3. A) It was initially used for fireworks but later adapted for warfare.

\subsection*{Part 2: Select All That Apply Questions}

4. A) Blockades aimed to starve defenders into submission.  
   B) Siege engines like trebuchets were used to breach walls.  
   C) Defenders poured boiling oil on attackers.  
   D) Disease was a constant threat during sieges.  

5. A) The type of milk affects flavor and texture.  
   B) Aging conditions, like humidity, shape the final product.  
   D) Regional practices led to unique cheeses like Roquefort.

6. A) Gunpowder led to the development of fire arrows and cannons.  
   B) Fireworks became a symbol of Chinese cultural celebrations.  
   C) Smaller armies could challenge larger, well-fortified forces.

\subsection*{Part 3: Short Answer Questions}

7. Medieval siege warfare drove innovations in both offensive and defensive strategies. Attackers developed siege engines like trebuchets, battering rams, and tunnels to breach walls, while defenders used boiling oil, arrows, and other weapons to resist. The development of concentric castles made defenses even stronger. This constant back-and-forth innovation shaped both offense and defense.

8. The cheesemaking process demonstrates ancient ingenuity through the discovery of fermentation and its adaptation into a preservation method. Modern adaptations include technological advancements in mass production, yet traditional methods are still used by artisans, showing the blend of ancient and modern practices in cheesemaking.

\subsection*{Part 4: Fill in the Blank Questions}

9. opinions  
10. different
\end{document}
\end{document}
