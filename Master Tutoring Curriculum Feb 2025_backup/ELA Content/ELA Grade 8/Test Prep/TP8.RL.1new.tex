\documentclass[12pt]{article}

\usepackage[a4paper, top=0.8in, bottom=0.7in, left=0.7in, right=0.7in]{geometry}
\usepackage{amsmath}
\usepackage{graphicx}
\usepackage{fancyhdr}
\usepackage{tcolorbox}
\usepackage[defaultfam,tabular,lining]{montserrat}
\usepackage[T1]{fontenc}
\renewcommand*\oldstylenums[1]{{\fontfamily{Montserrat-TOsF}\selectfont #1}}
\renewcommand{\familydefault}{\sfdefault}
\usepackage{enumitem}
\usepackage{setspace}

\setlength{\parindent}{0pt}
\hyphenpenalty=10000
\exhyphenpenalty=10000

\pagestyle{fancy}
\fancyhf{}
\fancyhead[L]{\textbf{8.RL.1: Textual Evidence and Inference Practice}}
\fancyhead[R]{\includegraphics[width=1cm]{Round Logo.png}}
\fancyfoot[C]{\footnotesize Study Smart Tutors}

\begin{document}

\subsection*{Citing Evidence to Support Explicit and Inferred Meanings}
\onehalfspacing

\begin{tcolorbox}[colframe=black!40, colback=gray!0, title=Learning Objective]
\textbf{Objective:} Cite the strongest textual evidence to support analysis of what the text says explicitly and inferences drawn from the text.
\end{tcolorbox}

\subsection*{Part 1: Multiple-Choice Questions}

1. \textbf{What inference can be made about Liam’s attitude toward change in the community?}\\
"Liam leaned on the fence, watching the construction trucks roll in and out of the field that had once been a central part of his childhood. The sharp hum of machinery filled the air, drowning out the birdsong that used to make the mornings here feel peaceful. The empty field where he played soccer for years was now a maze of mounds, steel beams, and trucks. He sighed, his hand gripping the top of the fence tightly. 'I get it,' he muttered to his best friend, Jay, who stood silently beside him. 'They need more houses, and people have to live somewhere, but… it feels wrong, you know?' Jay nodded but didn’t say much. Liam kicked at the gravel under his feet, the loose stones scattering with each hit. He shook his head. 'This was the one place where everything felt free, like it was ours.' He turned toward the field and stared at it, as though his gaze could bring it back to the way it had been. When a gust of wind blew a stray piece of paper across the ground, Liam’s eyes followed it, and he sighed again. 'I guess nothing stays the same forever,' he whispered, more to himself than to Jay, his voice heavy with sadness."  
\begin{enumerate}[label=\Alph*.]
    \item Liam is excited about the new housing project.  
    \item Liam feels indifferent about the changes in his community.  
    \item Liam is nostalgic and upset about losing the field.  
    \item Liam feels hopeful about what the future might bring.  
\end{enumerate}

\vspace{1cm}
\newpage
2. \textbf{Which evidence best supports the idea that Maria values her education?}\\
"Maria’s alarm rang at 5 AM sharp, and without hesitation, she slid out of bed. Her desk was already set up with her books and notebooks from the night before. She made a quick cup of tea and sat down to study while the rest of her family was still asleep. Her younger siblings often joked about her dedication, calling her 'the early bird.' But Maria didn’t mind. Her notebooks were meticulously organized, each subject marked with colorful tabs and sections. She highlighted key points in different colors to ensure everything was easy to review. When her teachers asked her how she managed to stay so focused, Maria replied with a confident smile, \\'Education is my way to a better future.' After school, when her friends gathered at the park, Maria would head straight to the library. Her dream of becoming a doctor fueled her determination. On particularly tough days, Maria reminded herself of her goals, whispering, 'One day, all this will be worth it.' Even her family noticed her unwavering dedication. Her mother often said, 'Maria’s got her head in the books, but her heart’s in the right place.' Whether it was exams or projects, Maria tackled every challenge with the same level of discipline and passion."  
\begin{enumerate}[label=\Alph*.]
    \item Maria woke up at 5 AM every day to study.  
    \item Maria spent time at the park with her friends.  
    \item Maria’s teachers praised her efforts.  
    \item Maria always highlighted her notes in multiple colors.  
\end{enumerate}
\newpage
\vspace{1cm}

3. \textbf{What explicit evidence shows that the team respected Coach Ellis?}\\
"Coach Ellis wasn’t the loudest or the most demanding coach the team had ever had. But what set him apart was his genuine belief in each player’s potential. During practices, he never singled anyone out to criticize. Instead, he pulled players aside privately to offer tips for improvement. After practice, he often stayed late to help those who wanted to work on specific skills, no matter how tired he was. On game days, Coach Ellis would gather the team in a circle and deliver motivational speeches. 'This game is about heart,' he’d say. 'Win or lose, I want to see you give it everything you’ve got.' The players listened intently, nodding along as if his words were a mantra. One player said, 'Coach Ellis makes you feel like you’re more than just a part of the team—you’re essential.' After their championship victory, the team didn’t just celebrate the trophy; they celebrated their coach. They hoisted the trophy high and then began chanting his name: 'Ellis! Ellis!' The sound echoed through the stadium, and for a moment, Coach Ellis simply stood there, overcome with emotion. He later told a reporter, 'Seeing the respect and love from these players means more to me than any championship ever could.'"  
\begin{enumerate}[label=\Alph*.]
    \item Coach Ellis stayed late to help players practice.  
    \item The team chanted Coach Ellis's name after their victory.  
    \item Coach Ellis gave motivational speeches before games.  
    \item Coach Ellis wasn’t the loudest or most demanding.  
\end{enumerate}



\newpage
\subsection*{Part 2: Select All That Apply Questions}

4. Select \textbf{all} details in the passage from question 1 that show Liam feels nostalgic about the field:  
\begin{enumerate}[label=\Alph*.]
    \item Liam leaned on the fence, watching the construction trucks.  
    \item Liam sighed and ran a hand through his hair.  
    \item Liam said, 'This place used to be ours.'  
    \item Liam smiled and welcomed the changes.  
\end{enumerate}

\vspace{1cm}

5. Which details in the passage from question 2 support the idea that Maria values education?  
\begin{enumerate}[label=\Alph*.]
    \item Maria woke up at 5 AM every morning.  
    \item Maria spent time organizing her notes.  
    \item Maria thanked her teachers for their praise.  
    \item Maria always went straight to the library after school.  
\end{enumerate}

\vspace{1cm}

6. Select \textbf{all} details in the passage from question 3 that show the team respected Coach Ellis:  
\begin{enumerate}[label=\Alph*.]
    \item The team stayed late to help Coach Ellis.  
    \item The players chanted, 'Ellis! Ellis!'  
    \item The team believed Coach Ellis was fair and dedicated.  
    \item The team ignored Coach Ellis’s advice.  
\end{enumerate}
\newpage
\subsection*{Part 3: Short Answer Questions}

7. What does Liam’s behavior suggest about his feelings toward the changes in his community? Use evidence from the passage from question 1 to support your answer.  
\vspace{4cm}

8. Based on the story about Maria in question 2, explain how her actions reflect her values. Use specific evidence from the passage.  
\vspace{4cm}

\subsection*{Part 4: Fill in the Blank Questions}
\vspace{1cm}
9. Textual evidence is used to support both \underline{\hspace{4cm}} statements\\ and \underline{\hspace{4cm}} drawn from the text.  
\vspace{2cm}

10. When analyzing a text, strong evidence should be both \\\underline{\hspace{4cm}} and \underline{\hspace{4cm}}.  
\vspace{2cm}
\newpage
\section*{Answer Key}

\subsection*{Part 1: Multiple-Choice Questions}

1. \textbf{What inference can be made about Liam’s attitude toward change in the community?}
\begin{enumerate}[label=\Alph*.]
    \item \textbf{C} Liam is nostalgic and upset about losing the field.
\end{enumerate}

2. \textbf{Which evidence best supports the idea that Maria values her education?}
\begin{enumerate}[label=\Alph*.]
    \item \textbf{A} Maria woke up at 5 AM every day to study.
\end{enumerate}

3. \textbf{What explicit evidence shows that the team respected Coach Ellis?}
\begin{enumerate}[label=\Alph*.]
    \item \textbf{B} The team chanted Coach Ellis's name after their victory.
\end{enumerate}

\subsection*{Part 2: Select All That Apply Questions}

4. Select \textbf{all} details in the passage from question 1 that show Liam feels nostalgic about the field:
\begin{enumerate}[label=\Alph*.]
    \item \textbf{C} Liam said, 'This place used to be ours.'
\end{enumerate}

5. Which details in the passage from question 2 support the idea that Maria values education?
\begin{enumerate}[label=\Alph*.]
    \item \textbf{A} Maria woke up at 5 AM every morning.
    \item \textbf{B} Maria spent time organizing her notes.
    \item \textbf{D} Maria always went straight to the library after school.
\end{enumerate}

6. Select \textbf{all} details in the passage from question 3 that show the team respected Coach Ellis:
\begin{enumerate}[label=\Alph*.]
    \item \textbf{B} The players chanted, 'Ellis! Ellis!'
    \item \textbf{C} The team believed Coach Ellis was fair and dedicated.
\end{enumerate}

\subsection*{Part 3: Short Answer Questions}

7. \textbf{What does Liam’s behavior suggest about his feelings toward the changes in his community? Use evidence from the passage from question 1 to support your answer.}
\textbf{Sample Answer:} Liam's behavior shows that he is upset and nostalgic about the changes in his community. He is seen sighing, gripping the fence tightly, and reflecting on how the field used to be a place where everything felt free, showing his sadness and discomfort with the loss of the field.

8. \textbf{Based on the story about Maria in question 2, explain how her actions reflect her values. Use specific evidence from the passage.}
\textbf{Sample Answer:} Maria's actions reflect her value of education through her dedication and discipline. She wakes up early to study, organizes her notes with care, and goes to the library after school instead of spending time with friends, demonstrating her commitment to her academic goals and her desire for a better future.

\subsection*{Part 4: Fill in the Blank Questions}

9. Textual evidence is used to support both \underline{explicit} statements and \underline{inferences} drawn from the text.

10. When analyzing a text, strong evidence should be both \underline{relevant} and \underline{credible}.

\end{document}

\end{document}

\subsection*{Part 2: Select All That Apply Questions}

4. Select \textbf{all} details in the passage from question 1 that show Liam feels nostalgic about the field:  
\begin{enumerate}[label=\Alph*.]
    \item Liam leaned on the fence, watching the construction trucks.  
    \item Liam sighed and ran a hand through his hair.  
    \item Liam said, 'This place used to be ours.'  
    \item Liam smiled and welcomed the changes.  
\end{enumerate}

\vspace{1cm}

5. Which details in the passage from question 2 support the idea that Maria values education?  
\begin{enumerate}[label=\Alph*.]
    \item Maria woke up at 5 AM every morning.  
    \item Maria spent time organizing her notes.  
    \item Maria thanked her teachers for their praise.  
    \item Maria always went straight to the library after school.  
\end{enumerate}

\vspace{1cm}

6. Select \textbf{all} details in the passage from question 3 that show the team respected Coach Ellis:  
\begin{enumerate}[label=\Alph*.]
    \item The team stayed late to help Coach Ellis.  
    \item The players chanted, 'Ellis! Ellis!'  
    \item The team believed Coach Ellis was fair and dedicated.  
    \item The team ignored Coach Ellis’s advice.  
\end{enumerate}

\subsection*{Part 3: Short Answer Questions}

7. What does Liam’s behavior suggest about his feelings toward the changes in his community? Use evidence from the passage to support your answer.  
\vspace{4cm}

8. Based on the story about Maria, explain how her actions reflect her values. Use specific evidence from the passage.  
\vspace{4cm}

\subsection*{Part 4: Fill in the Blank Questions}

9. Textual evidence is used to support both \underline{\hspace{4cm}} statements and \underline{\hspace{4cm}} drawn from the text.  
\vspace{2cm}

10. When analyzing a text, strong evidence should be both \underline{\hspace{4cm}} and \underline{\hspace{4cm}}.  
\vspace{2cm}

\end{document}
