\documentclass[12pt]{article}

\usepackage[a4paper, top=0.8in, bottom=0.7in, left=0.7in, right=0.7in]{geometry}
\usepackage{amsmath}
\usepackage{graphicx}
\usepackage{fancyhdr}
\usepackage{tcolorbox}
\usepackage{multicol}
\usepackage{pifont} % For checkboxes
\usepackage[defaultfam,tabular,lining]{montserrat} %% Option 'defaultfam'
\usepackage[T1]{fontenc}
\renewcommand*\oldstylenums[1]{{\fontfamily{Montserrat-TOsF}\selectfont #1}}
\renewcommand{\familydefault}{\sfdefault}
\usepackage{enumitem}
\usepackage{setspace}
\usepackage{parcolumns}
\usepackage{tabularx}

\setlength{\parindent}{0pt}
\hyphenpenalty=10000
\exhyphenpenalty=10000

\pagestyle{fancy}
\fancyhf{}
%\fancyhead[L]{\textbf{4.RI.3: Text Relationships Practice}}
\fancyhead[R]{\includegraphics[width=1cm]{Round Logo.png}}
\fancyfoot[C]{\footnotesize Study Smart Tutors}

\begin{document}

\subsection*{Analyzing Connections Between Events, Ideas, and Concepts}
\onehalfspacing

\begin{tcolorbox}[colframe=black!40, colback=gray!0, title=Learning Objective]
\textbf{Objective:} Explain connections and relationships between events, ideas, or concepts in a text using language that pertains to cause/effect and comparison/contrast.
\end{tcolorbox}

\subsection*{Part 1: Multiple-Choice Questions}

1. How does the passage explain the relationship between clean energy and pollution?\\
"Clean energy sources like solar and wind power produce electricity without releasing harmful pollutants into the air. Fossil fuels, on the other hand, emit greenhouse gases and toxins when burned. This pollution contributes to climate change and poor air quality. By transitioning to clean energy, communities can reduce their carbon footprint and improve public health. Governments are investing in renewable energy projects to promote cleaner environments and sustainable growth."\\
\begin{enumerate}[label=\Alph*.]
    \item Clean energy creates more pollution than fossil fuels.  
    \item Clean energy helps reduce pollution and improve air quality.  
    \item Clean energy is too expensive to reduce pollution.  
    \item Clean energy and fossil fuels create the same amount of pollution.  
\end{enumerate}

\vspace{1cm}

2. What is the cause of the declining bee population, according to the passage?\\
"Bees play a critical role in pollination, but their numbers are decreasing. Habitat loss, pesticide use, and climate change are key factors driving this decline. Without bees, many crops and wild plants would struggle to reproduce. Efforts to protect bees include creating wildflower habitats, reducing pesticide use, and raising \\awareness about their importance."\\
\begin{enumerate}[label=\Alph*.]
    \item An increase in bee habitats.  
    \item Habitat loss, pesticide use, and climate change.  
    \item More bees working in pollination.  
    \item A decrease in awareness about bees.  
\end{enumerate}

\vspace{1cm}

3. How does the text compare solar and wind energy?\\
"Solar and wind energy are both renewable sources of power that help reduce \\greenhouse gas emissions. Solar panels capture energy from the sun, making them effective in sunny climates. Wind turbines, however, harness energy from wind, which is more abundant in open, flat areas. While both energy sources are sustainable, solar panels are easier to install on homes, while wind turbines are often used for large-scale power generation."\\
\begin{enumerate}[label=\Alph*.]
    \item Solar energy is less sustainable than wind energy.  
    \item Solar and wind energy are similar but suited to different environments.  
    \item Wind energy is more efficient than solar energy in sunny areas.  
    \item Solar and wind energy are identical in their use and function.  
\end{enumerate}




\subsection*{Part 2: Short Answer Questions}

4. Based on the passage below, explain how trees help prevent soil erosion: \\
"Trees play a vital role in stabilizing soil. Their roots anchor the soil, preventing it from being washed away by rain or blown away by wind. In areas where trees are removed, soil erosion can lead to loss of fertile land and increased sediment in rivers, harming aquatic ecosystems. Reforestation efforts help restore these benefits by planting new trees in deforested areas."\\
\vspace{3cm}

5. Summarize the relationship between exercise and mental health: \\
"Regular exercise has been shown to improve mental health by reducing stress and anxiety. Physical activity releases endorphins, chemicals in the brain that promote feelings of happiness. Additionally, exercise improves sleep patterns and boosts self-esteem. People who engage in regular physical activity often report feeling more energetic and focused."\\
\vspace{4cm}

\subsection*{Part 3: Select All That Apply}

6. Select \textbf{all} cause-and-effect relationships from the passage: \\
"Deforestation leads to habitat loss for countless species, many of which face extinction as a result. Additionally, removing trees reduces the amount of carbon dioxide absorbed from the atmosphere, contributing to climate change. Efforts to combat deforestation include promoting sustainable farming and logging practices and creating protected areas."\\
\begin{enumerate}[label=\Alph*.]
    \item Deforestation causes habitat loss.  
    \item Protected areas reduce habitat loss.  
    \item Deforestation increases carbon dioxide in the atmosphere.  
    \item Deforestation improves farming practices.  
\end{enumerate}

\vspace{1cm}

7. Which details describe the benefits of clean energy?\\
\begin{enumerate}[label=\Alph*.]
    \item Clean energy reduces greenhouse gas emissions.  
    \item Clean energy creates pollution.  
    \item Clean energy improves public health.  
    \item Clean energy is a renewable resource.  
\end{enumerate}

\vspace{1cm}


8. Select \textbf{all} connections that help explain relationships between events:\\
\begin{enumerate}[label=\Alph*.]
    \item Deforestation reduces biodiversity and causes extinction.  
    \item Exercise improves both physical and mental health.  
    \item Clean energy increases air pollution.  
    \item Trees prevent soil erosion by anchoring the soil.  
\end{enumerate}

\vspace{1cm}

\subsection*{Part 4: Fill in the Blank}
9. Comparing and contrasting ideas helps readers understand their\\ \underline{\hspace{4cm}} and differences.

\vspace{3cm}

10. Understanding \underline{\hspace{4cm}} relationships helps explain why events\\ in a text occur.

\vspace{3cm}
% \newpage
% \section*{Answer Key}

% \subsection*{Part 1: Multiple-Choice Questions}

% B. Clean energy helps reduce pollution and improve air quality.

% B. Habitat loss, pesticide use, and climate change.

% B. Solar and wind energy are similar but suited to different environments.

% \subsection*{Part 2: Short Answer Questions}

% Answer: Trees help prevent soil erosion by anchoring the soil with their roots. This prevents the soil from being washed away by rain or blown away by wind. Without trees, soil erosion can lead to the loss of fertile land and contribute to sediment build-up in rivers, which harms aquatic ecosystems.

% Answer: Exercise improves mental health by reducing stress and anxiety. It releases endorphins in the brain that promote feelings of happiness. Additionally, exercise helps improve sleep patterns, boosts self-esteem, and enhances focus and energy levels.

% \subsection*{Part 3: Select All That Apply}

% A, C.

% Deforestation causes habitat loss.
% Deforestation increases carbon dioxide in the atmosphere.
% A, C, D.

% Clean energy reduces greenhouse gas emissions.
% Clean energy improves public health.
% Clean energy is a renewable resource.
% A, B, D.

% Deforestation reduces biodiversity and causes extinction.
% Exercise improves both physical and mental health.
% Trees prevent soil erosion by anchoring the soil.
% \subsection*{Part 4: Fill in the Blank}

% Comparing and contrasting ideas helps readers understand their \underline{similarities} and differences.

% Understanding \underline{cause-and-effect} relationships helps explain why events in a text occur.
\end{document}
