\documentclass[12pt]{article}

\usepackage[a4paper, top=0.8in, bottom=0.7in, left=0.7in, right=0.7in]{geometry}
\usepackage{amsmath}
\usepackage{graphicx}
\usepackage{fancyhdr}
\usepackage{tcolorbox}
\usepackage{multicol}
\usepackage{pifont} % For checkboxes
%\usepackage{tgadventor}
\usepackage[defaultfam,tabular,lining]{montserrat} %% Option 'defaultfam'
\usepackage[T1]{fontenc}
\renewcommand*\oldstylenums[1]{{\fontfamily{Montserrat-TOsF}\selectfont #1}}
\renewcommand{\familydefault}{\sfdefault}
\usepackage{enumitem}
\usepackage{setspace}
\usepackage{parcolumns}
\usepackage{tabularx}

\setlength{\parindent}{0pt}
\hyphenpenalty=10000
\exhyphenpenalty=10000

\pagestyle{fancy}
\fancyhf{}
%\fancyhead[L]{\textbf{4.RL.4: Vocabulary in Context}}
\fancyhead[R]{\includegraphics[width=1cm]{Round Logo.png}}
\fancyfoot[C]{\footnotesize Study Smart Tutors}

\begin{document}

\onehalfspacing

% Fictional Text - The Lost Treasure

\subsection*{Fictional Text: The Lost Treasure}

\begin{tcolorbox}[colframe=black!40, colback=gray!5]

\begin{spacing}{1.15}
    Emily and her brother, Sam, were exploring the old attic in their grandmother’s house when they found a dusty, leather-bound book. The cover was decorated with strange symbols, and the pages inside were filled with mysterious maps and riddles. "This could be a treasure map!" Sam exclaimed. 

    Emily frowned. "But how can we be sure? This book is ancient, and the map might be out of date." 

    Sam smiled. "Let’s take a closer look. We have nothing to lose!" The two children decided to follow the map, even though it was clear that the journey would be difficult. The clues led them to the nearby forest, where the map said a treasure chest was buried under a large oak tree.

    After hours of searching, Emily finally shouted, "I found it!" She pulled a rusty old chest from the ground, and inside, they discovered gold coins and sparkling jewels. "We did it!" Sam yelled. "We found the treasure!"

\end{spacing}

\end{tcolorbox}

\vspace{4.5cm}

% Multiple Choice Questions

\subsection*{Questions}

\begin{enumerate}

    \item What does the word “frowned” mean in the sentence: “Emily frowned. ‘But how can we be sure?’” 
    \begin{enumerate}[label=\Alph*.]
        \item She smiled happily
        \item She looked confused or unhappy
        \item She laughed loudly
        \item She jumped in excitement
    \end{enumerate}
    \vspace{0.5cm}

    \item In the context of the story, what does the word "ancient" mean?
    \begin{enumerate}[label=\Alph*.]
        \item Old and from the past
        \item New and shiny
        \item Broken and useless
        \item Fun and exciting
    \end{enumerate}
    \vspace{0.5cm}

    \item What does the phrase “take a closer look” mean in the sentence: “Let’s take a closer look”?
    \begin{enumerate}[label=\Alph*.]
        \item Look from a distance
        \item Examine carefully
        \item Ignore the map
        \item Rush through it quickly
    \end{enumerate}
    \vspace{0.5cm}

    \item What does the word “journey” mean in the sentence: “The clues led them to the nearby forest, where the map said a treasure chest was buried under a large oak tree.” 
    \begin{enumerate}[label=\Alph*.]
        \item A long and difficult trip or adventure
        \item A quick walk around the house
        \item A short visit to a friend
        \item A place to sleep for the night
    \end{enumerate}
    \vspace{0.5cm}

    \item What does the word “rusty” mean in the sentence: “She pulled a rusty old chest from the ground”?
    \begin{enumerate}[label=\Alph*.]
        \item New and clean
        \item Old and covered with rust
        \item Bright and shiny
        \item Made of wood
    \end{enumerate}
    \vspace{0.5cm}

    \item In the context of the story, what does the word “discovered” mean in the sentence: “...they discovered gold coins and sparkling jewels”?
    \begin{enumerate}[label=\Alph*.]
        \item They found something
        \item They hid something
        \item They ignored something
        \item They broke something
    \end{enumerate}
    \vspace{0.5cm}

\vspace{3cm}
    \item What does the word “mysterious” mean in the sentence: “The cover was decorated with strange symbols, and the pages inside were filled with mysterious maps and riddles”?
    \begin{enumerate}[label=\Alph*.]
        \item Easy to understand
        \item Full of questions and puzzles
        \item Bright and colorful
        \item Boring and simple
    \end{enumerate}
    \vspace{0.5cm}

    \item In the context of the story, what does the word “exclaimed” mean in the sentence: “Sam exclaimed, ‘This could be a treasure map!’”?
    \begin{enumerate}[label=\Alph*.]
        \item Said in a loud, excited way
        \item Whispered quietly
        \item Said angrily
        \item Asked politely
    \end{enumerate}
    \vspace{0.5cm}

    \item What does the word “treasure” mean in the sentence: “The two children decided to follow the map, even though it was clear that the journey would be difficult”?
    \begin{enumerate}[label=\Alph*.]
        \item A large house
        \item Valuable things like gold and jewels
        \item A piece of clothing
        \item A type of tree
    \end{enumerate}
    \vspace{0.5cm}

    \item What does the word “shouted” mean in the sentence: “Emily finally shouted, ‘I found it!’”?
    \begin{enumerate}[label=\Alph*.]
        \item Whispered
        \item Yelled loudly with excitement
        \item Said quietly
        \item Stayed silent
    \end{enumerate}
    \vspace{0.5cm}


\vspace{3cm}
    \item What does the word “sparkling” mean in the sentence: “...they discovered gold coins and sparkling jewels”?
    \begin{enumerate}[label=\Alph*.]
        \item Dull and plain
        \item Bright and shining with light
        \item Covered in dust
        \item Hard and rough
    \end{enumerate}
    \vspace{0.5cm}

    \item What does the word “symbols” mean in the sentence: “The cover was decorated with strange symbols”?
    \begin{enumerate}[label=\Alph*.]
        \item Words that are easy to understand
        \item Pictures or signs that represent something
        \item Numbers in order
        \item Directions on how to find something
    \end{enumerate}
    \vspace{0.5cm}

    \item What does the word “find” mean in the sentence: “Emily and Sam found a dusty, leather-bound book”?
    \begin{enumerate}[label=\Alph*.]
        \item Lost
        \item Discovered something they didn’t know was there
        \item Ignored
        \item Broke something
    \end{enumerate}
    \vspace{0.5cm}

    \item What does the word “decided” mean in the sentence: “The two children decided to follow the map”?
    \begin{enumerate}[label=\Alph*.]
        \item Chose something
        \item Argued about it
        \item Ignored the map
        \item Forgot to follow the map
    \end{enumerate}
    \vspace{0.5cm}

\end{enumerate}
\newpage
% Answer Key

\section*{Answer Key}

\begin{enumerate}

    \item B. She looked confused or unhappy
    \item A. Old and from the past
    \item B. Examine carefully
    \item A. A long and difficult trip or adventure
    \item B. Old and covered with rust
    \item A. They found something
    \item B. Full of questions and puzzles
    \item A. Said in a loud, excited way
    \item B. Valuable things like gold and jewels
    \item B. Yelled loudly with excitement
    \item B. Bright and shining with light
    \item B. Pictures or signs that represent something
    \item B. Discovered something they didn’t know was there
    \item A. Chose something

\end{enumerate}

\end{document}

