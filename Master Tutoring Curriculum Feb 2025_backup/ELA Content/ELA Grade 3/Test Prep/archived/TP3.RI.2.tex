\documentclass[12pt]{article}

\usepackage[a4paper, top=0.8in, bottom=0.7in, left=0.7in, right=0.7in]{geometry}
\usepackage{amsmath}
\usepackage{graphicx}
\usepackage{fancyhdr}
\usepackage{tcolorbox}
\usepackage{multicol}
\usepackage{pifont} % For checkboxes
%\usepackage{tgadventor}
\usepackage[defaultfam,tabular,lining]{montserrat} %% Option 'defaultfam'
\usepackage[T1]{fontenc}
\renewcommand*\oldstylenums[1]{{\fontfamily{Montserrat-TOsF}\selectfont #1}}
\renewcommand{\familydefault}{\sfdefault}
\usepackage{enumitem}
\usepackage{setspace}
\usepackage{parcolumns}
\usepackage{tabularx}

\setlength{\parindent}{0pt}
\hyphenpenalty=10000
\exhyphenpenalty=10000

\pagestyle{fancy}
\fancyhf{}
%\fancyhead[L]{\textbf{3.RI.2: Identifying Main Idea and Details in Informational Text}}
\fancyhead[R]{\includegraphics[width=1cm]{Round Logo.png}}
\fancyfoot[C]{\footnotesize Study Smart Tutors}

\begin{document}

\onehalfspacing

% Passage

\subsection*{Wolves: Fascinating Creatures of the Wild}

\begin{tcolorbox}[colframe=black!40, colback=gray!5]

\begin{spacing}{1.15}
        Wolves are powerful and intelligent animals found in many parts of the world, and they live in groups called packs. A wolf pack usually consists of a family: parents and their pups, as well as older wolves. The alpha wolf leads the pack and makes important decisions, like where to hunt and when to rest.

    Wolves are carnivores, which means they eat meat. They often hunt in groups to catch large animals like deer, elk, and bison. Wolves use teamwork to surround and bring down their prey. Their strong jaws and sharp teeth help them tear meat. Wolves also communicate with each other through howling, barking, and growling. Howling is especially important because it helps the wolves stay connected to other members of the pack, especially when they are far apart.

    Wolves are excellent hunters, but they are also skilled travelers. They can travel long distances in search of food. Some wolves can cover up to 30 miles in a single day. Because wolves need a lot of space to roam, they live in large areas, from forests to grasslands. However, wolves face many challenges in the wild. Habitat loss, hunting, and human activities have caused the wolf population to decline in some areas.

    Despite their strength and ability to hunt, wolves are often misunderstood and feared. Some people think wolves are dangerous to humans, but this is not true. Wolves usually avoid people and are not a threat unless they feel threatened. However, humans often see wolves as a threat to livestock, which leads to hunting and trapping.

    Wolves play an important role in the ecosystem. They help control populations of herbivores like deer and elk. Without wolves, these animals would overgraze the land, causing harm to plants and other animals. By keeping the balance in nature, wolves are essential for healthy ecosystems.

    Wolves are also social animals. They care for their young, and the pack works together to protect each other. In some places, wolves have been reintroduced to help restore the balance of nature. This shows how important they are to the environment and why it is necessary to protect them.

    Many people are working to protect wolves by creating safe areas for them to live and educating the public about the importance of these animals. By helping to protect wolves, we can make sure that they continue to thrive in the wild.

\end{spacing}

\end{tcolorbox}

\vspace{1.5cm}
% Worksheet Questions

\vspace{0.1cm}

\subsection*{Questions}

\begin{enumerate}

    % Question 1
    \item What is the main idea of the passage?

    \begin{enumerate}[label=\Alph*.]
        \item Wolves are dangerous to humans.
        \item Wolves are important animals that live in packs and face challenges in the wild.
        \item Wolves can fly.
        \item Wolves are the biggest predators in the world.
    \end{enumerate}

    \vspace{0.5cm}

    % Question 2
    \item What do wolves live in?

    \begin{enumerate}[label=\Alph*.]
        \item Homes
        \item Packs
        \item Caves
        \item Families
    \end{enumerate}

    \vspace{0.5cm}

    % Question 3
    \item Who leads the wolf pack?

    \begin{enumerate}[label=\Alph*.]
        \item The strongest wolf
        \item The oldest wolf
        \item The alpha wolf
        \item The fastest wolf
    \end{enumerate}

    \vspace{0.5cm}

    % Question 4
    \item What do wolves eat?

    \begin{enumerate}[label=\Alph*.]
        \item Plants
        \item Insects
        \item Meat
        \item Fruits
    \end{enumerate}

    \vspace{0.1cm}

    % Question 5
    \item How do wolves communicate with each other?

    \begin{enumerate}[label=\Alph*.]
        \item By singing
        \item By howling, barking, and growling
        \item By dancing
        \item By drawing pictures
    \end{enumerate}

    \vspace{0.5cm}

    % Question 6
    \item Why do wolves hunt in groups?

    \begin{enumerate}[label=\Alph*.]
        \item To find food
        \item To stay warm
        \item To play games
        \item To find other wolves
    \end{enumerate}

    \vspace{0.5cm}

    % Question 7
    \item What is the job of the alpha wolf?

    \begin{enumerate}[label=\Alph*.]
        \item To play with the pups
        \item To make decisions for the pack
        \item To hunt alone
        \item To run the fastest
    \end{enumerate}

    \vspace{0.5cm}

    % Question 8
    \item What can wolves eat when they hunt?

    \begin{enumerate}[label=\Alph*.]
        \item Deer and elk
        \item Fish and birds
        \item Insects and berries
        \item Grass and leaves
    \end{enumerate}

    \vspace{0.5cm}

    % Question 9
    \item What do wolves use their strong teeth and jaws for?

    \begin{enumerate}[label=\Alph*.]
        \item To tear meat
        \item To protect themselves from other animals
        \item To dig holes
        \item To build homes
    \end{enumerate}

    \vspace{0.5cm}

    % Question 10
    \item Why are wolves facing challenges in the wild?

    \begin{enumerate}[label=\Alph*.]
        \item There are too many wolves
        \item Wolves are not strong enough
        \item Habitat loss, hunting, and human activities
        \item Wolves are too friendly
    \end{enumerate}

    \vspace{0.5cm}

    % Question 11
    \item How can people help wolves?

    \begin{enumerate}[label=\Alph*.]
        \item By hunting them
        \item By protecting their habitats
        \item By keeping them in zoos
        \item By feeding them
    \end{enumerate}

    \vspace{0.5cm}

    % Question 12
    \item What does howling help wolves do?

    \begin{enumerate}[label=\Alph*.]
        \item Stay in touch with other pack members
        \item Scare away other animals
        \item Attract other animals to hunt
        \item Warn other animals to stay away
    \end{enumerate}

    \vspace{0.5cm}

    % Question 13
    \item What type of animal is a wolf?

    \begin{enumerate}[label=\Alph*.]
        \item A mammal
        \item A bird
        \item A reptile
        \item An amphibian
    \end{enumerate}

    \vspace{0.5cm}

    % Question 14
    \item What is sometimes misunderstood about wolves?

    \begin{enumerate}[label=\Alph*.]
        \item They are dangerous to humans.
        \item They prefer to live and hunt alone.
        \item They are not very strong.
        \item They are not good at hunting.
    \end{enumerate}

    \vspace{0.5cm}

    % Question 15
    \item Why is it important to protect wolves?

    \begin{enumerate}[label=\Alph*.]
        \item Because they are the strongest animals
        \item Because they are endangered due to habitat loss and human activities
        \item Because they are good at swimming
        \item Because they can talk to humans
    \end{enumerate}
    \vspace{0.5cm}
\end{enumerate}
% Answer Key Section
    \newpage
    \section*{Answer Key}
    \begin{enumerate}

        \item B. Wolves are important animals that live in packs and face challenges in the wild.
        \item B. Packs
        \item C. The alpha wolf
        \item C. Meat
        \item B. By howling, barking, and growling
        \item A. To find food
        \item B. To make decisions for the pack
        \item A. Deer and elk
        \item A. To tear meat
        \item C. Habitat loss, hunting, and human activities
        \item B. By protecting their habitats
        \item A. Stay in touch with other pack members
        \item A. A mammal
        \item A. They are dangerous to humans.
        \item B. Because they are endangered due to habitat loss and human activities

    \end{enumerate}

\end{document}

