\documentclass[12pt]{article}

\usepackage[a4paper, top=0.8in, bottom=0.7in, left=0.7in, right=0.7in]{geometry}
\usepackage{amsmath}
\usepackage{graphicx}
\usepackage{fancyhdr}
\usepackage{tcolorbox}
\usepackage{multicol}
\usepackage{pifont} % For checkboxes
%\usepackage{tgadventor}
\usepackage[defaultfam,tabular,lining]{montserrat} %% Option 'defaultfam'
\usepackage[T1]{fontenc}
\renewcommand*\oldstylenums[1]{{\fontfamily{Montserrat-TOsF}\selectfont #1}}
\renewcommand{\familydefault}{\sfdefault}
\usepackage{enumitem}
\usepackage{setspace}
\usepackage{parcolumns}
\usepackage{tabularx}

\setlength{\parindent}{0pt}
\hyphenpenalty=10000
\exhyphenpenalty=10000

\pagestyle{fancy}
\fancyhf{}
%\fancyhead[L]{\textbf{3.RI.8: Identifying the Author's Purpose}}
\fancyhead[R]{\includegraphics[width=1cm]{Round Logo.png}}
\fancyfoot[C]{\footnotesize Study Smart Tutors}

\begin{document}

\onehalfspacing

% Informational Text - The Solar System

\subsection*{Informational Text: The Solar System}

\begin{tcolorbox}[colframe=black!40, colback=gray!5]

\begin{spacing}{1.15}
    Our solar system is made up of the Sun and everything that orbits around it. This includes eight planets, their moons, comets, asteroids, and other space objects. The largest planet in our solar system is Jupiter, while the smallest is Mercury. The planets are divided into two groups: the inner planets and the outer planets. The inner planets—Mercury, Venus, Earth, and Mars—are rocky and closer to the Sun. The outer planets—Jupiter, Saturn, Uranus, and Neptune—are gas giants, much larger and farther from the Sun.

    Earth is the only planet known to support life. The atmosphere, the air that surrounds Earth, contains oxygen and other gases that allow living things to breathe. The Moon is Earth’s only natural satellite, and it orbits around our planet. Other planets, like Mars, have moons too, but none of them are like Earth’s Moon. 

    The Sun is the center of our solar system, and it provides the heat and light needed for life on Earth. It is a star, a giant ball of burning gases. Without the Sun, there would be no life on Earth. The Sun's gravity keeps all the planets and other objects in orbit around it.

    The solar system is also home to many other objects. Comets are icy objects that travel around the Sun. Some comets have long, bright tails that can be seen from Earth. Asteroids are rocky objects that are found mostly in the asteroid belt between Mars and Jupiter. These objects provide important clues about the early solar system.
\end{spacing}

\end{tcolorbox}

\vspace{0.5cm}

% Multiple Choice Questions

\subsection*{Questions}

\begin{enumerate}

    \item What is the center of our solar system?
    \begin{enumerate}[label=\Alph*.]
        \item The Earth
        \item The Sun
        \item Jupiter
        \item Mars
    \end{enumerate}
    \vspace{0.5cm}

    \item Which planet is the smallest in the solar system?
    \begin{enumerate}[label=\Alph*.]
        \item Mercury
        \item Venus
        \item Earth
        \item Mars
    \end{enumerate}
    \vspace{0.5cm}

    \item What is unique about Earth compared to other planets?
    \begin{enumerate}[label=\Alph*.]
        \item Earth is the closest planet to the Sun.
        \item Earth is the only planet known to support life.
        \item Earth is a gas giant.
        \item Earth has no atmosphere.
    \end{enumerate}
    \vspace{0.5cm}

    \item Which group of planets includes Mercury, Venus, Earth, and Mars?
    \begin{enumerate}[label=\Alph*.]
        \item Outer planets
        \item Inner planets
        \item Gas giants
        \item Dwarf planets
    \end{enumerate}
    \vspace{0.5cm}

    \item Which planet is the largest in the solar system?
    \begin{enumerate}[label=\Alph*.]
        \item Mars
        \item Jupiter
        \item Saturn
        \item Uranus
    \end{enumerate}
    \vspace{0.5cm}

    \item What does the Sun provide for life on Earth?
    \begin{enumerate}[label=\Alph*.]
        \item Water
        \item Heat and light
        \item Oxygen
        \item Food
    \end{enumerate}
    \vspace{0.5cm}

    \item What keeps the planets in orbit around the Sun?
    \begin{enumerate}[label=\Alph*.]
        \item The planets' gravity
        \item The Sun's gravity
        \item The Moon's gravity
        \item The Earth's atmosphere
    \end{enumerate}
    \vspace{6.5cm}

    \item What is the Moon's relationship with Earth?
    \begin{enumerate}[label=\Alph*.]
        \item The Moon orbits Earth
        \item The Moon is a planet
        \item The Moon is a satellite of Mars
        \item The Moon is a comet
    \end{enumerate}
    \vspace{0.5cm}

    \item What is the difference between the inner and outer planets?
    \begin{enumerate}[label=\Alph*.]
        \item Inner planets are made of gas, outer planets are rocky.
        \item Inner planets are smaller and rocky, outer planets are larger and made of gas.
        \item Inner planets are farther from the Sun, outer planets are closer.
        \item There is no difference between inner and outer planets.
    \end{enumerate}
    \vspace{0.5cm}

    \item Where is the asteroid belt located?
    \begin{enumerate}[label=\Alph*.]
        \item Between Earth and Mars
        \item Between Mars and Jupiter
        \item Between Jupiter and Saturn
        \item Between Neptune and Pluto
    \end{enumerate}
    \vspace{0.5cm}

    \item What is a comet made of?
    \begin{enumerate}[label=\Alph*.]
        \item Ice and gas
        \item Rock and dust
        \item Ice and rock
        \item Gas and dust
    \end{enumerate}
    \vspace{0.5cm}

    \item Which of the following is NOT a planet in our solar system?
    \begin{enumerate}[label=\Alph*.]
        \item Mercury
        \item Mars
        \item Pluto
        \item Saturn
    \end{enumerate}
    \vspace{0.5cm}

    \item What is the main characteristic of gas giants?
    \begin{enumerate}[label=\Alph*.]
        \item They are made mostly of gas and are very large.
        \item They are rocky and small.
        \item They are close to the Sun.
        \item They have many moons.
    \end{enumerate}
    \vspace{0.5cm}

    \item Which planet has the most moons?
    \begin{enumerate}[label=\Alph*.]
        \item Earth
        \item Jupiter
        \item Saturn
        \item Mars
    \end{enumerate}
    \vspace{0.5cm}

\end{enumerate}
% Answer Key Section
\newpage
\subsection*{Answer Key}

\begin{enumerate}
    \item B. The Sun
    \item A. Mercury
    \item B. Earth is the only planet known to support life.
    \item B. Inner planets
    \item B. Jupiter
    \item B. Heat and light
    \item B. The Sun's gravity
    \item A. The Moon orbits Earth
    \item B. Inner planets are smaller and rocky, outer planets are larger and made of gas.
    \item B. Between Mars and Jupiter
    \item C. Ice and rock
    \item C. Pluto
    \item A. They are made mostly of gas and are very large.
    \item B. Jupiter
\end{enumerate}

\end{document}

