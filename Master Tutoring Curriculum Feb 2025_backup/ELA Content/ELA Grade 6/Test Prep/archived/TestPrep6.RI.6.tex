\documentclass[12pt]{article}
\usepackage[a4paper, top=0.8in, bottom=0.7in, left=0.7in, right=0.7in]{geometry}
\usepackage{amsmath}
\usepackage{graphicx}
\usepackage{fancyhdr}
\usepackage{tcolorbox}
\usepackage{multicol}
\usepackage{pifont} % For checkboxes
\usepackage[defaultfam,tabular,lining]{montserrat} %% Option 'defaultfam'
\usepackage[T1]{fontenc}
\renewcommand*\oldstylenums[1]{{\fontfamily{Montserrat-TOsF}\selectfont #1}}
\renewcommand{\familydefault}{\sfdefault}
\usepackage{enumitem}
\usepackage{setspace}
\usepackage{parcolumns}
\usepackage{tabularx}

\setlength{\parindent}{0pt}

\hyphenpenalty=10000
\exhyphenpenalty=10000

\pagestyle{fancy}
\fancyhf{}
%\fancyhead[L]{\textbf{6.RI.6: Author's Purpose and Point of View}}
\fancyhead[R]{\includegraphics[width=1cm]{Round Logo.png}}
\fancyfoot[C]{\footnotesize Study Smart Tutors}

\begin{document}

\onehalfspacing

% Passage
\subsection*{The Importance of Bees in Our Ecosystem}
\begin{tcolorbox}[colframe=black!40, colback=gray!5]
\begin{spacing}{1.15}
    Bees are more than just insects that buzz around flowers. They play a critical role in the environment by pollinating plants, which helps produce the food we eat. Without bees, the food chain would be severely disrupted, as many crops depend on bees for pollination. In fact, one-third of the food we consume, including fruits, vegetables, and nuts, relies on bee pollination.

    Bees also help maintain biodiversity by pollinating a variety of plants, which in turn supports ecosystems. They contribute to the survival of many different species by helping plants reproduce. Without bees, many plants would not be able to grow, leading to a loss of food for both animals and humans.

    Unfortunately, bee populations are declining. Pesticides, habitat loss, and climate change are among the major threats to bee survival. If we do not act quickly, we risk losing these essential pollinators and causing irreparable damage to our food supply and ecosystems.

    It is important for people to understand the role of bees in the environment and to take steps to protect them. Planting bee-friendly flowers, avoiding harmful pesticides, and supporting policies that protect bee habitats are just a few ways we can help save bees and ensure a healthy, sustainable future for all.
\end{spacing}
\end{tcolorbox}

% Worksheet Questions
\subsection*{Questions}
\begin{enumerate}

    % Multiple Choice Questions
    \item What is the main purpose of the passage?
    \begin{enumerate}[label=\Alph*.]
        \item To explain how bees are important to the environment
        \item To encourage people to stop using pesticides
        \item To describe how bees are harmful to plants
        \item To explain why bees are the only pollinators
    \end{enumerate}

    \vspace{0.5cm}

    \item According to the passage, what would happen if bees were not around?
    \begin{enumerate}[label=\Alph*.]
        \item The food supply would remain unaffected.
        \item Many plants would not be able to reproduce, harming food chains.
        \item Bees would still pollinate plants through artificial means.
        \item The food supply would increase.
    \end{enumerate}

    \vspace{0.5cm}

    \item What is the author's perspective on the importance of bees?
    \begin{enumerate}[label=\Alph*.]
        \item Bees are unimportant to the environment.
        \item Bees are essential to the survival of ecosystems and food supply.
        \item Bees only help in pollinating flowers, not crops.
        \item Bees should be kept away from human activities.
    \end{enumerate}

    \vspace{0.5cm}

    \item Why does the author mention pesticides in the passage?
    \begin{enumerate}[label=\Alph*.]
        \item To suggest that pesticides help increase the bee population
        \item To explain that pesticides are one of the threats to bees
        \item To show how pesticides can improve bee health
        \item To describe how bees can avoid pesticides
    \end{enumerate}

    \vspace{0.5cm}

    \item What does the author hope readers will do after reading the passage?
    \begin{enumerate}[label=\Alph*.]
        \item Stop using all pesticides immediately
        \item Understand the importance of bees and help protect them
        \item Only plant flowers that bees don’t like
        \item Support the hunting of bees to control their population
    \end{enumerate}

    \vspace{0.5cm}

    \item Which of the following is most likely the author's attitude toward the decline in bee populations?
    \begin{enumerate}[label=\Alph*.]
        \item The author is indifferent and does not care.
        \item The author is alarmed and urges action to protect bees.
        \item The author believes the decline is part of nature.
        \item The author thinks it is too late to do anything about it.
    \end{enumerate}

    \vspace{0.5cm}

    \item What is the author's point of view about the role of bees in pollination?
    \begin{enumerate}[label=\Alph*.]
        \item Bees are the only pollinators that matter.
        \item Pollination by bees is not important to food production.
        \item Bees play a vital role in pollinating plants and ensuring food production.
        \item Other animals are more important than bees for pollination.
    \end{enumerate}

    \vspace{0.5cm}

    \item Why does the author emphasize the need for people to act quickly?
    \begin{enumerate}[label=\Alph*.]
        \item Because bee populations are declining and the situation is urgent
        \item Because people are planting too many flowers for bees
        \item Because the decline of bees is not a serious issue
        \item Because humans should not be concerned with nature
    \end{enumerate}

    \vspace{0.5cm}

    \item According to the passage, how can people help save bees?
    \begin{enumerate}[label=\Alph*.]
        \item By planting bee-friendly flowers and avoiding harmful pesticides
        \item By cutting down bee habitats to reduce overcrowding
        \item By using more pesticides to control the bee population
        \item By importing bees from other countries
    \end{enumerate}

    \vspace{0.5cm}

    \item What does the author mean by "bee-friendly flowers"?
    \begin{enumerate}[label=\Alph*.]
        \item Flowers that attract bees and help them thrive
        \item Flowers that are harmful to bees
        \item Flowers that bees do not like
        \item Flowers that are designed to control the bee population
    \end{enumerate}

    \vspace{0.5cm}

    \item What does the author suggest as a possible consequence of not protecting bees?
    \begin{enumerate}[label=\Alph*.]
        \item The decline in bees will have no significant impact.
        \item Bee populations will thrive on their own without human help.
        \item We could face severe disruption to food production and ecosystems.
        \item The problem will be solved by planting more flowers.
    \end{enumerate}

    \vspace{0.5cm}

    \item How does the author support the idea that bees are essential to ecosystems?
    \begin{enumerate}[label=\Alph*.]
        \item By explaining how bees help pollinate plants that sustain biodiversity
        \item By focusing only on the negative aspects of bees
        \item By saying that other animals are more important than bees
        \item By giving examples of bee populations thriving
    \end{enumerate}

    \vspace{0.5cm}

    \item What does the author mean by "irreparable damage"?
    \begin{enumerate}[label=\Alph*.]
        \item That the damage to the environment caused by bee loss can be undone
        \item That the damage to the environment from bee loss cannot be fixed
        \item That the damage will not affect food production
        \item That the damage will only affect certain plants
    \end{enumerate}

    \vspace{0.5cm}

    \item What is the author's primary goal in writing this passage?
    \begin{enumerate}[label=\Alph*.]
        \item To inform readers about the importance of bees and raise awareness of their decline
        \item To convince readers to stop eating fruits and vegetables
        \item To describe the life cycle of bees in great detail
        \item To teach readers how to avoid using all pesticides
    \end{enumerate}

    \vspace{0.5cm}

    \item According to the passage, what role do bees play in biodiversity?
    \begin{enumerate}[label=\Alph*.]
        \item Bees help plants reproduce, supporting biodiversity.
        \item Bees reduce biodiversity by killing plants.
        \item Bees are not important for biodiversity.
        \item Bees only pollinate flowers, not crops.
    \end{enumerate}

    \vspace{0.5cm}

    \item How does the author suggest we can protect bees and their habitats?
    \begin{enumerate}[label=\Alph*.]
        \item By cutting down trees and flowers
        \item By planting flowers and supporting policies that protect their habitats
        \item By moving bees to new locations
        \item By stopping all farming activities
    \end{enumerate}

    \vspace{0.5cm}

    \item What tone does the author use throughout the passage?
    \begin{enumerate}[label=\Alph*.]
        \item Urgent and concerned
        \item Lighthearted and humorous
        \item Angry and frustrated
        \item Neutral and indifferent
    \end{enumerate}

    \vspace{0.5cm}

    \item What can be inferred about the author's attitude toward the use of pesticides?
    \begin{enumerate}[label=\Alph*.]
        \item The author believes pesticides are harmless to bees.
        \item The author believes pesticides contribute to the decline in bee populations.
        \item The author thinks pesticides are helpful for farming.
        \item The author does not mention pesticides in the passage.
    \end{enumerate}

\end{enumerate}

\end{document}
