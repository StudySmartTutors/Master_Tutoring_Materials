\documentclass[12pt]{article}

\usepackage[a4paper, top=0.8in, bottom=0.7in, left=0.7in, right=0.7in]{geometry}
\usepackage{amsmath}
\usepackage{graphicx}
\usepackage{fancyhdr}
\usepackage{tcolorbox}
\usepackage{multicol}
\usepackage{pifont} % For checkboxes
%\usepackage{tgadventor}
\usepackage[defaultfam,tabular,lining]{montserrat} %% Option 'defaultfam'
\usepackage[T1]{fontenc}
\renewcommand*\oldstylenums[1]{{\fontfamily{Montserrat-TOsF}\selectfont #1}}
\renewcommand{\familydefault}{\sfdefault}
\usepackage{enumitem}
\usepackage{setspace}
\usepackage{parcolumns}
\usepackage{tabularx}

\setlength{\parindent}{0pt}
\hyphenpenalty=10000
\exhyphenpenalty=10000

\pagestyle{fancy}
\fancyhf{}
%\fancyhead[L]{\textbf{6.L.1: Language Standards Worksheet}}
\fancyhead[R]{\includegraphics[width=1cm]{Round Logo.png}}
\fancyfoot[C]{\footnotesize Study Smart Tutors}

\begin{document}

\onehalfspacing

% Passage
\subsection*{Language Standards - Multiple Choice Questions}

\begin{tcolorbox}[colframe=black!40, colback=gray!5]
\begin{spacing}{1.15}

\noindent Answer the following multiple-choice questions based on your understanding of language standards. Choose the best answer for each question. 

\end{spacing}
\end{tcolorbox}

% Worksheet Questions
\subsection*{Questions}

\begin{enumerate}

    % Question 1
    \item What is the main purpose of using figurative language in a text?

    \begin{enumerate}[label=\Alph*.]
        \item To provide factual information
        \item To create a vivid image or emotional response
        \item To simplify complex ideas
        \item To make the text shorter
    \end{enumerate}

    \vspace{0.5cm}

    % Question 2
    \item Which of the following is an example of a simile?

    \begin{enumerate}[label=\Alph*.]
        \item The moon shone brightly in the night sky.
        \item The grass was as green as emeralds.
        \item He ran as fast as lightning.
        \item She is a shining star.
    \end{enumerate}

    \vspace{0.5cm}

    % Question 3
    \item Which of the following is an example of personification?

    \begin{enumerate}[label=\Alph*.]
        \item The tree swayed in the wind.
        \item The wind whispered through the leaves.
        \item The sun was setting in the sky.
        \item The clouds moved quickly across the sky.
    \end{enumerate}

    \vspace{0.5cm}

    % Question 4
    \item What does the word "metaphor" mean?

    \begin{enumerate}[label=\Alph*.]
        \item A comparison using "like" or "as"
        \item A description of an object in great detail
        \item A comparison without using "like" or "as"
        \item A type of exaggeration
    \end{enumerate}

    \vspace{0.5cm}

    % Question 5
    \item Which of the following sentences contains an example of hyperbole?

    \begin{enumerate}[label=\Alph*.]
        \item He is as tall as a giant.
        \item I have a ton of homework.
        \item She walked through the door.
        \item The dog barked loudly.
    \end{enumerate}

    \vspace{0.5cm}

    % Question 6
    \item What is the purpose of a preposition in a sentence?

    \begin{enumerate}[label=\Alph*.]
        \item To describe an action
        \item To show the relationship between a noun or pronoun and another word
        \item To join two independent clauses
        \item To express a feeling
    \end{enumerate}

    \vspace{0.5cm}

    % Question 7
    \item Which of the following is an example of an idiom?

    \begin{enumerate}[label=\Alph*.]
        \item The ball bounced across the field.
        \item She has a heart of gold.
        \item He was as strong as a lion.
        \item The wind howled through the trees.
    \end{enumerate}

    \vspace{0.5cm}

    % Question 8
    \item Which of these sentences is a compound sentence?

    \begin{enumerate}[label=\Alph*.]
        \item I like reading books.
        \item I want to go outside, but it’s raining.
        \item The dog runs fast.
        \item She plays soccer.
    \end{enumerate}

    \vspace{0.5cm}

    % Question 9
    \item What is the meaning of the word "context"?

    \begin{enumerate}[label=\Alph*.]
        \item The part of the text that explains the main idea
        \item The information surrounding a word or passage that helps explain its meaning
        \item The conclusion of a passage
        \item A dictionary definition of a word
    \end{enumerate}

    \vspace{0.5cm}

    % Question 10
    \item What is the correct definition of "tone" in writing?

    \begin{enumerate}[label=\Alph*.]
        \item The emotion or attitude expressed in the writing
        \item The vocabulary used in the text
        \item The central idea of the passage
        \item The grammatical structure of the text
    \end{enumerate}

    \vspace{0.5cm}

    % Question 11
    \item Which of the following words is an example of a conjunction?

    \begin{enumerate}[label=\Alph*.]
        \item And
        \item Quickly
        \item Tall
        \item Desk
    \end{enumerate}

    \vspace{0.5cm}

    % Question 12
    \item Which of the following is an example of an alliteration?

    \begin{enumerate}[label=\Alph*.]
        \item Peter Piper picked a peck of pickled peppers.
        \item The dog barked loudly.
        \item The sky is blue and clear.
        \item The rain fell gently on the ground.
    \end{enumerate}

    \vspace{0.5cm}

    % Question 13
    \item What is the function of an adverb in a sentence?

    \begin{enumerate}[label=\Alph*.]
        \item To describe a noun
        \item To describe a verb, adjective, or another adverb
        \item To join two sentences together
        \item To show possession
    \end{enumerate}

    \vspace{0.5cm}

    % Question 14
    \item What does the word "synonym" mean?

    \begin{enumerate}[label=\Alph*.]
        \item A word that has the opposite meaning of another word
        \item A word that sounds similar to another word
        \item A word that has the same or similar meaning as another word
        \item A word that describes an action
    \end{enumerate}

    \vspace{0.5cm}

    % Question 15
    \item Which of the following sentences uses the correct punctuation?

    \begin{enumerate}[label=\Alph*.]
        \item I went to the store, and I bought some apples.
        \item I went to the store and, I bought some apples.
        \item I went, to the store and bought, some apples.
        \item I went to the store and bought, some apples.
    \end{enumerate}

    \vspace{0.5cm}

    % Question 16
    \item Which of the following is an example of an antonym?

    \begin{enumerate}[label=\Alph*.]
        \item Hot and cold
        \item Tall and wide
        \item Beautiful and pretty
        \item Dark and light
    \end{enumerate}

    \vspace{0.5cm}

    % Question 17
    \item What does the word "allusion" refer to in literature?

    \begin{enumerate}[label=\Alph*.]
        \item A brief reference to a well-known event or figure
        \item The main idea of the story
        \item A description of the setting
        \item The central character of the story
    \end{enumerate}

    \vspace{0.5cm}

    % Question 18
    \item Which of the following is an example of an imperative sentence?

    \begin{enumerate}[label=\Alph*.]
        \item Please close the door.
        \item I love ice cream.
        \item He is my friend.
        \item We are going to the park.
    \end{enumerate}

    \vspace{0.5cm}

    % Question 19
    \item Which part of speech is the word "quickly"?

    \begin{enumerate}[label=\Alph*.]
        \item Noun
        \item Verb
        \item Adjective
        \item Adverb
    \end{enumerate}

    \vspace{0.5cm}

    % Question 20
    \item What is a complex sentence?

    \begin{enumerate}[label=\Alph*.]
        \item A sentence with one independent clause
        \item A sentence with more than one independent clause
        \item A sentence with one independent clause and at least one dependent clause
        \item A sentence with no subject
    \end{enumerate}

    \vspace{0.5cm}

    % Question 21
    \item What is the function of an interjection in a sentence?

    \begin{enumerate}[label=\Alph*.]
        \item To express emotion or surprise
        \item To describe a noun
        \item To join sentences together
        \item To explain something
    \end{enumerate}

    \vspace{0.5cm}

    % Question 22
    \item Which of the following sentences contains a gerund?

    \begin{enumerate}[label=\Alph*.]
        \item Running is fun.
        \item I will run tomorrow.
        \item She runs fast.
        \item We ran all day.
    \end{enumerate}

    \vspace{0.5cm}

    % Question 23
    \item Which of the following sentences is correct?

    \begin{enumerate}[label=\Alph*.]
        \item They walked to the store, and they bought some milk.
        \item They walked to the store and bought, some milk.
        \item They walked, to the store and bought some milk.
        \item They walked to the store bought, some milk.
    \end{enumerate}

    \vspace{0.5cm}

    % Question 24
    \item What is the role of the subject in a sentence?

    \begin{enumerate}[label=\Alph*.]
        \item It describes the action
        \item It tells who or what the sentence is about
        \item It tells when the action takes place
        \item It shows how the action is done
    \end{enumerate}

    \vspace{0.5cm}

    % Question 25
    \item Which of the following sentences uses an exclamation mark correctly?

    \begin{enumerate}[label=\Alph*.]
        \item Wow, that was an amazing performance!
        \item I am excited to see the movie!
        \item He ran so fast!
        \item All of the above
    \end{enumerate}

\end{enumerate}

\end{document}
