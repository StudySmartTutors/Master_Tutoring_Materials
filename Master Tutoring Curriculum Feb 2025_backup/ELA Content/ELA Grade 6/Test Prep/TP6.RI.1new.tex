\documentclass[12pt]{article}

\usepackage[a4paper, top=0.8in, bottom=0.7in, left=0.7in, right=0.7in]{geometry}
\usepackage{amsmath}
\usepackage{graphicx}
\usepackage{fancyhdr}
\usepackage{tcolorbox}
\usepackage[defaultfam,tabular,lining]{montserrat} %% Option 'defaultfam'
\usepackage[T1]{fontenc}
\renewcommand*\oldstylenums[1]{{\fontfamily{Montserrat-TOsF}\selectfont #1}}
\renewcommand{\familydefault}{\sfdefault}
\usepackage{enumitem}
\usepackage{setspace}

\setlength{\parindent}{0pt}
\hyphenpenalty=10000
\exhyphenpenalty=10000

\pagestyle{fancy}
\fancyhf{}
%\fancyhead[L]{\textbf{6.RI.1: Analyzing Informational Text Practice}}
\fancyhead[R]{\includegraphics[width=1cm]{Round Logo.png}}
\fancyfoot[C]{\footnotesize Study Smart Tutors}

\begin{document}

\subsection*{Citing Evidence in Informational Texts}
\onehalfspacing

\begin{tcolorbox}[colframe=black!40, colback=gray!0, title=Learning Objective]
\textbf{Objective:} Cite textual evidence to support analysis of what the text says explicitly and inferences drawn from the text.
\end{tcolorbox}

\subsection*{Part 1: Multiple-Choice Questions}

1. \textbf{What inference can be made from the passage below?\\}
"Coral reefs are often called the 'rainforests of the sea' because they support an incredible diversity of life. Fish, turtles, and crustaceans make their homes in reefs, while larger animals like sharks rely on reefs for hunting. Reefs also provide benefits to humans, such as protection from coastal erosion and sources for medicine. However, coral reefs face growing threats. Climate change increases ocean temperatures, leading to coral bleaching, where coral expels the algae it needs to survive. Pollution from industrial and agricultural runoff introduces toxins into reef ecosystems, harming marine life. Overfishing depletes fish populations and disrupts ecological balance. To combat these challenges, researchers are developing methods like coral gardening, where fragments of healthy coral are grown in nurseries and replanted. International agreements and local conservation efforts aim to reduce human impact by regulating fishing and limiting coastal development. Saving coral reefs is vital not only for marine life but also for the millions of people who depend on them for livelihoods and food security. These efforts demonstrate the global importance of preserving these ecosystems."  
\begin{enumerate}[label=\Alph*.]
    \item Coral reefs only benefit humans.  
    \item Coral reefs are not important to marine life.  
    \item Coral reefs are vital ecosystems that face serious threats.  
    \item Coral reefs are unaffected by climate change.  
\end{enumerate}

\vspace{1cm}
\newpage
2. \textbf{Which statement is supported by evidence in the passage below?\\}
"Bees are vital to ecosystems and food production. As pollinators, they ensure the reproduction of many plants, including fruits, vegetables, and nuts. Bees are directly responsible for the survival of over 80 percent of flowering plant species. Without bees, food supply chains would collapse, leading to shortages of key crops like apples, almonds, and blueberries. In addition to food production, bees contribute to \\biodiversity by supporting ecosystems that depend on pollination. Unfortunately, bee populations are declining at alarming rates due to habitat loss, pesticide use, and climate change. Urbanization has replaced wildflower meadows with concrete landscapes, depriving bees of food sources. Pesticides used in agriculture weaken bees' immune systems and disrupt their ability to forage. Climate change alters blooming cycles, creating mismatches between bees and the flowers they pollinate. To protect bees, conservationists encourage planting native wildflowers, reducing pesticide use, and supporting local beekeeping initiatives. These efforts help maintain ecological balance and food security."  
\begin{enumerate}[label=\Alph*.]
    \item Bees are not important for food production.  
    \item Habitat destruction has no impact on bees.  
    \item Protecting bees is essential to maintaining food supplies.  
    \item Climate change does not affect bee populations.  
\end{enumerate}

\vspace{1cm}
\newpage
3. \textbf{What explicit detail from the passage supports the claim that "Renewable energy benefits the environment"?\\}
"Renewable energy, such as solar and wind power, is a sustainable alternative to fossil fuels. Solar panels capture sunlight to generate electricity, while wind turbines harness wind energy to power homes and businesses. Unlike fossil fuels, renewables do not emit greenhouse gases, which helps improve air quality and combat climate change. Additionally, renewable energy reduces the extraction of nonrenewable resources, such as coal and oil, which often involve environmentally \\destructive practices. Transitioning to renewable energy creates economic \\opportunities, including jobs in manufacturing, installation, and maintenance. \\Despite the upfront costs, renewable energy technologies become more affordable over time, making them accessible to a wider population. Countries investing in renewables have seen long-term benefits, including energy independence and \\reduced reliance on imported fuels. Public policies, like tax incentives and subsidies, further encourage the adoption of renewable energy. As global energy demand rises, renewable energy plays a critical role in creating a cleaner, more sustainable future."  
\begin{enumerate}[label=\Alph*.]
    \item Renewable energy creates job opportunities.  
    \item Solar panels generate electricity from sunlight.  
    \item Renewable energy improves air quality by reducing emissions.  
    \item Fossil fuels do not impact climate change.  
\end{enumerate}



\subsection*{Part 2: Select All That Apply Questions}

4. Select \textbf{all} statements that are supported by evidence in the text:  
"Plastic pollution harms marine ecosystems. Sea turtles often mistake plastic bags for jellyfish, \\swallowing them and blocking their digestive systems. Microplastics, tiny fragments of plastic, are ingested by fish and enter the food chain, affecting human health. Furthermore, discarded fishing nets trap and kill marine animals, while floating debris damages coral reefs. Reducing plastic use and cleaning up beaches can help protect marine life."  
\begin{enumerate}[label=\Alph*.]
    \item Microplastics can harm humans who consume fish.  
    \item Plastic debris has no impact on coral reefs.  
    \item Reducing plastic use can help protect marine ecosystems.  
    \item Plastic pollution benefits sea turtles.  
\end{enumerate}

\vspace{1cm}

5. Which details explain the threats faced by coral reefs?  
"Coral reefs are dying at an alarming rate. Rising ocean temperatures cause coral bleaching, a condition that weakens reefs by expelling the algae they depend on for energy. Pollution from agricultural runoff leads to nutrient imbalances, harming marine life. Overfishing disrupts reef ecosystems, while coastal development destroys habitats. Conservation efforts, like marine protected areas and reef restoration projects, aim to preserve these ecosystems."  
\begin{enumerate}[label=\Alph*.]
    \item Coral bleaching is caused by rising ocean temperatures.  
    \item Agricultural runoff has no impact on reefs.  
    \item Overfishing disrupts reef ecosystems.  
    \item Coastal development destroys habitats.  
\end{enumerate}

\vspace{1cm}

6. Select \textbf{all} statements that cite explicit evidence from the text:  
"Forests provide numerous benefits to the environment. They absorb carbon dioxide, releasing oxygen into the atmosphere. Forests also prevent soil erosion, regulate water cycles, and support biodiversity by providing habitats for countless species. However, \\deforestation threatens these benefits, contributing to climate change and habitat loss. Reforestation programs and sustainable logging practices are crucial for \\protecting forests."  
\begin{enumerate}[label=\Alph*.]
    \item Forests release oxygen and absorb carbon dioxide.  
    \item Deforestation helps maintain biodiversity.  
    \item Forests prevent soil erosion.  
    \item Reforestation programs protect forest ecosystems.  
\end{enumerate}

\vspace{1cm}
\newpage
\subsection*{Part 3: Short Answer Questions}

7. Based on the passage about bees from question 2, what is one reason their protection is critical? Cite specific evidence from the text.  
\vspace{4cm}

8. How does renewable energy help reduce climate change? Provide textual evidence from the passage from question 3 to support your answer.  
\vspace{4cm}

\subsection*{Part 4: Fill in the Blank Question}
\vspace{1cm}
9. Evidence from a text that is \underline{\hspace{4cm}} is directly stated in the text.
\vspace{2cm}

10. An  \underline{\hspace{4cm}} is an educated guess based on details from the text.
% \newpage
% \subsection*{Answer Key}

% \textbf{Part 1: Multiple-Choice Questions}

% 1. C  
% 2. C  
% 3. C  

% \textbf{Part 2: Select All That Apply Questions}

% 4. A, C  
% 5. A, C, D  
% 6. A, C, D  

% \textbf{Part 3: Short Answer Questions}

% 7. Bees are critical because they ensure the reproduction of many plants and are directly responsible for the survival of over 80 percent of flowering plant species. Without bees, food supply chains would collapse, leading to shortages of crops like apples and almonds.  

% 8. Renewable energy helps reduce climate change by not emitting greenhouse gases, improving air quality, and reducing the extraction of nonrenewable resources. This contributes to combating climate change and creating a sustainable future.  

% \textbf{Part 4: Fill in the Blank Question}

% 9. explicit  
% 10. inference

\end{document}

